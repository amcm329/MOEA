%Autor: Aarón Martín Castillo Medina.
%Asesora: Dra. Katya Rodríguez Vázquez
%Contacto: katya.rodriguez@iimas.unam.mx; amcm329@hotmail.com

%Este archivo contiene información relacionada con la Introducción,
%la cual manifiesta la motivación y contenido del trabajo de tesis,
%en éste último se detallan los temas vistos en cada capítulo.


%Se indica que el documento es de tipo reporte bajo el paquete standalone.
\documentclass[class=report, crop=false]{standalone}

%Se cargan todos los paquetes que residen en el archivo packages_used_standard.sty
\usepackage{packages_used_standard}

%Comienza el documento.
\begin{document}

%Se reinicia la numeración de paginación a 1.
\pagenumbering{gobble}

%Dado que el archivo .tex anterior se encuentra en numeración romana,
%con esta directiva se indica que se comienza con la numeración arábiga.
\pagenumbering{arabic} 

%Tomando en cuenta que la Tabla de Contenido tiene prioridad y por
%ende se le asigna la página 1, con esta directiva a la tabla se le 
%asigna la página 0 y entonces el capítulo empieza con la página 1.
\setcounter{page}{0}

%Comienza el capítulo relacionado con la Introducción de la obra escrita,
%indicando su contenido y detalle de cada sección incluida.
\chapter{Introducción}
El presente documento tiene como meta otorgar un preámbulo 
preciso relativo a los Algoritmos Evolutivos Multiobjetivo \textbf{(en inglés Multi-Objective Evolutionary Algorithms ó
simplemente M.O.E.A.s)} haciendo énfasis en su descripción, 
modelado y análisis mediante evaluadores de desempeño, 
apoyándose en un programa propiamente creado para este 
trabajo nombrado \textbf{M.O.E.A. Software}.\medskip\break
La motivación detrás de dicho desarrollo se debe a que, 
desde el punto de vista del autor, la inmersión a este tema 
con un enfoque pragmático enriquece los comentarios, discusiones 
y retroalimentaciones suscitadas alrededor de este proyecto; 
por otra parte la visualización de resultados de manera expedita 
facilita la comprensión de los elementos que conforman tanto 
a este documento como al producto de software.\medskip\break
Entonces, concretando ideas, se persiguen principalmente dos 
objetivos:

\begin{itemize}
\item Crear un producto de software que emule las características 
y funcionalidades de los M.O.E.A.s \textbf{(M.O.E.A. Software)} 
con la finalidad de incentivar a las personas a acercarse y 
operar con este tipo de técnicas.
\item Usando el producto de software antes mencionado, realizar 
un análisis básico sobre los resultados derivados del uso de 
los M.O.E.A.s para determinar su comportamiento y desempeño 
tanto teórico como práctico, tomando como fundamento las 
métricas desarrolladas para esta finalidad.
\end{itemize}

Con respecto del primer objetivo, es importante añadir que, si 
bien el producto de software por sí solo representa un tópico 
por separado, para fines de este proyecto se le tratará como 
el medio y no el fin; no obstante se agregan algunos recursos 
que ilustran su construcción y uso con la finalidad de que 
quien así lo desee pueda modificar o potenciar su estructura.\break
Lo anterior se encuentra orientado hacia una perspectiva técnica.\medskip\break 
Apuntando al segundo objetivo, para lograr un acercamiento preciso 
del tema sin saturar de información al lector se lleva a cabo 
la revisión de los cuatro M.O.E.A.s más representativos, los 
cuales son:

\begin{itemize}
\item {V.E.G.A. \textbf{(Vector Evaluated Genetic Algorithm)}.}
\item {M.O.G.A. \textbf{(Multi-Objective Genetic Algorithm)}.}
\item {S.P.E.A. 2 \textbf{(Strength Pareto Evolutionary Algorithm 2)}.}
\item {N.S.G.A. II \textbf{(Non-Dominated Sorting Genetic Algorithm II)}.}
\end{itemize}

La selección determinada con anterioridad implica la existencia 
de un listado extenso de técnicas que si bien son importantes, 
no son imprescindibles para una primera aproximación por parte 
del lector.\break 
A pesar de esto, se proporcionarán las referencias correspondientes 
para que se puedan estudiar adicionalmente.\break
El procedimiento es similar para con las métricas de desempeño.\medskip\break
Un detalle importante a considerar es que durante el desarrollo 
del escrito se usarán los términos ``lector'' y ``usuario'' \textbf{(este último sobre todo en la 
sección \hyperref[sec:c1_1]{\textcolor{blue}{Apéndice}})}; para 
estos fines ambos términos se referirán a la persona que incursione 
en este escrito independientemente de su curiosidad teórica o 
práctica.\break
Análogamente dentro de la parte técnica los términos ``programa'', 
``producto'' y ``producto de software'' se considerarán sinónimos, 
ya que éstos describen a \textbf{M.O.E.A Software}.\medskip\break
Recapitulando, la obra completa consta de los siguientes elementos:

\begin{itemize}
\item Trabajo Escrito.
\item M.O.E.A. Software \textbf{(código fuente y ejecutables, para mayores detalles 
véase la subsección \hyperref[sec:c1_2]{\textcolor{blue}{Características Técnicas}} 
del Apéndice)}
\item Manual Técnico de M.O.E.A. Software \textbf{(anexado en un documento aparte)}.
\end{itemize}
 
Cada uno de estos elementos funge como un complemento de los 
demás para garantizar la dispersión adecuada de información 
sin redundancias innecesarias, además en éstos se proporciona 
terminología exclusiva que pudiera entorpecer la comprensión 
del tema si se localizara en otras secciones.\medskip\break
Es menester mencionar que, el Manual Técnico abarca única 
y exclusivamente el contenido de M.O.E.A Software, indicando 
al usuario los componentes, relaciones con los demás y el 
detalle de sus funcionalidades.\medskip\break
En lo concerniente al contenido de la parte escrita, el lector 
notará la existencia de 4 capítulos adicionales, la Bibliografía 
y el Apéndice.\break
En el Capítulo 2 se muestran todos los fundamentos que dan pie 
a los Algoritmos Evolutivos Multiobjectivo, es decir, los temas 
que, combinados entre sí originan el tema de estudio principal 
de este trabajo de tesis.\break
Debido a que cada uno por sí mismo se considera un tema aparte 
sólo se tratarán las características necesarias para poder 
enlazar adecuadamente un tópico con los demás y así converger 
apropiadamente al tema en cuestión.\medskip\break
El Capítulo 3 es la sección dedicada enteramente al M.O.E.A., 
en ésta se encuentran algunos elementos adicionales como 
son ciertos criterios de selección específicos \textbf{(Shared Fitness ó Fitness Compartido)}, 
evaluadores de desempeño, los algoritmos más representativos 
mencionados con anterioridad, los conjuntos de pruebas 
para determinar su desempeño \textbf{(M.O.P, Multi-Objective Problem ó Problema Multiobjetivo)}
así como una introducción al uso del producto de software.\medskip\break
El Capítulo 4 está dirigido hacia las pruebas y análisis de 
los algoritmos con el uso del programa tomando como punto de 
referencia las pautas de desempeño establecidos en el Capítulo 3.\medskip\break
En el Capítulo 5 se localizan las conclusiones y manifiestos 
del trabajo futuro no sólo con base en los resultados del 
Capítulo 4, sino en general con los elementos de toda la obra.\medskip\break
Con respecto de la Bibliografía es preciso detallar que también 
contiene Mesografía ya que una considerable cantidad de las 
fuentes utilizadas se obtuvo mediante medios electrónicos.\medskip\break
Finalmente el Apéndice contiene las características técnicas 
alusivas al producto de software, indicando los paquetes y 
versiones empleadas para la construcción de dicho programa, 
así como una ligera introducción al diseño del mismo; para 
referencias técnicas más sucintas es recomendable que se 
revise el Manual Técnico.\medskip\break
En lo relativo a los conocimientos previos a este trabajo que 
se deben adquirir, aún tomando en cuenta el hecho de que se 
abordarán los conceptos necesarios con una explicación suficiente, 
es recomendable que el lector indague un poco en tópicos 
relativos a conjuntos, funciones y operaciones elementales con 
vectores, lo anterior para poder comprender más a profundidad 
las explicaciones propiciadas.\medskip\break
En lo que concierne al producto de software sólo es necesario 
un conocimiento básico en el lenguaje de programación Python 
y por ende en las sentencias elementales de codificación; 
lo anterior es relevante sólo si el usuario decide introducirse 
en el código fuente proporcionado ya que de hecho se incluyen 
programas ejecutables cuya función es la de otorgar un ambiente 
de software totalmente separado de aspectos técnicos y de esta 
manera el usuario únicamente se avoque a la parte analítica.

%Termina el documento.
\end{document}
