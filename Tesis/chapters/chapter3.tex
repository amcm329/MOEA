%Autor: Aarón Martín Castillo Medina.
%Asesora: Dra. Katya Rodríguez Vázquez
%Contacto: katya.rodriguez@iimas.unam.mx; amcm329@hotmail.com

%Este archivo contiene información relacionada con las características,
%algoritmos y aplicaciones relacionados con los M.O.E.A.'s así como las 
%métricas de desempeño que se utilizarán e incluso una introducción al
%producto de software elaborado con base en todo lo anterior.


%Se indica que el documento es de tipo reporte bajo el paquete standalone.
\documentclass[class=report, crop=false]{standalone}

%Se añaden los paquetes a usarse localizados en packages_used_standard.sty
\usepackage{packages_used_standard}

%Comienza el documento.
\begin{document}

%Comienza el capítulo relacionado con la estructura interna,
%ejemplos y métricas alusivas a un M.O.E.A.
\chapter {M.O.E.A.}


Una vez llevada a cabo la definición de los dos preceptos más importantes
de M.O.E.A. entonces podemos definir a un Multi-Objective Evolutionary 
Algorithm como un problema de optimizacón multiobjetivo donde para su
resolución se ocupan algoritmos evolutivos.

Como se ha visto con anterioridad, la gama de técnicas y en general de opciones
ya sea de un lado o del otro es muy amplia, y por ende existe también aquí
una variaedad grande*** de algoritmos que se verificarán con detenimiento.

Hasta este punto se tiene que comentar que el proceso evolutivo tiene que cambiar
ligeramente ya que como se mencionó en el capítulo anterior 
%%%%%%%%%%%%%%%%%%%%%%%%%%%%%%%%%%%%%%%%%%%%%%%%%%%%%%%%%%%%%%



\section{Rankeo}

\section{Aptitud Compartida}  
También llamado Shared Fitness

\section{Algoritmos}
Con base en el capítulo anterior, el funcionamiento de un M.O.E.A. 
\textbf{(resolver un problema de optimización multiobjetivo usando 
algoritmos evolutivos)} generalmente se lleva a cabo de la siguiente 
manera:

\begin{enumerate}[1.]
\item Usando una Representación Cromosómica, crear la Población Padre y evaluar cada uno de los Individuos respecto a las funciones objetivo.

\item Asignar un Ranking a los Individuos de la Población Padre.  

\item Con base en el Ranking, asignar la Aptitud \textbf{(en inglés Fitness)} a cada uno de los Individuos.

\item Tomando en cuenta el Fitness, aplicar las operaciones de Selección, Cruza y Mutación con la finalidad de crear una Población Hija. Todos los métodos empleados en este punto deben funcionar acorde a la Representación Cromosómica del punto 1.

\item \textbf{(Opcional)} Utilizar el Fitness Compartido \textbf{(en inglés Shared Fitness)} para aplicar una elección más minuciosa de los mejores Individuos en la Población Hija. 

\item Designar a la población Hija como la nueva población Padre.

\item Repetir los pasos 2 a 6 hasta haber alcanzado un número límite de generaciones \textbf{(iteraciones)}. 
\end{enumerate}

A grandes rasgos la diferencia entre un M.O.E.A. y otro es la Presión Selectiva 
\textbf{(en inglés Selective Pressure)} que se aplica durante el procedimiento, para fines de este proyecto
se trata de la tolerancia para seleccionar a los Individuos de calidad media o baja frente a los
mejores. Una baja Presión Selectiva permite elegir Individuos no tan aptos; el caso es análogo para
una alta Presión Selectiva.



\subsection{V.E.G.A.}
La forma de proceder del algoritmo es la siguiente:

\begin{enumerate} 
\item Se crea la Población Padre (de tamaño \(n\)).
\item Tomando en cuenta las \(k\) funciones objetivo y la Población Padre, se crean \(k\) subpoblaciones de tamaño \(n/k\) cada una, si este número llega a ser irracional se pueden hacer ajustes con respecto de la distribución de los Individuos.
\item Por cada subpoblación, se aplica la técnica de Selección y obtienen los \(n/k\) Individuos, terminado esto se deben unificar todos los seleccionados de nuevo en una súper Población.
\item Con la súper Población del paso 3, se crea a la población Hija, la cual pasará a convertirse en la la nueva Población Padre.
\item Se repiten los pasos 2 a 4 hasta haber alcanzado el número de generaciones \textbf{(iteraciones)} límite.
\end{enumerate}

Como se puede apreciar es una implementación muy sencilla 
de optimización multiobjetivo, sin embargo el inconveniente 
que tiene es la fácil pérdida de material genético valioso.\break
Lo anterior significa que un Individuo que en una generación 
previa era el mejor para una función objetivo \(i\) al momento 
de ser separado y seleccionado en una subpoblación \(j\) \textbf{(y por ende analizado bajo la función objetivo \(j\))} 
puede ser muy malo en calidad y por tanto no ser seleccionado;
perdiéndose la ganancia genética hasta el momento obtenida para 
la función objetivo \(i;\ i \neq j\).\medskip\break
Por ello es que se puede decir que V.E.G.A. genera soluciones 
promedio que destacan con una calidad media para todas las 
funciones objetivo.\medskip\break
Finalmente hay que comentar que para este algoritmo no se requiere 
aplicar un Ranking específico, no obstante, se ha decidido utilizar 
el de Fonseca \& Flemming \textbf{(véase Model/Community/Community.py)} 
pues es el más sencillo de implementar.

\subsection{M.O.G.A.}
Su funcionamiento es el siguiente:

\begin{enumerate} 
\item Se crea la Población Padre, se evalúan las funciones objetivo de sus correspondientes Individuos.
\item Se asigna a los Individuos un Ranking \textbf{(Fonseca \& Flemming)} y posteriormente se calcula el Niche Count de la Población Padre.
\item Tomando en cuenta los valores del punto 2 se obtiene el Fitness para cada Individuo y posteriormente su Shared Fitness.
\item Se aplica el operador de selección sobre la Población Padre para determinar los elegidos para dejar descendencia.
\item Se crea la Población Hija, se evalúan las funciones objetivo de sus correspondientes Individuos.
\item Se asigna a los Individuos un Ranking \textbf{(Fonseca \& Flemming)} y posteriormente se calcula el Niche Count de la Población Hija.
\item Tomando en cuenta los valores del punto 6 se obtiene el Fitness para cada Individuo y posteriormente su Shared Fitness.
\item La Población Hija pasará a ser la nueva Población Padre.
\item Se repiten los pasos 4 a 8 hasta que se haya alcanzado el número límite de generaciones \textbf{(iteraciones)}.
\end{enumerate}

Como se puede apreciar, la implementación de este algoritmo es 
muy sencilla, además se rige casi en su totalidad por el 
Shared Fitness \textbf{(ó Fitness Compartido)}, por lo que 
la Presión Selectiva \textbf{(ó Selective Pressure)} incluida 
dependerá en gran medida de la función de Distancia que se utilice, 
así como de la magnitud indicada por el usuario.\medskip\break
Finalmente es menester mencionar que para esta implementación el 
Ranking utilizado debe ser estrictamente el de Fonseca \& Flemming 

\subsection{S.P.E.A. 2}
Se desarrolla la implementación de la técnica M.O.E.A. 
conocida como S.P.E.A. II \textbf{(Strength Pareto Evolutionary Algorithm ó Algoritmo Evolutivo de Fuerza de Pareto)}.\break
El funcionamiento del algoritmo es el siguiente:

\begin{enumerate}
\item Se inicializa una población llamada \emph{P} y un conjunto inicialmente vacío llamado \emph{E} \textbf{(E albergará Individuos también)}; ambos son de tamaño n.
\item Se asigna el Fitness a los Individuos de \emph{P} y \emph{E} \textbf{(para ello se evalúan las funciones objetivo de los Individuos de ambos conjuntos y se asigna el Ranking Zitzler \& Thiele)}.
\item A continuación se funden \emph{P} y \emph{E} en una súper Población \textbf{(llamémosle S también señalado en el algoritmo como Mating Pool, de tamaño n)}.Para ello primero se añaden los Individuos \emph{NO DOMINADOS} de \emph{P} en \emph{S} y posteriormente los \emph{NO DOMINADOS} de \emph{E} en \emph{S}.\break 
Aquí se distinguen dos casos:
      \begin{itemize}
      \item Si llegasen a faltar Individuos se añaden al azar Individuos \emph{DOMINADOS} de \emph{P} en \emph{S} hasta completar la demanda.
      \item Si después de la fusión el número de Individuos supera a n, entonces se hace un truncamiento en \emph{S} hasta ajustar su tamaño a n.
      \end{itemize}
\item \emph{S} será la nueva \emph{E}, además se crea la población Hija de la recién creada \emph{E} \textbf{(E-Child)}.
\item E-Child será la nueva P.
\item Se repiten los pasos 2 a 5 hasta que se haya alcanzado el límite de generaciones \textbf{(iteraciones)}.
\end{enumerate}

Finalmente lo que se regresa es \emph{E}, ya que ahí es 
donde se han almacenado los mejores Individuos de todas 
las generaciones.\medskip\break
La característica de este algoritmo es que tiene una 
Presión Selectiva alta ya que se da prioridad a los 
Individuos no dominados \textbf{(de ahí el nombre de Fuerza de Pareto ó los más fuertes con respecto al principio de Pareto)},
y el hecho de mezclar a \emph{E} y \emph{P} en una 
súper Población garantiza la conservación de los mejores 
Individuos sin importar el transcurso de las generaciones \textbf{(a eso se le conoce como Elitismo)}, 
pero también da una tolerancia, aunque mínima, a los Individuos 
de menor calidad como en el punto 3.\break
Además al momento de actualizar \emph{S} a \emph{E} y 
E-Child a \emph{P} se tiene una especie de seguro de vida, 
es decir, si en algún momento la población E-Child llegara a
tener una calidad baja se tiene el respaldo de \emph{E} 
para una generación posterior para formar \emph{S}.\medskip\break
Se debe tener en cuenta que el algoritmo originalmente no 
contempla ni una súper Población \emph{S} ni E-Child 
sino que en los pasos 3 y 4 se utiliza solamente \emph{E} 
para referirse tanto a E-child como a \emph{S}, sin embargo 
para no confundir al usuario en la funcionalidad del método 
se decidió colocar contenedores extra para poder diferenciar 
más precisamente a los elementos involucrados.\medskip\break
Algo muy importante a mencionar es que en el paso 1 y al momento 
de crear la población E-Child es necesario evaluar las funciones 
objetivo, asignar un Ranking y posteriormente un Fitness para 
que se puedan aplicar los operadores geneticos \textbf{(véase Model/GeneticOperator)}, 
para este caso el Ranking es estrictamente el de Zitzler \& Thiele; 

\subsection{N.S.G.A. II}
La forma de proceder del método es la siguiente:

\begin{enumerate}
\item Se crea una Población Padre \textbf{(de tamaño n)}, a la cual se le evalúan las funciones objetivo de sus Individuos, se les asigna un Ranking \textbf{(Goldberg)} y posteriormente se les otorga un Fitness.
\item Con base en la Población Padre se aplica el operador de Selección para elegir a los Individuos que serán aptos para reproducirse.
\item Usando a los elementos del punto 2, se crea una Población Hija \textbf{(de tamaño n)}.
\item Se crea una súper Población \textbf{(llamémosle S, de tamaño 2n)} que albergará todos los Individuos tanto de la Población Padre como Hija; a \emph{S} se le evalúan las funciones objetivo de sus Individuos, se les asigna un Ranking \textbf{(Goldberg)} y posteriormente se les otorga un Fitness.
\item La súper Población \emph{S} se divide en subcategorías de acuerdo a los niveles de dominancia que existan, es decir, existirá la categoría de dominancia 0, la cual almacena Individuos que tengan una dominancia de 0 Individuos \textbf{(ningún Individuo los domina)}, existirá la categoría de dominancia 1 con el significado análogo y así sucesivamente hasta haber cubierto todos los niveles de dominancia existentes.
\item Se construye la nueva Población Padre, pare ello constará de los Individuos de \emph{S} donde la prioridad será el nivel de dominancia, es decir, primero se añaden los elementos del nivel 0,luego los del nivel 1 y así en lo sucesivo hasta haber adquirido n elementos.
Se debe aclarar que la adquisición de Individuos por nivel debe ser total, esto significa que no se pueden dejar Individuos sueltos para el mismo nivel de dominancia.\break
Supongamos que a un nivel k existen tantos Individuos que su presunta adquisición supera el tamaño n, en este caso se debe hacer lo siguiente:
      \begin{enumerate}
      \item Se crea una Población provisional \textbf{(Prov)} con los Individuos del nivel k, se evalúan las funciones objetivo a cada uno de sus Individuos, se les asigna un Ranking \textbf{(Goldberg)} y posteriormente se les asigna el Fitness.\break
            Con los valores anteriores se calcula el Niche Count \textbf{(véase Model/SharingFunction)} de los Individuos; una vez hecho ésto se seleccionan desde Prov los Individuos faltantes con los mayores Niche Count, esto hasta completar el tamaño n de la nueva Población Padre.
      \end{enumerate}
\item Al haber conformado la nueva Población Padre, se evalúan las funciones objetivo de sus Individuos, se les asigna el Ranking correspondiente \textbf{(Goldberg)} y se les atribuye su Fitness.
\item Se repiten los pasos 2 a 7 hasta haber alcanzado el límite de generaciones \textbf{(iteraciones)}.
\end{enumerate}

Como su nombre lo indica, la característica de este algoritmo es 
la clasificación de los Individuos en niveles para su posterior 
selección.\break
Esto al principio propicia una Presión Selectiva moderada por 
la ausencia de elementos con dominancia baja que suele existir 
en las primeras generaciones, sin embargo en iteraciones posteriores 
se agudiza la Presión Selectiva ya que eventualmente la mayoría de 
los Individuos serán alojados en las primeras categorías de dominancia, 
cubriendo casi instantáneamente la demanda de Individuos necesaria en 
el paso 6, por lo que las categorías posteriores serán cada vez 
menos necesarias con el paso de los ciclos.\medskip\break
Por otra parte la fusión de las Poblaciones en \emph{S} garantiza 
que siempre se conserven a los mejores Individuos independientemente 
de la generación transcurrida, a eso se le llama Elitismo.\break
Por cierto que en el algoritmo original no existe un nombre oficial 
para \emph{S} sino más bien se señala como una estructura genérica, 
sin embargo se le ha formalizado con un identificador para guiar 
apropiadamente al usuario en el flujo del algoritmo.\medskip\break
Para finalizar se señala que el uso del ranking de Goldberg 

\section{M.O.P.}      
\section{Indicadores de Desempeño}
\section{Introducción a M.O.E.A Software}
Habiendo detallado todos los elementos 

M.O.E.A. Software surge como una solución ante la problemática blablabla.

Para abordar detalles de índole más técnica se recomienda visitar la sección **Características Técnicas** perteneciente al Apéndice.

\label{sec:p_1}    

%Termina el documento.
\end{document}
