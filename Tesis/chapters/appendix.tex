%Autor: Aarón Martín Castillo Medina.
%Asesora: Dra. Katya Rodríguez Vázquez
%Contacto: katya.rodriguez@iimas.unam.mx; amcm329@hotmail.com

%Este archivo contiene información relativa al Apéndice de la tesis,
%más en específico se muestran las característivas básicas del programa
%M.O.E.A. Software.
%Para profundizar en detalles relacionados con el contenido y organización
%del código fuente se recomienda ver el Manual Técnico.


%Se indica que el documento es de tipo reporte bajo el paquete standalone.
\documentclass[class=report, crop=false]{standalone}

%Se cargan todos los paquetes que residen en el archivo packages_used_standard.sty
\usepackage{packages_used_standard}

%Comienza el documento.
\begin{document}

%Empieza el capítulo relacionado con el Apéndice.
%Se coloca la estructura \chapter* para ocultar 
%el capítulo actual en la Tabla de Contenido y coloca
%una versión personalizada.
\chapter*{Apéndice}

%Con esta línea se añade el capítulo personalizado
%del que se habló previamente.
\addcontentsline{toc}{chapter}{Apéndice}\markboth{Apéndice}{}

%Se coloca el vínculo interno procedente del capítulo 1 (c1_1).
\label{sec:c1_1}

%Para que la sección del Apéndice en la Tabla de Contenido
%NO tenga subsecciones visibles.
\addtocontents{toc}{\protect\setcounter{tocdepth}{-1}}
Esta sección corresponde a la incursión de detalles elementales 
relacionados con el producto de software denominado \textbf{M.O.E.A. Software}, 
la cual recopila las características de dicho producto tales 
como el diseño y datos alusivos a la construcción y ejecución 
del programa.\break
Con base en lo anterior, es preciso mencionar que en este 
apartado existe terminología meramente técnica que se contempla 
para garantizar una mayor comprensión del proyecto.\medskip\break
Habiendo proporcionado el anterior preámbulo, la finalidad es, 
a grandes rasgos, otorgar una guía rápida y concisa al usuario 
en lo concerniente a la motivación e implementación del producto 
de software, la idea detrás de ésto radica en notificar al usuario 
de todos los pormenores encontrados en el desarrollo del programa 
y de esta manera lograr que su interacción con el producto sea 
afable, evitando para ello la mayor cantidad de contratiempos 
posible.\medskip\break
Es importante mencionar que en caso de querer ahondar más 
minuciosamente en los componentes del programa se ofrece un 
\textbf{Manual Técnico} que contiene precisadas todas las funcionalidades 
con su respectiva explicación, las cuales han sido creadas para 
poder erigir el producto de software .\break
Dicho manual se ha constituido en un documento independiente al 
trabajo de tesis que, aunque guarda una cierta distancia con los 
temas que se revisan aquí al estar asociado a un panorama más 
técnico, también contiene referencias de carácter analítico para 
poder enlazar adecuadamente lo estipulado en este medio con aquél.\medskip\break
A continuación se abarca más detalladamente cada uno de los rasgos 
del producto de software.

%Ahora comienza la sección enfocada hacia las Características de 
%Diseño, las cuales son las bases lógicas en las que se basa el
%producto de software.
\section{Características de Diseño}
En esta parte se considera todo lo relacionado con la arquitectura 
de diseño.\break
Este concepto corresponde al tipo de organización que se suele 
emplear para agrupar y comunicar apropiadamente cada uno de los 
componentes del producto de software con la finalidad de minimizar 
los tiempos de corrección y actualización del código, así como 
proporcionar una presentación digerible para cualquiera que desee 
familiarizarse con las partes codificadas.\medskip\break
Para este proyecto se ha elegido la arquitectura denominada MVC 
\textbf{(Model-View-Controller ó Modelo-Vista-Controlador)}.\break
Siguiendo este tipo de organización, se colocan las funcionalidades 
en tres categorías principales, que son:

\begin{itemize}
\item \textbf{Model (ó Modelo)}, se almacenan todos los elementos 
que realizan el proceso analítico, en este caso todo lo relacionado 
con la ejecución de M.O.E.A.’s y la recolección de resultados.
\item \textbf{View (ó Vista)}, se coloca todo aquéllo asociado a 
la interfaz gráfica del programa y en el caso del proyecto, la 
graficación apropiada de resultados.
\item \textbf{Controller (ó Controlador)}, se guarda toda la 
parafernalia relativa al control de las comunicaciones entre la 
Vista y el Modelo.
\end{itemize}

El proceso usual de interacción entre dichas categorías es el 
siguiente:

\begin{enumerate}
\item El usuario inserta las configuraciones pertinentes en la 
Vista, las cuales permitirán obtener resultados detallados del 
M.O.E.A. que se fuera a ejecutar.
\item El Controlador obtiene las configuraciones previamente 
insertadas por el usuario; durante esta etapa se realiza una 
verificación y saneamiento de dichas configuraciones. Si el 
proceso fue exitoso se procede ir al paso \textbf{(3)}, en 
cualquier otro caso se retrocede al paso \textbf{(1)} con una 
notificación de error.
\item El Modelo se encarga de ejecutar el algoritmo precisado 
por el usuario en \textbf{(1)}, para ello se le proporcionan 
todas las configuraciones adjuntas. Una vez concluido el proceso 
el Modelo le regresa los resultados al Controlador.
\item El Controlador toma los resultados y a su vez los transfiere 
a la Vista, la cual se encarga de mostrar al usuario los datos 
finales de manera gráfica.
\end{enumerate}

Señalando nuevamente al Manual Técnico, en éste el usuario notará 
que las funcionalidades están colocadas siguiendo un listado 
basado en la arquitectura antes mencionada.\break
Esto permite vislumbrar de manera secundaria los elementos y 
sus relaciones.

%Comienza la sección concerniente a las Características Técnicas,
%esto es, la mención de las herramientas que se utilizaron para
%la elaboración del programa.
\section{Características Técnicas}
%Se coloca el vínculo interno procedente del capítulo 1 (c1_2).
\label{sec:c1_2}
Adentrándose en una perspectiva tecnológica el programa 
\textbf{M.O.E.A. Software} está elaborado usando el lenguaje 
de programación Python en su versión 2.7.3. 
\textbf{\href{http://www.python.org}{\break\textcolor{blue}{(http://www.python.org)}}}\medskip\break
Conviene primeramente explicar la motivación detrás del uso de 
dicho lenguaje; desde el punto de vista del autor éste es 
sintácticamente fácil de aprender lo cual implica un esfuerzo 
reducido en la lectura y comprensión del código fuente por 
lo que de esta manera el usuario puede familiarizarse 
rápidamente con las funcionalidades elaboradas.\break
Por otra parte, al ser un lenguaje de alto nivel, éste permite 
la agrupación de varias instrucciones en pocos comandos 
\textbf{(a esto se le conoce también como azúcar sintáctica)}, 
como resultado se crea un código no tan extenso que de nueva 
cuenta agiliza su interpretación  por parte del usuario.\medskip\break
Llegados a este punto se pudiera uno cuestionar la existencia 
de otros productos desoftware cuyo fin se pareciera o fuera 
el mismo que el que se muestra en el presente trabajo de tesis.\break
Si bien existen paquetes que realizan operaciones relacionadas 
tanto con la Optimización Multiobjetivo como con los Algoritmos 
Evolutivos, como los siguientes:
\begin{itemize}
\item \textbf{DEAP, Distributed Evolutionary Algorithms in Python}\break\textbf{\href{https://pypi.python.org/pypi/deap}{\textcolor{blue}{(https://pypi.python.org/pypi/deap)}}}.
\item \textbf{PyGMO, Python Parallel Global Multiobjective Optimizer}\break\textbf{\href{http://esa.github.io/pygmo/}{\textcolor{blue}{(http://esa.github.io/pygmo/)}}}.
\item \textbf{jMetal, Metaheuristic Algorithms in Java}\break\textbf{\href{http://jmetal.sourceforge.net/}{\textcolor{blue}{(http://jmetal.sourceforge.net/)}}}.
\item \textbf{MOEA Framework} \textbf{\href{http://moeaframework.org/}{\textcolor{blue}{(http://moeaframework.org/)}}}.
\item \textbf{Borg MOEA} \textbf{\href{http://borgmoea.org/}{\textcolor{blue}{(http://borgmoea.org/)}}}.
\end{itemize}

Se pueden percibir ciertas diferencias, en primera instancia 
podemos señalar el lenguaje de programación utilizado, pues 
jMetal y MOEA Framework se han elaborado usando el lenguaje 
de programación Java; desde otra perspectiva el paquete PyGMO 
aunque ofrece técnicas de Optimización Multiobjetivo éstas no 
se centran en el uso de Algoritmos Evolutivos precisamente sino 
que ofrece una gama aparte de métodosrelacionados con el primer 
tema.\break
Enfocándose en la disponibilidad Borg MOEA resulta ser un 
programa muy completo, sin embargo su uso se restringe sólo a 
ciertas operaciones de tipo comercial o estudiantil por lo que 
en ese sentido su extensión se encuentra limitada.\break
Finalmente el paquete que más se acerca, DEAP, aunque maneja un 
catálogo mayor de Algoritmos Evolutivos no considera la 
graficación de resultados además de que es preciso aprender la 
sintaxis misma del paquete para poder operar con las funcionalidades 
y ello implica que el uso está dirigido hacia un público más 
especializado en el tema.\medskip\break
Dicho lo anterior, sin desmerecer el esfuerzo y dedicación que 
se ha puesto en cada uno de los trabajos antes mencionados, 
con base en estas comparaciones se podrá aseverar que el desarrollo 
de \textbf{M.O.E.A. Software} tiene finalidades distintas, ya 
que, como se ha mencionado con anterioridad, su uso se enfoca 
hacia el acercamiento inicial de los Algoritmos Evolutivos 
Multiobjetivo desde un panorama gráfico donde para ello el usuario 
tiene total libertad de revisar y/o modificar el código fuente 
anexado sin necesidad de ninguna capa intermedia de aprendizaje 
debido a que todo el proyecto ha sido escrito usando la sintaxis 
más simple.\medskip\break
En lo concerniente al contenido del producto de software, lo que 
se incluye en el proyecto son tanto los ejecutables como el código 
fuente.\break 
La principal diferencia entre éstos radica en que los primeros 
son programas que contienen todas las dependencias necesarias 
ya anexadas para que el usuario únicamente se enfoque en la 
ejecución e interacción con el producto de software, por otra 
parte el código fuente almacena las funcionalidades desarrolladas 
de \textbf{M.O.E.A. Software} pero para su ejecución es preciso 
que el usuario instale algunas dependencias en su sistema 
operativo.\medskip\break
Para los ejecutables se ha determinado enfocarse en sistemas 
operativos Windows y GNU/Linux por su popularidad \cite{b8}, 
sin embargo, debido a una falla de origen con una de las 
dependencias usadas en el producto de software \textbf{(Matplotlib)} 
se ha tenido la necesidad de crear un ejecutable por cada 
sistema operativo. De esta manera los que se encuentran soportados 
son:

\begin{itemize}
\item \textbf{Windows}.
\item \textbf{Debian}.
\item \textbf{Ubuntu}.
\item \textbf{CentOS}.
\item \textbf{Fedora}.
\end{itemize}

No importa la versión utilizada siempre y cuando sea del mismo 
linaje de sistema operativo empleado.\medskip\break
Algo muy importante a mencionar es que los ejecutables han sido 
creados para sistemas operativos de 32 bits, esto ya que en 
teoría los programas hechos para sistemas de 32 bits son admisibles 
en sistemas de 64 bits, no así el caso contrario.\break
A pesar de esto, mientras este factor en los sistemas operativos 
Windows no presenta problema, en los relativos a GNU/Linux de 
64 bits sí puesto que se ha detectado que éstos no contienen los 
paquetes necesarios para poder ejecutar programas de 32 bits.\break
Dado que este proyecto no contempla este tipo de paquetes y 
hasta el término de este proyecto no se han podido integrar a 
los ejecutables es imprescindible alertar al usuario sobre estas 
condiciones.\break
Lo ideal sería el uso de un sistema operativo de 32 bits pero en 
caso de no ser posible se deben instalar los paquetes que permitan 
la ejecución de programas de 32 bits.\medskip\break
Es menester mencionar el programa utilizado para la elaboración 
de los ejecutables, para este proyecto se ha utilizado la herramienta 
\textbf{PyInstaller (versión 3.2) \href{http://www.pyinstaller.org}{\break\textcolor{blue}{(http://www.pyinstaller.org)}}}.\break
A grandes rasgos lo que realiza este ejecutable es introducir el 
código fuente,paqueterías y demás aditamentos necesarios dentro 
de un contenedor que eventualmente llega a ser el ejecutable.\break
Dicho de otra forma, PyInstaller crea un ínfimo ambiente controlado 
en el quese puede ejecutar \textbf{M.O.E.A. Software} dando la 
ilusión de que se ha creado un programa ejecutable independiente.\medskip\break
En el peor de los escenarios, si no se tiene la posibilidad de 
operar con los archivos ejecutables ya sea por el problema antes 
mencionado o porque los sistemas soportados no coinciden con el 
sistema del usuario se ofrece el código fuente, al estar elaborado 
en Python y éste a su vez ser un lenguaje multiplataforma 
\textbf{(se puede instalar en cualquier sistema operativo)} como 
consecuencia lógica se indaga que el código fuente puede ser 
ejecutado en cualquier sistema si se cuenta con las dependencias 
adecuadas.\medskip\break
Como se ha mencionado anteriormente el producto de software ha 
sido escrito usando la sintaxis más simple y ello implica que se 
ha construido utilizando sólo las dependencias necesarias para 
garantizar que aún empleando el código fuente se conciba la máxima 
independencia posible del producto de software.\medskip\break
Para utilizar el código fuente es necesario, además de la instalación 
obligatoria de Python 2.7.3 \textbf{(otras versiones causarían problemas de incompatibilidad)}
las siguientes dependencias:

\begin{itemize}
\item \textbf{Tkinter}, versión 8.5 \textbf{\break\href{http://tkinter.unpythonic.net/wiki/How\_to\_install\_Tkinter}{\textcolor{blue}{(http://tkinter.unpythonic.net/wiki/How\_to\_install\_Tkinter)}}}.
\item \textbf{Matplotlib}, versión 1.1.1rc2 \textbf{\href{http://matplotlib.org/}{\textcolor{blue}{(http://matplotlib.org/)}}}.
\end{itemize}

La preferencia por las versiones de las dependencias no es 
estricta, no obstante se recomienda seguir estas indicaciones 
lo más fielmente posible para garantizar resultados satisfactorios.\break
De manera similar para con los archivos ejecutables, dado que 
los sistemas operativos en los que se use el programa pueden 
variar sólo se indican las bibliotecas empleadas, le corresponde 
al usuario instalarlas en su sistema operativo.\medskip\break
Por otra parte el desarrollo del programa se llevó a cabo 
primordialmente usando entornos virtuales con la ayuda de 
la herramienta \textbf{VirtualBox \break\href{https://www.virtualbox.org/}{\textcolor{blue}{(https://www.virtualbox.org/)}}}\break
Entonces, los sistemas operativos creados fueron:

\begin{itemize}
\item \textbf{Windows 7 \textit{Home Premium} (32 bits)}; 1GB Memoria RAM, Procesador Intel Core i5.
\item \textbf{Debian 7 \textit{Wheezy} (32 bits)}; 1GB Memoria RAM, Procesador Intel Core i5.
\end{itemize}

Tanto la construcción del código como las pruebas consecuentes 
fueron hechas en estos sistemas debido a que se necesitaba 
asegurarse que el producto de software funcionara en los entornos 
principales Windows y GNU/Linux.\medskip\break
Se considera importante mencionar las características de los 
sistemas operativos creados con la finalidad de enfatizar la 
elaboración del producto de software bajo escenarios con 
condiciones paupérrimas; de esta manera se garantiza un rendimiento 
favorable en sistemas con los mismos o mayores recursos.\medskip\break
Para construir los ejecutables de los demás sistemas operativos 
se crearonpor igual sistemas virtualizados, sin embargo dado 
que sólo fueron utilizados para los fines señalados no se 
considera importante profundizar más en sus características.\medskip\break
Finalmente es menester mencionar que las instrucciones y 
aditamentos ilustrativos en la interfaz gráfica del programa 
se encuentran plasmados en el idioma inglés,el objetivo de tal 
acción tiene dos metas: pulir las habilidades lingüísticas del 
autor e incentivar a los estudiantes con la práctica de dicho 
idioma, de todas formas la gramática y vocabulario empleados 
son sencillos y no se encontrará dificultad alguna en la comprensión 
del tema.

%Termina el documento.
\end{document}
