%Autor: Aarón Martín Castillo Medina.
%Asesora: Dra. Katya Rodríguez Vázquez
%Contacto: katya.rodriguez@iimas.unam.mx; amcm329@hotmail.com

%Este archivo contiene información relacionada con la Presentación,
%de la tesis que incluye el resumen del trabajo escrito, los agradecimientos
%y la dedicatoria.


%Se indica que el documento es de tipo reporte bajo el paquete standalone.
\documentclass[class=report, crop=false]{standalone}

%Se cargan todos los paquetes que residen en el archivo packages_used_standard.sty
\usepackage{packages_used_standard}

%Empieza el documento.
\begin{document}

%Con esto se indica que el número de página va escrito en numeración
%romana.
\pagenumbering{roman}

%Comienza el capítulo dedicado al Resumen, es decir, muestra un esbozo
%de todo el contenido que posteriormente se desglosará.
\chapter*{\centerline{Resumen}}
Se proporciona una introducción concisa relativa a la definición, 
constitución y ejemplificación de los Algoritmos Evolutivos 
Multiobjetivo \textbf{(en inglés Multi-Objective Evolutionary Algorithms ó simplemente M.O.E.A.s)} 
así como sus evaluadores de desempeño.\medskip\break
Para poder realizar las operaciones pertinentes se toma como 
base un entorno gráfico específicamente elaborado para este 
trabajo, el cual lleva por denominación \textbf{M.O.E.A. Software}; 
con base en éste se realizan pruebas para analizar y corroborar 
las medidas de desempeño contempladas en la teoría.\medskip\break
A su vez el producto de software en sí constituye una pieza 
de divulgación con la finalidad de acercar a los interesados 
en el estudio de este tipo de técnicas a través de un ambiente 
dinámico; por ende se incluye información relacionada con la 
construcción y ejecución del programa antes mencionado para 
guiar a aquéllos que deseen adentrarse en lo concerniente al 
ámbito técnico del producto.

%Después del Resumen se agrega una página en blanco.
\newpage\null\thispagestyle{empty}\newpage

%Comienza el capítulo dedicado a los Agradecimientos tanto académicos
%como personales.
\chapter*{\centerline{Agradecimientos}}

%En construcción
%\begin{comment}

\vspace{-1.4cm}
Primero que nada en el escenario académico quiero darle 
las gracias a la Dra. Katya Rodríguez Vázquez por su magna 
paciencia al momento de dirigir y corregir mi tesis, amén 
de su disposición para reforzar el conocimiento adquirido 
e incluso la transmisión de consejos a través de su experiencia.\break
Además incluyo a los profesores/doctores X Y Z por sus 
invaluables aportes a mi trabajo escrito.\break
También deseo señalar al Dr. Otto Hahn Herrera y a Emiliano 
Valdés Guerrero, por ser un modelo de inspiración y ambición 
cultural y por incluirme en los proyectos en donde obtuve 
sólidas bases profesionales.\break
Nombro a la profesora M. en C. Adriana Espinosa Contreras 
por el conocimiento facilitado, solidaridad, franqueza y 
por haber creído en mí a través de mi primera oportunidad 
laboral.\medskip\break
Saliendo del plano académico y entrando al plano laboral 
quiero mencionar a mi jefe Rubén Sergio Paniagua Aguilar 
por su tolerancia para conmigo al poder prescindir de mis 
actividades momentáneamente y dedicarme al avance de esta 
obra, así como por su visión al permitirme compartir este 
tipo de temas y considerarlos en el entorno en el que nos 
desenvolvemos.\break
Por otra parte anhelo recordar a Bianca García Álvarez y 
Juan Carlos Torres Patiño porque son un claro ejemplo de 
que aún en los lugares más inhóspitos se puede cultivar una 
firme y fructífera amistad.\medskip\break
En un nivel más personal brindo reconocimiento a Manuel y 
Logan por su fraternidad, palabras de aliento y la voluntad 
de escuchar mi tema de especialidad aún cuando nuestras 
disciplinas son distintas.\break
A Karla y Dinorah por el apoyo otorgado y las incontables 
horas de pláticas en las que me expresaban su cariño y 
estímulos.\break
A Teresa, Paula y Kathy por las sugerencias concisas y los 
momentos sublimes que, aunque escasos, han sido motivo de 
lucha y persistencia.\break
A Paulina Segovia por las muestras de afecto que indican 
que no importa la distancia para sentir aprecio por una 
persona.\break
A la profesora Laura Gasparyan,\selectlanguage{russian} огромное 
спасибо за помощь и советы, я никогда не забуду ваши слова.\selectlanguage{spanish}\break
A Lyudmila Tolstaya,\selectlanguage{russian} я тебя чувствую, 
я всегда с тобой.\selectlanguage{spanish}\break
A Yulia Boroshko,\selectlanguage{russian} наша дружба это 
очень важная для меня, спасибо.\selectlanguage{spanish}

%\end{comment}

%Después de los Agradecimientos se añade una página en blanco.
\newpage\null\thispagestyle{empty}\newpage

%Ahora se construye la parte correspondiente a la Dedicatoria.
\chapter*{\centerline{Dedicatoria}}

%En construcción
%\begin{comment}

\textit{A mis padres Rosa María y Martín; no hay palabras que 
describan todo el amor y gratificación que tengo para con 
ustedes. Merezcan ésto}.\bigskip

\textit{A mis hermanos Rosalba y Víctor Eduardo; dicen que un 
hermano es un amigo proporcionado por la naturaleza. Es cierto. 
Los quiero}.\bigskip

\textit{A mis abuelos María Concepción} \textbf{(\scalebox{0.8}{\Cross})} 
\textit{y Pedro} \textbf{(\scalebox{0.8}{\Cross})}\textit{; en donde quiera 
que estén espero se sientan orgullosos. Sepan que les estoy 
muy agradecido por todo}.\bigskip

\textit{A mis abuelos Cristina y Ramón; la fortaleza y 
perseverancia son características que he asimilado de ustedes 
y las tengo siempre presentes}.\bigskip

\textit{A mis tíos Luis Alberto, Arturo, Juan Manuel, Carlos, 
Juan Carlos, Ricardo \textbf{(\scalebox{0.8}{\Cross})} y Óscar; 
de cada uno he tomado muestras de dedicación, esfuerzo y valor 
para enfrentar las vicisitudes más acérrimas}.

%\end{comment}

%Después de la Dedicatoria se añade una página en blanco.
\newpage\null\thispagestyle{empty}\newpage

%Termina el documento.
\end{document}
