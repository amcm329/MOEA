%Autor: Aarón Martín Castillo Medina.
%Asesora: Dra. Katya Rodríguez Vázquez
%Contacto: katya.rodriguez@iimas.unam.mx; amcm329@hotmail.com

%Este archivo almacena las conclusiones obtenidas a partir de las
%pruebas elaboradas en el capítulo previo; también se hace mención
%de las posibilidades de expansión y enriquecimiento del trabajo para
%el futuro.


%Se indica que el documento es de tipo reporte bajo el paquete standalone. 
\documentclass[class=report, crop=false]{standalone}

%Se cargan todos los paquetes que residen en el archivo packages_used_standard.sty
\usepackage{packages_used_standard}

%Empieza el documento.
\begin{document}

\chapter{Conclusiones}
Con base en los resultados anteriores se puede 
primeramente verificar que

\section{Trabajo Futuro}
Una vez que se han concretado los méritos y metas cumplidas del 
presente trabajo escrito conviene mencionar los posibles escenarios 
de expansión del proyecto, ante lo cual se pueden dividir en dos 
categorías: analítica y técnica.

En lo concerniente a la primera

Ahora en consideración a la segunda el tratamiento se enfoca 
principalmente en las características relacionadas con M.O.E.A. 
Software.

En primer lugar, debido a la carencia de tiempo y limitaciones 
tecnológicas, las restricciones que están sujetas a las 
variables de decisión hasta el cierre de este trabajo corresponden 
unicamente a valores escalares aún teniendo en cuenta que las 
definiciones vistasen un principio contemplan el uso de funciones. 
Por este motivo se debe tener en mente la inclusión de esta 
caracterítica para futuras versiones del programa.

Se manifiesta un escenario parecido con las funciones objetivo 
ya que el producto de software no considera aquéllas definidas a 
trozos que, aunque no forman parte de la versión actual, son 
imprescindibles puesto que muchos de los M.O.P.’s adicionales 
localizados en las fuentes de consulta****** contienen funciones 
de este tipo.
Desde una perspectiva gráfica, la interfaz elaborada hasta el 
momento resulta insuficiente debido a que durante el proceso se 
hallaron varias fallas originadas en el uso de bibliotecas que 
carecían de mantenimiento, no obstante se utilizaron debido a su 
portabilidad con todos los sistemas operativos.
Es por esta razón que se puede sugerir el uso de alguna otra 
tecnología gráfica pudiendo incluso ser considerado algún 
microservicio en la red (con Node.js por ejemplo) o aplicación 
móvil (usando Android o Swift) con la finalidad de hacer mas 
eficiente la interacción con el usuario y por otro lado garantizar 
una mayor distribución en la divulgación de las técnicas mostradas 
durante el desarrollo de este trabajo.

%Termina documento.
\end{document}
