% Generated by Sphinx.
\def\sphinxdocclass{report}
\newif\ifsphinxKeepOldNames \sphinxKeepOldNamestrue
\documentclass[letterpaper,10pt,english]{sphinxmanual}
\usepackage{iftex}

\ifPDFTeX
  \usepackage[utf8]{inputenc}
\fi
\ifdefined\DeclareUnicodeCharacter
  \DeclareUnicodeCharacter{00A0}{\nobreakspace}
\fi
\usepackage{cmap}
\usepackage[T1]{fontenc}
\usepackage{amsmath,amssymb,amstext}
\usepackage{babel}
\usepackage{times}
\usepackage[Sonny]{fncychap}
\usepackage{longtable}
\usepackage{sphinx}
\usepackage{multirow}
\usepackage{eqparbox}


\addto\captionsenglish{\renewcommand{\figurename}{Fig.\@ }}
\addto\captionsenglish{\renewcommand{\tablename}{Table }}
\SetupFloatingEnvironment{literal-block}{name=Listing }

\addto\extrasenglish{\def\pageautorefname{page}}




\title{M.O.E.A. Software Documentation}
\date{Jan 03, 2017}
\release{1.0}
\author{Aarón Martín Castillo Medina}
\newcommand{\sphinxlogo}{}
\renewcommand{\releasename}{Release}
\makeindex

\makeatletter
\def\PYG@reset{\let\PYG@it=\relax \let\PYG@bf=\relax%
    \let\PYG@ul=\relax \let\PYG@tc=\relax%
    \let\PYG@bc=\relax \let\PYG@ff=\relax}
\def\PYG@tok#1{\csname PYG@tok@#1\endcsname}
\def\PYG@toks#1+{\ifx\relax#1\empty\else%
    \PYG@tok{#1}\expandafter\PYG@toks\fi}
\def\PYG@do#1{\PYG@bc{\PYG@tc{\PYG@ul{%
    \PYG@it{\PYG@bf{\PYG@ff{#1}}}}}}}
\def\PYG#1#2{\PYG@reset\PYG@toks#1+\relax+\PYG@do{#2}}

\expandafter\def\csname PYG@tok@gd\endcsname{\def\PYG@tc##1{\textcolor[rgb]{0.63,0.00,0.00}{##1}}}
\expandafter\def\csname PYG@tok@gu\endcsname{\let\PYG@bf=\textbf\def\PYG@tc##1{\textcolor[rgb]{0.50,0.00,0.50}{##1}}}
\expandafter\def\csname PYG@tok@gt\endcsname{\def\PYG@tc##1{\textcolor[rgb]{0.00,0.27,0.87}{##1}}}
\expandafter\def\csname PYG@tok@gs\endcsname{\let\PYG@bf=\textbf}
\expandafter\def\csname PYG@tok@gr\endcsname{\def\PYG@tc##1{\textcolor[rgb]{1.00,0.00,0.00}{##1}}}
\expandafter\def\csname PYG@tok@cm\endcsname{\let\PYG@it=\textit\def\PYG@tc##1{\textcolor[rgb]{0.25,0.50,0.56}{##1}}}
\expandafter\def\csname PYG@tok@vg\endcsname{\def\PYG@tc##1{\textcolor[rgb]{0.73,0.38,0.84}{##1}}}
\expandafter\def\csname PYG@tok@vi\endcsname{\def\PYG@tc##1{\textcolor[rgb]{0.73,0.38,0.84}{##1}}}
\expandafter\def\csname PYG@tok@mh\endcsname{\def\PYG@tc##1{\textcolor[rgb]{0.13,0.50,0.31}{##1}}}
\expandafter\def\csname PYG@tok@cs\endcsname{\def\PYG@tc##1{\textcolor[rgb]{0.25,0.50,0.56}{##1}}\def\PYG@bc##1{\setlength{\fboxsep}{0pt}\colorbox[rgb]{1.00,0.94,0.94}{\strut ##1}}}
\expandafter\def\csname PYG@tok@ge\endcsname{\let\PYG@it=\textit}
\expandafter\def\csname PYG@tok@vc\endcsname{\def\PYG@tc##1{\textcolor[rgb]{0.73,0.38,0.84}{##1}}}
\expandafter\def\csname PYG@tok@il\endcsname{\def\PYG@tc##1{\textcolor[rgb]{0.13,0.50,0.31}{##1}}}
\expandafter\def\csname PYG@tok@go\endcsname{\def\PYG@tc##1{\textcolor[rgb]{0.20,0.20,0.20}{##1}}}
\expandafter\def\csname PYG@tok@cp\endcsname{\def\PYG@tc##1{\textcolor[rgb]{0.00,0.44,0.13}{##1}}}
\expandafter\def\csname PYG@tok@gi\endcsname{\def\PYG@tc##1{\textcolor[rgb]{0.00,0.63,0.00}{##1}}}
\expandafter\def\csname PYG@tok@gh\endcsname{\let\PYG@bf=\textbf\def\PYG@tc##1{\textcolor[rgb]{0.00,0.00,0.50}{##1}}}
\expandafter\def\csname PYG@tok@ni\endcsname{\let\PYG@bf=\textbf\def\PYG@tc##1{\textcolor[rgb]{0.84,0.33,0.22}{##1}}}
\expandafter\def\csname PYG@tok@nl\endcsname{\let\PYG@bf=\textbf\def\PYG@tc##1{\textcolor[rgb]{0.00,0.13,0.44}{##1}}}
\expandafter\def\csname PYG@tok@nn\endcsname{\let\PYG@bf=\textbf\def\PYG@tc##1{\textcolor[rgb]{0.05,0.52,0.71}{##1}}}
\expandafter\def\csname PYG@tok@no\endcsname{\def\PYG@tc##1{\textcolor[rgb]{0.38,0.68,0.84}{##1}}}
\expandafter\def\csname PYG@tok@na\endcsname{\def\PYG@tc##1{\textcolor[rgb]{0.25,0.44,0.63}{##1}}}
\expandafter\def\csname PYG@tok@nb\endcsname{\def\PYG@tc##1{\textcolor[rgb]{0.00,0.44,0.13}{##1}}}
\expandafter\def\csname PYG@tok@nc\endcsname{\let\PYG@bf=\textbf\def\PYG@tc##1{\textcolor[rgb]{0.05,0.52,0.71}{##1}}}
\expandafter\def\csname PYG@tok@nd\endcsname{\let\PYG@bf=\textbf\def\PYG@tc##1{\textcolor[rgb]{0.33,0.33,0.33}{##1}}}
\expandafter\def\csname PYG@tok@ne\endcsname{\def\PYG@tc##1{\textcolor[rgb]{0.00,0.44,0.13}{##1}}}
\expandafter\def\csname PYG@tok@nf\endcsname{\def\PYG@tc##1{\textcolor[rgb]{0.02,0.16,0.49}{##1}}}
\expandafter\def\csname PYG@tok@si\endcsname{\let\PYG@it=\textit\def\PYG@tc##1{\textcolor[rgb]{0.44,0.63,0.82}{##1}}}
\expandafter\def\csname PYG@tok@s2\endcsname{\def\PYG@tc##1{\textcolor[rgb]{0.25,0.44,0.63}{##1}}}
\expandafter\def\csname PYG@tok@nt\endcsname{\let\PYG@bf=\textbf\def\PYG@tc##1{\textcolor[rgb]{0.02,0.16,0.45}{##1}}}
\expandafter\def\csname PYG@tok@nv\endcsname{\def\PYG@tc##1{\textcolor[rgb]{0.73,0.38,0.84}{##1}}}
\expandafter\def\csname PYG@tok@s1\endcsname{\def\PYG@tc##1{\textcolor[rgb]{0.25,0.44,0.63}{##1}}}
\expandafter\def\csname PYG@tok@ch\endcsname{\let\PYG@it=\textit\def\PYG@tc##1{\textcolor[rgb]{0.25,0.50,0.56}{##1}}}
\expandafter\def\csname PYG@tok@m\endcsname{\def\PYG@tc##1{\textcolor[rgb]{0.13,0.50,0.31}{##1}}}
\expandafter\def\csname PYG@tok@gp\endcsname{\let\PYG@bf=\textbf\def\PYG@tc##1{\textcolor[rgb]{0.78,0.36,0.04}{##1}}}
\expandafter\def\csname PYG@tok@sh\endcsname{\def\PYG@tc##1{\textcolor[rgb]{0.25,0.44,0.63}{##1}}}
\expandafter\def\csname PYG@tok@ow\endcsname{\let\PYG@bf=\textbf\def\PYG@tc##1{\textcolor[rgb]{0.00,0.44,0.13}{##1}}}
\expandafter\def\csname PYG@tok@sx\endcsname{\def\PYG@tc##1{\textcolor[rgb]{0.78,0.36,0.04}{##1}}}
\expandafter\def\csname PYG@tok@bp\endcsname{\def\PYG@tc##1{\textcolor[rgb]{0.00,0.44,0.13}{##1}}}
\expandafter\def\csname PYG@tok@c1\endcsname{\let\PYG@it=\textit\def\PYG@tc##1{\textcolor[rgb]{0.25,0.50,0.56}{##1}}}
\expandafter\def\csname PYG@tok@o\endcsname{\def\PYG@tc##1{\textcolor[rgb]{0.40,0.40,0.40}{##1}}}
\expandafter\def\csname PYG@tok@kc\endcsname{\let\PYG@bf=\textbf\def\PYG@tc##1{\textcolor[rgb]{0.00,0.44,0.13}{##1}}}
\expandafter\def\csname PYG@tok@c\endcsname{\let\PYG@it=\textit\def\PYG@tc##1{\textcolor[rgb]{0.25,0.50,0.56}{##1}}}
\expandafter\def\csname PYG@tok@mf\endcsname{\def\PYG@tc##1{\textcolor[rgb]{0.13,0.50,0.31}{##1}}}
\expandafter\def\csname PYG@tok@err\endcsname{\def\PYG@bc##1{\setlength{\fboxsep}{0pt}\fcolorbox[rgb]{1.00,0.00,0.00}{1,1,1}{\strut ##1}}}
\expandafter\def\csname PYG@tok@mb\endcsname{\def\PYG@tc##1{\textcolor[rgb]{0.13,0.50,0.31}{##1}}}
\expandafter\def\csname PYG@tok@ss\endcsname{\def\PYG@tc##1{\textcolor[rgb]{0.32,0.47,0.09}{##1}}}
\expandafter\def\csname PYG@tok@sr\endcsname{\def\PYG@tc##1{\textcolor[rgb]{0.14,0.33,0.53}{##1}}}
\expandafter\def\csname PYG@tok@mo\endcsname{\def\PYG@tc##1{\textcolor[rgb]{0.13,0.50,0.31}{##1}}}
\expandafter\def\csname PYG@tok@kd\endcsname{\let\PYG@bf=\textbf\def\PYG@tc##1{\textcolor[rgb]{0.00,0.44,0.13}{##1}}}
\expandafter\def\csname PYG@tok@mi\endcsname{\def\PYG@tc##1{\textcolor[rgb]{0.13,0.50,0.31}{##1}}}
\expandafter\def\csname PYG@tok@kn\endcsname{\let\PYG@bf=\textbf\def\PYG@tc##1{\textcolor[rgb]{0.00,0.44,0.13}{##1}}}
\expandafter\def\csname PYG@tok@cpf\endcsname{\let\PYG@it=\textit\def\PYG@tc##1{\textcolor[rgb]{0.25,0.50,0.56}{##1}}}
\expandafter\def\csname PYG@tok@kr\endcsname{\let\PYG@bf=\textbf\def\PYG@tc##1{\textcolor[rgb]{0.00,0.44,0.13}{##1}}}
\expandafter\def\csname PYG@tok@s\endcsname{\def\PYG@tc##1{\textcolor[rgb]{0.25,0.44,0.63}{##1}}}
\expandafter\def\csname PYG@tok@kp\endcsname{\def\PYG@tc##1{\textcolor[rgb]{0.00,0.44,0.13}{##1}}}
\expandafter\def\csname PYG@tok@w\endcsname{\def\PYG@tc##1{\textcolor[rgb]{0.73,0.73,0.73}{##1}}}
\expandafter\def\csname PYG@tok@kt\endcsname{\def\PYG@tc##1{\textcolor[rgb]{0.56,0.13,0.00}{##1}}}
\expandafter\def\csname PYG@tok@sc\endcsname{\def\PYG@tc##1{\textcolor[rgb]{0.25,0.44,0.63}{##1}}}
\expandafter\def\csname PYG@tok@sb\endcsname{\def\PYG@tc##1{\textcolor[rgb]{0.25,0.44,0.63}{##1}}}
\expandafter\def\csname PYG@tok@k\endcsname{\let\PYG@bf=\textbf\def\PYG@tc##1{\textcolor[rgb]{0.00,0.44,0.13}{##1}}}
\expandafter\def\csname PYG@tok@se\endcsname{\let\PYG@bf=\textbf\def\PYG@tc##1{\textcolor[rgb]{0.25,0.44,0.63}{##1}}}
\expandafter\def\csname PYG@tok@sd\endcsname{\let\PYG@it=\textit\def\PYG@tc##1{\textcolor[rgb]{0.25,0.44,0.63}{##1}}}

\def\PYGZbs{\char`\\}
\def\PYGZus{\char`\_}
\def\PYGZob{\char`\{}
\def\PYGZcb{\char`\}}
\def\PYGZca{\char`\^}
\def\PYGZam{\char`\&}
\def\PYGZlt{\char`\<}
\def\PYGZgt{\char`\>}
\def\PYGZsh{\char`\#}
\def\PYGZpc{\char`\%}
\def\PYGZdl{\char`\$}
\def\PYGZhy{\char`\-}
\def\PYGZsq{\char`\'}
\def\PYGZdq{\char`\"}
\def\PYGZti{\char`\~}
% for compatibility with earlier versions
\def\PYGZat{@}
\def\PYGZlb{[}
\def\PYGZrb{]}
\makeatother

\renewcommand\PYGZsq{\textquotesingle}

\begin{document}

\maketitle
\tableofcontents
\phantomsection\label{index::doc}\begin{wrapfigure}{l}{None}
\centering
\noindent\sphinxincludegraphics{{unam_shield}.png}
\end{wrapfigure}
\begin{wrapfigure}{r}{None}
\centering
\noindent\sphinxincludegraphics{{sciences_shield}.png}
\end{wrapfigure}


\chapter{Contenido:}
\label{index:contenido}

\section{Begin (script)}
\label{Begin::doc}\label{Begin:begin-script}
Este archivo funge como un launcher \textbf{(disparador)}
el cual simplemente crea y muestra la ventana principal
\textbf{(véase la sección View)}.
\phantomsection\label{Begin:module-Begin}\index{Begin (module)}

\section{Model (sección)}
\label{Model/Model:model-seccion}\label{Model/Model::doc}
La sección Model \textbf{(ó Modelo)} contiene toda la base lógica del programa, más en específico, todas las
características para poder ejecutar MOEA's apropiadamente alimentados con los datos obtenidos por
la sección View \textbf{(ó Vista)} usando la sección Controller \textbf{(ó Controlador)} como intermediaria.
Una vez que se obtenga algún resultado, éste será transmitido a la sección View a través del Controller.

A continuación se observan las subcategorías que conforman a la sección en cuestión:


\subsection{ChromosomalRepresentation (módulo)}
\label{Model/ChromosomalRepresentation/ChromosomalRepresentation::doc}\label{Model/ChromosomalRepresentation/ChromosomalRepresentation:chromosomalrepresentation-modulo}
Ofrece elementos para elaborar una codificación adecuada.

Entiéndase por codificación a la forma de determinar el cromosoma y sus propiedades;
cabe mencionar que el cromosoma será portado por los Individuals \textbf{(ó Individuos)}.

Es importante mencionar que cualquier codificación implementada debe ser sustentada
en los métodos correspondientes al Crossover \textbf{(ó Cruza)} y Mutation \textbf{(ó Mutación)},
ésto porque dichas operaciones funcionan con cromosomas.

De esta manera, la idea es que el usuario pueda crear sus propias codificaciones,
por lo que, además de agregar la descripción de la codificación a Controller/XML/Features.xml
\textbf{(véase el archivo mencionado en la sección de código)}, deberá implementar por
lo menos las siguientes funciones:


\begin{fulllineitems}
\pysigline{\sphinxbfcode{calculate\_length\_subchromosomes(vector\_variables,number\_of\_decimals,representation\_parameters):}}
Por cada variable de decisión se crea una porción del cromosoma, entonces en esta función se calcula
el tamaño de cada porción \textbf{(ó subcromosoma)}, ya que al final las operaciones de cruza y mutación se realizarán
sobre el súper crosomoma, el cual es la concatenacion de todos los subcromosomas. Por eso es importante identificar
el tamaño de cada subcromosoma, así como sus límites dentro del súper cromosoma.
\begin{quote}\begin{description}
\item[{Parameters}] \leavevmode\begin{itemize}
\item {} 
\textbf{\texttt{vector\_variables}} (\emph{\texttt{List}}) -- El vector de variables de decisión, donde cada variable trae consigo sus límites inferior
y superior.

\item {} 
\textbf{\texttt{number\_of\_decimals}} (\emph{\texttt{Integer}}) -- El número de decimales que deberá traer cada variable de decisión.

\item {} 
\textbf{\texttt{representation\_parameters}} (\emph{\texttt{Dictionary}}) -- Un diccionario que contiene todas las opciones adicionales para cada tipo de
codificación.

\end{itemize}

\item[{Returns}] \leavevmode
Una lista que contiene el tamaño del cromosoma por cada variable de decisión. Dado que el orden de las
variables de decisión es inmutable, se preserva el mismo y por ello la lista contiene sólo los tamaños.

\item[{Return type}] \leavevmode
List

\end{description}\end{quote}

\end{fulllineitems}



\begin{fulllineitems}
\pysigline{\sphinxbfcode{create\_chromosome(length\_subchromosomes,vector\_variables,number\_of\_decimals,representation\_parameters):}}
Función que crea el cromosoma. Se usa la como apoyo el método \textbf{calculate\_length\_subchromosomes} descrito con
anterioridad.
\begin{quote}\begin{description}
\item[{Parameters}] \leavevmode\begin{itemize}
\item {} 
\textbf{\texttt{length\_subchromosomes}} (\emph{\texttt{List}}) -- La lista que contiene los tamaños de cada variable de decisión.

\item {} 
\textbf{\texttt{vector\_variables}} (\emph{\texttt{List}}) -- La lista que contiene las variables de decisión, así como sus rangos.

\item {} 
\textbf{\texttt{number\_of\_decimals}} (\emph{\texttt{Integer}}) -- El número de decimales que traerá cada variable de decisión.

\item {} 
\textbf{\texttt{representation\_parameters}} (\emph{\texttt{Dictionary}}) -- Un diccionario que contiene todas las opciones adicionales para cada tipo de
codificación.

\end{itemize}

\item[{Returns}] \leavevmode
El cromosoma devuelto en forma de lista.

\item[{Return type}] \leavevmode
List

\end{description}\end{quote}

\end{fulllineitems}



\begin{fulllineitems}
\pysigline{\sphinxbfcode{evaluate\_subchromosomes(complete\_chromosome,length\_subchromosomes,vector\_variables,number\_of\_decimals,representation\_parameters):}}
Tomando en cuenta que el cromosoma ya ha sido creado usando los tamaños de los subcromosomas,
en esta función se procede a evaluar el súper cromosoma partiéndolo en los subcromosomas pertinentes y evaluando
individualmente cada uno de éstos.
\begin{quote}\begin{description}
\item[{Parameters}] \leavevmode\begin{itemize}
\item {} 
\textbf{\texttt{complete\_chromosome}} (\emph{\texttt{List}}) -- El súper cromosoma a ser evaluado.

\item {} 
\textbf{\texttt{length\_subchromosomes}} (\emph{\texttt{List}}) -- La lista que contiene los tamaños de cada variable de decisión.

\item {} 
\textbf{\texttt{vector\_variables}} (\emph{\texttt{List}}) -- La lista que contiene las variables de decisión, así como sus rangos.

\item {} 
\textbf{\texttt{number\_of\_decimals}} (\emph{\texttt{Integer}}) -- El número de decimales que traerá cada variable de decisión.

\item {} 
\textbf{\texttt{representation\_parameters}} (\emph{\texttt{Dictionary}}) -- Un diccionario que contiene todas las opciones adicionales para cada tipo de
codificación.

\end{itemize}

\item[{Returns}] \leavevmode
Un diccionario que contiene como llave la variable de decisión y como valor la evaluación del
subcromosoma correspondiente.

\item[{Return type}] \leavevmode
Dictionary

\end{description}\end{quote}

\end{fulllineitems}


A continuación se develan los elementos que constituyen a este módulo:


\subsubsection{BinaryRepresentation (script)}
\label{Model/ChromosomalRepresentation/BinaryRepresentation:binaryrepresentation-script}\label{Model/ChromosomalRepresentation/BinaryRepresentation::doc}
\begin{DUlineblock}{0em}
\item[] Contiene todas las funcionalidades requeridas para que se pueda hacer uso de una
codificación de tipo Binary \textbf{(ó Binaria)}; ésto significa que los alelos que
conforman al cromosoma serán exclusivamente 0 ó 1.
\end{DUlineblock}
\phantomsection\label{Model/ChromosomalRepresentation/BinaryRepresentation:module-Model.ChromosomalRepresentation.BinaryRepresentation}\index{Model.ChromosomalRepresentation.BinaryRepresentation (module)}\index{binary\_to\_decimal() (in module Model.ChromosomalRepresentation.BinaryRepresentation)}

\begin{fulllineitems}
\phantomsection\label{Model/ChromosomalRepresentation/BinaryRepresentation:Model.ChromosomalRepresentation.BinaryRepresentation.binary_to_decimal}\pysiglinewithargsret{\sphinxbfcode{binary\_to\_decimal}}{\emph{chromosome}}{}~
\begin{DUlineblock}{0em}
\item[] Método que convierte un número binario a decimal.
\item[] Este es un ejemplo de método que se puede agregar
adicionalmente siempre y cuando se implementen las 
funciones que se han mencionado ya.
\end{DUlineblock}
\begin{quote}\begin{description}
\item[{Parameters}] \leavevmode
\textbf{\texttt{chromosome}} (\emph{\texttt{List}}) -- El cromosoma sobre el cual se hará
la evaluación.

\item[{Returns}] \leavevmode
La representación en decimal del número binario.

\item[{Return type}] \leavevmode
Integer

\end{description}\end{quote}

\end{fulllineitems}

\index{calculate\_length\_subchromosomes() (in module Model.ChromosomalRepresentation.BinaryRepresentation)}

\begin{fulllineitems}
\phantomsection\label{Model/ChromosomalRepresentation/BinaryRepresentation:Model.ChromosomalRepresentation.BinaryRepresentation.calculate_length_subchromosomes}\pysiglinewithargsret{\sphinxbfcode{calculate\_length\_subchromosomes}}{\emph{vector\_variables}, \emph{number\_of\_decimals}, \emph{representation\_parameters}}{}~
\begin{DUlineblock}{0em}
\item[] Esta es la implementación del método para la codificación en binario.
A grandes rasgos primero se determina el número de bits que se deben tomar en cuenta
para representar la magnitud de una determinada variable de decisión.
\item[] Haciendo esto para todas las variables de decisión se obtienen las longitudes
de todos los subcromosomas. 
\item[] Esta función se implementa obligatoriamente.
\end{DUlineblock}

\end{fulllineitems}

\index{create\_chromosome() (in module Model.ChromosomalRepresentation.BinaryRepresentation)}

\begin{fulllineitems}
\phantomsection\label{Model/ChromosomalRepresentation/BinaryRepresentation:Model.ChromosomalRepresentation.BinaryRepresentation.create_chromosome}\pysiglinewithargsret{\sphinxbfcode{create\_chromosome}}{\emph{length\_subchromosomes}, \emph{vector\_variables}, \emph{number\_of\_decimals}, \emph{representation\_parameters}}{}~
\begin{DUlineblock}{0em}
\item[] Crea un cromosoma binario completo con base en las longitudes de los subcromosmas.
\item[] Este método debe implementarse obligatoriamente.
\end{DUlineblock}

\end{fulllineitems}

\index{evaluate\_subchromosomes() (in module Model.ChromosomalRepresentation.BinaryRepresentation)}

\begin{fulllineitems}
\phantomsection\label{Model/ChromosomalRepresentation/BinaryRepresentation:Model.ChromosomalRepresentation.BinaryRepresentation.evaluate_subchromosomes}\pysiglinewithargsret{\sphinxbfcode{evaluate\_subchromosomes}}{\emph{complete\_chromosome}, \emph{length\_subchromosomes}, \emph{vector\_variables}, \emph{number\_of\_decimals}, \emph{representation\_parameters}}{}~
\begin{DUlineblock}{0em}
\item[] Realiza una evaluación de los subcromosomas para la codificación binaria \textbf{(ó Binary)}.
\item[] En términos generales se toma cada porción del subcrosomoma \textbf{(tomando en 
cuenta que previamente se calcularon sus longitudes)} y así se convierte a 
decimal, considerando la expansión numérica.
\item[] Posteriormente para obtener el número final se hace lo siguiente:
\end{DUlineblock}

\begin{center}\(Conversi\acute{o}n(subcromosoma) = A + DN(subcromosoma) \cdot \frac{B - A}{2^M - 1}\)
\end{center}
\begin{DUlineblock}{0em}
\item[] Donde:
\item[]
\begin{DUlineblock}{\DUlineblockindent}
\item[] \textbf{A} es el límite inferior que toma la variable de decisión.
\item[] \textbf{B} es el límite superior que toma la variable de decisión.
\item[] \textbf{M} es la longitud del subcromosoma asociado a la variable de decisión.
\item[] \textbf{DN (Decimal number)} es el número en decimal del subcromosoma asociado a la variable de decisión.
\end{DUlineblock}
\end{DUlineblock}

\end{fulllineitems}



\subsubsection{FloatPointRepresentation (script)}
\label{Model/ChromosomalRepresentation/FloatPointRepresentation:floatpointrepresentation-script}\label{Model/ChromosomalRepresentation/FloatPointRepresentation::doc}
\begin{DUlineblock}{0em}
\item[] Este script contiene todas las funcionalidades requeridas para que se pueda hacer uso de una
codificación de tipo Float Point \textbf{(ó Punto Flotante)}; ésto significa que los alelos que
conforman al cromosoma serán números de punto flotante.
\item[] Un número de punto flotante es aquél que tiene una parte entera y una decimal; cabe mencionar que
si el número en cuestión no tiene expansión decimal, se le considera un número de representación Integer
\textbf{(ó Entera)}; ésto porque en algunas fuentes se manejan la representación de Punto Flotante y
Entera por separado.
\end{DUlineblock}
\phantomsection\label{Model/ChromosomalRepresentation/FloatPointRepresentation:module-Model.ChromosomalRepresentation.FloatPointRepresentation}\index{Model.ChromosomalRepresentation.FloatPointRepresentation (module)}\index{calculate\_length\_subchromosomes() (in module Model.ChromosomalRepresentation.FloatPointRepresentation)}

\begin{fulllineitems}
\phantomsection\label{Model/ChromosomalRepresentation/FloatPointRepresentation:Model.ChromosomalRepresentation.FloatPointRepresentation.calculate_length_subchromosomes}\pysiglinewithargsret{\sphinxbfcode{calculate\_length\_subchromosomes}}{\emph{vector\_variables}, \emph{number\_of\_decimals}, \emph{representation\_parameters}}{}~
\begin{DUlineblock}{0em}
\item[] Realiza el cálculo de subcromosomas de acuerdo a la representación Float Point \textbf{(ó Punto Flotante)}.
\item[] Esta función es de aquéllas que se tienen que implementar obligatoriamente.
\end{DUlineblock}

\end{fulllineitems}

\index{create\_chromosome() (in module Model.ChromosomalRepresentation.FloatPointRepresentation)}

\begin{fulllineitems}
\phantomsection\label{Model/ChromosomalRepresentation/FloatPointRepresentation:Model.ChromosomalRepresentation.FloatPointRepresentation.create_chromosome}\pysiglinewithargsret{\sphinxbfcode{create\_chromosome}}{\emph{length\_subchromosomes}, \emph{vector\_variables}, \emph{number\_of\_decimals}, \emph{representation\_parameters}}{}~
\begin{DUlineblock}{0em}
\item[] Crea un cromosoma con contenido de punto flotante.
\item[] Esta función es de aquéllas que se tienen que implementar obligatoriamente.
\end{DUlineblock}

\end{fulllineitems}

\index{evaluate\_subchromosomes() (in module Model.ChromosomalRepresentation.FloatPointRepresentation)}

\begin{fulllineitems}
\phantomsection\label{Model/ChromosomalRepresentation/FloatPointRepresentation:Model.ChromosomalRepresentation.FloatPointRepresentation.evaluate_subchromosomes}\pysiglinewithargsret{\sphinxbfcode{evaluate\_subchromosomes}}{\emph{complete\_chromosome}, \emph{length\_subchromosomes}, \emph{vector\_variables}, \emph{number\_of\_decimals}, \emph{representation\_parameters}}{}
Toma cada porción de cromosoma y la evalúa para luego ser asignada
a la variable de decisión correspondiente. Este método es de los que se
debe de implementar obligatoriamente.

\end{fulllineitems}



\subsection{Community (clase)}
\label{Model/Community/Community:module-Model.Community.Community}\label{Model/Community/Community:community-clase}\label{Model/Community/Community::doc}\index{Model.Community.Community (module)}\index{Community (class in Model.Community.Community)}

\begin{fulllineitems}
\phantomsection\label{Model/Community/Community:Model.Community.Community.Community}\pysiglinewithargsret{\sphinxstrong{class }\sphinxbfcode{Community}}{\emph{vector\_functions}, \emph{vector\_variables}, \emph{available\_expressions}, \emph{number\_of\_decimals}, \emph{representation\_instance}, \emph{representation\_parameters}, \emph{fitness\_instance}, \emph{fitness\_parameters}, \emph{sharing\_function\_instance}, \emph{sharing\_function\_parameters}, \emph{selection\_instance}, \emph{selection\_parameters}, \emph{crossover\_instance}, \emph{crossover\_parameters}, \emph{mutation\_instance}, \emph{mutation\_parameters}}{}~
\begin{DUlineblock}{0em}
\item[] Proporciona toda la infraestructura lógica para poder construir poblaciones y operar con éstas,
además de transacciones relacionadas con sus elementos de manera individual.
\item[] Se le llama Community porque aludiendo a su significado una Community \textbf{(ó Comunidad)}
consta de al menos una Population \textbf{(o Población)}. De esta manera se deduce que en algún momento
habrán métodos que involucren a más de una población.
\end{DUlineblock}
\begin{quote}\begin{description}
\item[{Parameters}] \leavevmode\begin{itemize}
\item {} 
\textbf{\texttt{vector\_functions}} (\emph{\texttt{List}}) -- Lista que contiene las funciones objetivo previamente 
saneadas por Controller/Controller.py.

\item {} 
\textbf{\texttt{vector\_variables}} (\emph{\texttt{List}}) -- Lista que contiene las variables de decisión previamente 
saneadas por Controller/Controller.py.

\item {} 
\textbf{\texttt{available\_expressions}} (\emph{\texttt{Dictionary}}) -- Diccionario que contiene algunas funciones escritas como azúcar sintáctica
para que puedan ser utilizadas más fácilmente por el usuario y evaluadas
más ŕapidamente en el programa \textbf{(véase Controller/XML/PythonExpressions.xml)}.

\item {} 
\textbf{\texttt{number\_of\_decimals}} (\emph{\texttt{Integer}}) -- El número de decimales que tendrán las soluciones; con este número se determina
en gran medida el tamaño del cromosoma.

\item {} 
\textbf{\texttt{representation\_instance}} (\emph{\texttt{Instance}}) -- Instancia de la técnica de representación que eligió el usuario
\textbf{(véase Controller/Verifier.py)}.

\item {} 
\textbf{\texttt{representation\_parameters}} (\emph{\texttt{Dictionary}}) -- Diccionario que contiene todos los parámetros adicionales a la técnica
de representación considerada por el usuario.

\item {} 
\textbf{\texttt{fitness\_instance}} (\emph{\texttt{Instance}}) -- Instancia de la técnica de Fitness que eligió el usuario
\textbf{(véase Controller/Verifier.py)}.

\item {} 
\textbf{\texttt{fitness\_parameters}} (\emph{\texttt{Dictionary}}) -- Diccionario que contiene todos los parámetros adicionales a la técnica
de Fitness seleccionada por el usuario.

\item {} 
\textbf{\texttt{sharing\_function\_instance}} (\emph{\texttt{Instance}}) -- Instancia de la técnica de Sharing Function seleccionada por el usuario
\textbf{(véase Controller/Verifier.py)}.

\item {} 
\textbf{\texttt{sharing\_function\_parameters}} (\emph{\texttt{Dictionary}}) -- Diccionario que contiene todos los parámetros adicionales a la técnica
de Fitness seleccionada por el usuario.

\item {} 
\textbf{\texttt{selection\_instance}} (\emph{\texttt{Instance}}) -- Instancia de la técnica de selección \textbf{(Selection)} elegida por el usuario
\textbf{(véase Controller/Verifier.py)}.

\item {} 
\textbf{\texttt{selection\_parameters}} (\emph{\texttt{Dictionary}}) -- Diccionario que contiene todos los parámetros adicionales a la técnica
de selección \textbf{(Selection)} usada por el usuario.

\item {} 
\textbf{\texttt{crossover\_instance}} (\emph{\texttt{Instance}}) -- Instancia de la técnica de cruza \textbf{(Crossover)} tomada por el usuario
\textbf{(véase Controller/Verifier.py)}.

\item {} 
\textbf{\texttt{crossover\_parameters}} (\emph{\texttt{Dictionary}}) -- Diccionario que contiene todos los parámetros adicionales a la técnica
de cruza \textbf{(Crossover)} manejada por el usuario.

\item {} 
\textbf{\texttt{mutation\_instance}} (\emph{\texttt{Instance}}) -- Instancia de la técnica de mutación \textbf{(Mutation)} tomada por el usuario
\textbf{(véase Controller/Verifier.py)}.

\item {} 
\textbf{\texttt{mutation\_parameters}} (\emph{\texttt{Dictionary}}) -- Diccionario que contiene todos los parámetros adicionales a la técnica
de mutación \textbf{(Mutation)} seleccionada por el usuario.

\end{itemize}

\item[{Returns}] \leavevmode
Model.Community.Community

\item[{Return type}] \leavevmode
Instance

\end{description}\end{quote}
\index{\_Community\_\_compare\_dominance() (Community method)}

\begin{fulllineitems}
\phantomsection\label{Model/Community/Community:Model.Community.Community.Community._Community__compare_dominance}\pysiglinewithargsret{\sphinxbfcode{\_Community\_\_compare\_dominance}}{\emph{current}, \emph{challenger}, \emph{allowed\_functions}}{}~
\begin{notice}{note}{Note:}
Este método es privado.
\end{notice}

Permite realizar la comparación de las funciones objetivo de los 
individuos current y challenger tomadas una a una para indicar así quién es el dominado y quién
es el que domina. Cabe mencionar que más apropiadamente se le conoce como dominancia
fuerte de Pareto.
\begin{quote}\begin{description}
\item[{Parameters}] \leavevmode\begin{itemize}
\item {} 
\textbf{\texttt{current}} (\emph{\texttt{Instance}}) -- El Individuo inicial para comprobar dominancia.

\item {} 
\textbf{\texttt{challenger}} (\emph{\texttt{Instance}}) -- El Individuo que reta al inicial para comprobar dominancia.

\item {} 
\textbf{\texttt{allowed\_functions}} (\emph{\texttt{List}}) -- Lista que indica cuáles son las funciones objetivo que deben 
compararse.

\end{itemize}

\item[{Returns}] \leavevmode
True si current domina a challenger, False en otro caso.

\item[{Return type}] \leavevmode
Boolean

\end{description}\end{quote}

\end{fulllineitems}

\index{\_Community\_\_get\_best\_individual\_results() (Community method)}

\begin{fulllineitems}
\phantomsection\label{Model/Community/Community:Model.Community.Community.Community._Community__get_best_individual_results}\pysiglinewithargsret{\sphinxbfcode{\_Community\_\_get\_best\_individual\_results}}{\emph{population}}{}~
\begin{notice}{note}{Note:}
Este método es privado.
\end{notice}

Obtiene los valores de las variables de decisión y de las funciones objetivo
por cada individuo.
\begin{quote}\begin{description}
\item[{Parameters}] \leavevmode
\textbf{\texttt{population}} (\emph{\texttt{List}}) -- Una lista que contiene los mejores individuos por generación.

\item[{Returns}] \leavevmode
Una lista que contiene por un lado la tupla (generacion, funciones)
y por otro la tupla (generación, variables). Esto por cada generación.

\item[{Return type}] \leavevmode
List

\end{description}\end{quote}

\end{fulllineitems}

\index{\_Community\_\_get\_pareto\_results() (Community method)}

\begin{fulllineitems}
\phantomsection\label{Model/Community/Community:Model.Community.Community.Community._Community__get_pareto_results}\pysiglinewithargsret{\sphinxbfcode{\_Community\_\_get\_pareto\_results}}{\emph{population}}{}~
\begin{notice}{note}{Note:}
Este método es privado.
\end{notice}

\begin{DUlineblock}{0em}
\item[] Obtiene el frente de Pareto, el complemento del frente de Pareto y el óptimo de Pareto.
\item[] Para una mejor orientación léase la parte escrita del proyecto.
\end{DUlineblock}
\begin{quote}\begin{description}
\item[{Parameters}] \leavevmode
\textbf{\texttt{population}} (\emph{\texttt{Instance}}) -- La población sobre la cual se obtendrán estos elementos.

\item[{Returns}] \leavevmode
Una lista que contiene el frente de Pareto, su complemento y el óptimo de Pareto.

\item[{Return type}] \leavevmode
List

\end{description}\end{quote}

\end{fulllineitems}

\index{\_Community\_\_using\_sharing\_function() (Community method)}

\begin{fulllineitems}
\phantomsection\label{Model/Community/Community:Model.Community.Community.Community._Community__using_sharing_function}\pysiglinewithargsret{\sphinxbfcode{\_Community\_\_using\_sharing\_function}}{\emph{individual\_i}, \emph{individual\_j}, \emph{alpha\_share}, \emph{sigma\_share}}{}~
\begin{notice}{note}{Note:}
Este método es privado.
\end{notice}

\begin{DUlineblock}{0em}
\item[] Devuelve un valor que ayuda al cálculo del Sharing Function.
\item[] A grandes rasgos el sharing function sirve para hacer una selección más precisa de los
mejores Individuos cuando se da el caso de que tienen el mismo número de Individuos dominados.
\end{DUlineblock}
\begin{quote}\begin{description}
\item[{Parameters}] \leavevmode\begin{itemize}
\item {} 
\textbf{\texttt{individual\_i}} (\emph{\texttt{Instance}}) -- Individuo sobre el que se hará la operación.

\item {} 
\textbf{\texttt{individual\_j}} (\emph{\texttt{Instance}}) -- Individuo sobre el que se hará la operación.

\item {} 
\textbf{\texttt{alpha\_share}} (\emph{\texttt{Float}}) -- El valor necesario para poder calcular la distancia entre Individuos.

\item {} 
\textbf{\texttt{sigma\_share}} (\emph{\texttt{Float}}) -- El valor necesario para poder calcular la distancia entre Individuos.

\end{itemize}

\item[{Returns}] \leavevmode
El resultado que contribuirá al sharing function.

\item[{Return type}] \leavevmode
Float

\end{description}\end{quote}

\end{fulllineitems}

\index{assign\_fonseca\_and\_flemming\_pareto\_rank() (Community method)}

\begin{fulllineitems}
\phantomsection\label{Model/Community/Community:Model.Community.Community.Community.assign_fonseca_and_flemming_pareto_rank}\pysiglinewithargsret{\sphinxbfcode{assign\_fonseca\_and\_flemming\_pareto\_rank}}{\emph{population}, \emph{allowed\_functions='All'}}{}~
\begin{DUlineblock}{0em}
\item[] Asigna una puntuación \textbf{(ó rank)} a cada uno de los Individuos de una Población 
con base en su dominancia de Pareto.
\item[] A grandes rasgos, el algoritmo asigna un rank que consiste en:
\end{DUlineblock}

\begin{center}\(rank(Individuo) = n\acute{u}mero\_soluciones\_que\_dominan(Individuo) + 1\)
\end{center}
\begin{DUlineblock}{0em}
\item[] Esta técnica es usada principalmente por M.O.G.A.
\end{DUlineblock}
\begin{quote}\begin{description}
\item[{Parameters}] \leavevmode\begin{itemize}
\item {} 
\textbf{\texttt{population}} (\emph{\texttt{Instance}}) -- La Población sobre la que se hará la operación.

\item {} 
\textbf{\texttt{allowed\_functions}} (\emph{\texttt{List}}) -- Lista que contiene las posiciones de las funciones que son admisibles 
para hacer comparaciones. Por defecto tiene el valor ``All''.

\end{itemize}

\end{description}\end{quote}

\end{fulllineitems}

\index{assign\_goldberg\_pareto\_rank() (Community method)}

\begin{fulllineitems}
\phantomsection\label{Model/Community/Community:Model.Community.Community.Community.assign_goldberg_pareto_rank}\pysiglinewithargsret{\sphinxbfcode{assign\_goldberg\_pareto\_rank}}{\emph{population}, \emph{additional\_info=False}, \emph{allowed\_functions='All'}}{}~
\begin{DUlineblock}{0em}
\item[] Asigna una puntuación \textbf{(ó rank)} a cada uno de los Individuos de una Población con base en su dominancia
de Pareto.
\item[] En términos generales, el algoritmo trabaja con niveles, es decir, primero toma los Individuos no
dominados y les asigna un valor 0, luego los elimina del conjunto y nuevamente aplica la 
operación sobre los no dominados del nuevo conjunto, a los que les asigna el valor 1, y así
sucesivamente hasta no quedar Individuos.
\item[] Esta técnica es usada principalmente por N.S.G.A. II.
\end{DUlineblock}
\begin{quote}\begin{description}
\item[{Parameters}] \leavevmode\begin{itemize}
\item {} 
\textbf{\texttt{population}} (\emph{\texttt{Instance}}) -- La Población sobre la que se hará la operación.

\item {} 
\textbf{\texttt{additional\_info}} (\emph{\texttt{Boolean}}) -- Un valor que le indica a la función que debe regresar información 
adicional.

\item {} 
\textbf{\texttt{allowed\_functions}} (\emph{\texttt{List}}) -- Lista que contiene las posiciones de las funciones que son admisibles 
para hacer comparaciones. Por defecto tiene el valor ``All''.

\end{itemize}

\item[{Returns}] \leavevmode
Si additional\_info es True: un arreglo con dos elementos: en el primero 
se almacena una lista con los niveles de dominancia disponibles, mientras que el 
segundo consta de una estructura que contiene todos los posibles niveles y asociados 
a éstos, los cromosomas de los Individuos que los conforman.
Si additional\_info es False: el método es void \textbf{(no regresa nada)}.

\item[{Return type}] \leavevmode
List

\end{description}\end{quote}

\end{fulllineitems}

\index{assign\_population\_fitness() (Community method)}

\begin{fulllineitems}
\phantomsection\label{Model/Community/Community:Model.Community.Community.Community.assign_population_fitness}\pysiglinewithargsret{\sphinxbfcode{assign\_population\_fitness}}{\emph{population}}{}
Aplica la asignación de Fitness para una Población dada usando como
base el Ranking de cada Individuo \textbf{(véase Model/Fitness)}.
\begin{quote}\begin{description}
\item[{Parameters}] \leavevmode
\textbf{\texttt{population}} (\emph{\texttt{Instance}}) -- La población sobre la que se hará la operación.

\end{description}\end{quote}

\end{fulllineitems}

\index{assign\_zitzler\_and\_thiele\_pareto\_rank() (Community method)}

\begin{fulllineitems}
\phantomsection\label{Model/Community/Community:Model.Community.Community.Community.assign_zitzler_and_thiele_pareto_rank}\pysiglinewithargsret{\sphinxbfcode{assign\_zitzler\_and\_thiele\_pareto\_rank}}{\emph{population}, \emph{allowed\_functions='All'}}{}~
\begin{DUlineblock}{0em}
\item[] Asigna una puntuación (rank) a cada uno de los Individuos de una Población con 
base en su dominancia de Pareto.
\item[] A manera de esbozo, el algoritmo asigna un rank que consiste en una razón:
\end{DUlineblock}

\begin{center}\(rank(Individuo) = \frac{n\acute{u}mero\_soluciones\_dominadas(Individuo)}{tama\tilde{n}o\_poblaci\acute{o}n} + 1\)
\end{center}
\begin{DUlineblock}{0em}
\item[] Esta técnica es usada principalmente por S.P.E.A. II
\end{DUlineblock}
\begin{quote}\begin{description}
\item[{Parameters}] \leavevmode\begin{itemize}
\item {} 
\textbf{\texttt{population}} (\emph{\texttt{Instance}}) -- La población sobre la que se hará la operación.

\item {} 
\textbf{\texttt{allowed\_functions}} (\emph{\texttt{List}}) -- Lista que contiene las posiciones de las funciones que son admisibles 
para hacer comparaciones. Por defecto tiene el valor ``All''.

\end{itemize}

\end{description}\end{quote}

\end{fulllineitems}

\index{calculate\_population\_niche\_count() (Community method)}

\begin{fulllineitems}
\phantomsection\label{Model/Community/Community:Model.Community.Community.Community.calculate_population_niche_count}\pysiglinewithargsret{\sphinxbfcode{calculate\_population\_niche\_count}}{\emph{population}}{}
Calcula el valor conocido como niche count que no es mas que la suma de los sharing function
de todos los individuos j con el individuo i, con i != j.
\begin{quote}\begin{description}
\item[{Parameters}] \leavevmode
\textbf{\texttt{population}} (\emph{\texttt{Instance}}) -- Conjunto sobre el que se hará la operación.

\end{description}\end{quote}

\end{fulllineitems}

\index{calculate\_population\_pareto\_dominance() (Community method)}

\begin{fulllineitems}
\phantomsection\label{Model/Community/Community:Model.Community.Community.Community.calculate_population_pareto_dominance}\pysiglinewithargsret{\sphinxbfcode{calculate\_population\_pareto\_dominance}}{\emph{population}, \emph{allowed\_functions}}{}
Realiza la comparación de dominancia entre todos los elementos de la Población con base
en la evaluación de sus funciones objetivo.
\begin{quote}\begin{description}
\item[{Parameters}] \leavevmode\begin{itemize}
\item {} 
\textbf{\texttt{population}} (\emph{\texttt{Instance}}) -- La Población sobre la que se hará la operación.

\item {} 
\textbf{\texttt{allowed\_functions}} (\emph{\texttt{List}}) -- Lista que indica las funciones objetivo permitidas para hacer la 
comparación.

\end{itemize}

\end{description}\end{quote}

\end{fulllineitems}

\index{calculate\_population\_shared\_fitness() (Community method)}

\begin{fulllineitems}
\phantomsection\label{Model/Community/Community:Model.Community.Community.Community.calculate_population_shared_fitness}\pysiglinewithargsret{\sphinxbfcode{calculate\_population\_shared\_fitness}}{\emph{population}}{}
Calcula el Shared Fitness \textbf{(ó Fitness Compartido)} de cada uno
de los Individuos de la Población.
\begin{quote}\begin{description}
\item[{Parameters}] \leavevmode
\textbf{\texttt{population}} (\emph{\texttt{Instance}}) -- Conjunto sobre el que se hará la operación.

\end{description}\end{quote}

\end{fulllineitems}

\index{create\_population() (Community method)}

\begin{fulllineitems}
\phantomsection\label{Model/Community/Community:Model.Community.Community.Community.create_population}\pysiglinewithargsret{\sphinxbfcode{create\_population}}{\emph{set\_chromosomes}}{}
Crea una población usando un conjunto de cromosomas como base.
\begin{quote}\begin{description}
\item[{Parameters}] \leavevmode
\textbf{\texttt{set\_chromosomes}} (\emph{\texttt{List}}) -- Conjunto de cromosomas.

\item[{Returns}] \leavevmode
Model.Community.Population

\item[{Return type}] \leavevmode
Instance

\end{description}\end{quote}

\end{fulllineitems}

\index{evaluate\_population\_functions() (Community method)}

\begin{fulllineitems}
\phantomsection\label{Model/Community/Community:Model.Community.Community.Community.evaluate_population_functions}\pysiglinewithargsret{\sphinxbfcode{evaluate\_population\_functions}}{\emph{population}}{}~
\begin{DUlineblock}{0em}
\item[] Evalúa cada uno de los subcromosomas de los individuos de la 
población \textbf{(Population)}.
\item[] De manera adicional obtiene el listado de los valores extremos tanto
de variables de decisión como de funciones objetivo para el 
cálculo del sigma share \textbf{(véase el método \_\_using\_sharing\_function)}. 
\end{DUlineblock}
\begin{quote}\begin{description}
\item[{Parameters}] \leavevmode
\textbf{\texttt{population}} (\emph{\texttt{Instance}}) -- La población sobre la que se hará la operación.

\end{description}\end{quote}

\end{fulllineitems}

\index{execute\_crossover\_and\_mutation() (Community method)}

\begin{fulllineitems}
\phantomsection\label{Model/Community/Community:Model.Community.Community.Community.execute_crossover_and_mutation}\pysiglinewithargsret{\sphinxbfcode{execute\_crossover\_and\_mutation}}{\emph{selected\_parents\_chromosomes}}{}
Realiza la cruza y mutación de los individuos. Para el caso de la cruza ésta se lleva a cabo siempre
entre dos individuos, mientras que la mutación es unaria.
\begin{quote}\begin{description}
\item[{Parameters}] \leavevmode
\textbf{\texttt{selected\_parents\_chromosomes}} (\emph{\texttt{List}}) -- El conjunto de cromosomas sobre los cuales se aplicarán dichos operadores genéticos.

\item[{Returns}] \leavevmode
Una instancia del tipo Model.Community.Population.

\item[{Return type}] \leavevmode
Instance

\end{description}\end{quote}

\end{fulllineitems}

\index{execute\_selection() (Community method)}

\begin{fulllineitems}
\phantomsection\label{Model/Community/Community:Model.Community.Community.Community.execute_selection}\pysiglinewithargsret{\sphinxbfcode{execute\_selection}}{\emph{parents}}{}
Realiza la ejecución de la técnica de selección por medio de una instancia que
se creó previamente \textbf{(véase Controller/Verifier.py)}.
\begin{quote}\begin{description}
\item[{Parameters}] \leavevmode
\textbf{\texttt{parents}} (\emph{\texttt{Instance}}) -- El conjunto de individuos sobre el cual se aplicará la técnica

\item[{Returns}] \leavevmode
Una lista con los cromosomas de aquellos individuos seleccionados.

\item[{Return type}] \leavevmode
List

\end{description}\end{quote}

\end{fulllineitems}

\index{get\_best\_individual() (Community method)}

\begin{fulllineitems}
\phantomsection\label{Model/Community/Community:Model.Community.Community.Community.get_best_individual}\pysiglinewithargsret{\sphinxbfcode{get\_best\_individual}}{\emph{population}}{}
Obtiene el mejor individuo dentro de una población. Para estos fines el mejor individuo es aquél que
tenga mejor dominancia.
\begin{quote}\begin{description}
\item[{Parameters}] \leavevmode
\textbf{\texttt{population}} (\emph{\texttt{Instance}}) -- La población sobre la cual se hará la búsqueda.

\item[{Returns}] \leavevmode
El individuo que cumple con la característica de la mayor dominancia.

\item[{Return type}] \leavevmode
Instance

\end{description}\end{quote}

\end{fulllineitems}

\index{get\_results() (Community method)}

\begin{fulllineitems}
\phantomsection\label{Model/Community/Community:Model.Community.Community.Community.get_results}\pysiglinewithargsret{\sphinxbfcode{get\_results}}{\emph{best\_individual\_along\_generations}, \emph{external\_set\_population}}{}
Recolecta la información y la almacena en una estructura que contiene dos categorías principales: 
funciones objetivo y variables de decisión. Por cada una existen las subcategorías Pareto y mejor 
individuo, en referencia al óptimo o frente de Pareto \textbf{(según corresponda)} y a los valores del mejor 
individuo por generación \textbf{(véase View/Additional/ResultsGrapher/GraphFrame.py)}.
\begin{quote}\begin{description}
\item[{Parameters}] \leavevmode\begin{itemize}
\item {} 
\textbf{\texttt{best\_individual\_along\_generations}} (\emph{\texttt{List}}) -- Una lista que contiene los mejores individuos por generación.

\item {} 
\textbf{\texttt{external\_set\_population}} (\emph{\texttt{Instance}}) -- La población sobre la cual se efectuarán las operaciones.

\end{itemize}

\item[{Returns}] \leavevmode
Un diccionario con los elementos mostrados en la descripción.

\item[{Return type}] \leavevmode
Dictionary

\end{description}\end{quote}

\end{fulllineitems}

\index{init\_population() (Community method)}

\begin{fulllineitems}
\phantomsection\label{Model/Community/Community:Model.Community.Community.Community.init_population}\pysiglinewithargsret{\sphinxbfcode{init\_population}}{\emph{population\_size}}{}
Crea una población de manera aleatoria.
\begin{quote}\begin{description}
\item[{Parameters}] \leavevmode
\textbf{\texttt{population\_size}} (\emph{\texttt{Integer}}) -- El tamaño de la población.

\item[{Returns}] \leavevmode
Model.Community.Community.Population

\item[{Return type}] \leavevmode
Instance

\end{description}\end{quote}

\end{fulllineitems}


\end{fulllineitems}


La clase en cuestión se apoya de los siguientes elementos:


\subsubsection{Population (clase)}
\label{Model/Community/Population/Population:population-clase}\label{Model/Community/Population/Population:module-Model.Community.Population.Population}\label{Model/Community/Population/Population::doc}\index{Model.Community.Population.Population (module)}\index{Population (class in Model.Community.Population.Population)}

\begin{fulllineitems}
\phantomsection\label{Model/Community/Population/Population:Model.Community.Population.Population.Population}\pysiglinewithargsret{\sphinxstrong{class }\sphinxbfcode{Population}}{\emph{population\_size}, \emph{vector\_functions}, \emph{vector\_variables}, \emph{available\_expressions}, \emph{number\_of\_decimals}}{}
Consiste en un conjunto de instancias de la clase Individual, proporcionando además métodos y atributos que 
se manifiestan tanto en grupo como de manera individual.
\begin{quote}\begin{description}
\item[{Parameters}] \leavevmode\begin{itemize}
\item {} 
\textbf{\texttt{population\_size}} (\emph{\texttt{Integer}}) -- El tamaño de la población.

\item {} 
\textbf{\texttt{vector\_functions}} (\emph{\texttt{List}}) -- Lista con las funciones objetivo.

\item {} 
\textbf{\texttt{vector\_variables}} (\emph{\texttt{List}}) -- Lista con las variables de decisión y sus rangos.

\item {} 
\textbf{\texttt{available\_expressions}} (\emph{\texttt{Dictionary}}) -- Diccionario que contiene algunas funciones escritas como azúcar sintáctica
para que puedan ser utilizadas más fácilmente por el usuario y evaluadas
más ŕapidamente en el programa \textbf{(véase Controller/XML/PythonExpressions.xml)}.

\item {} 
\textbf{\texttt{number\_of\_decimals}} (\emph{\texttt{Integer}}) -- Número de decimales que tendrá cada solución \textbf{(tanto en variables de decisión como
funciones objetivo)}.

\end{itemize}

\item[{Returns}] \leavevmode
Model.Community.Population

\item[{Return type}] \leavevmode
Instance

\end{description}\end{quote}
\index{add\_individual() (Population method)}

\begin{fulllineitems}
\phantomsection\label{Model/Community/Population/Population:Model.Community.Population.Population.Population.add_individual}\pysiglinewithargsret{\sphinxbfcode{add\_individual}}{\emph{position}, \emph{complete\_chromosome}}{}
Añade un individuo a la Población.
\begin{quote}\begin{description}
\item[{Parameters}] \leavevmode\begin{itemize}
\item {} 
\textbf{\texttt{position}} (\emph{\texttt{Integer}}) -- La posición dentro del arreglo de individuos 
donde se colocará el nuevo elemento.

\item {} 
\textbf{\texttt{complete\_chromosome}} (\emph{\texttt{Array}}) -- El cromosoma del Individuo.

\end{itemize}

\end{description}\end{quote}

\end{fulllineitems}

\index{calculate\_population\_properties() (Population method)}

\begin{fulllineitems}
\phantomsection\label{Model/Community/Population/Population:Model.Community.Population.Population.Population.calculate_population_properties}\pysiglinewithargsret{\sphinxbfcode{calculate\_population\_properties}}{}{}
Calcula atributos individuales con base en los valores de toda la Población.

\end{fulllineitems}

\index{get\_decision\_variables\_extreme\_values() (Population method)}

\begin{fulllineitems}
\phantomsection\label{Model/Community/Population/Population:Model.Community.Population.Population.Population.get_decision_variables_extreme_values}\pysiglinewithargsret{\sphinxbfcode{get\_decision\_variables\_extreme\_values}}{}{}
Regresa el listado de los valores máximo y mínimo de las 
variables de decisión para el cálculo de sigma share.
\begin{quote}\begin{description}
\item[{Returns}] \leavevmode
Una colección con los valores máximo y mínimo para las
variables de decisión.

\item[{Return type}] \leavevmode
Dictionary

\end{description}\end{quote}

\end{fulllineitems}

\index{get\_individuals() (Population method)}

\begin{fulllineitems}
\phantomsection\label{Model/Community/Population/Population:Model.Community.Population.Population.Population.get_individuals}\pysiglinewithargsret{\sphinxbfcode{get\_individuals}}{}{}
Regresa los individuos de la Población.
\begin{quote}\begin{description}
\item[{Returns}] \leavevmode
Estructura que contiene a los Individuos de la Población.

\item[{Return type}] \leavevmode
Array

\end{description}\end{quote}

\end{fulllineitems}

\index{get\_length\_vector\_functions() (Population method)}

\begin{fulllineitems}
\phantomsection\label{Model/Community/Population/Population:Model.Community.Population.Population.Population.get_length_vector_functions}\pysiglinewithargsret{\sphinxbfcode{get\_length\_vector\_functions}}{}{}
Regresa el número de elementos del vector de funciones objetivo.
\begin{quote}\begin{description}
\item[{Returns}] \leavevmode
Número de funciones objetivo.

\item[{Return type}] \leavevmode
Integer

\end{description}\end{quote}

\end{fulllineitems}

\index{get\_objective\_functions\_extreme\_values() (Population method)}

\begin{fulllineitems}
\phantomsection\label{Model/Community/Population/Population:Model.Community.Population.Population.Population.get_objective_functions_extreme_values}\pysiglinewithargsret{\sphinxbfcode{get\_objective\_functions\_extreme\_values}}{}{}
Regresa el listado de los valores máximo y mínimo de las 
funciones objetivo para el cálculo de sigma share.
\begin{quote}\begin{description}
\item[{Returns}] \leavevmode
El listado con los valores máximo y mínimo para las
funciones objetivo.

\item[{Return type}] \leavevmode
List

\end{description}\end{quote}

\end{fulllineitems}

\index{get\_size() (Population method)}

\begin{fulllineitems}
\phantomsection\label{Model/Community/Population/Population:Model.Community.Population.Population.Population.get_size}\pysiglinewithargsret{\sphinxbfcode{get\_size}}{}{}
Otorga el tamaño de la Población.
\begin{quote}\begin{description}
\item[{Returns}] \leavevmode
El tamaño de la Población.

\item[{Return type}] \leavevmode
Integer

\end{description}\end{quote}

\end{fulllineitems}

\index{get\_total\_expected\_value() (Population method)}

\begin{fulllineitems}
\phantomsection\label{Model/Community/Population/Population:Model.Community.Population.Population.Population.get_total_expected_value}\pysiglinewithargsret{\sphinxbfcode{get\_total\_expected\_value}}{}{}
Regresa el valor esperado de la Población.
\begin{quote}\begin{description}
\item[{Returns}] \leavevmode
El valor esperado.

\item[{Return type}] \leavevmode
Float

\end{description}\end{quote}

\end{fulllineitems}

\index{get\_total\_fitness() (Population method)}

\begin{fulllineitems}
\phantomsection\label{Model/Community/Population/Population:Model.Community.Population.Population.Population.get_total_fitness}\pysiglinewithargsret{\sphinxbfcode{get\_total\_fitness}}{}{}
Captura el Fitness total de la Población.
\begin{quote}\begin{description}
\item[{Returns}] \leavevmode
El valor del Fitness poblacional.

\item[{Return type}] \leavevmode
Float

\end{description}\end{quote}

\end{fulllineitems}

\index{get\_vector\_variables() (Population method)}

\begin{fulllineitems}
\phantomsection\label{Model/Community/Population/Population:Model.Community.Population.Population.Population.get_vector_variables}\pysiglinewithargsret{\sphinxbfcode{get\_vector\_variables}}{}{}
Regresa el vector de variables de decisión.
\begin{quote}\begin{description}
\item[{Returns}] \leavevmode
Conjunto que contiene las variables de decisión con sus rangos.

\item[{Return type}] \leavevmode
List

\end{description}\end{quote}

\end{fulllineitems}

\index{print\_info() (Population method)}

\begin{fulllineitems}
\phantomsection\label{Model/Community/Population/Population:Model.Community.Population.Population.Population.print_info}\pysiglinewithargsret{\sphinxbfcode{print\_info}}{}{}
Imprime en texto las características de los Individuos
de la Población, tanto grupales como individuales \textbf{(en consola)}.

\end{fulllineitems}

\index{set\_decision\_variables\_extreme\_values() (Population method)}

\begin{fulllineitems}
\phantomsection\label{Model/Community/Population/Population:Model.Community.Population.Population.Population.set_decision_variables_extreme_values}\pysiglinewithargsret{\sphinxbfcode{set\_decision\_variables\_extreme\_values}}{\emph{decision\_variables\_extreme\_values}}{}
Actualiza el listado de valores máximo y mínimo de las
variables de decisión para el cálculo de sigma share.
\begin{quote}\begin{description}
\item[{Parameters}] \leavevmode
\textbf{\texttt{decision\_variables\_extreme\_values}} (\emph{\texttt{Dictionary}}) -- Un conjunto con los valores máximo y mínimo
de cada una de las variables de decisión.

\end{description}\end{quote}

\end{fulllineitems}

\index{set\_objective\_functions\_extreme\_values() (Population method)}

\begin{fulllineitems}
\phantomsection\label{Model/Community/Population/Population:Model.Community.Population.Population.Population.set_objective_functions_extreme_values}\pysiglinewithargsret{\sphinxbfcode{set\_objective\_functions\_extreme\_values}}{\emph{objective\_functions\_extreme\_values}}{}
Actualiza el listado de valores máximo y mínimo de las
funciones objetivo para el cálculo de sigma share.
\begin{quote}\begin{description}
\item[{Parameters}] \leavevmode
\textbf{\texttt{objective\_functions\_extreme\_values}} (\emph{\texttt{List}}) -- Una lista con los valores máximo y mínimo
de cada una de las funciones objetivo.

\end{description}\end{quote}

\end{fulllineitems}

\index{set\_total\_fitness() (Population method)}

\begin{fulllineitems}
\phantomsection\label{Model/Community/Population/Population:Model.Community.Population.Population.Population.set_total_fitness}\pysiglinewithargsret{\sphinxbfcode{set\_total\_fitness}}{\emph{value}}{}
Actualiza el Fitness total de la Población.
\begin{quote}\begin{description}
\item[{Parameters}] \leavevmode
\textbf{\texttt{value}} (\emph{\texttt{Float}}) -- El valor a actualizar.

\end{description}\end{quote}

\end{fulllineitems}

\index{shuffle\_individuals() (Population method)}

\begin{fulllineitems}
\phantomsection\label{Model/Community/Population/Population:Model.Community.Population.Population.Population.shuffle_individuals}\pysiglinewithargsret{\sphinxbfcode{shuffle\_individuals}}{}{}
Desordena los elementos de la Población.

\end{fulllineitems}

\index{sort\_individuals() (Population method)}

\begin{fulllineitems}
\phantomsection\label{Model/Community/Population/Population:Model.Community.Population.Population.Population.sort_individuals}\pysiglinewithargsret{\sphinxbfcode{sort\_individuals}}{\emph{method}, \emph{is\_descendent}}{}
Ordena los Individuos de acuerdo a algún criterio dado.
\begin{quote}\begin{description}
\item[{Parameters}] \leavevmode\begin{itemize}
\item {} 
\textbf{\texttt{method}} (\emph{\texttt{String}}) -- El método o atributo sobre el cual se hará la comparación.

\item {} 
\textbf{\texttt{is\_descendent}} (\emph{\texttt{Boolean}}) -- Indica si el ordenamiento es ascendente o descendente.

\end{itemize}

\end{description}\end{quote}

\end{fulllineitems}


\end{fulllineitems}


La clase actual tiene como base el siguiente elemento:


\paragraph{Individual (clase)}
\label{Model/Community/Population/Individual/Individual:individual-clase}\label{Model/Community/Population/Individual/Individual:module-Model.Community.Population.Individual.Individual}\label{Model/Community/Population/Individual/Individual::doc}\index{Model.Community.Population.Individual.Individual (module)}\index{Individual (class in Model.Community.Population.Individual.Individual)}

\begin{fulllineitems}
\phantomsection\label{Model/Community/Population/Individual/Individual:Model.Community.Population.Individual.Individual.Individual}\pysiglinewithargsret{\sphinxstrong{class }\sphinxbfcode{Individual}}{\emph{complete\_chromosome}, \emph{vector\_functions}, \emph{available\_expressions}, \emph{number\_of\_decimals}}{}~
\begin{DUlineblock}{0em}
\item[] La base de toda operación lógica.
\item[] Consiste en una abstracción de un elemento simple en función de un ecosistema.
Si bien la parte esencial es el cromosoma, en esta implementación se añaden algunos elementos extra
con la finalidad de facilitar ciertas operaciones.
\end{DUlineblock}
\begin{quote}\begin{description}
\item[{Parameters}] \leavevmode\begin{itemize}
\item {} 
\textbf{\texttt{complete\_chromosome}} (\emph{\texttt{Array}}) -- El cromosoma que conformará al Individuo.

\item {} 
\textbf{\texttt{vector\_functions}} (\emph{\texttt{List}}) -- Lista que contiene a las funciones objetivo.

\item {} 
\textbf{\texttt{available\_expressions}} (\emph{\texttt{Dictionary}}) -- Diccionario que contiene algunas funciones escritas como azúcar sintáctica
para que puedan ser utilizadas más fácilmente por el usuario y evaluadas
más ŕapidamente en el programa \textbf{(véase Controller/XML/PythonExpressions.xml)}.

\item {} 
\textbf{\texttt{number\_of\_decimals}} (\emph{\texttt{Integer}}) -- El número de decimales que deberá tener cada solución, influye en el
comportamiento del cromosoma.

\end{itemize}

\item[{Returns}] \leavevmode
Model.Community.Population.Individual

\item[{Return type}] \leavevmode
Instance

\end{description}\end{quote}
\index{\_Individual\_\_evaluate\_single\_function() (Individual method)}

\begin{fulllineitems}
\phantomsection\label{Model/Community/Population/Individual/Individual:Model.Community.Population.Individual.Individual.Individual._Individual__evaluate_single_function}\pysiglinewithargsret{\sphinxbfcode{\_Individual\_\_evaluate\_single\_function}}{\emph{function}, \emph{expressions}}{}~
\begin{notice}{note}{Note:}
Este método es privado.
\end{notice}

Evalúa una función objetivo.
\begin{quote}\begin{description}
\item[{Parameters}] \leavevmode\begin{itemize}
\item {} 
\textbf{\texttt{function}} (\emph{\texttt{String}}) -- La función que será evaluada.

\item {} 
\textbf{\texttt{expressions}} (\emph{\texttt{Dictionary}}) -- El diccionario que ayuda a evaluar la función.
Expressions = variables + constantes + funciones built-in.

\end{itemize}

\item[{Returns}] \leavevmode
La función evaluada.

\item[{Return type}] \leavevmode
Float

\end{description}\end{quote}

\end{fulllineitems}

\index{evaluate\_functions() (Individual method)}

\begin{fulllineitems}
\phantomsection\label{Model/Community/Population/Individual/Individual:Model.Community.Population.Individual.Individual.Individual.evaluate_functions}\pysiglinewithargsret{\sphinxbfcode{evaluate\_functions}}{\emph{decision\_variables}}{}
Evalúa todas las funciones objetivo.
\begin{quote}\begin{description}
\item[{Parameters}] \leavevmode
\textbf{\texttt{decision\_variables}} (\emph{\texttt{List}}) -- El vector de variables de decisión.

\end{description}\end{quote}

\end{fulllineitems}

\index{get\_complete\_chromosome() (Individual method)}

\begin{fulllineitems}
\phantomsection\label{Model/Community/Population/Individual/Individual:Model.Community.Population.Individual.Individual.Individual.get_complete_chromosome}\pysiglinewithargsret{\sphinxbfcode{get\_complete\_chromosome}}{}{}
Regresa el cromosoma del Individuo.
\begin{quote}\begin{description}
\item[{Returns}] \leavevmode
El cromosoma.

\item[{Return type}] \leavevmode
Array

\end{description}\end{quote}

\end{fulllineitems}

\index{get\_decision\_variables() (Individual method)}

\begin{fulllineitems}
\phantomsection\label{Model/Community/Population/Individual/Individual:Model.Community.Population.Individual.Individual.Individual.get_decision_variables}\pysiglinewithargsret{\sphinxbfcode{get\_decision\_variables}}{}{}
Da el vector de variables de decisión.
\begin{quote}\begin{description}
\item[{Returns}] \leavevmode
El vector de variables de decisión.

\item[{Return type}] \leavevmode
List

\end{description}\end{quote}

\end{fulllineitems}

\index{get\_evaluated\_functions() (Individual method)}

\begin{fulllineitems}
\phantomsection\label{Model/Community/Population/Individual/Individual:Model.Community.Population.Individual.Individual.Individual.get_evaluated_functions}\pysiglinewithargsret{\sphinxbfcode{get\_evaluated\_functions}}{}{}
Regresa las funciones objetivo evaluadas.
\begin{quote}\begin{description}
\item[{Returns}] \leavevmode
Las funciones objetivo evaluadas.

\item[{Return type}] \leavevmode
List

\end{description}\end{quote}

\end{fulllineitems}

\index{get\_expected\_value() (Individual method)}

\begin{fulllineitems}
\phantomsection\label{Model/Community/Population/Individual/Individual:Model.Community.Population.Individual.Individual.Individual.get_expected_value}\pysiglinewithargsret{\sphinxbfcode{get\_expected\_value}}{}{}~
\begin{DUlineblock}{0em}
\item[] Se obtiene el valor esperado del Individuo.
\item[] Por definición, el valor esperado es el número de posibles
hijos que puede tener un Individuo. Mientras más apto, más 
hijos.
\end{DUlineblock}
\begin{quote}\begin{description}
\item[{Returns}] \leavevmode
El valor esperado.

\item[{Return type}] \leavevmode
Float

\end{description}\end{quote}

\end{fulllineitems}

\index{get\_fitness() (Individual method)}

\begin{fulllineitems}
\phantomsection\label{Model/Community/Population/Individual/Individual:Model.Community.Population.Individual.Individual.Individual.get_fitness}\pysiglinewithargsret{\sphinxbfcode{get\_fitness}}{}{}
Regresa el Fitness del Individuo.
\begin{quote}\begin{description}
\item[{Returns}] \leavevmode
El Fitness.

\item[{Return type}] \leavevmode
Float

\end{description}\end{quote}

\end{fulllineitems}

\index{get\_niche\_count() (Individual method)}

\begin{fulllineitems}
\phantomsection\label{Model/Community/Population/Individual/Individual:Model.Community.Population.Individual.Individual.Individual.get_niche_count}\pysiglinewithargsret{\sphinxbfcode{get\_niche\_count}}{}{}
Regresa el valor niche para el Individuo.
\begin{quote}\begin{description}
\item[{Returns}] \leavevmode
El tamaño niche.

\item[{Return type}] \leavevmode
Float

\end{description}\end{quote}

\end{fulllineitems}

\index{get\_pareto\_dominated() (Individual method)}

\begin{fulllineitems}
\phantomsection\label{Model/Community/Population/Individual/Individual:Model.Community.Population.Individual.Individual.Individual.get_pareto_dominated}\pysiglinewithargsret{\sphinxbfcode{get\_pareto\_dominated}}{}{}
Regresa el número de soluciones que dominan al 
Individuo actual.
\begin{quote}\begin{description}
\item[{Returns}] \leavevmode
El número de soluciones que dominan a la actual.

\item[{Return type}] \leavevmode
Integer

\end{description}\end{quote}

\end{fulllineitems}

\index{get\_pareto\_dominates() (Individual method)}

\begin{fulllineitems}
\phantomsection\label{Model/Community/Population/Individual/Individual:Model.Community.Population.Individual.Individual.Individual.get_pareto_dominates}\pysiglinewithargsret{\sphinxbfcode{get\_pareto\_dominates}}{}{}
Regresa el número de soluciones que son dominadas por 
el actual Individuo.
\begin{quote}\begin{description}
\item[{Returns}] \leavevmode
El número de soluciones dominadas.

\item[{Return type}] \leavevmode
Integer

\end{description}\end{quote}

\end{fulllineitems}

\index{get\_rank() (Individual method)}

\begin{fulllineitems}
\phantomsection\label{Model/Community/Population/Individual/Individual:Model.Community.Population.Individual.Individual.Individual.get_rank}\pysiglinewithargsret{\sphinxbfcode{get\_rank}}{}{}
Regresa la puntuación \textbf{(rank)} que se le designó al Individuo
\textbf{(véase Model/Community/Community.py)}.
\begin{quote}\begin{description}
\item[{Returns}] \leavevmode
El rango.

\item[{Return type}] \leavevmode
Float

\end{description}\end{quote}

\end{fulllineitems}

\index{get\_vector\_functions() (Individual method)}

\begin{fulllineitems}
\phantomsection\label{Model/Community/Population/Individual/Individual:Model.Community.Population.Individual.Individual.Individual.get_vector_functions}\pysiglinewithargsret{\sphinxbfcode{get\_vector\_functions}}{}{}
Obtiene el vector de funciones objetivo.
\begin{quote}\begin{description}
\item[{Returns}] \leavevmode
El vector de funciones objetivo.

\item[{Return type}] \leavevmode
List

\end{description}\end{quote}

\end{fulllineitems}

\index{print\_info() (Individual method)}

\begin{fulllineitems}
\phantomsection\label{Model/Community/Population/Individual/Individual:Model.Community.Population.Individual.Individual.Individual.print_info}\pysiglinewithargsret{\sphinxbfcode{print\_info}}{}{}
Imprime las características básicas del Individuo \textbf{(en consola)}.

\end{fulllineitems}

\index{set\_expected\_value() (Individual method)}

\begin{fulllineitems}
\phantomsection\label{Model/Community/Population/Individual/Individual:Model.Community.Population.Individual.Individual.Individual.set_expected_value}\pysiglinewithargsret{\sphinxbfcode{set\_expected\_value}}{\emph{value}}{}
Actualiza el valor esperado del Individuo.
\begin{quote}\begin{description}
\item[{Parameters}] \leavevmode
\textbf{\texttt{value}} (\emph{\texttt{Float}}) -- El valor a actualizar.

\end{description}\end{quote}

\end{fulllineitems}

\index{set\_fitness() (Individual method)}

\begin{fulllineitems}
\phantomsection\label{Model/Community/Population/Individual/Individual:Model.Community.Population.Individual.Individual.Individual.set_fitness}\pysiglinewithargsret{\sphinxbfcode{set\_fitness}}{\emph{value}}{}
Actualiza el valor del Fitness.
\begin{quote}\begin{description}
\item[{Parameters}] \leavevmode
\textbf{\texttt{value}} (\emph{\texttt{Float}}) -- El valor a actualizar.

\end{description}\end{quote}

\end{fulllineitems}

\index{set\_niche\_count() (Individual method)}

\begin{fulllineitems}
\phantomsection\label{Model/Community/Population/Individual/Individual:Model.Community.Population.Individual.Individual.Individual.set_niche_count}\pysiglinewithargsret{\sphinxbfcode{set\_niche\_count}}{\emph{value}}{}
Actualiza el valor niche.
\begin{quote}\begin{description}
\item[{Parameters}] \leavevmode
\textbf{\texttt{value}} (\emph{\texttt{Float}}) -- El valor a actualizar.

\end{description}\end{quote}

\end{fulllineitems}

\index{set\_pareto\_dominated() (Individual method)}

\begin{fulllineitems}
\phantomsection\label{Model/Community/Population/Individual/Individual:Model.Community.Population.Individual.Individual.Individual.set_pareto_dominated}\pysiglinewithargsret{\sphinxbfcode{set\_pareto\_dominated}}{\emph{value}}{}
Actualiza el número de soluciones que dominan a la
solución actual.
\begin{quote}\begin{description}
\item[{Parameters}] \leavevmode
\textbf{\texttt{value}} (\emph{\texttt{Integer}}) -- El valor a actualizar.

\end{description}\end{quote}

\end{fulllineitems}

\index{set\_pareto\_dominates() (Individual method)}

\begin{fulllineitems}
\phantomsection\label{Model/Community/Population/Individual/Individual:Model.Community.Population.Individual.Individual.Individual.set_pareto_dominates}\pysiglinewithargsret{\sphinxbfcode{set\_pareto\_dominates}}{\emph{value}}{}
Actualiza el número de soluciones dominadas por el
Individuo actual.
\begin{quote}\begin{description}
\item[{Parameters}] \leavevmode
\textbf{\texttt{value}} (\emph{\texttt{Integer}}) -- El valor a actualizar.

\end{description}\end{quote}

\end{fulllineitems}

\index{set\_rank() (Individual method)}

\begin{fulllineitems}
\phantomsection\label{Model/Community/Population/Individual/Individual:Model.Community.Population.Individual.Individual.Individual.set_rank}\pysiglinewithargsret{\sphinxbfcode{set\_rank}}{\emph{rank}}{}
Actualiza el rango del Individuo.
\begin{quote}\begin{description}
\item[{Parameters}] \leavevmode
\textbf{\texttt{rank}} (\emph{\texttt{Float}}) -- El valor a actualizar.

\end{description}\end{quote}

\end{fulllineitems}


\end{fulllineitems}



\subsection{Fitness (módulo)}
\label{Model/Fitness/Fitness::doc}\label{Model/Fitness/Fitness:fitness-modulo}
Este módulo provee técnicas que calculan el Fitness \textbf{(ó Aptitud)} de los
Individuals \textbf{(ó Individuos)} de una Population \textbf{(ó Población)}.

Se entiende por Fitness a un número que indica la calidad del
Individuo \textbf{(en particular de sus variables de decisión)} frente a
las funciones objetivo al momento de ser evaluadas, esto es, a mayor Fitness, mayor
es la probalidad de que las variables de decisión del Individuo sean la solución óptima
para las funciones objetivo.

La asignación del Fitness depende en gran medida del ranking que se les haya
otorgado a los Individuos previamente \textbf{(véase Model/Community/Community.py)}.

Indirectamente, esto nos indica que un Individuo con un Fitness alto
tiene más probabilidades de ser elegido en los métodos de Selection \textbf{(ó Selección)}
y propagar su carga genética; así en las funciones de dicha sección \textbf{(Model/Operator/Selection)}
el criterio para escoger a un Individuo está basado comúnmente en su Fitness.

Al final la meta es que el usuario cree sus propias versiones de asignación
de Fitness, para lo cual es imperativo que, además de agregar la descripción de
la codificación a Controller/XML/Features.xml \textbf{(véase el archivo mencionado en la sección de código)},
se implemente la siguiente función:


\begin{fulllineitems}
\pysigline{\sphinxbfcode{assign\_fitness(population,fitness\_parameters):}}~
\begin{DUlineblock}{0em}
\item[] Realiza la asignación de Fitness de los individuos.
\item[] Dentro de esta se suelen usar métodos de la clase Population \textbf{(véase Model/Community/Population/Population.py)}
y de la clase Individual \textbf{(véase Model/Community/Population/Individual/Individual.py)}, por lo que es
muy recomendable que el usuario verifique las funciones disponibles. Algunas de las que se ocupan
más frecuentemente son:
\item[]
\begin{DUlineblock}{\DUlineblockindent}
\item[] \textbf{get\_rank (Individual)}
\item[] \textbf{set\_fitness (Individual).}
\item[] \textbf{set\_total\_fitness (Population)}
\item[] \textbf{calculate\_population\_properties (Population)}
\end{DUlineblock}
\end{DUlineblock}
\begin{quote}\begin{description}
\item[{Parameters}] \leavevmode\begin{itemize}
\item {} 
\textbf{\texttt{population}} (\emph{\texttt{Instance}}) -- La Población sobre la cual se hará el cálculo de Fitness por cada Individuo.

\item {} 
\textbf{\texttt{fitness\_parameters}} (\emph{\texttt{Dictionary}}) -- Un diccionario que puede contener opciones adicionales para el cálculo
de Fitness.

\end{itemize}

\end{description}\end{quote}

\end{fulllineitems}


Ahora se muestran los elementos que conforman el módulo actual:


\subsubsection{LinearRankingFitness (script)}
\label{Model/Fitness/LinearRankingFitness:linearrankingfitness-script}\label{Model/Fitness/LinearRankingFitness::doc}
\begin{DUlineblock}{0em}
\item[] Se implementa la asignación de Fitness conocida como Linear Ranking \textbf{(ó Ranking Lineal)}.
\item[] Es denominada así porque el Fitness se asigna con una función lineal que tiene como
fundamento la posición que ocupa el Individuo dentro de la Población.
\item[] El procedimiento es: tomando en cuenta el ranking asignado a los Individuals
\textbf{(ó Individuos)} por la clase Community \textbf{(véase Model/Community/Community.py)}
se ordenan de acuerdo a este valor y entonces el Fitness se basa en la posición
que cada uno de los Individuos ocupa. Más en específico, el Fitness está proporcionado
por la siguiente fórmula:
\end{DUlineblock}

\begin{center}\(Fitness(Individuo) = 2 - SP + \frac{2 \cdot (SP - 1) \cdot posici\acute{o}n(Individuo)}{tama\tilde{n}o\_poblaci\acute{o}n - 1}\)
\end{center}
\begin{DUlineblock}{0em}
\item[] Donde:
\item[]
\begin{DUlineblock}{\DUlineblockindent}
\item[] \textbf{SP (Selective Pressure ó Presión Selectiva)} es un valor que oscila entre 1 y 2.
\item[] \textbf{Posición(Individuo)} es la que ocupa el Individuo de acuerdo al rank.
\item[] 
\end{DUlineblock}
\item[] Haciendo un análisis somero en la fórmula, se puede apreciar que los
Individuos con mejor Fitness serán aquéllos que se encuentren en las últimas posiciones,
sin embargo los rankings que se manejan en este proyecto son inversamente proporcionales
a la calidad de los Individuos \textbf{(véase Model/Community/Community.py)};
por ello es importante ordenar a los Individuos de manera descendente para que la operación tenga sentido.
\item[] La función encargada de esto se llama sort\_individuals y está en \textbf{Model/Community/Population/Population.py}.
\end{DUlineblock}
\phantomsection\label{Model/Fitness/LinearRankingFitness:module-Model.Fitness.LinearRankingFitness}\index{Model.Fitness.LinearRankingFitness (module)}\index{assign\_fitness() (in module Model.Fitness.LinearRankingFitness)}

\begin{fulllineitems}
\phantomsection\label{Model/Fitness/LinearRankingFitness:Model.Fitness.LinearRankingFitness.assign_fitness}\pysiglinewithargsret{\sphinxbfcode{assign\_fitness}}{\emph{population}, \emph{fitness\_parameters}}{}
Se realiza la implementación del Fitness de tipo Linear Ranking
\textbf{(ó Ranking Lineal)} tomando en cuenta los datos proporcionados
en la parte superior.

\end{fulllineitems}



\subsubsection{NonLinearRankingFitness (script)}
\label{Model/Fitness/NonLinearRankingFitness::doc}\label{Model/Fitness/NonLinearRankingFitness:nonlinearrankingfitness-script}
\begin{DUlineblock}{0em}
\item[] Se implementa la asignación de Fitness conocida como Non-Linear Ranking
\textbf{(ó Ranking No Lineal)} que, a diferencia de los demás métodos, la aplica
usando como base la posición del Individual \textbf{(ó Individuo)} en la Population
\textbf{(ó Población)} como resultado de las operaciones de ranking
\textbf{(véase Model/Community/Community.py)}.
\item[] Posteriormente el Fitness se constituye tomando la posición del Individuo y
una función polinomial \textbf{(la cual es una función no lineal, de ahí el nombre)}.
\item[] La fórmula es la siguiente:
\end{DUlineblock}

\begin{center}\(Fitness(Individuo) = \frac{TP \cdot X^{posici\acute{o}n(Individuo)}}{\sum_{i=1}^{TP}X^{i - 1}}\)
\end{center}
\begin{DUlineblock}{0em}
\item[] Donde:
\item[]
\begin{DUlineblock}{\DUlineblockindent}
\item[] \textbf{TP} es el tamaño de la Población.
\item[] \textbf{Posición(Individuo)} es la que ocupa éste de acuerdo al ranking previo.
\item[] \textbf{X} es la solución al polinomio: \((SP - TP) \cdot X^{TP - 1} + SP \cdot X^{TP - 2} + ... + SP \cdot X + SP = 0\)
\item[] \textbf{SP (Selective Pressure ó Presión Selectiva)} varía entre 1 y 2.
\item[] 
\end{DUlineblock}
\item[] Haciendo un análisis somero en la fórmula, se puede apreciar que los
Individuos con mejor Fitness serán aquéllos que se encuentren en las últimas posiciones,
sin embargo los rankings que se manejan en este proyecto son inversamente proporcionales
a la calidad de los Individuos \textbf{(véase Model/Community/Community.py)};
por ello es importante ordenar a los Individuos de manera descendente para que la operación tenga sentido.
\item[] La función encargada de esto se llama sort\_individuals y está en \textbf{Model/Community/Population/Population.py}.
\end{DUlineblock}
\phantomsection\label{Model/Fitness/NonLinearRankingFitness:module-Model.Fitness.NonLinearRankingFitness}\index{Model.Fitness.NonLinearRankingFitness (module)}\index{assign\_fitness() (in module Model.Fitness.NonLinearRankingFitness)}

\begin{fulllineitems}
\phantomsection\label{Model/Fitness/NonLinearRankingFitness:Model.Fitness.NonLinearRankingFitness.assign_fitness}\pysiglinewithargsret{\sphinxbfcode{assign\_fitness}}{\emph{population}, \emph{fitness\_parameters}}{}
Utilizando la explicación concretada al principio, se realiza
la implementación de la asignación de Non-Linear Ranking Fitness \textbf{(ó Fitness de Ranking No Lineal)}.

\end{fulllineitems}

\index{calculate\_root() (in module Model.Fitness.NonLinearRankingFitness)}

\begin{fulllineitems}
\phantomsection\label{Model/Fitness/NonLinearRankingFitness:Model.Fitness.NonLinearRankingFitness.calculate_root}\pysiglinewithargsret{\sphinxbfcode{calculate\_root}}{\emph{polynome}, \emph{x\_0}, \emph{epsilon}}{}~\begin{description}
\item[{Calcula la solución de un polinomio usando el método Newton-Raphson.}] \leavevmode
A grandes rasgos el funcionamiento es el siguiente:

\end{description}

\begin{DUlineblock}{0em}
\item[] Tomando como base el punto \(x_0\) se obtiene \(x_1\) así:
\end{DUlineblock}

\begin{center}\(x_1 = x_0 - \frac{f(x_0)}{f'(x_0)}\)
\end{center}
\begin{DUlineblock}{0em}
\item[] Una vez obtenido \(x_1\) se calcula \(x_2\) de la misma manera:
\end{DUlineblock}

\begin{center}\(x_2 = x_1 - \frac{f(x_1)}{f'(x_1)}\)
\end{center}
\begin{DUlineblock}{0em}
\item[] El proceso se repite para `n' iteraciones hasta que el valor alcance la
precisión de epsilon ó el polinomio ya no tenga más derivadas. Concretando lo anterior:
\end{DUlineblock}

\begin{center}\(x_{n+1} = x_n - \frac{f(x_n)}{f'(x_n)}\)
\end{center}
\begin{DUlineblock}{0em}
\item[] Cuando \(x_{n+1}\) se acerque a epsilon ó cuando el
polinomio no sea más derivable el método se detendrá.
\end{DUlineblock}
\begin{quote}\begin{description}
\item[{Parameters}] \leavevmode\begin{itemize}
\item {} 
\textbf{\texttt{polynome}} (\emph{\texttt{List}}) -- El polinomio en el que se buscará la solución.

\item {} 
\textbf{\texttt{x\_0}} (\emph{\texttt{Float}}) -- el punto sobre el que se hará la evaluación del polinomio.

\item {} 
\textbf{\texttt{epsilon}} (\emph{\texttt{Float}}) -- La precisión decimal que se necesita para poder devolver
el resultado.

\end{itemize}

\item[{Returns}] \leavevmode
La solución al polinomio.

\item[{Return type}] \leavevmode
Float

\end{description}\end{quote}

\end{fulllineitems}

\index{derivate() (in module Model.Fitness.NonLinearRankingFitness)}

\begin{fulllineitems}
\phantomsection\label{Model/Fitness/NonLinearRankingFitness:Model.Fitness.NonLinearRankingFitness.derivate}\pysiglinewithargsret{\sphinxbfcode{derivate}}{\emph{polynome}}{}
Método que calcula la derivada de un polinomio, modificando
directamente éste sin regresar nada.
\begin{quote}\begin{description}
\item[{Parameters}] \leavevmode
\textbf{\texttt{polynome}} (\emph{\texttt{List}}) -- El polinomio inicial.

\end{description}\end{quote}

\end{fulllineitems}

\index{evaluate\_polynome() (in module Model.Fitness.NonLinearRankingFitness)}

\begin{fulllineitems}
\phantomsection\label{Model/Fitness/NonLinearRankingFitness:Model.Fitness.NonLinearRankingFitness.evaluate_polynome}\pysiglinewithargsret{\sphinxbfcode{evaluate\_polynome}}{\emph{polynome}, \emph{x}}{}
Evalúa un polinomio en un cierto valor.
\begin{quote}\begin{description}
\item[{Parameters}] \leavevmode\begin{itemize}
\item {} 
\textbf{\texttt{polynome}} (\emph{\texttt{List}}) -- El polinomio a evaluar.

\item {} 
\textbf{\texttt{x}} (\emph{\texttt{Float}}) -- El valor sobre el que se evaluará el polinomio.

\end{itemize}

\item[{Returns}] \leavevmode
La evaluación del polinomio.

\item[{Return type}] \leavevmode
Float

\end{description}\end{quote}

\end{fulllineitems}



\subsubsection{ProportionalFitness (script)}
\label{Model/Fitness/ProportionalFitness:proportionalfitness-script}\label{Model/Fitness/ProportionalFitness::doc}
\begin{DUlineblock}{0em}
\item[] Se desarrolla la asignación de Fitness conocida como Proportional \textbf{(ó Proporcional)}.
\item[] La función \textbf{(ó fórmula)} utilizada es la siguiente:
\end{DUlineblock}

\begin{center}\(Fitness(Individuo) = \frac{F_0(Individuo)}{\sum_{i=1}^{tama\tilde{n}o\_poblaci\acute{o}n}F_0(Individuo_i)}\)
\end{center}
\begin{DUlineblock}{0em}
\item[] Donde:
\item[]
\begin{DUlineblock}{\DUlineblockindent}
\item[] \(F_0\) es conocido como el valor de la función objetivo del Individuo. Nótese
\item[] que \(F_0\) debe ser proporcional al Fitness del Individuo.
\item[] 
\end{DUlineblock}
\item[] De acuerdo a la información provista anteriormente, la asignación es llamada así porque,
como dice el nombre, el Fitness de un Individuo corresponde a la parte proporcional
de la cantidad total de \(F_0\) de la Population \textbf{(ó Población)}.
\item[] De esta manera es posible ajustar los valores para que no existan Fitness dispares.
\item[] Con respecto de \(F_0\) es importante considerar que, dado que se está manejando
un sistema multi objetivo puede haber más de un valor en existencia,  por ello se necesita
una cantidad que conjunte estas evaluaciones el cual es el rank, sin embargo el rank es inversamente
proporcional a la calidad de un Individuo.
\item[] Entonces se debe hacer una modificación para garantizar que exista un valor
proporcional al Fitness del Individuo, por lo cual \(F_0\) se reescribe así:
\end{DUlineblock}

\begin{center}\(F_0(Individuo) = tama\tilde{n}o\_poblaci\acute{o}n - rank(Individuo)\)
\end{center}
\begin{DUlineblock}{0em}
\item[] Reescribiendo la fórmula inicial se tiene lo siguiente:
\end{DUlineblock}

\begin{center}\(Fitness(Individuo) = \frac{tama\tilde{n}o\_poblaci\acute{o}n - rank(Individuo)}{\sum_{i=1}^{tama\tilde{n}o\_poblaci\acute{o}n}[tama\tilde{n}o\_poblaci\acute{o}n - rank(Individuo_i)]}\)
\end{center}
\begin{DUlineblock}{0em}
\item[] Con esta actualización ya es posible obtener un Fitness acorde al rank del Individuo sin alterar
la esencia de la técnica.
\end{DUlineblock}
\phantomsection\label{Model/Fitness/ProportionalFitness:module-Model.Fitness.ProportionalFitness}\index{Model.Fitness.ProportionalFitness (module)}\index{assign\_fitness() (in module Model.Fitness.ProportionalFitness)}

\begin{fulllineitems}
\phantomsection\label{Model/Fitness/ProportionalFitness:Model.Fitness.ProportionalFitness.assign_fitness}\pysiglinewithargsret{\sphinxbfcode{assign\_fitness}}{\emph{population}, \emph{fitness\_parameters}}{}
Se implementa la asignación de Proportional Fitness \textbf{(ó Fitness Proporcional)}
con base en la información especificada con anterioridad.

\end{fulllineitems}



\subsection{Operator (módulo)}
\label{Model/Operator/Operator:operator-modulo}\label{Model/Operator/Operator::doc}
\begin{DUlineblock}{0em}
\item[] En éste se encuentran implementadas todas aquellas funcionalidades que intervengan
en el proceso de la creación de una nueva Population \textbf{(ó Población)} hija.
\item[] La finalidad de ésto es propagar y realizar combinaciones de la carga genética de
los Individuals \textbf{(ó Individuos)} más aptos mediante el cromosoma \textbf{(véase Model/ChromosomalRepresentation)} para obtener soluciones con una mejor calidad
que sus predecesoras.
\item[] Para este punto es importante mencionar que la calidad de un Individuo es directamente proporcional a su Fitness
\textbf{(véase Model/Fitness)}
\item[] En términos generales, la manera de construir una Población hija es la siguiente:
\end{DUlineblock}
\begin{itemize}
\item {} 
De la Población actual y tomando como base el Fitness de cada Individuo se seleccionan aquéllos que serán los elegidos para reproducirse. Nótese que un Individuo puede ser tomado en cuenta más de una vez si se da el caso.

\item {} 
Con base en los elegidos, se toman sus respectivos cromosomas y se realiza la operación de Crossover \textbf{(ó Cruza)}. Ésta es una simulación de una reproducción de tipo sexual donde se toman dos padres para ``procrear'' dos hijos. Las características de los hijos dependerán de las técnicas usadas \textbf{(véase Model/Operator/Crossover)}.

\item {} 
Se toman los hijos y uno a uno se les aplica la operación de mutación.

\end{itemize}

Al final Población hija constará de los hijos ``mutados''.

A continuación se muestran las siguientes subcategorías correspondientes a los pasos descritos anteriormente, cada una con sus respectivas técnicas desarrolladas:


\subsubsection{Selection (módulo)}
\label{Model/Operator/Selection/Selection:selection-modulo}\label{Model/Operator/Selection/Selection::doc}
\begin{DUlineblock}{0em}
\item[] En esta sección se encuentran implementadas todas las técnicas relacionadas con
la selección de Individuos.
\item[] Como se ha mencionado antes, durante dicha operación la importancia de la elección
radica en el Fitness de cada Individuo, además un Individuo puede ser seleccionado más de una vez
si la causa lo amerita.
\item[] Así, se elegirán tantos Individuos como elementos haya en la Población.
\item[] El objetivo radica en mantener el equilibrio entre una ``selección justa'' y
la oportunidad de permitir a los Individuos con una calidad media o baja la propagación de su carga genética.
\item[] 
\item[] Al final se busca que el usuario desarrolle sus propias técnicas de selección, por lo cual, además
de añadir el método en el listado localizado en \textbf{Controller/XML/Features.xml}, deberá implementar la siguiente
función:
\end{DUlineblock}


\begin{fulllineitems}
\pysigline{\sphinxbfcode{execute\_selection\_technique(population,selection\_parameters):}}~
\begin{DUlineblock}{0em}
\item[] Lleva a cabo la selección de Individuos de una Población. Es importante recalcar que, la función que
más se ocupa es:
\end{DUlineblock}

\begin{center}\textbf{get\_fitness (Model/Community/Population/Individual.py)}
\end{center}
\begin{DUlineblock}{0em}
\item[] Aunque existen otras que pueden tener relevancia para el usuario \textbf{(véase Model/Community/Population.py)}.
\item[] Como medida adicional, para los eventos de Crossover \textbf{(ó Cruza)} y Mutation \textbf{(ó Mutación)} se recomienda
ampliamente que este método regrese únicamente los cromosomas asociados a los Individuos, ya que ésto facilita sobremanera
las operaciones mencionadas.
\end{DUlineblock}
\begin{quote}\begin{description}
\item[{Parameters}] \leavevmode\begin{itemize}
\item {} 
\textbf{\texttt{population}} (\emph{\texttt{Instance}}) -- La Población sobre la cual se se seleccionarán los Individuos.

\item {} 
\textbf{\texttt{selection\_parameters}} (\emph{\texttt{Dictionary}}) -- Un diccionario que puede contener opciones adicionales para la
selección de Individuos.

\end{itemize}

\item[{Returns}] \leavevmode
Una lista que contiene los cromosomas de los Individuos seleccionados.

\item[{Return type}] \leavevmode
List

\end{description}\end{quote}

\end{fulllineitems}


A continuación se vislumbran los elementos característicos de este módulo:


\paragraph{Roulette (script)}
\label{Model/Operator/Selection/Roulette::doc}\label{Model/Operator/Selection/Roulette:roulette-script}
\begin{DUlineblock}{0em}
\item[] Se implementa el método de selección conocido como Roulette \textbf{(ó Ruleta)}. También es llamado
Proportional Selection \textbf{(ó Selección Proporcional)}.
\item[] En la función se distinguen dos etapas principales: construir la ruleta y ``ponerla a girar'' para que se elija el elemento.
\end{DUlineblock}
\begin{itemize}
\item {} 
Para la primera etapa se toma como base el Valor Esperado \textbf{(ó Expected Value)} de cada Individuo \textbf{(véase Model/Community/Population/Individual.py)}.

\end{itemize}
\begin{quote}

El Valor Esperado para fines de este proyecto es el número de ``hijos'' que un Individuo puede ofrecer. Éste se calcula de la siguiente forma:
\end{quote}

\begin{center}\(Valor\_Esperado(Individuo) = \frac{tama\tilde{n}o\_poblaci\acute{o}n \cdot Fitness(Individuo)}{\sum_{i=1}^{tama\tilde{n}o\_poblaci\acute{o}n}Fitness(Individuo_i)}\)
\end{center}
\begin{DUlineblock}{0em}
\item[] Al final aquéllos con Valores Esperados altos tendrán lugar a mayores espacios en la ruleta y por ende su probabilidad de elección aumenta.
\end{DUlineblock}
\begin{itemize}
\item {} 
Para recorrer la ruleta en realidad se toma un valor aleatorio entre 0 y la suma de los Valores Esperados. Entonces se van sumando los Valores Esperados de los Individuos hasta que se exceda el valor aleatorio mencionado antes. Aquel elemento cuyo Valor Esperado haya excedido la suma se considera el elegido y es seleccionado para la etapa de cruza.

\end{itemize}

\begin{DUlineblock}{0em}
\item[] Para la selección de Individuos se efectúa la segunda operación tantas veces como el tamaño de la Población.
\item[] Cabe mencionar que el Valor Esperado ya se calcula de manera automática en este proyecto \textbf{(véase Model/Community/Population/Population.py)}.
\end{DUlineblock}
\phantomsection\label{Model/Operator/Selection/Roulette:module-Model.Operator.Selection.Roulette}\index{Model.Operator.Selection.Roulette (module)}\index{execute\_selection\_technique() (in module Model.Operator.Selection.Roulette)}

\begin{fulllineitems}
\phantomsection\label{Model/Operator/Selection/Roulette:Model.Operator.Selection.Roulette.execute_selection_technique}\pysiglinewithargsret{\sphinxbfcode{execute\_selection\_technique}}{\emph{population}, \emph{selection\_parameters}}{}
De acuerdo al proceso descrito anteriormente, se implementa
la técnica conocida como Roulette \textbf{(ó Ruleta)}.

\end{fulllineitems}



\paragraph{ProbabilisticTournament (script)}
\label{Model/Operator/Selection/ProbabilisticTournament::doc}\label{Model/Operator/Selection/ProbabilisticTournament:probabilistictournament-script}
\begin{DUlineblock}{0em}
\item[] Se desarrolla la técnica conocida como Torneo Probabilístico \textbf{(ó Probabilistic Tournament)}.
\item[] Tal como lo sugiere el nombre, la selección será llevada a cabo en forma de competencia
directa entre los Individuos.
\item[] Tradicionalmente se comparan sus Fitness y de esta manera el Individuo ganador es aquél con
la cantidad mayor de Fitness, pero dado que se maneja un esquema probabilístico la decisión
no depende totalmente del factor antes mencionado.
\item[] 
\item[] De esta manera se pueden recapitular los siguientes pasos:
\end{DUlineblock}
\begin{itemize}
\item {} 
Tomar k \((2 \leqslant k \leqslant tama\tilde{n}o\_poblaci\acute{o}n)\) Individuos de la Población.

\item {} 
Realizar el torneo de manera secuencial entre los elementos seleccionados anteriormente, esto es, tomar el elemento A y enfrentarlo con B, al resultado de la batalla anterior enfrentarlo con C y así sucesivamente.

\end{itemize}
\begin{quote}

Para ello por cada encuentro se crea un número aleatorio entre 0 y 1, si el número es menor a 0.5
se toma al elemento con menor Fitness, de lo contrario se elige al de mayor Fitness.
La operación se lleva a cabo hasta que se tenga un ganador de los k Individuos.
\end{quote}

\begin{DUlineblock}{0em}
\item[] Los dos pasos anteriores se repiten hasta que se hayan obtenido tantos Individuos
como el tamaño de la Población.
\end{DUlineblock}
\phantomsection\label{Model/Operator/Selection/ProbabilisticTournament:module-Model.Operator.Selection.ProbabilisticTournament}\index{Model.Operator.Selection.ProbabilisticTournament (module)}\index{execute\_selection\_technique() (in module Model.Operator.Selection.ProbabilisticTournament)}

\begin{fulllineitems}
\phantomsection\label{Model/Operator/Selection/ProbabilisticTournament:Model.Operator.Selection.ProbabilisticTournament.execute_selection_technique}\pysiglinewithargsret{\sphinxbfcode{execute\_selection\_technique}}{\emph{population}, \emph{selection\_parameters}}{}
Tomando en cuenta las bases descritas previamente, se implementa
el método conocido como Probabilistic Tournament \textbf{(ó Torneo Probabilístico)}.

\end{fulllineitems}



\paragraph{StochasticUniversalSampling (script)}
\label{Model/Operator/Selection/StochasticUniversalSampling::doc}\label{Model/Operator/Selection/StochasticUniversalSampling:stochasticuniversalsampling-script}
\begin{DUlineblock}{0em}
\item[] Se determina la técnica conocida como Stochastic Universal Sampling
\textbf{(ó Muestreo Estocástico Universal)}.
\item[] Primero que nada es menester mencionar que es necesario el uso del Expected Value \textbf{(ó Valor Esperado)} de cada Individuo.
\item[] Para fines concernientes a este proyecto, se trata del número de ``hijos'' que un Individuo puede ofrecer. Éste se calcula de la siguiente forma:
\end{DUlineblock}

\begin{center}\(Valor\_Esperado(Individuo) = \frac{tama\tilde{n}o\_poblaci\acute{o}n \cdot Fitness(Individuo)}{\sum_{i=1}^{tama\tilde{n}o\_poblaci\acute{o}n}Fitness(Individuo_i)}\)
\end{center}
\begin{DUlineblock}{0em}
\item[] Con base a lo anterior, el método consiste en lo siguiente:
\end{DUlineblock}
\begin{itemize}
\item {} 
Se selecciona un valor aleatorio entre 0 y 1, a éste se le llamará Pointer \textbf{(ó Puntero)}

\item {} 
De manera secuencial se seleccionarán tantos Individuos como el tamaño de la población, los cuales deben estar igualmente espaciados en su Valor Esperado tomando como referencia el valor de Pointer.

\end{itemize}

\begin{DUlineblock}{0em}
\item[] Es importante aclarar el segundo punto, así que se abordará desde una perspectiva computacional:
\end{DUlineblock}
\begin{itemize}
\item {} 
Se deben tener variables adicionales que indiquen la acumulación tanto del Pointer \textbf{(CP, Cumulative Pointers)} como de los Valores Esperados \textbf{(CEV, Cumulative Expected Value)} así como al Individuo actual que está siendo seleccionado \textbf{(I)}.

\item {} 
Para averiguar si un Individuo está igualmente espaciado en su Valor Esperado con respecto de los demás basándose en Pointer, basta con corroborar que:

\end{itemize}

\begin{center}\(CP + Pointer > CEV + EV\)
\end{center}\begin{itemize}
\item {} 
Si la condición descrita es verdadera los valores EV e I deben actualizarse \textbf{(I se ajusta al siguiente Individuo)} ya que esto indica que se buscará al siguiente Individuo espaciado equitativamente con el valor Pointer. No se hace nada si la condición es falsa.

\item {} 
Independientemente del valor de la condición anterior, CP y CEV deben actualizarse durante todo el ciclo.

\end{itemize}

\begin{DUlineblock}{0em}
\item[] Cabe mencionar que si la lista de Individuos se agota, se puede volver a iterar
desde el inicio teniendo cautela en conservar CEV y CP.
\end{DUlineblock}
\phantomsection\label{Model/Operator/Selection/StochasticUniversalSampling:module-Model.Operator.Selection.StochasticUniversalSampling}\index{Model.Operator.Selection.StochasticUniversalSampling (module)}\index{execute\_selection\_technique() (in module Model.Operator.Selection.StochasticUniversalSampling)}

\begin{fulllineitems}
\phantomsection\label{Model/Operator/Selection/StochasticUniversalSampling:Model.Operator.Selection.StochasticUniversalSampling.execute_selection_technique}\pysiglinewithargsret{\sphinxbfcode{execute\_selection\_technique}}{\emph{population}, \emph{selection\_parameters}}{}
De acuerdo a la información provista anteriormente, se implementa
el método conocido como Stochastic Universal Sampling \textbf{(ó Muestreo Estocástico Universal)}.

\end{fulllineitems}



\subsubsection{Crossover (módulo)}
\label{Model/Operator/Crossover/Crossover::doc}\label{Model/Operator/Crossover/Crossover:crossover-modulo}
\begin{DUlineblock}{0em}
\item[] Aquí se desarrollan las técnicas de Crossover \textbf{(ó Cruza)}.
\item[] Prosiguiendo con el ciclo de creación de una nueva Población, es en este apartado donde
se lleva a cabo la concepción de nuevos Individuos.
\item[] 
\item[] Debido a esto se busca crear ``hijos'' más aptos que respondan mejor ante la problemática
fundamentada, es decir, concebir soluciones que se adapten mejor a los criterios establecidos
por el usuario desde un inicio basándose en las soluciones predecesoras.
\item[] 
\item[] Es menester mencionar que esta función es meramente binaria, lo cual significa que siempre
deben haber dos padres, además se debe hacer hincapié en que la Cruza se ejecuta a nivel
cromosómico \textbf{(véase Model/ChromosomalRepresentation)},por lo que se debe tener mesura
con el tratamiento de los métodos, dicho de otra manera, cada Representación Cromosómica debe ir
acompañada de al menos una función de Cruza.
\item[] 
\item[] Como dato para posteriores referencias, un gen hace referencia a una casilla del cromosoma,
mientras que un alelo es el valor que puede existir en un gen.
\item[] 
\item[] Entonces se persigue que el usuario construya sus propias funciones de Cruza, para lo cual,
además de añadir el método en el listado localizado en \textbf{Controller/XML/Features.xml}, deberá
implementar la siguiente función:
\end{DUlineblock}


\begin{fulllineitems}
\pysigline{\sphinxbfcode{execute\_crossover\_technique(chromosome\_a,chromosome\_b,crossover\_parameters):}}~
\begin{DUlineblock}{0em}
\item[] Lleva a cabo la cruza de dos Individuos a nivel cromosómico.
\item[] Además esta función debe retornar siempre dos hijos los cuales serán la cruza de A con B y
la cruza de B con A, esto nos indica que, con el objetivo de incrementar la calidad de los Individuos
sin perder la carga genética ganada o introducir elementos riesgosos, la cruza consiste en generar un
nuevo Individuo y su recíproco; así se garantiza una adecuada y controlada descendencia.
\item[] Finalmente, esta función debe contar con la probabilidad de Cruza, la cual indica si se debe o
no hacer la operación cromosómica; en caso de ser la respuesta negativa los hijos resultan en copias
idénticas de los padres.
\end{DUlineblock}
\begin{quote}\begin{description}
\item[{Parameters}] \leavevmode\begin{itemize}
\item {} 
\textbf{\texttt{chromosome\_a}} (\emph{\texttt{List}}) -- El cromosoma del Individuo A.

\item {} 
\textbf{\texttt{chromosome\_b}} (\emph{\texttt{List}}) -- El cromosoma del Individuo B.

\item {} 
\textbf{\texttt{crossover\_parameters}} (\emph{\texttt{Dictionary}}) -- Un diccionario que puede contener opciones adicionales para la
cruza de Individuos.

\end{itemize}

\item[{Returns}] \leavevmode
Un arreglo con dos cromososomas, el primero es la cruza de A con B, mientras que el segundo
es la cruza de B con A.

\item[{Return type}] \leavevmode
Array

\end{description}\end{quote}

\end{fulllineitems}


Se colocan los elementos alusivos a este módulo:


\paragraph{NPointsCrossover (script)}
\label{Model/Operator/Crossover/NPointsCrossover:npointscrossover-script}\label{Model/Operator/Crossover/NPointsCrossover::doc}
\begin{DUlineblock}{0em}
\item[] Se implementa el método que lleva por nombre N-Points Crossover \textbf{(ó Cruza en N-Puntos)}.
Para comenzar, esta técnica está elaborada para usarse tanto por Representación Cromosómica
\textbf{(véase Model/ChromosomalRepresentation)} de tipo FloatPoint \textbf{(ó de Punto Flotante)} como
Binary \textbf{(ó Binaria)}.
\item[] Su funcionamiento consiste en construir a los descendientes usando sub-bloques de cromosomas de cada
uno de los padres, determinados éstos por una cierta cantidad de puntos de corte, de ahí el nombre.
\item[] Aterrizando lo anterior de una manera concisa se tiene lo siguiente:
\end{DUlineblock}
\begin{itemize}
\item {} 
Consideremos a los cromosomas de los padres Padre I: \(I_1I_2...I_n\)

\end{itemize}
\begin{quote}

y Padre J: \(J_1J_2...J_n\)
\end{quote}
\begin{itemize}
\item {} 
Posteriormente se determinan aleatoriamente los puntos de corte, cabe mencionar que si los cromosomas son de tamaño n, pueden existir máximo n - 1 puntos. Supongamos que se crean k puntos \((1 \leqslant k \leqslant n - 1)\) y por lo tanto cada cromosoma queda separado en k + 1 bloques.

\end{itemize}
\begin{quote}

De esta manera obtenemos:
Padre I en bloques \textbf{(BI)}: \(BI_1BI_2...BI_{k + 1}\);
Padre J en bloques \textbf{(BJ)}: \(BJ_1BJ_2...BJ_{k + 1}\).
\end{quote}
\begin{itemize}
\item {} 
Finalmente cada hijo constará de la alternancia de bloques de manera secuencial comenzando por el bloque inicial de un padre determinado, dicho de otra forma, los hijos estarán constituidos de la siguiente manera:

\end{itemize}
\begin{itemize}
\item {} 
Para el hijo \(H_1\): \(BI_1BJ_2...BI_{k + 1}\)

\item {} 
Para el hijo \(H_2\): \(BJ_1BI_2...BJ_{k + 1}\)

\end{itemize}

\begin{DUlineblock}{0em}
\item[] Sólo queda mencionar que hasta el cierre de este proyecto no existe una manera
transparente desde el View \textbf{(ó Vista)} de conocer, dada una representación Binaria
y un conjunto de variables de decisión y funciones objetivo, el número máximo de puntos
de corte permitidos para este procedimiento, sin embargo, una manera de mitigar esta situación
fue contemplar algún posible caso de error en esta sección y mandar un mensaje de error a la Vista
por si llegase a suceder algún desperfecto durante el proceso.
\end{DUlineblock}
\phantomsection\label{Model/Operator/Crossover/NPointsCrossover:module-Model.Operator.Crossover.NPointsCrossover}\index{Model.Operator.Crossover.NPointsCrossover (module)}\index{execute\_crossover\_technique() (in module Model.Operator.Crossover.NPointsCrossover)}

\begin{fulllineitems}
\phantomsection\label{Model/Operator/Crossover/NPointsCrossover:Model.Operator.Crossover.NPointsCrossover.execute_crossover_technique}\pysiglinewithargsret{\sphinxbfcode{execute\_crossover\_technique}}{\emph{chromosome\_a}, \emph{chromosome\_b}, \emph{crossover\_parameters}}{}
Usando como base la información proporcionada anteriormente, se implementa
el método conocido como N-Points Crossover \textbf{(ó Cruza en `N' Puntos)}.

\end{fulllineitems}



\paragraph{UniformCrossover (script)}
\label{Model/Operator/Crossover/UniformCrossover:uniformcrossover-script}\label{Model/Operator/Crossover/UniformCrossover::doc}
\begin{DUlineblock}{0em}
\item[] Se lleva a cabo la implementación de la técnica conocida como Uniform Crossover
\textbf{(ó Cruza Uniforme)}.
\item[] Primero que nada esta operación está fabricada para usarse tanto con la Representación
Cromosómica \textbf{(véase Model/ChromosomalRepresentation)}
de tipo FloatPoint \textbf{(ó Punto Flotante)} como Binary \textbf{(ó Binaria)}.
\item[] 
\item[] La característica de este procedimiento es crear nuevos Individuos intercambiando
secuencialmente los genes de sus padres; visto de una manera más estructurada consiste en lo siguiente:
\end{DUlineblock}
\begin{itemize}
\item {} 
Tenemos a los cromosomas de los padres Padre A: \(A_1A_2...A_n\)

\end{itemize}
\begin{quote}

y Padre B: \(B_1B_2...B_n\)
\end{quote}
\begin{itemize}
\item {} 
Ahora, cada hijo será construido con genes de uno y sólo uno de los padres a menos que se indique lo contrario; este movimiento será posible con una variable denominada Pmask \textbf{(Pm)} que toma valores de 0 a 1 y una probabilidad de Pmask \textbf{(Pp)} que también toma valores de 0 a 1. Entonces lo anterior se puede declarar así:

\end{itemize}
\begin{quote}
\begin{itemize}
\item {} 
Para el hijo \((H_1)\) que tomará sus genes del padre A \textbf{(PA)}:

\end{itemize}
\begin{quote}

Si \(Pm \leqslant Pp\ entonces\ H_1(i) = A_i,\ en\ otro\ caso\ H_1(i) = B_i; 1 \leqslant i \leqslant n\)
\end{quote}
\begin{itemize}
\item {} 
Para el hijo \((H_2)\) que tomará sus genes del padre B \textbf{(PB)}:

\end{itemize}
\begin{quote}

Si \(Pm \leqslant Pp\ entonces\ H_2(i) = B_i,\ en\ otro\ caso\ H_1(i) = A_i; 1 \leqslant i \leqslant n\)
\end{quote}
\end{quote}
\phantomsection\label{Model/Operator/Crossover/UniformCrossover:module-Model.Operator.Crossover.UniformCrossover}\index{Model.Operator.Crossover.UniformCrossover (module)}\index{execute\_crossover\_technique() (in module Model.Operator.Crossover.UniformCrossover)}

\begin{fulllineitems}
\phantomsection\label{Model/Operator/Crossover/UniformCrossover:Model.Operator.Crossover.UniformCrossover.execute_crossover_technique}\pysiglinewithargsret{\sphinxbfcode{execute\_crossover\_technique}}{\emph{chromosome\_a}, \emph{chromosome\_b}, \emph{crossover\_parameters}}{}
Tomando en cuenta la información proporcionada con anterioridad, 
se implementa el método conocido como Uniform Crossover \textbf{(ó Cruza Uniforme)}.

\end{fulllineitems}



\subsubsection{Mutation (módulo)}
\label{Model/Operator/Mutation/Mutation:mutation-modulo}\label{Model/Operator/Mutation/Mutation::doc}
\begin{DUlineblock}{0em}
\item[] En esta parte se encuentran detalladas las técnicas relacionadas con
Mutation \textbf{(ó Mutación)}.
\item[] Retomando el proceso de creación de una nueva Población, es aquí donde
una vez obtenidos los hijos, se modifican pequeñas porciones \textbf{(genes)} de sus cromosomas
de manera individual.
\item[] Con ésto se persigue principalmente que estas ínfimas alteraciones permitan
incrementar la exploración del material genético y por ende otorgar Individuos
aún más aptos sin caer en el peligro de perder características valiosas en la
Población.
\item[] 
\item[] Considerando lo anterior, lo primero que hay que tomar en cuenta es que
la operación de Mutación es unaria, esto significa que sólo se puede mutar
el cromosoma de un Individuo a la vez.
\item[] También y reiterando la información pasada, la Mutación es una operación
que se lleva a cabo a nivel cromosómico \textbf{(véase Model/ChromosomalRepresentation)},
por lo que se debe tener mesura  con el tratamiento de los métodos, dicho de otra manera,
cada Representación Cromosómica debe ir acompañada de al menos una función de Mutación.
\item[] 
\item[] Como dato para posteriores referencias, un gen hace referencia a una casilla del cromosoma,
mientras que un alelo es el valor que puede existir en un gen.
\item[] 
\item[] Así, se invita a que el usuario construya sus propias versiones
de Mutación, por lo cual, además de añadir el método en el listado localizado en
\textbf{Controller/XML/Features.xml}, deberá implementar la siguiente función:
\end{DUlineblock}


\begin{fulllineitems}
\pysigline{\sphinxbfcode{execute\_mutation\_technique(chromosome,mutation\_parameters):}}~
\begin{DUlineblock}{0em}
\item[] Lleva a cabo mutación del Individuo a nivel cromosómico.
\item[] A grandes rasgos, modifica los alelos de los genes tomando en cuenta
la gama de valores a los que se pueden transformar \textbf{(por ejemplo, una mutación de representación
Binaria puede transformarse sólo en valores 0 ó 1)}.
\item[] El método debe retornar siempre el cromosoma mutado.
\item[] Finalmente, esta función debe contar con la probabilidad de Mutación,
la cual indica si se debe o no hacer la operación cromosómica por cada gen; en caso
de ser la respuesta negativa el Individuo no experimenta modificación alguna en el gen y
se pasa al siguiente y así sucesivamente.
\end{DUlineblock}
\begin{quote}\begin{description}
\item[{Parameters}] \leavevmode\begin{itemize}
\item {} 
\textbf{\texttt{chromosome}} (\emph{\texttt{List}}) -- El cromosoma para ser mutado.

\item {} 
\textbf{\texttt{mutation\_parameters}} (\emph{\texttt{Dictionary}}) -- Un diccionario que puede contener opciones adicionales para la
mutación del cromosoma del Individuo.

\end{itemize}

\item[{Returns}] \leavevmode
El cromosoma modificado.

\item[{Return type}] \leavevmode
List

\end{description}\end{quote}

\end{fulllineitems}


A continuación se muestran los elementos concernientes a este módulo:


\paragraph{BinaryMutation (script)}
\label{Model/Operator/Mutation/BinaryMutation:binarymutation-script}\label{Model/Operator/Mutation/BinaryMutation::doc}
\begin{DUlineblock}{0em}
\item[] Se implementa el método conocido como Binary Mutation \textbf{(ó Mutación Binaria)}.
\item[] El procedimiento es el siguiente:
\end{DUlineblock}
\begin{itemize}
\item {} 
Se trata cada gen individualmente y se modifica de acuerdo a una probabilidad de Mutación asignada, si ésta es suficiente se procede a hacer el cambio, en otro caso se deja el alelo asociado al gen intacto.

\item {} 
Retomando el caso en que se puede modificar el alelo del gen se verifica su valor actual y ya que se maneja una representación Binaria su transformación es muy simple: si se encuentra un 0, el alelo toma el valor 1 y viceversa.

\end{itemize}
\phantomsection\label{Model/Operator/Mutation/BinaryMutation:module-Model.Operator.Mutation.BinaryMutation}\index{Model.Operator.Mutation.BinaryMutation (module)}\index{execute\_mutation\_technique() (in module Model.Operator.Mutation.BinaryMutation)}

\begin{fulllineitems}
\phantomsection\label{Model/Operator/Mutation/BinaryMutation:Model.Operator.Mutation.BinaryMutation.execute_mutation_technique}\pysiglinewithargsret{\sphinxbfcode{execute\_mutation\_technique}}{\emph{chromosome}, \emph{mutation\_parameters}}{}
Usando la información mostrada anteriormente, se desarrolla la función
conocida como Binary Mutation \textbf{(ó Mutación Binaria)}.

\end{fulllineitems}



\paragraph{FloatPointMutation (script)}
\label{Model/Operator/Mutation/FloatPointMutation:floatpointmutation-script}\label{Model/Operator/Mutation/FloatPointMutation::doc}
\begin{DUlineblock}{0em}
\item[] Se concreta el método conocido como Float Point Mutation \textbf{(ó Mutación de Punto Flotante)}.
\item[] El procedimiento es el siguiente:
\end{DUlineblock}
\begin{itemize}
\item {} 
Se trata cada gen individualmente y se modifica de acuerdo a una probabilidad de Mutación asignada, si ésta es suficiente se procede a hacer el cambio, en otro caso se deja el alelo asociado al gen intacto.

\item {} 
Retomando el caso en que se puede modificar el alelo del gen se verifica los límites de la variable de decisión que está ligada a éste, así como la precisión decimal. Entonces se crea el nuevo número con la precisión decimal requerida y se sustituye por el anterior.

\end{itemize}
\phantomsection\label{Model/Operator/Mutation/FloatPointMutation:module-Model.Operator.Mutation.FloatPointMutation}\index{Model.Operator.Mutation.FloatPointMutation (module)}\index{execute\_mutation\_technique() (in module Model.Operator.Mutation.FloatPointMutation)}

\begin{fulllineitems}
\phantomsection\label{Model/Operator/Mutation/FloatPointMutation:Model.Operator.Mutation.FloatPointMutation.execute_mutation_technique}\pysiglinewithargsret{\sphinxbfcode{execute\_mutation\_technique}}{\emph{chromosome}, \emph{mutation\_parameters}}{}
Utilizando los datos de la parte superior, se desarrolla la función
conocida como Float Point Mutation \textbf{(ó Mutación de Punto Flotante)}.

\end{fulllineitems}



\subsection{SharingFunction (módulo)}
\label{Model/SharingFunction/SharingFunction:sharingfunction-modulo}\label{Model/SharingFunction/SharingFunction::doc}
\begin{DUlineblock}{0em}
\item[] En esta sección se almacenan las técnicas relativas al Sharing Function
\textbf{(ó Función de Compartición)}.
\item[] El objetivo de estas técnicas se delega a un rol secundario pero aún así
muy importante y consiste en realizar un filtrado más minucioso de los mejores Individuos
y así tomar a los candidatos elegidos para dejar descendencia.
\end{DUlineblock}

\begin{DUlineblock}{0em}
\item[] La operación es útil en casos en el que la calidad de los Individuos es muy similar y entonces
se desea seleccionar a los que son superiores, sin embargo, es menester mencionar que, en exceso, dicha selección
parsimoniosa puede dar lugar a un efecto negativo del Selective Pressure \textbf{(ó Presión Selectiva, véase Model/MOEA)}.
\item[] Esto provoca que, lejos de dar una Población de elementos óptimos, los Indviduos se queden estancados
puesto que al tener todos cargas genéticas muy similares, existe una pobre exploración genética en
sus cromosomas y entonces no se llegará a una optimización de funciones objetivo adecuada.
\end{DUlineblock}

\begin{DUlineblock}{0em}
\item[] Es por ello que no todos los MOEAS \textbf{(véase Model/MOEA)} lo utilizan, sin embargo se decidió implementar
esta sección ya que extrapolando las circunstancias, en cualquier momento se puede hacer uso de técnicas
de esta índole.
\end{DUlineblock}

\begin{DUlineblock}{0em}
\item[] Haciendo énfasis en la parte matemática, el Sharing Function funciona así:
\end{DUlineblock}

\begin{DUlineblock}{0em}
\item[] Cada Individual \textbf{(ó Individuo)} tendrá asociado un Shared Fitness \textbf{(ó Fitness Compartido)} que fungirá como el
Fitness original asignado a cada Individo y el cual será obtenido de la siguiente manera:
\end{DUlineblock}

\begin{center}\(SharedFitness(Individuo) = \frac{Fitness(Individuo)}{NicheCount(Individuo)}\)
\end{center}
\begin{DUlineblock}{0em}
\item[] Para fines de implementación el Shared Fitness será colocado en la misma variable utilizada para almacenar el Fitness
original, esto por cada Individuo.
\item[] El Niche Count es un valor que indica qué tan cercano en calidad se encuentra un Individuo con respecto de los demás.
La forma de calcularlo es la siguiente:
\end{DUlineblock}

\begin{center}\(NicheCount(Individuo) = \sum_{j=1}^{tama\tilde{n}o\_poblaci\acute{o}n}SF(D(Individuo,Individuo_j))\)
\end{center}
\begin{DUlineblock}{0em}
\item[] Donde \(D(Individuo_i,Individuo_j)\) es la distancia que existe entre el Individuo i y el Individuo j;
mientras que el SF es el Sharing Function.
\item[] Entonces el SF se define como:
\end{DUlineblock}

\begin{center}\(SF(D(Individuo_i,Individuo_j)) = \left\{ \begin{array}{lcc}
              1 - (\frac{D(Individuo_i,Individuo_j)}{\sigma_{share}})^{\alpha},\ \ si\ \ D < \sigma_{share}. \\
              \\ 0,\ \ en\ cualquier\ otro\ caso. \\
         \end{array}
\right.\)
\end{center}
\begin{DUlineblock}{0em}
\item[] Donde \(\alpha\) es una variable que casi siempre se asigna a 1 \textbf{(aunque en este proyecto se le
da la libertad al usuario de seleccionar valores distintos)} y \(\sigma_{share}\) marca el límite en el
cual dos Individuos se consideran cercanos en calidad, es decir, viven en el mismo Niche.
\item[] 
\item[] Llegados a este punto, si bien la parte que se utilizará finalmente es el Shared Fitness,
sólo las técnicas concernientes a \(D(Individuo_i,Individuo_j)\)
serán las que se implementen en esta sección, pues lo demás siempre se mantendrá estático.
\item[] 
\item[] Siendo más específicos con base en lo anterior, existen dos tipos de funciones de Distancia:
\end{DUlineblock}
\begin{itemize}
\item {} 
De Similaridad Genotípica \textbf{(ó Genotypic Similarity)}.

\item {} 
De Similaridad Fenotípica \textbf{(ó Phenotypic Similarity)}.

\end{itemize}

\begin{DUlineblock}{0em}
\item[] La primera indica en pocas palabras que la comparación se hará usando únicamente características relacionadas
con el cromosoma, mientras que la segunda implicará la comparación de características externas como las funciones objetivo
evaluadas con las variables de decisión de cada Individuo ó las variables de decisión por sí solas.
\item[] 
\item[] Eventualmente se desea que el usuario implemente sus propias funciones, por ello es que, además de añadir
el método en el listado localizado en \textbf{Controller/XML/Features.xml}, deberá implementar las siguiente funciones:
\end{DUlineblock}


\begin{fulllineitems}
\pysigline{\sphinxbfcode{calculate\_sigma\_share(population,sharing\_function\_parameters):}}~
\begin{DUlineblock}{0em}
\item[] Realiza el cálculo del factor \(\sigma_{share}\) sobre el cual se hará el cuestionamiento
de Individuos cercanos en calidad.
\item[] Es importante mencionar que la función debe regresar un escalar que representa el límite
máximo para el cual dos Individuos se consideran en el mismo Niche.
\end{DUlineblock}
\begin{quote}\begin{description}
\item[{Parameters}] \leavevmode\begin{itemize}
\item {} 
\textbf{\texttt{population}} (\emph{\texttt{Instance}}) -- La Población sobre la cual se hará el cálculo correspondiente.

\item {} 
\textbf{\texttt{sharing\_function\_parameters}} (\emph{\texttt{Dictionary}}) -- Un diccionario que puede contener opciones adicionales para
el cálculo de la distancia entre Individuos.

\end{itemize}

\item[{Returns}] \leavevmode
Un valor escalar que representa el límite de cercanía para cualesquiera dos Individuos
de una Población.

\item[{Return type}] \leavevmode
Float

\end{description}\end{quote}

\end{fulllineitems}



\begin{fulllineitems}
\pysigline{\sphinxbfcode{calculate\_distance(individual\_i,individual\_j,sharing\_function\_parameters):}}~
\begin{DUlineblock}{0em}
\item[] Calcula la distancia de calidad que existe entre dos Individuos cualesquiera.
\item[] Dada la simpleza del método, se puede usar independientemente de las categorías
antes especificadas.
\item[] Es importante resaltar que la función debe regresar un escalar que aluda a la distancia
entre los Individuos.
\end{DUlineblock}
\begin{quote}\begin{description}
\item[{Parameters}] \leavevmode\begin{itemize}
\item {} 
\textbf{\texttt{individual\_i}} (\emph{\texttt{Instance}}) -- El Individuo para calcular distancia.

\item {} 
\textbf{\texttt{individual\_j}} (\emph{\texttt{Instance}}) -- El Individuo para calcular distancia.

\item {} 
\textbf{\texttt{sharing\_function\_parameters}} (\emph{\texttt{Dictionary}}) -- Un diccionario que puede contener opciones adicionales para
el cálculo de la distancia entre Individuos.

\end{itemize}

\item[{Returns}] \leavevmode
Un valor escalar que indica la distancia entre los Individuos.

\item[{Return type}] \leavevmode
Float

\end{description}\end{quote}

\end{fulllineitems}


A continuación se muestran las subcategorías correspondientes:


\subsubsection{GenotypicSimilarity (módulo)}
\label{Model/SharingFunction/GenotypicSimilarity/GenotypicSimilarity::doc}\label{Model/SharingFunction/GenotypicSimilarity/GenotypicSimilarity:genotypicsimilarity-modulo}
\begin{DUlineblock}{0em}
\item[] La similaridad Genotípica \textbf{(ó Genotypic Similarity)}, es una subcategoría
que calcula las distancias entre dos Individuos cualesquiera usando para ello
características Genotípicas de éstos, lo cual quiere decir que se emplearán rasgos
meramente internos endémicos de los Individuos.
\item[] 
\item[] Para fines del proyecto típicamente se utiliza el cromosoma y/o sus características asociadas,
no obstante siendo sensatos con el término, el cromosoma no es la única herramienta
que se puede usar sino cualquier rasgo interno.
\item[] 
\item[] Ahora se muestran los elementos implementados para esta subcategoría:
\end{DUlineblock}


\paragraph{HammingDistance (script)}
\label{Model/SharingFunction/GenotypicSimilarity/HammingDistance:hammingdistance-script}\label{Model/SharingFunction/GenotypicSimilarity/HammingDistance::doc}
\begin{DUlineblock}{0em}
\item[] La Distancia de Hamming \textbf{(ó Hamming Distance)} es una implementación
perteneciente a la subcategoría Genotypic Similarity \textbf{(ó Similaridad Genotípica)}.
\item[] Esta consiste en comparar los alelos entre los cromosomas de los Individuos y devolver un valor numérico
que indica en cuántos alelos los cromosomas de los Individuos resultaron tener valores diferentes.
\item[] Como consecuencia lógica, la magnitud de la Distancia de Hamming es inversamente proporcional a la calidad
de los Individuos.
\item[] 
\item[] Es ampliamente usada para la Representación Cromosómica \textbf{(véase Model/ChromosomalRepresentation)}
de tipo Binario \textbf{(ó Binary)}, aunque su uso no se limita sólo a esta codificación.
\item[] 
\item[] Con respecto del cálculo del \(\sigma_{share}\), éste se hace tomando en cuenta el número máximo
permitido de genes diferentes entre dos cromosomas cualesquiera.
\item[] Dicha cantidad es deducida solicitándole al usuario únicamente el porcentaje máximo permitido, con base
en éste se determina entonces el número en concreto.
\end{DUlineblock}
\phantomsection\label{Model/SharingFunction/GenotypicSimilarity/HammingDistance:module-Model.SharingFunction.GenotypicSimilarity.HammingDistance}\index{Model.SharingFunction.GenotypicSimilarity.HammingDistance (module)}\index{calculate\_distance() (in module Model.SharingFunction.GenotypicSimilarity.HammingDistance)}

\begin{fulllineitems}
\phantomsection\label{Model/SharingFunction/GenotypicSimilarity/HammingDistance:Model.SharingFunction.GenotypicSimilarity.HammingDistance.calculate_distance}\pysiglinewithargsret{\sphinxbfcode{calculate\_distance}}{\emph{individual\_i}, \emph{individual\_j}, \emph{sharing\_function\_parameters}}{}
Con base en la información proporcionada anteriormente, se implementa
el cálculo de la distancia entre dos Individuos apoyándose de la técnica
conocida como Distancia de Hamming \textbf{(ó Hamming Distance)}.

\end{fulllineitems}

\index{calculate\_sigma\_share() (in module Model.SharingFunction.GenotypicSimilarity.HammingDistance)}

\begin{fulllineitems}
\phantomsection\label{Model/SharingFunction/GenotypicSimilarity/HammingDistance:Model.SharingFunction.GenotypicSimilarity.HammingDistance.calculate_sigma_share}\pysiglinewithargsret{\sphinxbfcode{calculate\_sigma\_share}}{\emph{population}, \emph{sharing\_function\_parameters}}{}
Basándose en las indicaciones mencionadas anteriormente, se
lleva a cabo la implementación de la obtención del valor Sigma Share.

\end{fulllineitems}



\subsubsection{PhenotypicSimilarity (módulo)}
\label{Model/SharingFunction/PhenotypicSimilarity/PhenotypicSimilarity::doc}\label{Model/SharingFunction/PhenotypicSimilarity/PhenotypicSimilarity:phenotypicsimilarity-modulo}
\begin{DUlineblock}{0em}
\item[] La Similaridad Fenotípica \textbf{(ó Phenotypic Similarity)} es una subcategoría
que calcula las distancias entre cualesquiera dos Individuos usando características
concernientes al Fenotipo, es decir, rasgos exteriores de los Individuos.
\item[] Para fines relativos al proyecto, dichos atributos tradicionalmente no son otra cosa que las
funciones objetivo evaluadas de cada Individuo, usando para ello las variables de decisión
que cada uno lleva consigo.
\item[] 
\item[] Aún considerando lo anterior, siendo más generales, cualquier característica externa que se relacione
con el Individuo puede ser utilizada.
\item[] 
\item[] El presente módulo consta de los siguientes scripts:
\end{DUlineblock}


\paragraph{EuclideanDistance (script)}
\label{Model/SharingFunction/PhenotypicSimilarity/EuclideanDistance:euclideandistance-script}\label{Model/SharingFunction/PhenotypicSimilarity/EuclideanDistance::doc}
\begin{DUlineblock}{0em}
\item[] La Distancia Euclidiana \textbf{(ó Euclidean Distance)} es una implementación
de cálculo de distancia entre dos Individuos que pertenece a la subcategoría
Phenotypic Similarity \textbf{(ó Similaridad Fenotípica)}.
\item[] Esta versión está dirigida para las Funciones Objetivo \textbf{(ó Objective Functions)}
que poseen cada uno de los Individuos \textbf{(ó Individuals)} de una Población \textbf{(ó Population)}.
\item[] 
\item[] Primero que nada para obtener el cálculo de \(\sigma_{share}\) la operación está regida por la siguiente
fórmula:
\end{DUlineblock}

\begin{center}\(\sigma_{share} = \frac{\sum_{j=1}^{n\acute{u}m\_funciones\_objetivo}(max(F_j) - min(F_j))}{tama\tilde{n}o\_poblaci\acute{o}n - 1}\)
\end{center}
\begin{DUlineblock}{0em}
\item[] Lo anterior significa que se van a obtener los valores máximo y mínimo de cada función objetivo,
se restan entre sí y al resultado anterior se le divide entre el tamaño de la población menos uno; esto por cada generación.
\item[] 
\item[] La forma de hacer el cálculo de la distancia es la siguiente:
\item[] Supongamos que tenemos los vectores \(U = (u_1,u_2,...,u_n)\) y \(V = (v_1,v_2,...,v_n)\). Entonces la Distancia Euclidiana se define como:
\end{DUlineblock}

\begin{center}\(d_E(U,V) = \sqrt{(v_1 - u_1)^2 + (v_2 - u_2)^2 + ... + (v_n - u_n)^2}\)
\end{center}
\begin{DUlineblock}{0em}
\item[] Para los fines que nos conciernen, los vectores \(U\ y\ V\) serán las evaluaciones en las funciones objetivos
de cada Individuo participante.
\item[] 
\item[] Finalmente es menester mencionar que, aunque tradicionalmente esta técnica se usa para Representaciones Cromosómicas
\textbf{(véase Model/ChromosomalRepresentation)} de tipo FloatPoint \textbf{(ó Punto Flotante)}, en sentido estricto no se encuentra
limitada sólo a este tipo de codificación.
\end{DUlineblock}
\phantomsection\label{Model/SharingFunction/PhenotypicSimilarity/EuclideanDistance:module-Model.SharingFunction.PhenotypicSimilarity.EuclideanDistance}\index{Model.SharingFunction.PhenotypicSimilarity.EuclideanDistance (module)}\index{calculate\_distance() (in module Model.SharingFunction.PhenotypicSimilarity.EuclideanDistance)}

\begin{fulllineitems}
\phantomsection\label{Model/SharingFunction/PhenotypicSimilarity/EuclideanDistance:Model.SharingFunction.PhenotypicSimilarity.EuclideanDistance.calculate_distance}\pysiglinewithargsret{\sphinxbfcode{calculate\_distance}}{\emph{individual\_i}, \emph{individual\_j}, \emph{sharing\_function\_parameters}}{}
Apoyándose de la técnica conocida como Distancia Euclidiana 
\textbf{(ó Euclidean Distance)} se implementa el cálculo de la distancia 
para dos Individuos cualesquiera.

\end{fulllineitems}

\index{calculate\_sigma\_share() (in module Model.SharingFunction.PhenotypicSimilarity.EuclideanDistance)}

\begin{fulllineitems}
\phantomsection\label{Model/SharingFunction/PhenotypicSimilarity/EuclideanDistance:Model.SharingFunction.PhenotypicSimilarity.EuclideanDistance.calculate_sigma_share}\pysiglinewithargsret{\sphinxbfcode{calculate\_sigma\_share}}{\emph{population}, \emph{sharing\_function\_parameters}}{}
Tomando como referencia la información antes mencionada para el cálculo del 
Sigma Share, se realiza la implementación correspondiente.

\end{fulllineitems}



\subsection{MOEA (módulo)}
\label{Model/MOEA/MOEA::doc}\label{Model/MOEA/MOEA:moea-modulo}
\begin{DUlineblock}{0em}
\item[] En esta parte se encuentran desarrolladas todas las técnicas
concernientes al uso de M.O.E.A.'s \textbf{(Multi-Objective Evolutionary Algorithms
ó Algoritmos Evolutivos Multiobjetivo)}.
\item[] 
\item[] Un M.O.E.A. es la convergencia y culminación de todas las técnicas que se
han implementado en la sección Model \textbf{(ó Modelo)} con la finalidad de
ofrecer una solución óptima ante un problema multiobjetivo mediante el
uso de Algoritmos Evolutivos.
\item[] 
\item[] Primero, solucionar un problema multiobjetivo aterrizado en un lenguaje matemático consiste en lo siguiente:
\item[] Tenemos un vector de funciones objetivo:
\end{DUlineblock}

\begin{center}\(F(\vec{x}) = [f_1(\vec{x}),f_2(\vec{x}),...,f_n(\vec{x})]^T;\ con\ n \geqslant 1.\)
\end{center}
\begin{DUlineblock}{0em}
\item[] Donde:
\end{DUlineblock}

\begin{center}\(\vec{x} = [x_1,x_2,...,x_k]^T;\ k \geqslant 1.\)
\end{center}
\begin{DUlineblock}{0em}
\item[] Representa al vector de variables de decisión que ``alimenta'' a cada función objetivo.
\item[] La meta es encontrar un vector especial de variables de decisión, llamémosle:
\end{DUlineblock}

\begin{center}\(\vec{x*} = [x_1*,x_2*,...,x_k*]^T;\ k \geqslant 1.\)
\end{center}
\begin{DUlineblock}{0em}
\item[] Tal que:
\end{DUlineblock}

\begin{center}\(f_i(\vec{x*}) \leqslant f_i(\vec{x});\ 1 \leqslant i \leqslant n;\ \forall f \in F\).
\end{center}
\begin{DUlineblock}{0em}
\item[] Dicho de otra forma, se debe encontrar el vector de variables de decisión que minimize todas y cada una de las
funciones objetivo en existencia.
\item[] Adicionalmente, todo vector de variables de decisión debe estar sujeto a las restricciones:
\end{DUlineblock}

\begin{center}\(h_i(\vec{x}) = 0;\ 1 \leqslant i \leqslant p\ \ (restricciones\ de\ igualdad).\)
\end{center}
\begin{center}\(g_i(\vec{x}) \leqslant 0;\ 1 \leqslant i \leqslant m\ \ (restricciones\ de\ desigualdad).\)
\end{center}
\begin{DUlineblock}{0em}
\item[] Las cuales para fines de este proyecto son aquéllas a las que se encuentran afianzadas
las variables de decisión \textbf{(véase View/Main/DecisionVariable/VariableFrame.py)}
\item[] 
\item[] Una definición adicional que sin lugar a dudas se verá utilizada es la de \emph{dominancia} entre vectores de variables de decisión,
para ello tomemos dos vectores \(U = (u_1,u_2,...,u_k)\) y \(V = (v_1,v_2,...,v_k)\), se dice
que \textbf{U domina a V ó V es dominada por U} si:
\end{DUlineblock}

\begin{center}\(\forall i \in \{1,...,k\}\ \ u_i \leqslant v_i \land \exists i \in \{1,...,k\}; \ \ u_i < v_i\).
\end{center}
\begin{DUlineblock}{0em}
\item[] Lo anterior significa que \(U\) debe ser mejor que \(V\) en cada uno de sus componentes para garantizar la dominancia.
\item[] La simbología que se suele usar para identificar este hecho es \(u \succ v\).
\item[] 
\item[] Algo importante a mencionar es que en las definiciones se trata únicamente la minimización
de funciones objetivo porque, en caso de querer la maximización, simplemente se realiza la
sustitución:
\end{DUlineblock}

\begin{center}\(f'_i(\vec{x}) = -f_i(\vec{x});\ 1 \leqslant i \leqslant n,\ para\ alguna\ f \in F.\)
\end{center}
\begin{DUlineblock}{0em}
\item[] Es decir, minimizando la función negativa se obtiene el máximo. El proyecto ya contempla este tipo
de casos \textbf{(véase View/Main/ObjectiveFunction/FunctionFrame)}.
\item[] Como dato adicional, es menester añadir que, en un escenario típico muchas de las funciones
objetivo entrarán en conflicto, esto quiere decir que en algunas se buscará el mínimo mientras
que en otras, el máximo.
\item[] 
\item[] Con base en lo anterior, el funcionamiento de un M.O.E.A. \textbf{(resolver un problema de optimización
multiobjetivo usando algoritmos genéticos)} generalmente se lleva a cabo de la siguiente manera:
\item[] 
\item[] 1.- Usando una Representación Cromosómica \textbf{(véase Model/ChromosomalRepresentation)}, crear la Población Padre y evaluar cada uno de los Individuos respecto a las funciones objetivo.
\end{DUlineblock}

\begin{DUlineblock}{0em}
\item[] 2.- Asignar un Ranking a los Individuos de la Población Padre \textbf{(véase Model/Community/Community.py)}.
\end{DUlineblock}

\begin{DUlineblock}{0em}
\item[] 3.- Con base en el Ranking, asignar el Fitness a cada uno de los Individuos \textbf{(véase Model/Fitness)}.
\end{DUlineblock}

\begin{DUlineblock}{0em}
\item[] 4.- Tomando en cuenta el Fitness, aplicar las operaciones de Selección, Cruza y Mutación con la finalidad de crear una Población Hija \textbf{(véase Model/GeneticOperator)}. Todos los métodos empleados en este punto deben funcionar acorde a la Representación Cromosómica del punto 1.
\end{DUlineblock}

\begin{DUlineblock}{0em}
\item[] 5- \textbf{(Opcional)} Utilizar el Fitness Compartido para aplicar una elección más minuciosa de los mejores Individuos en la Población Hija \textbf{(véase Model/SharingFunction)}.
\end{DUlineblock}

\begin{DUlineblock}{0em}
\item[] 6.- Designar a la población Hija como la nueva población Padre.
\end{DUlineblock}

\begin{DUlineblock}{0em}
\item[] 7.- Repetir los pasos 2 a 6 hasta haber alcanzado un número límite de generaciones \textbf{(iteraciones)}.
\item[] 
\item[] A grandes rasgos la diferencia entre un M.O.E.A. y otro es la Presión Selectiva
\textbf{(ó Selective Pressure)} que se aplica durante el procedimiento, para fines de este proyecto
se trata de la tolerancia para seleccionar a los Individuos de calidad media o baja frente a los
mejores. Una baja Presión Selectiva permite elegir Individuos no tan aptos; el caso es análogo para
una alta Presión Selectiva.
\item[] Es por eso que se han tomado los M.O.E.A.'s más representativos, pues se desea ilustrar la
consistencia y eficacia de dichos métodos en general a través de variadas circunstancias.
\item[] 
\item[] Tomando en cuenta lo anterior, la finalidad es que el usuario desarrolle
sus propios M.O.E.A.'s, por ello es que, además de  además de añadir
el método en el listado localizado en \textbf{Controller/XML/Features.xml}, deberá implementar la siguiente función:
\end{DUlineblock}


\begin{fulllineitems}
\pysigline{\sphinxbfcode{execute\_moea(execution\_task\_count,generations\_queue,generations,population\_size,vector\_functions,vector\_variables,available\_expressions,number\_of\_decimals,}}\pysigline{\sphinxbfcode{community\_instance,algorithm\_parameters,representation\_instance,representation\_parameters,fitness\_instance,fitness\_parameters,}}\pysigline{\sphinxbfcode{sharing\_function\_instance,sharing\_function\_parameters,selection\_instance,selection\_parameters,crossover\_instance,crossover\_parameters,}}\pysigline{\sphinxbfcode{mutation\_instance,mutation\_parameters):}}~
\begin{DUlineblock}{0em}
\item[] Devuelve la solución óptima para un conjunto de funciones objetivo \textbf{vector\_functions} ligadas
a un conjunto de restricciones \textbf{vector\_variables} tomando como fundamento el uso de algoritmos genéticos.
\item[] El método se apoya de las características subyacentes; en lo concerniente a la devolución de resultados
se recomienda ver el método \textbf{get\_results} localizado en \textbf{Model/Community/Community.py}.
\end{DUlineblock}
\begin{quote}\begin{description}
\item[{Parameters}] \leavevmode\begin{itemize}
\item {} 
\textbf{\texttt{execute\_task\_count}} (\emph{\texttt{Integer}}) -- El identificador que se utiliza para orquestar el orden en que el método será ejecutado
con respecto de los demás \textbf{(véase View/Additional/ResultsGrapher/ResultsGrapherTopLevel.py)}.

\item {} 
\textbf{\texttt{generations\_queue}} (\emph{\texttt{Instance}}) -- Una estructura auxiliar \textbf{(Queue o Cola)} que es necesaria para indicar a la interfaz gráfica el progreso del método
\textbf{(véase Controller/Controller.py, View/MainWindow.py, View/Additional/ResultsGrapher/ResultsGrapherTopLevel.py)} .

\item {} 
\textbf{\texttt{generations}} (\emph{\texttt{Integer}}) -- El número de generaciones \textbf{(iteraciones)} que se emplearán para la ejecución del método.

\item {} 
\textbf{\texttt{population\_size}} (\emph{\texttt{Integer}}) -- El tamaño de la Población \textbf{(número de Individuos)}.

\item {} 
\textbf{\texttt{vector\_functions}} (\emph{\texttt{List}}) -- El vector con las funciones objetivo insertadas por el usuario.

\item {} 
\textbf{\texttt{vector\_variables}} (\emph{\texttt{List}}) -- El vector con las variables de decisión ingresadas por el usuario.

\item {} 
\textbf{\texttt{available\_expressions}} (\emph{\texttt{Dictionary}}) -- Un diccionario con expresiones creadas para que la evaluación de funciones objetivo sea mucho más sencilla
\textbf{(véase Controller/Verifier.py, Controller/XML/PythonExpressions.xml, View/Additional/MenuInternalOption/InternalOptionTab/PythonExpressionFrame.py)}.

\item {} 
\textbf{\texttt{number\_of\_decimals}} (\emph{\texttt{Integer}}) -- La precisión decimal \textbf{(número de decimales)} que tendrán las soluciones inherentes a los Individuos.

\item {} 
\textbf{\texttt{community\_instance}} (\emph{\texttt{Instance}}) -- Una instancia de la clase Community
\textbf{(véase Controller/Verifier.py, Model/Community/Community.py)}.

\item {} 
\textbf{\texttt{algorithm\_parameters}} (\emph{\texttt{Instance}}) -- Un diccionario para añadir opciones adicionales para los M.O.E.A.'s.

\item {} 
\textbf{\texttt{representation\_instance}} (\emph{\texttt{Instance}}) -- Una instancia de la técnica de Representación Cromosómica \textbf{(ó Chromosomal Representation)} usada
por el usuario \textbf{(véase Controller/Verifier.py, Model/ChromosomalRepresentation)}.

\item {} 
\textbf{\texttt{representation\_parameters}} (\emph{\texttt{Dictionary}}) -- Un diccionario con opciones adicionales a la técnica de Representación Cromosómica usada.

\item {} 
\textbf{\texttt{fitness\_instance}} (\emph{\texttt{Instance}}) -- Una instancia de la técnica de Fitness seleccionada por el usuario
\textbf{(véase Controller/Verifier.py, Model/Fitness)}.

\item {} 
\textbf{\texttt{fitness\_parameters}} (\emph{\texttt{Dictionary}}) -- Un diccionario con parámetros adicionales para la técnica de Fitness utilizada.

\item {} 
\textbf{\texttt{sharing\_function\_instance}} (\emph{\texttt{Instance}}) -- Una instancia de la técnica de Sharing Function \textbf{(ó Función de Compartición)} usada
por el usuario \textbf{(véase Controller/Verifier.py, Model/SharingFunction)}.

\item {} 
\textbf{\texttt{sharing\_function\_parameters}} (\emph{\texttt{Dictionary}}) -- Un diccionario con opciones adicionales para la técnica de Sharing Function seleccionada.

\item {} 
\textbf{\texttt{selection\_instance}} (\emph{\texttt{Instance}}) -- Una instancia de la técnica de Selection \textbf{(ó Selección)} seleccionada por el usuario
\textbf{(véase Controller/Verifier.py, Model/Operator/Selection)}.

\item {} 
\textbf{\texttt{selection\_parameters}} (\emph{\texttt{Dictionary}}) -- Un diccionario con opciones adicionales para la técnica de Selection empleada.

\item {} 
\textbf{\texttt{crossover\_instance}} (\emph{\texttt{Instance}}) -- Una instancia de la técnica de Crossover \textbf{(ó Cruza)} seleccionada por el usuario
\textbf{(véase Controller/Verifier.py, Model/Operator/Crossover)}.

\item {} 
\textbf{\texttt{crossover\_parameters}} (\emph{\texttt{Dictionary}}) -- Un diccionario con parámetros adicionales para la técnica de Cruza solicitada.

\item {} 
\textbf{\texttt{mutation\_instance}} (\emph{\texttt{Instance}}) -- Una instancia de la técnica de Mutation \textbf{(ó Mutación)} empleada por el usuario
\textbf{(véase Controller/Verifier.py, Model/Operator/Mutation)}.

\item {} 
\textbf{\texttt{mutation\_parameters}} -- Un diccionario con parámetros adicionales para la técnica de Mutación usada.

\end{itemize}

\item[{Returns}] \leavevmode
El diccionario que resulta de aplicar el método \textbf{get\_results} que se encuentra en \textbf{Model/Community/Community.py}.

\item[{Return type}] \leavevmode
Dictionary

\end{description}\end{quote}

\end{fulllineitems}


A continuación se muestra la lista de los M.O.E.A.'s implementados:


\subsubsection{VEGA (script)}
\label{Model/MOEA/VEGA:vega-script}\label{Model/MOEA/VEGA::doc}
\begin{DUlineblock}{0em}
\item[] Se implementa la técnica M.O.E.A conocida como V.E.G.A. \textbf{(Vector Evaluated Genetic
Algorithm ó Algoritmo Genético de Vectores Evaluados)}.
\item[] La forma de proceder del algoritmo es la siguiente:
\item[] 
\item[] 1.- Se crea la Población Padre (de tamaño \(n\)).
\end{DUlineblock}

\begin{DUlineblock}{0em}
\item[] 2.- Tomando en cuenta las \(k\) funciones objetivo y la Población Padre, se crean \(k\) subpoblaciones de tamaño \(n/k\) cada una, si este número llega a ser irracional se pueden hacer ajustes con respecto de la distribución de los Individuos.
\end{DUlineblock}

\begin{DUlineblock}{0em}
\item[] 3.- Por cada subpoblación, se aplica la técnica de Selección y obtienen los \(n/k\) Individuos, terminado esto se deben unificar todos los seleccionados de nuevo en una súper Población.
\end{DUlineblock}

\begin{DUlineblock}{0em}
\item[] 4.- Con la súper Población del paso 3, se crea a la población Hija, la cual pasará a convertirse en la la nueva Población Padre.
\end{DUlineblock}

\begin{DUlineblock}{0em}
\item[] 5.- Se repiten los pasos 2 a 4 hasta haber alcanzado el número de generaciones \textbf{(iteraciones)} límite.
\item[] 
\item[] Como se puede apreciar es una implementación muy sencilla de optimización multiobjetivo,
sin embargo el inconveniente que tiene es la fácil pérdida de material genético valioso.
\item[] Lo anterior significa que un Individuo que en una generación previa era el mejor para una
función objetivo \(i\) al momento de ser separado y seleccionado en una subpoblación \(j\)
(y por ende analizado bajo la función objetivo \(j\)) puede ser muy malo en calidad y por tanto no ser seleccionado;
perdiéndose la ganancia genética hasta el momento obtenida para la función objetivo \(i;\ i \neq j\).
\item[] 
\item[] Por ello es que se puede decir que V.E.G.A. genera soluciones promedio que destacan con una calidad media
para todas las funciones objetivo.
\item[] 
\item[] Finalmente hay que comentar que para este algoritmo no se requiere aplicar un Ranking específico, no obstante,
se ha decidido utilizar el de Fonseca \& Flemming \textbf{(véase Model/Community/Community.py)} pues es el más sencillo
de implementar.
\end{DUlineblock}
\phantomsection\label{Model/MOEA/VEGA:module-Model.MOEA.VEGA}\index{Model.MOEA.VEGA (module)}\index{create\_subpopulations() (in module Model.MOEA.VEGA)}

\begin{fulllineitems}
\phantomsection\label{Model/MOEA/VEGA:Model.MOEA.VEGA.create_subpopulations}\pysiglinewithargsret{\sphinxbfcode{create\_subpopulations}}{\emph{comunidad}, \emph{main\_population}}{}
Método que divide a la Población principal en subpoblaciones
de acuerdo al número de funciones objetivo.
\begin{quote}\begin{description}
\item[{Parameters}] \leavevmode\begin{itemize}
\item {} 
\textbf{\texttt{comunidad}} (\emph{\texttt{Instance}}) -- Una instancia de Community para poder crear
poblaciones..

\item {} 
\textbf{\texttt{main\_population}} (\emph{\texttt{Instance}}) -- La Población que será dividida.

\end{itemize}

\item[{Returns}] \leavevmode
Una lista con las subpoblaciones \textbf{(de tipo Population)}.

\item[{Return type}] \leavevmode
List

\end{description}\end{quote}

\end{fulllineitems}

\index{execute\_moea() (in module Model.MOEA.VEGA)}

\begin{fulllineitems}
\phantomsection\label{Model/MOEA/VEGA:Model.MOEA.VEGA.execute_moea}\pysiglinewithargsret{\sphinxbfcode{execute\_moea}}{\emph{execution\_task\_count}, \emph{generations\_queue}, \emph{generations}, \emph{population\_size}, \emph{vector\_functions}, \emph{vector\_variables}, \emph{available\_expressions}, \emph{number\_of\_decimals}, \emph{community\_instance}, \emph{algorithm\_parameters}, \emph{representation\_instance}, \emph{representation\_parameters}, \emph{fitness\_instance}, \emph{fitness\_parameters}, \emph{sharing\_function\_instance}, \emph{sharing\_function\_parameters}, \emph{selection\_instance}, \emph{selection\_parameters}, \emph{crossover\_instance}, \emph{crossover\_parameters}, \emph{mutation\_instance}, \emph{mutation\_parameters}}{}
De acuerdo a la información proporcionada con anterioridad, se 
implementa el método que representa a la técnica M.O.E.A. conocida 
como V.E.G.A. \textbf{(Vector Evaluated Genetic Algorithm ó Algoritmo 
Genético de Vectores Evaluados)}.

\end{fulllineitems}



\subsubsection{SPEAII (script)}
\label{Model/MOEA/SPEAII::doc}\label{Model/MOEA/SPEAII:speaii-script}
Se desarrolla la implementación de la técnica M.O.E.A. conocida como S.P.E.A. II
\textbf{(Strength Pareto Evolutionary Algorithm ó Algoritmo Evolutivo de Fuerza de Pareto)}.

El funcionamiento del algoritmo es el siguiente:

\begin{DUlineblock}{0em}
\item[] 1.- Se inicializa una población llamada \emph{P} y un conjunto inicialmente vacío llamado \emph{E} \textbf{(E albergará Individuos también)}; ambos son de tamaño n.
\end{DUlineblock}

\begin{DUlineblock}{0em}
\item[] 2.- Se asigna el Fitness a los Individuos de \emph{P} y \emph{E} \textbf{(para ello se evalúan las funciones objetivo de los Individuos de ambos conjuntos y se asigna el Ranking Zitzler \& Thiele)}.
\end{DUlineblock}

\begin{DUlineblock}{0em}
\item[] 3.- A continuación se funden \emph{P} y \emph{E} en una súper Población \textbf{(llamémosle S también señalado en el algoritmo como Mating Pool, de tamaño n)}.Para ello primero se añaden los Individuos \emph{NO DOMINADOS} de \emph{P} en \emph{S} y posteriormente los \emph{NO DOMINADOS} de \emph{E} en \emph{S}.
\item[] Aquí se distinguen dos casos:
\end{DUlineblock}
\begin{itemize}
\item {} 
Si llegasen a faltar Individuos se añaden al azar Individuos \emph{DOMINADOS} de \emph{P} en \emph{S} hasta completar la demanda.

\item {} 
Si después de la fusión el número de Individuos supera a n, entonces se hace un truncamiento en \emph{S} hasta ajustar su tamaño a n.

\end{itemize}

\begin{DUlineblock}{0em}
\item[] 4.- \emph{S} será la nueva \emph{E}, además se crea la población Hija de la recién creada \emph{E} \textbf{(E-Child)}.
\end{DUlineblock}

\begin{DUlineblock}{0em}
\item[] 5.- E-Child será la nueva P.
\end{DUlineblock}

\begin{DUlineblock}{0em}
\item[] 6.- Se repiten los pasos 2 a 5 hasta que se haya alcanzado el límite de generaciones \textbf{(iteraciones)}.
\end{DUlineblock}

\begin{DUlineblock}{0em}
\item[] Finalmente lo que se regresa es \emph{E}, ya que ahí es donde se han
almacenado los mejores Individuos de todas las generaciones.
\item[] 
\item[] La característica de este algoritmo es que tiene una Presión Selectiva alta ya que
se da prioridad a los Individuos no dominados \textbf{(de ahí el nombre de
Fuerza de Pareto ó los más fuertes con respecto al principio de Pareto)},
y el hecho de mezclar a \emph{E} y \emph{P} en una súper Población garantiza la conservación
de los mejores Individuos sin importar el transcurso de las generaciones
\textbf{(a eso se le conoce como Elitismo)}, pero también da una tolerancia, aunque mínima, a los
Individuos de menor calidad como en el punto 3.
\item[] Además al momento de actualizar \emph{S} a \emph{E} y E-Child a \emph{P} se tiene una especie de
seguro de vida, es decir, si en algún momento la población E-Child llegara a
tener una calidad baja se tiene el respaldo de \emph{E} para una generación posterior
para formar \emph{S}.
\item[] 
\item[] Se debe tener en cuenta que el algoritmo originalmente no contempla ni una súper
Población \emph{S} ni E-Child sino que en los pasos 3 y 4 se utiliza solamente \emph{E} para referirse tanto a E-child como a \emph{S},
sin embargo para no confundir al usuario en la funcionalidad del método se decidió colocar contenedores
extra para poder diferenciar más precisamente a los elementos involucrados.
\item[] 
\item[] Algo muy importante a mencionar es que en el paso 1 y al momento de crear la población E-Child
es necesario evaluar las funciones objetivo, asignar un Ranking y posteriormente un Fitness
para que se puedan aplicar los operadores geneticos \textbf{(véase Model/GeneticOperator)}, para este caso
el Ranking es estrictamente el de Zitzler \& Thiele; la descripción completa de éste se
encuentra en \textbf{Model/Community/Community.py}.
\end{DUlineblock}
\phantomsection\label{Model/MOEA/SPEAII:module-Model.MOEA.SPEAII}\index{Model.MOEA.SPEAII (module)}\index{execute\_moea() (in module Model.MOEA.SPEAII)}

\begin{fulllineitems}
\phantomsection\label{Model/MOEA/SPEAII:Model.MOEA.SPEAII.execute_moea}\pysiglinewithargsret{\sphinxbfcode{execute\_moea}}{\emph{execution\_task\_count}, \emph{generations\_queue}, \emph{generations}, \emph{population\_size}, \emph{vector\_functions}, \emph{vector\_variables}, \emph{available\_expressions}, \emph{number\_of\_decimals}, \emph{community\_instance}, \emph{algorithm\_parameters}, \emph{representation\_instance}, \emph{representation\_parameters}, \emph{fitness\_instance}, \emph{fitness\_parameters}, \emph{sharing\_function\_instance}, \emph{sharing\_function\_parameters}, \emph{selection\_instance}, \emph{selection\_parameters}, \emph{crossover\_instance}, \emph{crossover\_parameters}, \emph{mutation\_instance}, \emph{mutation\_parameters}}{}
Con base en la información señalada se lleva a cabo la implementación del
M.O.E.A. conocido como S.P.E.A. II \textbf{(Strength Pareto Evolutionary Algorithm ó
Algoritmo Evolutivo de Fuerza de Pareto)}.

\end{fulllineitems}



\subsubsection{MOGA (script)}
\label{Model/MOEA/MOGA::doc}\label{Model/MOEA/MOGA:moga-script}
\begin{DUlineblock}{0em}
\item[] Se desarrolla la técnica M.O.E.A. que lleva por nombre M.O.G.A.
\textbf{(Multi Objective Genetic Algorithm ó Algoritmo Genético Multi Objetivo)}.
\item[] Su funcionamiento es el siguiente:
\item[] 
\item[] 1.- Se crea la Población Padre, se evalúan las funciones objetivo de sus correspondientes Individuos.
\end{DUlineblock}

\begin{DUlineblock}{0em}
\item[] 2.- Se asigna a los Individuos un Ranking \textbf{(Fonseca \& Flemming)} y posteriormente se calcula el Niche Count de la Población Padre.
\end{DUlineblock}

\begin{DUlineblock}{0em}
\item[] 3.- Tomando en cuenta los valores del punto 2 se obtiene el Fitness para cada Individuo y posteriormente su Shared Fitness.
\end{DUlineblock}

\begin{DUlineblock}{0em}
\item[] 4.- Se aplica el operador de selección sobre la Población Padre para determinar los elegidos para dejar descendencia.
\end{DUlineblock}

\begin{DUlineblock}{0em}
\item[] 5.- Se crea la Población Hija, se evalúan las funciones objetivo de sus correspondientes Individuos.
\end{DUlineblock}

\begin{DUlineblock}{0em}
\item[] 6.- Se asigna a los Individuos un Ranking \textbf{(Fonseca \& Flemming)} y posteriormente se calcula el Niche Count de la Población Hija.
\end{DUlineblock}

\begin{DUlineblock}{0em}
\item[] 7.- Tomando en cuenta los valores del punto 6 se obtiene el Fitness para cada Individuo y posteriormente su Shared Fitness.
\end{DUlineblock}

\begin{DUlineblock}{0em}
\item[] 8.- La Población Hija pasará a ser la nueva Población Padre.
\end{DUlineblock}

\begin{DUlineblock}{0em}
\item[] 9.- Se repiten los pasos 4 a 8 hasta que se haya alcanzado el número límite de generaciones \textbf{(iteraciones)}.
\item[] 
\item[] Como se puede apreciar, la implementación de este algoritmo es muy sencilla, además
se rige casi en su totalidad por el Shared Fitness \textbf{(ó Fitness Compartido)}, por lo
que la Presión Selectiva \textbf{(ó Selective Pressure)} incluida dependerá en gran medida
de la función de Distancia que se utilice, así como de la magnitud indicada por
el usuario.
\item[] 
\item[] Finalmente es menester mencionar que para esta implementación el Ranking utilizado debe ser
estrictamente el de Fonseca \& Flemming \textbf{(véase Model/Community/Community.py)}.
\end{DUlineblock}
\phantomsection\label{Model/MOEA/MOGA:module-Model.MOEA.MOGA}\index{Model.MOEA.MOGA (module)}\index{execute\_moea() (in module Model.MOEA.MOGA)}

\begin{fulllineitems}
\phantomsection\label{Model/MOEA/MOGA:Model.MOEA.MOGA.execute_moea}\pysiglinewithargsret{\sphinxbfcode{execute\_moea}}{\emph{execution\_task\_count}, \emph{generations\_queue}, \emph{generations}, \emph{population\_size}, \emph{vector\_functions}, \emph{vector\_variables}, \emph{available\_expressions}, \emph{number\_of\_decimals}, \emph{community\_instance}, \emph{algorithm\_parameters}, \emph{representation\_instance}, \emph{representation\_parameters}, \emph{fitness\_instance}, \emph{fitness\_parameters}, \emph{sharing\_function\_instance}, \emph{sharing\_function\_parameters}, \emph{selection\_instance}, \emph{selection\_parameters}, \emph{crossover\_instance}, \emph{crossover\_parameters}, \emph{mutation\_instance}, \emph{mutation\_parameters}}{}
Tomando como referencia el pseudocódigo antes citado, se elabora la implementación
de M.O.G.A. \textbf{(Multi Objective Genetic Algorithm ó Algoritmo Genético Multi Objetivo)}.

\end{fulllineitems}



\subsubsection{NSGAII (script)}
\label{Model/MOEA/NSGAII:nsgaii-script}\label{Model/MOEA/NSGAII::doc}
En esta parte se lleva a cabo la implementación del M.O.E.A. denominado
N.S.G.A. II \textbf{(Non-dominated Sorting Genetic Algorithm ó Algoritmo Genético
de Ordenamiento No Dominado)}.

La forma de proceder del método es la siguiente:

1.- Se crea una Población Padre \textbf{(de tamaño n)}, a la cual se le evalúan las funciones objetivo de sus Individuos, se les asigna un Ranking \textbf{(Goldberg)}
y posteriormente se les otorga un Fitness.

2.- Con base en la Población Padre se aplica el operador de Selección para elegir a los Individuos que serán aptos para reproducirse.

3.- Usando a los elementos del punto 2, se crea una Población Hija \textbf{(de tamaño n)}.

4.- Se crea una súper Población \textbf{(llamémosle S, de tamaño 2n)} que albergará todos los Individuos tanto de la Población Padre como Hija; a \emph{S} se le evalúan las funciones objetivo de sus Individuos, se les asigna un Ranking \textbf{(Goldberg)} y posteriormente se les otorga un Fitness.

5.- La súper Población \emph{S} se divide en subcategorías de acuerdo a los niveles de dominancia que existan, es decir, existirá la categoría de dominancia 0, la cual almacena Individuos que tengan una dominancia de 0 Individuos \textbf{(ningún Individuo los domina)}, existirá la categoría de dominancia 1 con el significado análogo y así sucesivamente hasta haber cubierto todos los niveles de dominancia existentes.

6.- Se construye la nueva Población Padre, pare ello constará de los Individuos de \emph{S} donde la prioridad será el nivel de dominancia, es decir, primero se añaden los elementos del nivel 0,luego los del nivel 1 y así en lo sucesivo hasta haber adquirido n elementos.
Se debe aclarar que la adquisición de Individuos por nivel debe ser total, esto significa que no se pueden dejar Individuos sueltos para el mismo nivel de dominancia.

Supongamos que a un nivel k existen tantos Individuos que su presunta adquisición supera el tamaño n, en este caso se debe hacer lo siguiente:
\begin{quote}

6.1.- Se crea una Población provisional \textbf{(Prov)} con los Individuos del nivel k, se evalúan las funciones objetivo a cada uno de sus Individuos, se les asigna un Ranking \textbf{(Goldberg)} y posteriormente se les asigna el Fitness.

Con los valores anteriores se calcula el Niche Count \textbf{(véase Model/SharingFunction)} de los Individuos; una vez hecho ésto se seleccionan desde Prov los Individuos faltantes con los mayores Niche Count, esto hasta completar el tamaño n de la nueva Población Padre.
\end{quote}

7.- Al haber conformado la nueva Población Padre, se evalúan las funciones objetivo de sus Individuos, se les asigna el Ranking correspondiente \textbf{(Goldberg)} y se les atribuye su Fitness.

8.- Se repiten los pasos 2 a 7 hasta haber alcanzado el límite de generaciones \textbf{(iteraciones)}.

\begin{DUlineblock}{0em}
\item[] Como su nombre lo indica, la característica de este algoritmo es la clasificación
de los Individuos en niveles para su posterior selección.
\end{DUlineblock}

\begin{DUlineblock}{0em}
\item[] Esto al principio propicia una Presión Selectiva moderada por la ausencia de elementos
con dominancia baja que suele existir en las primeras generaciones, sin embargo en iteraciones
posteriores se agudiza la Presión Selectiva ya que eventualmente la mayoría de los Individuos
serán alojados en las primeras categorías de dominancia, cubriendo casi instantáneamente
la demanda de Individuos necesaria en el paso 6, por lo que las categorías posteriores serán
cada vez menos necesarias con el paso de los ciclos.
\end{DUlineblock}

\begin{DUlineblock}{0em}
\item[] Por otra parte la fusión de las Poblaciones en \emph{S} garantiza que siempre se conserven a
los mejores Individuos independientemente de la generación transcurrida, a eso se le llama Elitismo.
\item[] Por cierto que en el algoritmo original no existe un nombre oficial para \emph{S} sino más bien se señala como
una estructura genérica, sin embargo se le ha formalizado con un identificador para guiar apropiadamente al
usuario en el flujo del algoritmo.
\end{DUlineblock}

\begin{DUlineblock}{0em}
\item[] Para finalizar se señala que el uso del ranking de Goldberg \textbf{(véase Model/Community/Community.py)}
es indispensable.
\end{DUlineblock}
\phantomsection\label{Model/MOEA/NSGAII:module-Model.MOEA.NSGAII}\index{Model.MOEA.NSGAII (module)}\index{execute\_moea() (in module Model.MOEA.NSGAII)}

\begin{fulllineitems}
\phantomsection\label{Model/MOEA/NSGAII:Model.MOEA.NSGAII.execute_moea}\pysiglinewithargsret{\sphinxbfcode{execute\_moea}}{\emph{execution\_task\_count}, \emph{generations\_queue}, \emph{generations}, \emph{population\_size}, \emph{vector\_functions}, \emph{vector\_variables}, \emph{available\_expressions}, \emph{number\_of\_decimals}, \emph{community\_instance}, \emph{algorithm\_parameters}, \emph{representation\_instance}, \emph{representation\_parameters}, \emph{fitness\_instance}, \emph{fitness\_parameters}, \emph{sharing\_function\_instance}, \emph{sharing\_function\_parameters}, \emph{selection\_instance}, \emph{selection\_parameters}, \emph{crossover\_instance}, \emph{crossover\_parameters}, \emph{mutation\_instance}, \emph{mutation\_parameters}}{}
Con base en los datos recabados se desarrolla la técnica M.O.E.A.
que lleva por nombre N.S.G.A. II \textbf{(Non-dominated Sorting Genetic Algorithm ó
Algoritmo Genético de Ordenamiento No Dominado)}-

\end{fulllineitems}



\section{View (sección)}
\label{View/View:view-seccion}\label{View/View::doc}
La capa View \textbf{(ó Vista)} contiene todos los elementos que serán
alusivos a la interfaz gráfica. De acuerdo al modelo MVC \textbf{(Model-View-Controller)},
opera exclusivamente con la capa Controller \textbf{(ó Controlador)}.

A continuación se muestran los elementos que conforman esta sección.


\subsection{MainWindow (clase)}
\label{View/MainWindow::doc}\label{View/MainWindow:module-View.MainWindow}\label{View/MainWindow:mainwindow-clase}\index{View.MainWindow (module)}\index{MainWindow (class in View.MainWindow)}

\begin{fulllineitems}
\phantomsection\label{View/MainWindow:View.MainWindow.MainWindow}\pysigline{\sphinxstrong{class }\sphinxbfcode{MainWindow}}~
\begin{DUlineblock}{0em}
\item[] Mezcla todas las estructuras gráficas que forman parte de la sección View \textbf{(ó vista)}.
\item[] Se trata de una Ventana que contendrá todas las opciones necesarias para que
el usuario pueda ejecutar a voluntad M.O.E.A.'s \textbf{(Multi Objective Evolutionary Algorithm
ó Algoritmo Evolutivo Multi Objetivo)}
\item[] 
\item[] El flujo que se suele seguir es el siguiente:
\end{DUlineblock}
\begin{itemize}
\item {} 
El usuario ingresa las características que desea que contenga el M.O.E.A. que será ejecutado.

\item {} 
Posteriormente el Controller \textbf{(ó Controlador, véase Controller/Controller.py)} verifica la consistencia de los datos anteriores para que no haya conflicto en el lado del Model \textbf{(ó Modelo)}.

\item {} 
Si no existe problema alguno se prosigue con el proceso, en otro caso se arroja un mensaje de error.

\item {} 
Siguiendo con el flujo normal se ejecutará una instancia del M.O.E.A. solicitado en la capa de Model \textbf{(ó Modelo)}, la cual tendrá una ventana asociada en View \textbf{(ó Vista)} que indicará el progreso del primero.

\item {} 
Cuando una instancia termine de ejecutarse, la ventana del progreso desaparece y en su lugar se muestra otra conteniendo los resultados del M.O.E.A. \textbf{(véase View/Additional/ResultsGrapher/ResultsGrapherToplevel.py)}.

\end{itemize}

\begin{DUlineblock}{0em}
\item[] Es importante mencionar que esta clase y el proyecto en general están diseñados para que 
se puedan crear varias instancias simultáneamente, con ello se espera aprovechar al máximo los recursos
computacionales en los que el proyecto fuera a ejecutarse.
\end{DUlineblock}
\begin{quote}\begin{description}
\item[{Returns}] \leavevmode
Tkinter.Frame

\item[{Return type}] \leavevmode
Instance

\end{description}\end{quote}
\index{\_MainWindow\_\_change\_frame() (MainWindow method)}

\begin{fulllineitems}
\phantomsection\label{View/MainWindow:View.MainWindow.MainWindow._MainWindow__change_frame}\pysiglinewithargsret{\sphinxbfcode{\_MainWindow\_\_change\_frame}}{\emph{current\_frame\_name}}{}~
\begin{notice}{note}{Note:}
Este método es privado.
\end{notice}

Hace el cambio en la Ventana Principal ocultando un Frame y
mostrando otro.
\begin{quote}\begin{description}
\item[{Parameters}] \leavevmode
\textbf{\texttt{current\_frame\_name}} (\emph{\texttt{Tkinter.Frame}}) -- El Frame que se va a mostrar en 
la Ventana Principal.

\end{description}\end{quote}

\end{fulllineitems}

\index{\_MainWindow\_\_check\_queues() (MainWindow method)}

\begin{fulllineitems}
\phantomsection\label{View/MainWindow:View.MainWindow.MainWindow._MainWindow__check_queues}\pysiglinewithargsret{\sphinxbfcode{\_MainWindow\_\_check\_queues}}{}{}~
\begin{notice}{note}{Note:}
Este método es privado.
\end{notice}

Una vez iniciado un proceso que ejecuta un M.O.E.A., este método
revisa periódicamente las colas \textbf{(Queues)} sobre las cuales los procesos
escribirán todo tipo de información pertinente.

\end{fulllineitems}

\index{\_MainWindow\_\_get\_information() (MainWindow method)}

\begin{fulllineitems}
\phantomsection\label{View/MainWindow:View.MainWindow.MainWindow._MainWindow__get_information}\pysiglinewithargsret{\sphinxbfcode{\_MainWindow\_\_get\_information}}{}{}~
\begin{notice}{note}{Note:}
Este método es privado.
\end{notice}

Método que obtiene los datos ingresados por el usuario
de cada uno de los Frames asociados a esta Ventana Principal.
\begin{quote}\begin{description}
\item[{Returns}] \leavevmode
Un diccionario con toda las preferencias del usuario recolectadas
para cada uno de los Frames disponibles.

\item[{Return type}] \leavevmode
Dictionary

\end{description}\end{quote}

\end{fulllineitems}

\index{\_MainWindow\_\_init\_procedure() (MainWindow method)}

\begin{fulllineitems}
\phantomsection\label{View/MainWindow:View.MainWindow.MainWindow._MainWindow__init_procedure}\pysiglinewithargsret{\sphinxbfcode{\_MainWindow\_\_init\_procedure}}{\emph{event}}{}~
\begin{notice}{note}{Note:}
Este método es privado.
\end{notice}

\begin{DUlineblock}{0em}
\item[] Inicia el procedimiento para ejecutar un M.O.E.A.
\item[] Los pasos que se realizan son:
\end{DUlineblock}
\begin{itemize}
\item {} 
Recolecta las preferencias ingresadas por el usuario en los Frames que conforman la Ventana Principal.

\item {} 
Se sanitizan dichos datos con ayuda del Controller.

\item {} 
En caso de no haber problemas con la sanitización, se ejecuta el proceso alojándolo en un hilo \textbf{(Thread)} para que permita seguir teniendo acceso a la Ventana Principal; por el contrario si hubo alguna falla regresa un mensaje de error.

\end{itemize}

\begin{DUlineblock}{0em}
\item[] Gracias a este método el proyecto entero tiene la característica de ser Multi-Hilo \textbf{(ó Multi-Threading)}, es decir,
se pueden ejecutar varios procedimientos de manera independiente.
\end{DUlineblock}
\begin{quote}\begin{description}
\item[{Parameters}] \leavevmode
\textbf{\texttt{event}} (\emph{\texttt{String}}) -- El evento del elemento gráfico que activa esta función.

\end{description}\end{quote}

\end{fulllineitems}

\index{\_MainWindow\_\_initialize\_frames() (MainWindow method)}

\begin{fulllineitems}
\phantomsection\label{View/MainWindow:View.MainWindow.MainWindow._MainWindow__initialize_frames}\pysiglinewithargsret{\sphinxbfcode{\_MainWindow\_\_initialize\_frames}}{}{}~
\begin{notice}{note}{Note:}
Este método es privado.
\end{notice}

Método que inicializa los Frames que se 
colocarán en la Ventana Principal como opciones.

\end{fulllineitems}

\index{\_MainWindow\_\_load\_images() (MainWindow method)}

\begin{fulllineitems}
\phantomsection\label{View/MainWindow:View.MainWindow.MainWindow._MainWindow__load_images}\pysiglinewithargsret{\sphinxbfcode{\_MainWindow\_\_load\_images}}{}{}~
\begin{notice}{note}{Note:}
Este método es privado.
\end{notice}

Carga las imágenes que se encuentran en el 
directorio View/Images para que puedan ser
usadas por los Frames. Es importante recalcar que el método
sólo carga imágenee .gif ya que son la extensión más estable
para que se muestren las imágenes en la interfaz gráfica.
\begin{quote}\begin{description}
\item[{Returns}] \leavevmode
Un diccionario con todas las imágenes cargadas.

\item[{Return type}] \leavevmode
Dictionary

\end{description}\end{quote}

\end{fulllineitems}

\index{\_MainWindow\_\_obtain\_results() (MainWindow method)}

\begin{fulllineitems}
\phantomsection\label{View/MainWindow:View.MainWindow.MainWindow._MainWindow__obtain_results}\pysiglinewithargsret{\sphinxbfcode{\_MainWindow\_\_obtain\_results}}{\emph{execution\_task\_count}, \emph{generations\_queue}, \emph{gathered\_information}, \emph{sanitized\_information}}{}~
\begin{notice}{note}{Note:}
Este método es privado.
\end{notice}

Ejecuta un M.O.E.A.. Esta es la función que se coloca en un hilo
para ser llevada a cabo de manera independiente con la finalidad 
de dejar libre la Ventana Principal y de manera secundaria ejecutar
varios procedimientos simultáneamente.
\begin{quote}\begin{description}
\item[{Parameters}] \leavevmode\begin{itemize}
\item {} 
\textbf{\texttt{execution\_task\_count}} (\emph{\texttt{Integer}}) -- El número de proceso actual.

\item {} 
\textbf{\texttt{generations\_queue}} (\emph{\texttt{Instance}}) -- Una referencia a una cola \textbf{(Queue)} donde los procesos escribirán su avance
en cuanto a las generaciones transcurridas.

\item {} 
\textbf{\texttt{gathered\_information}} (\emph{\texttt{Dictionary}}) -- La información que el usuario ingresó al momento de iniciar el proceso actual.

\item {} 
\textbf{\texttt{sanitized\_information}} (\emph{\texttt{Dictionary}}) -- La información anterior sanitizada.

\end{itemize}

\end{description}\end{quote}

\end{fulllineitems}

\index{\_MainWindow\_\_restore\_settings() (MainWindow method)}

\begin{fulllineitems}
\phantomsection\label{View/MainWindow:View.MainWindow.MainWindow._MainWindow__restore_settings}\pysiglinewithargsret{\sphinxbfcode{\_MainWindow\_\_restore\_settings}}{\emph{event}}{}~
\begin{notice}{note}{Note:}
Este método es privado.
\end{notice}

Limpia y deja por defecto los valores estándar del Frame
mostrado actualmente en la Ventana Principal.
El método no aplica para regresar a M.O.P.'s \textbf{(Multi Objective Problems)}
cargados anteriormente.
\begin{quote}\begin{description}
\item[{Parameters}] \leavevmode
\textbf{\texttt{event}} (\emph{\texttt{String}}) -- El evento del elemento gráfico que acciona esta función.

\end{description}\end{quote}

\end{fulllineitems}

\index{\_MainWindow\_\_update\_frame() (MainWindow method)}

\begin{fulllineitems}
\phantomsection\label{View/MainWindow:View.MainWindow.MainWindow._MainWindow__update_frame}\pysiglinewithargsret{\sphinxbfcode{\_MainWindow\_\_update\_frame}}{\emph{event}}{}~
\begin{notice}{note}{Note:}
Este método es privado.
\end{notice}

Muestra en la Ventana Principal el Frame actual.
\begin{quote}\begin{description}
\item[{Parameters}] \leavevmode
\textbf{\texttt{event}} (\emph{\texttt{String}}) -- El evento del elemento gráfico que 
activa la función.

\end{description}\end{quote}

\end{fulllineitems}

\index{load\_mop\_example() (MainWindow method)}

\begin{fulllineitems}
\phantomsection\label{View/MainWindow:View.MainWindow.MainWindow.load_mop_example}\pysiglinewithargsret{\sphinxbfcode{load\_mop\_example}}{\emph{elements}}{}
Carga el M.O.P \textbf{(Multi Objective Problem)} seleccionado
a los Frames correspondientes \textbf{(Objective Functions y Decision Variables)}.
\begin{quote}\begin{description}
\item[{Parameters}] \leavevmode
\textbf{\texttt{elements}} (\emph{\texttt{Array}}) -- Un arreglo que contiene dos elementos, el primero son las
funciones objetivo precargadas mientras que el segundo son las
variables de decisión también precargadas. Ambas provienen del menú
secundario \textbf{(véase View/Additional/MenuInternalOption/InternalOptionTab/MOPExampleFrame.py,
Controller/XML/MOPExamples.xml)}.

\end{description}\end{quote}

\end{fulllineitems}

\index{resource\_path() (MainWindow method)}

\begin{fulllineitems}
\phantomsection\label{View/MainWindow:View.MainWindow.MainWindow.resource_path}\pysiglinewithargsret{\sphinxbfcode{resource\_path}}{\emph{relative\_path}}{}
Esta función se utiliza para poder crear ejecutables
apropiadamente.
A grandes rasgos el ejecutable se empaqueta en un directorio
llamado \_MEIPASS, entonces aquí se implementa la búsqueda
de dicho archivo devolviendo un path \textbf{(ruta)}.
\begin{quote}\begin{description}
\item[{Returns}] \leavevmode
La ruta del directorio \_MEIPASS.

\item[{Return type}] \leavevmode
String

\end{description}\end{quote}

\end{fulllineitems}

\index{run() (MainWindow method)}

\begin{fulllineitems}
\phantomsection\label{View/MainWindow:View.MainWindow.MainWindow.run}\pysiglinewithargsret{\sphinxbfcode{run}}{}{}
Lanza la Ventana Principal.

\end{fulllineitems}


\end{fulllineitems}



\subsection{Main (módulo)}
\label{View/Main/Main:main-modulo}\label{View/Main/Main::doc}
Contiene todos los elementos gráficos para que el usuario
pueda configurar los atributos que intervienen en la ejecución de un M.O.E.A.

A continuación se colocan todos los elementos que constituen el módulo en cuestión:


\subsubsection{Home (módulo)}
\label{View/Main/Home/Home::doc}\label{View/Main/Home/Home:home-modulo}
Contiene toda la información posible para poder describir tanto los elementos
que conforman el programa como su correcto uso.

Los elementos que constituyen a este módulo son:


\paragraph{HomeFrame (clase)}
\label{View/Main/Home/HomeFrame:module-View.Main.Home.HomeFrame}\label{View/Main/Home/HomeFrame::doc}\label{View/Main/Home/HomeFrame:homeframe-clase}\index{View.Main.Home.HomeFrame (module)}\index{HomeFrame (class in View.Main.Home.HomeFrame)}

\begin{fulllineitems}
\phantomsection\label{View/Main/Home/HomeFrame:View.Main.Home.HomeFrame.HomeFrame}\pysiglinewithargsret{\sphinxstrong{class }\sphinxbfcode{HomeFrame}}{\emph{parent}, \emph{features}}{}
Bases: \sphinxcode{Tkinter.Frame}

\begin{DUlineblock}{0em}
\item[] Unifica dos elementos: Canvas e IntroductionFrame.
\item[] La razón de haber hecho esto es que, cuando se añaden demasiados elementos al
IntroductionFrame, se tiene que agregar una barra de desplazamiento para poder
acceder a los que se encuentran hasta abajo. Dentro del ambiente de Tkinter, el
elemento más sencillo para lograr esto es un Canvas, por ello se anida el
IntroductionFrame al Canvas.
\end{DUlineblock}
\begin{quote}\begin{description}
\item[{Parameters}] \leavevmode\begin{itemize}
\item {} 
\textbf{\texttt{parent}} (\emph{\texttt{Tkinter.Frame}}) -- Frame padre al que pertenece.

\item {} 
\textbf{\texttt{features}} (\emph{\texttt{Dictionary}}) -- Conjunto de técnicas con sus respectivos parámetros para que
se puedan cargar automáticamente en este Frame \textbf{(véase
Controller/XMLParser.py)}.

\end{itemize}

\item[{Returns}] \leavevmode
Tkinter.Frame

\item[{Return type}] \leavevmode
Instance

\end{description}\end{quote}
\index{\_HomeFrame\_\_update\_scrollbar() (HomeFrame method)}

\begin{fulllineitems}
\phantomsection\label{View/Main/Home/HomeFrame:View.Main.Home.HomeFrame.HomeFrame._HomeFrame__update_scrollbar}\pysiglinewithargsret{\sphinxbfcode{\_HomeFrame\_\_update\_scrollbar}}{\emph{event=None}}{}~
\begin{notice}{note}{Note:}
Este método es privado.
\end{notice}

Actualiza la barra de desplazamiento de acuerdo al número de elementos
existentes en el Frame, esto para poder hacer un recorrido apropiado de 
la barra.
\begin{quote}\begin{description}
\item[{Parameters}] \leavevmode
\textbf{\texttt{event}} (\emph{\texttt{String}}) -- Elemento que ejecutó esta función.

\end{description}\end{quote}

\end{fulllineitems}

\index{move\_to\_section() (HomeFrame method)}

\begin{fulllineitems}
\phantomsection\label{View/Main/Home/HomeFrame:View.Main.Home.HomeFrame.HomeFrame.move_to_section}\pysiglinewithargsret{\sphinxbfcode{move\_to\_section}}{\emph{y\_coordinate}}{}
Mueve la barra de desplazamiento (y por ende el contenido)
con base en la coordenada (en Y) que se le pase como parámetro.
\begin{quote}\begin{description}
\item[{Parameters}] \leavevmode
\textbf{\texttt{y\_coordinate}} -- Coordenada que se necesita para hace el
desplazamiento. Oscila entre 0 y 1.

\end{description}\end{quote}

\end{fulllineitems}

\index{restore\_settings() (HomeFrame method)}

\begin{fulllineitems}
\phantomsection\label{View/Main/Home/HomeFrame:View.Main.Home.HomeFrame.HomeFrame.restore_settings}\pysiglinewithargsret{\sphinxbfcode{restore\_settings}}{}{}
Restaura la configuración del Frame a la que tenía por
defecto.

\end{fulllineitems}


\end{fulllineitems}


La clase actual se apoya del elemento mostrado a continuación:


\subparagraph{IntroductionFrame (clase)}
\label{View/Main/Home/IntroductionFrame:module-View.Main.Home.IntroductionFrame}\label{View/Main/Home/IntroductionFrame::doc}\label{View/Main/Home/IntroductionFrame:introductionframe-clase}\index{View.Main.Home.IntroductionFrame (module)}\index{IntroductionFrame (class in View.Main.Home.IntroductionFrame)}

\begin{fulllineitems}
\phantomsection\label{View/Main/Home/IntroductionFrame:View.Main.Home.IntroductionFrame.IntroductionFrame}\pysiglinewithargsret{\sphinxstrong{class }\sphinxbfcode{IntroductionFrame}}{\emph{parent}, \emph{canvas\_function}, \emph{features}}{}
Bases: \sphinxcode{Tkinter.Frame}

\begin{DUlineblock}{0em}
\item[] Contiene información básica y concisa sobre el producto de software,
la cual es organizada y mostrada de acuerdo al número de secciones existentes
en éste.
\item[] De manera secundaria proporciona la infraestructura para poder darle al usuario
un desplazamiento más rápido entre dichas secciones.
\end{DUlineblock}
\begin{quote}\begin{description}
\item[{Parameters}] \leavevmode\begin{itemize}
\item {} 
\textbf{\texttt{parent}} (\emph{\texttt{Tkinter.Frame}}) -- Frame padre al que pertenece.

\item {} 
\textbf{\texttt{canvas\_function}} (\emph{\texttt{Instance}}) -- Una función alusiva al funcionamiento del Canvas.

\item {} 
\textbf{\texttt{features}} (\emph{\texttt{Dictionary}}) -- Conjunto de técnicas con sus respectivos parámetros para que
se puedan cargar automáticamente en este Frame.

\end{itemize}

\item[{Returns}] \leavevmode
Tkinter.Frame

\item[{Return type}] \leavevmode
Instance

\end{description}\end{quote}
\index{\_IntroductionFrame\_\_go\_to\_selected\_section() (IntroductionFrame method)}

\begin{fulllineitems}
\phantomsection\label{View/Main/Home/IntroductionFrame:View.Main.Home.IntroductionFrame.IntroductionFrame._IntroductionFrame__go_to_selected_section}\pysiglinewithargsret{\sphinxbfcode{\_IntroductionFrame\_\_go\_to\_selected\_section}}{\emph{event}}{}~
\begin{notice}{note}{Note:}
Este método es privado.
\end{notice}

Con base en la liga elegida por el usuario, realiza el
desplazamiento hacia la sección correspondiente.
\begin{quote}\begin{description}
\item[{Parameters}] \leavevmode
\textbf{\texttt{event}} (\emph{\texttt{String}}) -- Elemento que ejecutó esta función.

\end{description}\end{quote}

\end{fulllineitems}


\end{fulllineitems}



\subsubsection{DecisionVariable (módulo)}
\label{View/Main/DecisionVariable/DecisionVariable:decisionvariable-modulo}\label{View/Main/DecisionVariable/DecisionVariable::doc}
Proporciona los elementos gráficos para que el usuario pueda
insertar, modificar y eliminar variables de decisión con sus respectivos
rangos.

Los elementos que constituyen al módulo son:


\paragraph{DecisionVariableFrame (clase)}
\label{View/Main/DecisionVariable/DecisionVariableFrame:module-View.Main.DecisionVariable.DecisionVariableFrame}\label{View/Main/DecisionVariable/DecisionVariableFrame:decisionvariableframe-clase}\label{View/Main/DecisionVariable/DecisionVariableFrame::doc}\index{View.Main.DecisionVariable.DecisionVariableFrame (module)}\index{DecisionVariableFrame (class in View.Main.DecisionVariable.DecisionVariableFrame)}

\begin{fulllineitems}
\phantomsection\label{View/Main/DecisionVariable/DecisionVariableFrame:View.Main.DecisionVariable.DecisionVariableFrame.DecisionVariableFrame}\pysiglinewithargsret{\sphinxstrong{class }\sphinxbfcode{DecisionVariableFrame}}{\emph{parent}, \emph{features}}{}
Bases: \sphinxcode{Tkinter.Frame}

\begin{DUlineblock}{0em}
\item[] Realiza la fusión de Canvas y VariableFrame, debido a que, cuando se agregan 
numerosas variables al VariableFrame, se debe insertar una barra de desplazamiento
para poder acceder a aquéllos que se encuentren hasta abajo. Dentro del ambiente
de Tkinter, el elemento más sencillo para lograr este efecto es un Canvas, por ello 
se anida el VariableFrame al Canvas.
\end{DUlineblock}
\begin{quote}\begin{description}
\item[{Parameters}] \leavevmode\begin{itemize}
\item {} 
\textbf{\texttt{parent}} (\emph{\texttt{Tkinter.Frame}}) -- Frame padre al que pertenece.

\item {} 
\textbf{\texttt{features}} (\emph{\texttt{Dictionary}}) -- Conjunto de técnicas con sus respectivos parámetros para que
se puedan cargar automáticamente en este Frame \textbf{(véase
Controller/XMLParser.py)}.

\end{itemize}

\item[{Returns}] \leavevmode
Tkinter.Frame

\item[{Return type}] \leavevmode
Instance

\end{description}\end{quote}
\index{\_DecisionVariableFrame\_\_activate\_scroll() (DecisionVariableFrame method)}

\begin{fulllineitems}
\phantomsection\label{View/Main/DecisionVariable/DecisionVariableFrame:View.Main.DecisionVariable.DecisionVariableFrame.DecisionVariableFrame._DecisionVariableFrame__activate_scroll}\pysiglinewithargsret{\sphinxbfcode{\_DecisionVariableFrame\_\_activate\_scroll}}{\emph{event}}{}~
\begin{notice}{note}{Note:}
Este método es privado.
\end{notice}

Actualiza la barra de desplazamiento y con base en esta acción
la activa o desactiva.
\begin{quote}\begin{description}
\item[{Parameters}] \leavevmode
\textbf{\texttt{event}} (\emph{\texttt{String}}) -- Elemento que ejecutó esta función.

\end{description}\end{quote}

\end{fulllineitems}

\index{\_DecisionVariableFrame\_\_update\_scrollbar() (DecisionVariableFrame method)}

\begin{fulllineitems}
\phantomsection\label{View/Main/DecisionVariable/DecisionVariableFrame:View.Main.DecisionVariable.DecisionVariableFrame.DecisionVariableFrame._DecisionVariableFrame__update_scrollbar}\pysiglinewithargsret{\sphinxbfcode{\_DecisionVariableFrame\_\_update\_scrollbar}}{\emph{event=None}}{}~
\begin{notice}{note}{Note:}
Este método es privado.
\end{notice}

Actualiza la barra de desplazamiento de acuerdo al número de elementos
existentes en el Frame, esto para poder hacer un recorrido apropiado de 
la barra.
\begin{quote}\begin{description}
\item[{Parameters}] \leavevmode
\textbf{\texttt{event}} (\emph{\texttt{String}}) -- Elemento que ejecutó esta función.

\end{description}\end{quote}

\end{fulllineitems}

\index{get\_information() (DecisionVariableFrame method)}

\begin{fulllineitems}
\phantomsection\label{View/Main/DecisionVariable/DecisionVariableFrame:View.Main.DecisionVariable.DecisionVariableFrame.DecisionVariableFrame.get_information}\pysiglinewithargsret{\sphinxbfcode{get\_information}}{}{}
Regresa la información recabada en el Frame.
\begin{quote}\begin{description}
\item[{Returns}] \leavevmode
Un diccionario que contiene una lista con las variables ingresadas.

\item[{Return type}] \leavevmode
Dictionary

\end{description}\end{quote}

\end{fulllineitems}

\index{insert\_mop\_example() (DecisionVariableFrame method)}

\begin{fulllineitems}
\phantomsection\label{View/Main/DecisionVariable/DecisionVariableFrame:View.Main.DecisionVariable.DecisionVariableFrame.DecisionVariableFrame.insert_mop_example}\pysiglinewithargsret{\sphinxbfcode{insert\_mop\_example}}{\emph{variables}}{}~
\begin{DUlineblock}{0em}
\item[] Inserta un M.O.P \textbf{(Multi Objective Problem ó Problema Multi Objetivo)}.
\item[] En este caso significa que se insertarán las variables con 
sus respectivos rangos en el Frame para poder hacer pruebas rápidas en el programa, habiendo
antes limpiado por completo el contenido del Frame.
\item[] \textbf{(véase Controller/XML/MOPExample.xml)}
\item[] \textbf{(véase View/Additional/MenuInternalOption/InternalOptionFrame.py)}.
\end{DUlineblock}
\begin{quote}\begin{description}
\item[{Parameters}] \leavevmode
\textbf{\texttt{functions}} (\emph{\texttt{List}}) -- Lista de variables para ser insertadas en el Frame.

\end{description}\end{quote}

\end{fulllineitems}

\index{restore\_settings() (DecisionVariableFrame method)}

\begin{fulllineitems}
\phantomsection\label{View/Main/DecisionVariable/DecisionVariableFrame:View.Main.DecisionVariable.DecisionVariableFrame.DecisionVariableFrame.restore_settings}\pysiglinewithargsret{\sphinxbfcode{restore\_settings}}{}{}
Restaura el contenido del Frame, en este caso significa que se eliminará
todo lo que esté en éste y se dejará una casilla vacía libre.

\end{fulllineitems}


\end{fulllineitems}


La clase actual se basa en el siguiente elemento:


\subparagraph{VariableFrame (clase)}
\label{View/Main/DecisionVariable/VariableFrame:module-View.Main.DecisionVariable.VariableFrame}\label{View/Main/DecisionVariable/VariableFrame::doc}\label{View/Main/DecisionVariable/VariableFrame:variableframe-clase}\index{View.Main.DecisionVariable.VariableFrame (module)}\index{VariableFrame (class in View.Main.DecisionVariable.VariableFrame)}

\begin{fulllineitems}
\phantomsection\label{View/Main/DecisionVariable/VariableFrame:View.Main.DecisionVariable.VariableFrame.VariableFrame}\pysiglinewithargsret{\sphinxstrong{class }\sphinxbfcode{VariableFrame}}{\emph{parent}, \emph{features}}{}
Bases: \sphinxcode{Tkinter.Frame}

\begin{DUlineblock}{0em}
\item[] Proporciona bases gráficas para que el usuario pueda insertar
variables de decisión, así como información relativa a éstas.
\item[] En términos generales, el usuario insertará casillas para ingresar variables
de decisión, indicando también el valor mínimo y máximo que podrán tener.
\item[] Es importante comentar que todas las variables de decisión deben contener
rangos finitos, es decir, no se contemplan valores infinitos, aunque algunos
M.O.P.'s \textbf{(Multi Objective Problems ó Problemas Multi Objetivo)} manejan este tipo de rangos.         
\end{DUlineblock}
\begin{quote}\begin{description}
\item[{Parameters}] \leavevmode\begin{itemize}
\item {} 
\textbf{\texttt{parent}} (\emph{\texttt{Tkinter.Frame}}) -- Frame padre al que pertenece.

\item {} 
\textbf{\texttt{features}} (\emph{\texttt{Dictionary}}) -- Conjunto de técnicas con sus respectivos parámetros para que
se puedan cargar automáticamente en este Frame \textbf{(véase
Controller/XMLParser.py)}.

\end{itemize}

\item[{Returns}] \leavevmode
Tkinter.Frame

\item[{Return type}] \leavevmode
Instance

\end{description}\end{quote}
\index{\_VariableFrame\_\_add\_variable() (VariableFrame method)}

\begin{fulllineitems}
\phantomsection\label{View/Main/DecisionVariable/VariableFrame:View.Main.DecisionVariable.VariableFrame.VariableFrame._VariableFrame__add_variable}\pysiglinewithargsret{\sphinxbfcode{\_VariableFrame\_\_add\_variable}}{\emph{event}}{}~
\begin{notice}{note}{Note:}
Este método es privado.
\end{notice}

Agrega una casilla al Frame. Esta función se usa si 
fue ejecutada por un evento.
\begin{quote}\begin{description}
\item[{Parameters}] \leavevmode
\textbf{\texttt{event}} (\emph{\texttt{String}}) -- Identificador del elemento gráfico que activó la función.

\end{description}\end{quote}

\end{fulllineitems}

\index{\_VariableFrame\_\_delete\_single\_variable() (VariableFrame method)}

\begin{fulllineitems}
\phantomsection\label{View/Main/DecisionVariable/VariableFrame:View.Main.DecisionVariable.VariableFrame.VariableFrame._VariableFrame__delete_single_variable}\pysiglinewithargsret{\sphinxbfcode{\_VariableFrame\_\_delete\_single\_variable}}{\emph{event}}{}~
\begin{notice}{note}{Note:}
Este método es privado.
\end{notice}

Elimina una casilla y todos los elementos gráficos que la acompañan.
También elimina todo rastro que se encuentre en las estructuras lógicas.
\begin{quote}\begin{description}
\item[{Parameters}] \leavevmode
\textbf{\texttt{event}} (\emph{\texttt{String}}) -- Identificador del elemento gráfico que activó la función.

\end{description}\end{quote}

\end{fulllineitems}

\index{\_VariableFrame\_\_grid\_widgets() (VariableFrame method)}

\begin{fulllineitems}
\phantomsection\label{View/Main/DecisionVariable/VariableFrame:View.Main.DecisionVariable.VariableFrame.VariableFrame._VariableFrame__grid_widgets}\pysiglinewithargsret{\sphinxbfcode{\_VariableFrame\_\_grid\_widgets}}{}{}~
\begin{notice}{note}{Note:}
Este método es privado.
\end{notice}

Coloca elementos en el Frame.

\end{fulllineitems}

\index{get\_current\_elements() (VariableFrame method)}

\begin{fulllineitems}
\phantomsection\label{View/Main/DecisionVariable/VariableFrame:View.Main.DecisionVariable.VariableFrame.VariableFrame.get_current_elements}\pysiglinewithargsret{\sphinxbfcode{get\_current\_elements}}{}{}
Regresa el número actual de casillas en el Frame.
\begin{quote}\begin{description}
\item[{Returns}] \leavevmode
Cantidad de elementos en la estructura rows, donde se guardan las casillas (Entries).

\item[{Return type}] \leavevmode
Int

\end{description}\end{quote}

\end{fulllineitems}

\index{get\_information() (VariableFrame method)}

\begin{fulllineitems}
\phantomsection\label{View/Main/DecisionVariable/VariableFrame:View.Main.DecisionVariable.VariableFrame.VariableFrame.get_information}\pysiglinewithargsret{\sphinxbfcode{get\_information}}{}{}
Toma la información del Frame y regresa las variables con sus rangos que 
el usuario ingresó.
\begin{quote}\begin{description}
\item[{Returns}] \leavevmode
Un diccionario que contiene una lista con las variables (y rangos) escritas.

\item[{Return type}] \leavevmode
Dictionary

\end{description}\end{quote}

\end{fulllineitems}

\index{insert\_mop\_example() (VariableFrame method)}

\begin{fulllineitems}
\phantomsection\label{View/Main/DecisionVariable/VariableFrame:View.Main.DecisionVariable.VariableFrame.VariableFrame.insert_mop_example}\pysiglinewithargsret{\sphinxbfcode{insert\_mop\_example}}{\emph{variables}}{}~
\begin{DUlineblock}{0em}
\item[] Inserta un M.O.P (Multi Objective Problem) que no es más que un conjunto de 
variables con sus rangos para que se pueda hacer más rápidamente una prueba.
\item[] Previo a ésto se limpia el Frame para insertar únicamente el M.O.P.
\item[] \textbf{(véase Controller/XML/MOPExample.xml)}
\item[] \textbf{(véase View/Additional/MenuInternalOption/InternalOptionFrame.py)}.
\end{DUlineblock}
\begin{quote}\begin{description}
\item[{Parameters}] \leavevmode
\textbf{\texttt{functions}} (\emph{\texttt{List}}) -- Conjunto de variables para insertar en el Frame.

\end{description}\end{quote}

\end{fulllineitems}

\index{insert\_variable() (VariableFrame method)}

\begin{fulllineitems}
\phantomsection\label{View/Main/DecisionVariable/VariableFrame:View.Main.DecisionVariable.VariableFrame.VariableFrame.insert_variable}\pysiglinewithargsret{\sphinxbfcode{insert\_variable}}{\emph{variable=None}}{}~
\begin{DUlineblock}{0em}
\item[] Coloca en el Frame una colección de elementos:
\item[] {[}casilla para insertar variable ,casilla de rango minimo, casilla de rango máximo, botón para eliminar{]}
\item[] Si el parámetro function es \textbf{None}, se añade la casilla vacía, de lo contrario se 
agrega ésta con la variable y sus rangos.
\end{DUlineblock}
\begin{quote}\begin{description}
\item[{Parameters}] \leavevmode
\textbf{\texttt{function}} (\emph{\texttt{String}}) -- Una terna (nombre de la variable, rango máximo, rango mínimo)
para ser insertada en las casillas correspondientes.

\end{description}\end{quote}

\end{fulllineitems}

\index{restore\_settings() (VariableFrame method)}

\begin{fulllineitems}
\phantomsection\label{View/Main/DecisionVariable/VariableFrame:View.Main.DecisionVariable.VariableFrame.VariableFrame.restore_settings}\pysiglinewithargsret{\sphinxbfcode{restore\_settings}}{}{}
Restaura el contenido del Frame a sus valores por defecto.
Esto significa que borrará cualquier contenido que se encuentre en existencia y 
dejará una casilla vacía.

\end{fulllineitems}


\end{fulllineitems}



\subsubsection{ObjectiveFunction (módulo)}
\label{View/Main/ObjectiveFunction/ObjectiveFunction:objectivefunction-modulo}\label{View/Main/ObjectiveFunction/ObjectiveFunction::doc}
Proporciona los elementos gráficos para que el usuario
pueda insertar, modificar y eliminar funciones objetivo.

Sus elementos que lo conforman son:


\paragraph{ObjectiveFunctionFrame (clase)}
\label{View/Main/ObjectiveFunction/ObjectiveFunctionFrame:objectivefunctionframe-clase}\label{View/Main/ObjectiveFunction/ObjectiveFunctionFrame::doc}\label{View/Main/ObjectiveFunction/ObjectiveFunctionFrame:module-View.Main.ObjectiveFunction.ObjectiveFunctionFrame}\index{View.Main.ObjectiveFunction.ObjectiveFunctionFrame (module)}\index{ObjectiveFunctionFrame (class in View.Main.ObjectiveFunction.ObjectiveFunctionFrame)}

\begin{fulllineitems}
\phantomsection\label{View/Main/ObjectiveFunction/ObjectiveFunctionFrame:View.Main.ObjectiveFunction.ObjectiveFunctionFrame.ObjectiveFunctionFrame}\pysiglinewithargsret{\sphinxstrong{class }\sphinxbfcode{ObjectiveFunctionFrame}}{\emph{parent}, \emph{features}}{}
Bases: \sphinxcode{Tkinter.Frame}

\begin{DUlineblock}{0em}
\item[] Unifica dos elementos: Canvas y FunctionFrame.
\item[] La razón de haber hecho esto es que, cuando se agregan muchas funciones al
FunctionFrame, se tiene que agregar una barra de desplazamiento para poder
acceder a los que se encuentran hasta abajo. Dentro del ambiente de Tkinter, el
elemento más sencillo para lograr esto es un Canvas, por ello se anida el
FunctionFrame al Canvas.
\end{DUlineblock}
\begin{quote}\begin{description}
\item[{Parameters}] \leavevmode\begin{itemize}
\item {} 
\textbf{\texttt{parent}} (\emph{\texttt{Tkinter.Frame}}) -- Frame padre al que pertenece.

\item {} 
\textbf{\texttt{features}} (\emph{\texttt{Dictionary}}) -- Conjunto de técnicas con sus respectivos parámetros para que
se puedan cargar automáticamente en este Frame \textbf{(véase
Controller/XMLParser.py)}.

\end{itemize}

\item[{Returns}] \leavevmode
Tkinter.Frame

\item[{Return type}] \leavevmode
Instance

\end{description}\end{quote}
\index{\_ObjectiveFunctionFrame\_\_activate\_scroll() (ObjectiveFunctionFrame method)}

\begin{fulllineitems}
\phantomsection\label{View/Main/ObjectiveFunction/ObjectiveFunctionFrame:View.Main.ObjectiveFunction.ObjectiveFunctionFrame.ObjectiveFunctionFrame._ObjectiveFunctionFrame__activate_scroll}\pysiglinewithargsret{\sphinxbfcode{\_ObjectiveFunctionFrame\_\_activate\_scroll}}{\emph{event}}{}~
\begin{notice}{note}{Note:}
Este método es privado.
\end{notice}

Actualiza la barra de desplazamiento y con base en esta acción
la activa o desactiva.
\begin{quote}\begin{description}
\item[{Parameters}] \leavevmode
\textbf{\texttt{event}} (\emph{\texttt{String}}) -- Elemento que ejecutó esta función.

\end{description}\end{quote}

\end{fulllineitems}

\index{\_ObjectiveFunctionFrame\_\_update\_scrollbar() (ObjectiveFunctionFrame method)}

\begin{fulllineitems}
\phantomsection\label{View/Main/ObjectiveFunction/ObjectiveFunctionFrame:View.Main.ObjectiveFunction.ObjectiveFunctionFrame.ObjectiveFunctionFrame._ObjectiveFunctionFrame__update_scrollbar}\pysiglinewithargsret{\sphinxbfcode{\_ObjectiveFunctionFrame\_\_update\_scrollbar}}{\emph{event=None}}{}~
\begin{notice}{note}{Note:}
Este método es privado.
\end{notice}

Actualiza la barra de desplazamiento de acuerdo al número de elementos
existentes en el Frame, esto para poder hacer un recorrido apropiado de 
la barra.
\begin{quote}\begin{description}
\item[{Parameters}] \leavevmode
\textbf{\texttt{event}} (\emph{\texttt{String}}) -- Elemento que ejecutó esta función.

\end{description}\end{quote}

\end{fulllineitems}

\index{get\_information() (ObjectiveFunctionFrame method)}

\begin{fulllineitems}
\phantomsection\label{View/Main/ObjectiveFunction/ObjectiveFunctionFrame:View.Main.ObjectiveFunction.ObjectiveFunctionFrame.ObjectiveFunctionFrame.get_information}\pysiglinewithargsret{\sphinxbfcode{get\_information}}{}{}
Regresa la información recabada en el Frame.
\begin{quote}\begin{description}
\item[{Returns}] \leavevmode
Un diccionario que contiene una lista con las funciones escritas.

\item[{Return type}] \leavevmode
Dictionary

\end{description}\end{quote}

\end{fulllineitems}

\index{insert\_mop\_example() (ObjectiveFunctionFrame method)}

\begin{fulllineitems}
\phantomsection\label{View/Main/ObjectiveFunction/ObjectiveFunctionFrame:View.Main.ObjectiveFunction.ObjectiveFunctionFrame.ObjectiveFunctionFrame.insert_mop_example}\pysiglinewithargsret{\sphinxbfcode{insert\_mop\_example}}{\emph{functions}}{}~
\begin{DUlineblock}{0em}
\item[] Inserta un M.O.P (Multi Objective Problem).
\item[] En este caso significa que se insertarán funciones
para poder hacer pruebas rápidas en el programa.
\item[] \textbf{(véase Controller/XML/MOPExample.xml)}
\item[] \textbf{(véase View/Additional/MenuInternalOption/InternalOptionFrame.py)}.
\end{DUlineblock}
\begin{quote}\begin{description}
\item[{Parameters}] \leavevmode
\textbf{\texttt{functions}} (\emph{\texttt{List}}) -- Lista de funciones para ser insertadas en el Frame.

\end{description}\end{quote}

\end{fulllineitems}

\index{restore\_settings() (ObjectiveFunctionFrame method)}

\begin{fulllineitems}
\phantomsection\label{View/Main/ObjectiveFunction/ObjectiveFunctionFrame:View.Main.ObjectiveFunction.ObjectiveFunctionFrame.ObjectiveFunctionFrame.restore_settings}\pysiglinewithargsret{\sphinxbfcode{restore\_settings}}{}{}
Restaura el contenido del Frame, en este caso significa que se eliminará
todo lo que esté en éste y se dejará una casilla vacía libre.

\end{fulllineitems}


\end{fulllineitems}


La clase actual toma como fundamento lo siguiente:


\subparagraph{FunctionFrame (clase)}
\label{View/Main/ObjectiveFunction/FunctionFrame::doc}\label{View/Main/ObjectiveFunction/FunctionFrame:module-View.Main.ObjectiveFunction.FunctionFrame}\label{View/Main/ObjectiveFunction/FunctionFrame:functionframe-clase}\index{View.Main.ObjectiveFunction.FunctionFrame (module)}\index{FunctionFrame (class in View.Main.ObjectiveFunction.FunctionFrame)}

\begin{fulllineitems}
\phantomsection\label{View/Main/ObjectiveFunction/FunctionFrame:View.Main.ObjectiveFunction.FunctionFrame.FunctionFrame}\pysiglinewithargsret{\sphinxstrong{class }\sphinxbfcode{FunctionFrame}}{\emph{parent}, \emph{features}}{}
Bases: \sphinxcode{Tkinter.Frame}

\begin{DUlineblock}{0em}
\item[] Esta clase proporciona una base gráfica para que el usuario pueda
agregar tantas functiones objetivo como desee.
\item[] A grandes rasgos el usuario podrá agregar casillas donde se colocarán las funciones
objetivo, esto utilizando un botón. De igual manera, las casillas pueden ser eliminadas
usando un ícono que estará cerca de cada una de éstas.
\item[] Importante es mencionar que las funciones deben estar escritas en sintaxis de Python.
\end{DUlineblock}
\begin{quote}\begin{description}
\item[{Parameters}] \leavevmode\begin{itemize}
\item {} 
\textbf{\texttt{parent}} (\emph{\texttt{Tkinter.Frame}}) -- Frame padre al que pertenece.

\item {} 
\textbf{\texttt{features}} (\emph{\texttt{Dictionary}}) -- Conjunto de técnicas con sus respectivos parámetros para que
se puedan cargar automáticamente en este Frame \textbf{(véase
Controller/XMLParser.py)}.

\end{itemize}

\item[{Returns}] \leavevmode
Tkinter.Frame

\item[{Return type}] \leavevmode
Instance

\end{description}\end{quote}
\index{\_FunctionFrame\_\_add\_function() (FunctionFrame method)}

\begin{fulllineitems}
\phantomsection\label{View/Main/ObjectiveFunction/FunctionFrame:View.Main.ObjectiveFunction.FunctionFrame.FunctionFrame._FunctionFrame__add_function}\pysiglinewithargsret{\sphinxbfcode{\_FunctionFrame\_\_add\_function}}{\emph{event}}{}~
\begin{notice}{note}{Note:}
Este método es privado.
\end{notice}

Agrega una casilla al Frame. Esta función se usa si 
fue ejecutada por un evento.
\begin{quote}\begin{description}
\item[{Parameters}] \leavevmode
\textbf{\texttt{event}} (\emph{\texttt{String}}) -- Identificador del elemento gráfico que activó la función.

\end{description}\end{quote}

\end{fulllineitems}

\index{\_FunctionFrame\_\_delete\_single\_function() (FunctionFrame method)}

\begin{fulllineitems}
\phantomsection\label{View/Main/ObjectiveFunction/FunctionFrame:View.Main.ObjectiveFunction.FunctionFrame.FunctionFrame._FunctionFrame__delete_single_function}\pysiglinewithargsret{\sphinxbfcode{\_FunctionFrame\_\_delete\_single\_function}}{\emph{event}}{}~
\begin{notice}{note}{Note:}
Este método es privado.
\end{notice}

Elimina una casilla y todos los elementos gráficos que la acompañan.
También elimina todo rastro que se encuentre en las estructuras lógicas.
\begin{quote}\begin{description}
\item[{Parameters}] \leavevmode
\textbf{\texttt{event}} (\emph{\texttt{String}}) -- Identificador del elemento gráfico que activó la función.

\end{description}\end{quote}

\end{fulllineitems}

\index{\_FunctionFrame\_\_grid\_widgets() (FunctionFrame method)}

\begin{fulllineitems}
\phantomsection\label{View/Main/ObjectiveFunction/FunctionFrame:View.Main.ObjectiveFunction.FunctionFrame.FunctionFrame._FunctionFrame__grid_widgets}\pysiglinewithargsret{\sphinxbfcode{\_FunctionFrame\_\_grid\_widgets}}{}{}~
\begin{notice}{note}{Note:}
Este método es privado.
\end{notice}

Coloca elementos en el Frame.

\end{fulllineitems}

\index{get\_current\_elements() (FunctionFrame method)}

\begin{fulllineitems}
\phantomsection\label{View/Main/ObjectiveFunction/FunctionFrame:View.Main.ObjectiveFunction.FunctionFrame.FunctionFrame.get_current_elements}\pysiglinewithargsret{\sphinxbfcode{get\_current\_elements}}{}{}
Regresa el número actual de casillas en el Frame.
\begin{quote}\begin{description}
\item[{Returns}] \leavevmode
Cantidad de elementos en la estructura rows, donde se guardan las casillas (Entries).

\item[{Return type}] \leavevmode
Int

\end{description}\end{quote}

\end{fulllineitems}

\index{get\_information() (FunctionFrame method)}

\begin{fulllineitems}
\phantomsection\label{View/Main/ObjectiveFunction/FunctionFrame:View.Main.ObjectiveFunction.FunctionFrame.FunctionFrame.get_information}\pysiglinewithargsret{\sphinxbfcode{get\_information}}{}{}
Toma la información del Frame y regresa las funciones objectivo que 
el usuario insertó.
\begin{quote}\begin{description}
\item[{Returns}] \leavevmode
Un diccionario que contiene una lista con las funciones escritas.

\item[{Return type}] \leavevmode
Dictionary

\end{description}\end{quote}

\end{fulllineitems}

\index{insert\_function() (FunctionFrame method)}

\begin{fulllineitems}
\phantomsection\label{View/Main/ObjectiveFunction/FunctionFrame:View.Main.ObjectiveFunction.FunctionFrame.FunctionFrame.insert_function}\pysiglinewithargsret{\sphinxbfcode{insert\_function}}{\emph{function=None}}{}~
\begin{DUlineblock}{0em}
\item[] Coloca en el Frame una colección de elementos:
\item[] {[}casilla para insertar funcion, opción de maximizar, opción de minimizar, botón para eliminar{]}
\item[] Si el parámetro function es \textbf{None}, se agrega la casilla vacía, de lo contrario se 
añade ésta con la función.
\end{DUlineblock}
\begin{quote}\begin{description}
\item[{Parameters}] \leavevmode
\textbf{\texttt{function}} (\emph{\texttt{String}}) -- Una función para ser insertada en el primer elemento de la colección.

\end{description}\end{quote}

\end{fulllineitems}

\index{insert\_mop\_example() (FunctionFrame method)}

\begin{fulllineitems}
\phantomsection\label{View/Main/ObjectiveFunction/FunctionFrame:View.Main.ObjectiveFunction.FunctionFrame.FunctionFrame.insert_mop_example}\pysiglinewithargsret{\sphinxbfcode{insert\_mop\_example}}{\emph{functions}}{}~
\begin{DUlineblock}{0em}
\item[] Inserta un M.O.P (Multi Objective Problem) que no es más que un conjunto de 
funciones para que se pueda hacer más rápidamente una prueba.
\item[] Previo a ésto se limpia el Frame para insertar únicamente el M.O.P.
\item[] \textbf{(véase Controller/XML/MOPExample.xml)}
\item[] \textbf{(véase View/Additional/MenuInternalOption/InternalOptionFrame.py)}.
\end{DUlineblock}
\begin{quote}\begin{description}
\item[{Parameters}] \leavevmode
\textbf{\texttt{functions}} (\emph{\texttt{List}}) -- Conjunto de funciones para insertar en el Frame.

\end{description}\end{quote}

\end{fulllineitems}

\index{restore\_settings() (FunctionFrame method)}

\begin{fulllineitems}
\phantomsection\label{View/Main/ObjectiveFunction/FunctionFrame:View.Main.ObjectiveFunction.FunctionFrame.FunctionFrame.restore_settings}\pysiglinewithargsret{\sphinxbfcode{restore\_settings}}{}{}
Restaura el contenido del Frame a sus valores por defecto.
Esto significa que borrará cualquier contenido que se encuentre en existencia y 
dejará una casilla vacía.

\end{fulllineitems}


\end{fulllineitems}



\subsubsection{Population (módulo)}
\label{View/Main/Population/Population::doc}\label{View/Main/Population/Population:population-modulo}
Proporciona las estructuras gráficas para que el usuario pueda
configurar atributos de la población.

Los elementos que conforman al módulo son los siguientes:


\paragraph{PopulationFrame (clase)}
\label{View/Main/Population/PopulationFrame:populationframe-clase}\label{View/Main/Population/PopulationFrame:module-View.Main.Population.PopulationFrame}\label{View/Main/Population/PopulationFrame::doc}\index{View.Main.Population.PopulationFrame (module)}\index{PopulationFrame (class in View.Main.Population.PopulationFrame)}

\begin{fulllineitems}
\phantomsection\label{View/Main/Population/PopulationFrame:View.Main.Population.PopulationFrame.PopulationFrame}\pysiglinewithargsret{\sphinxstrong{class }\sphinxbfcode{PopulationFrame}}{\emph{parent}, \emph{features}}{}
Bases: \sphinxcode{Tkinter.Frame}

Unifica y mantiene un control sobre las clases PopulaceFrame y 
FitnessFrame, esto con el fin de poder colocar los elementos apropiadamente y 
agilizar el intercambio de información con el usuario.
\begin{quote}\begin{description}
\item[{Parameters}] \leavevmode\begin{itemize}
\item {} 
\textbf{\texttt{parent}} (\emph{\texttt{Tkinter.Frame}}) -- Frame padre al que pertenece.

\item {} 
\textbf{\texttt{features}} (\emph{\texttt{Dictionary}}) -- Conjunto de técnicas con sus respectivos parámetros para que
se puedan cargar automáticamente en este Frame \textbf{(véase
Controller/XMLParser.py)}.

\end{itemize}

\item[{Returns}] \leavevmode
Tkinter.Frame

\item[{Return type}] \leavevmode
Instance

\end{description}\end{quote}
\index{get\_information() (PopulationFrame method)}

\begin{fulllineitems}
\phantomsection\label{View/Main/Population/PopulationFrame:View.Main.Population.PopulationFrame.PopulationFrame.get_information}\pysiglinewithargsret{\sphinxbfcode{get\_information}}{}{}
Toma la información propiciada en cada Frame y después
la unifica para regresar un sólo conjunto de información.
\begin{quote}\begin{description}
\item[{Returns}] \leavevmode
Un diccionario con la información de PopulaceFrame y FitnessFrame.

\item[{Return type}] \leavevmode
Dictionary

\end{description}\end{quote}

\end{fulllineitems}

\index{restore\_settings() (PopulationFrame method)}

\begin{fulllineitems}
\phantomsection\label{View/Main/Population/PopulationFrame:View.Main.Population.PopulationFrame.PopulationFrame.restore_settings}\pysiglinewithargsret{\sphinxbfcode{restore\_settings}}{}{}
Restaura los valores por defecto en ambos Frames.

\end{fulllineitems}


\end{fulllineitems}



\paragraph{TemplatePopulationFrame (clase)}
\label{View/Main/Population/TemplatePopulation/TemplatePopulationFrame:templatepopulationframe-clase}\label{View/Main/Population/TemplatePopulation/TemplatePopulationFrame::doc}\label{View/Main/Population/TemplatePopulation/TemplatePopulationFrame:module-View.Main.Population.TemplatePopulation.TemplatePopulationFrame}\index{View.Main.Population.TemplatePopulation.TemplatePopulationFrame (module)}\index{TemplatePopulationFrame (class in View.Main.Population.TemplatePopulation.TemplatePopulationFrame)}

\begin{fulllineitems}
\phantomsection\label{View/Main/Population/TemplatePopulation/TemplatePopulationFrame:View.Main.Population.TemplatePopulation.TemplatePopulationFrame.TemplatePopulationFrame}\pysiglinewithargsret{\sphinxstrong{class }\sphinxbfcode{TemplatePopulationFrame}}{\emph{parent}, \emph{name}, \emph{features}}{}
Bases: \sphinxcode{Tkinter.Frame}

\begin{DUlineblock}{0em}
\item[] Esta clase proporciona la infraestructura gráfica para que el usuario pueda 
elegir técnicas y configurar atributos concernientes al Fitness de una Población
y a la Población en general.
\item[] A grandes rasgos se trata de una plantilla que deberán implementar las clases
FitnessFrame y PopulaceFrame.
\item[] La clase permite la selección de cada posible técnica disponible y automáticamente 
se muestran los parámetros necesarios \textbf{(si los hay)} para cada una de éstas.
\end{DUlineblock}
\begin{quote}\begin{description}
\item[{Parameters}] \leavevmode\begin{itemize}
\item {} 
\textbf{\texttt{parent}} (\emph{\texttt{Tkinter.Frame}}) -- Frame padre al que pertenece.

\item {} 
\textbf{\texttt{name}} (\emph{\texttt{String}}) -- Identificador \textbf{(único)} que tendrá el Frame.

\item {} 
\textbf{\texttt{features}} (\emph{\texttt{Dictionary}}) -- Conjunto de técnicas con sus respectivos parámetros para que
se puedan cargar automáticamente en este frame \textbf{(véase
Controller/XMLParser.py)}.

\end{itemize}

\item[{Returns}] \leavevmode
Tkinter.Frame

\item[{Return type}] \leavevmode
Instance

\end{description}\end{quote}
\index{\_TemplatePopulationFrame\_\_create\_dynamic\_widgets() (TemplatePopulationFrame method)}

\begin{fulllineitems}
\phantomsection\label{View/Main/Population/TemplatePopulation/TemplatePopulationFrame:View.Main.Population.TemplatePopulation.TemplatePopulationFrame.TemplatePopulationFrame._TemplatePopulationFrame__create_dynamic_widgets}\pysiglinewithargsret{\sphinxbfcode{\_TemplatePopulationFrame\_\_create\_dynamic\_widgets}}{}{}~
\begin{notice}{note}{Note:}
Este método es privado.
\end{notice}

Inicializa los elementos dinámicos del Frame, esto es, de acuerdo al tipo 
que lleva cada parámetro se creará un widget diferente.

\end{fulllineitems}

\index{\_TemplatePopulationFrame\_\_update\_widgets() (TemplatePopulationFrame method)}

\begin{fulllineitems}
\phantomsection\label{View/Main/Population/TemplatePopulation/TemplatePopulationFrame:View.Main.Population.TemplatePopulation.TemplatePopulationFrame.TemplatePopulationFrame._TemplatePopulationFrame__update_widgets}\pysiglinewithargsret{\sphinxbfcode{\_TemplatePopulationFrame\_\_update\_widgets}}{\emph{event=None}}{}~
\begin{notice}{note}{Note:}
Este método es privado.
\end{notice}

\begin{DUlineblock}{0em}
\item[] Realiza solamente la actualización y colocación de elementos dinámicos 
en el Frame.
\item[] Si el parámetro event es distinto de \textbf{None}, significa que se lanzó 
un evento que provocará que se actualicen los parámetros de acuerdo con
la técnica seleccionada.
\end{DUlineblock}
\begin{quote}\begin{description}
\item[{Parameters}] \leavevmode
\textbf{\texttt{event}} (\emph{\texttt{String}}) -- Contiene el valor del elemento que ejecutó esta función.

\end{description}\end{quote}

\end{fulllineitems}

\index{get\_information() (TemplatePopulationFrame method)}

\begin{fulllineitems}
\phantomsection\label{View/Main/Population/TemplatePopulation/TemplatePopulationFrame:View.Main.Population.TemplatePopulation.TemplatePopulationFrame.TemplatePopulationFrame.get_information}\pysiglinewithargsret{\sphinxbfcode{get\_information}}{}{}
Recolecta la información que ha seleccionado e introducido el usuario,
también la organiza para que se pueda utilizar apropiadamente.
\begin{quote}\begin{description}
\item[{Returns}] \leavevmode

\begin{DUlineblock}{0em}
\item[] Un diccionario que contiene:
\item[] \textbf{Clase},
\item[] \textbf{Técnica},
\item[] \textbf{Parametros}
\end{DUlineblock}


\item[{Return type}] \leavevmode
Dictionary

\end{description}\end{quote}

\end{fulllineitems}

\index{grid\_widgets() (TemplatePopulationFrame method)}

\begin{fulllineitems}
\phantomsection\label{View/Main/Population/TemplatePopulation/TemplatePopulationFrame:View.Main.Population.TemplatePopulation.TemplatePopulationFrame.TemplatePopulationFrame.grid_widgets}\pysiglinewithargsret{\sphinxbfcode{grid\_widgets}}{}{}
Permite la colocación adecuada de elementos estáticos y dinámicos, 
considerando además el espacio o características necesarias de redimensionamiento 
para éstos últimos.

\end{fulllineitems}

\index{restore\_settings() (TemplatePopulationFrame method)}

\begin{fulllineitems}
\phantomsection\label{View/Main/Population/TemplatePopulation/TemplatePopulationFrame:View.Main.Population.TemplatePopulation.TemplatePopulationFrame.TemplatePopulationFrame.restore_settings}\pysiglinewithargsret{\sphinxbfcode{restore\_settings}}{}{}
Asigna los valores por defecto tanto de las técnicas como de sus 
respectivos parámetros, también limpia aquéllos en donde se hayan 
insertado valores.

\end{fulllineitems}


\end{fulllineitems}


Los siguientes elementos implementan la plantilla actual:


\subparagraph{PopulaceFrame (clase)}
\label{View/Main/Population/TemplatePopulation/PopulaceFrame:populaceframe-clase}\label{View/Main/Population/TemplatePopulation/PopulaceFrame::doc}\label{View/Main/Population/TemplatePopulation/PopulaceFrame:module-View.Main.Population.PopulaceFrame}\index{View.Main.Population.PopulaceFrame (module)}\index{PopulaceFrame (class in View.Main.Population.PopulaceFrame)}

\begin{fulllineitems}
\phantomsection\label{View/Main/Population/TemplatePopulation/PopulaceFrame:View.Main.Population.PopulaceFrame.PopulaceFrame}\pysiglinewithargsret{\sphinxstrong{class }\sphinxbfcode{PopulaceFrame}}{\emph{parent}, \emph{name}, \emph{features}}{}
Bases: {\hyperref[View/Main/Population/TemplatePopulation/TemplatePopulationFrame:View.Main.Population.TemplatePopulation.TemplatePopulationFrame.TemplatePopulationFrame]{\sphinxcrossref{\sphinxcode{View.Main.Population.TemplatePopulation.TemplatePopulationFrame.TemplatePopulationFrame}}}}

\begin{DUlineblock}{0em}
\item[] Esta clase proporciona la infraestructura gráfica para que el usuario pueda 
elegir métodos y características concernientes a la conformación de la población.
\item[] También hereda atributos de la clase TemplatePopulationFrame con el fin de 
establecer una forma más rápida y ordenada de colocar componentes y recolectar
la información de éstos.
\end{DUlineblock}
\begin{quote}\begin{description}
\item[{Parameters}] \leavevmode\begin{itemize}
\item {} 
\textbf{\texttt{parent}} (\emph{\texttt{Tkinter.Frame}}) -- Frame padre al que pertenece.

\item {} 
\textbf{\texttt{name}} (\emph{\texttt{String}}) -- Identificador \textbf{(único)} que tendrá el Frame.

\item {} 
\textbf{\texttt{features}} (\emph{\texttt{Dictionary}}) -- Conjunto de técnicas con sus respectivos parámetros para que
se puedan cargar automáticamente en este Frame \textbf{(véase
Controller/XMLParser.py)}.

\end{itemize}

\item[{Returns}] \leavevmode
Tkinter.Frame

\item[{Return type}] \leavevmode
Instance

\end{description}\end{quote}
\index{get\_information() (PopulaceFrame method)}

\begin{fulllineitems}
\phantomsection\label{View/Main/Population/TemplatePopulation/PopulaceFrame:View.Main.Population.PopulaceFrame.PopulaceFrame.get_information}\pysiglinewithargsret{\sphinxbfcode{get\_information}}{}{}
Recolecta la información genérica \textbf{(usando el método de la clase Padre)}, y también
se le añade aquélla recolectada exclusivamente en esta clase.
\begin{quote}\begin{description}
\item[{Returns}] \leavevmode

\begin{DUlineblock}{0em}
\item[] Un diccionario que contiene:
\item[] \textbf{Métodos genéricos,}
\item[] \textbf{Número de Generaciones,}
\item[] \textbf{Tamaño de la Población,}
\item[] \textbf{Número de Decimales.}
\end{DUlineblock}


\item[{Return type}] \leavevmode
Dictionary

\end{description}\end{quote}

\end{fulllineitems}

\index{restore\_settings() (PopulaceFrame method)}

\begin{fulllineitems}
\phantomsection\label{View/Main/Population/TemplatePopulation/PopulaceFrame:View.Main.Population.PopulaceFrame.PopulaceFrame.restore_settings}\pysiglinewithargsret{\sphinxbfcode{restore\_settings}}{}{}
Por un lado, restaura el contenido de los elementos pertenecientes sólo 
a esta clase, y por el otro, activa el método de la clase Padre que realiza
una restauración de los elementos genéricos.

\end{fulllineitems}


\end{fulllineitems}



\subparagraph{FitnessFrame (clase)}
\label{View/Main/Population/TemplatePopulation/FitnessFrame:fitnessframe-clase}\label{View/Main/Population/TemplatePopulation/FitnessFrame:module-View.Main.Population.FitnessFrame}\label{View/Main/Population/TemplatePopulation/FitnessFrame::doc}\index{View.Main.Population.FitnessFrame (module)}\index{FitnessFrame (class in View.Main.Population.FitnessFrame)}

\begin{fulllineitems}
\phantomsection\label{View/Main/Population/TemplatePopulation/FitnessFrame:View.Main.Population.FitnessFrame.FitnessFrame}\pysiglinewithargsret{\sphinxstrong{class }\sphinxbfcode{FitnessFrame}}{\emph{parent}, \emph{name}, \emph{features}}{}
Bases: {\hyperref[View/Main/Population/TemplatePopulation/TemplatePopulationFrame:View.Main.Population.TemplatePopulation.TemplatePopulationFrame.TemplatePopulationFrame]{\sphinxcrossref{\sphinxcode{View.Main.Population.TemplatePopulation.TemplatePopulationFrame.TemplatePopulationFrame}}}}

\begin{DUlineblock}{0em}
\item[] Esta clase proporciona la infraestructura gráfica para que el usuario pueda 
elegir métodos concernientes a la asignación del Fitness para la población.
\item[] Además hereda atributos de la clase TemplatePopulationFrame para facilitar
la colocacion y extracción de información pertinente para el usuario.
\end{DUlineblock}
\begin{quote}\begin{description}
\item[{Parameters}] \leavevmode\begin{itemize}
\item {} 
\textbf{\texttt{parent}} (\emph{\texttt{Tkinter.Frame}}) -- Frame padre al que pertenece.

\item {} 
\textbf{\texttt{name}} (\emph{\texttt{String}}) -- Identificador \textbf{(único)} que tendrá el Frame.

\item {} 
\textbf{\texttt{features}} (\emph{\texttt{Dictionary}}) -- Conjunto de técnicas con sus respectivos parámetros para que
se puedan cargar automáticamente en este Frame \textbf{(véase
Controller/XMLParser.py)}.

\end{itemize}

\item[{Returns}] \leavevmode
Tkinter.Frame

\item[{Return type}] \leavevmode
Instance

\end{description}\end{quote}
\index{get\_information() (FitnessFrame method)}

\begin{fulllineitems}
\phantomsection\label{View/Main/Population/TemplatePopulation/FitnessFrame:View.Main.Population.FitnessFrame.FitnessFrame.get_information}\pysiglinewithargsret{\sphinxbfcode{get\_information}}{}{}
Llama al método de la clase Padre, el cual recopila toda la información
elegida por el usuario y la regresa en forma de diccionario.
\begin{quote}\begin{description}
\item[{Returns}] \leavevmode
Diccionario con información de los métodos genéricos.

\item[{Return type}] \leavevmode
Dictionary

\end{description}\end{quote}

\end{fulllineitems}

\index{restore\_settings() (FitnessFrame method)}

\begin{fulllineitems}
\phantomsection\label{View/Main/Population/TemplatePopulation/FitnessFrame:View.Main.Population.FitnessFrame.FitnessFrame.restore_settings}\pysiglinewithargsret{\sphinxbfcode{restore\_settings}}{}{}
Llamar al método de la clase Padre, el cual restaura los valores por 
defecto de los elementos dinámicos y estáticos del Frame.

\end{fulllineitems}


\end{fulllineitems}



\subsubsection{GeneticOperator (módulo)}
\label{View/Main/GeneticOperator/GeneticOperator:geneticoperator-modulo}\label{View/Main/GeneticOperator/GeneticOperator::doc}
Proporciona los elementos gráficos para que el usuario pueda
realizar operaciones relacionadas con la Selección, Cruza y Mutación de
individuos de una población.

Los elementos que lo conforman son:


\paragraph{GeneticOperatorFrame (clase)}
\label{View/Main/GeneticOperator/GeneticOperatorFrame:geneticoperatorframe-clase}\label{View/Main/GeneticOperator/GeneticOperatorFrame::doc}\label{View/Main/GeneticOperator/GeneticOperatorFrame:module-View.Main.GeneticOperator.GeneticOperatorFrame}\index{View.Main.GeneticOperator.GeneticOperatorFrame (module)}\index{GeneticOperatorFrame (class in View.Main.GeneticOperator.GeneticOperatorFrame)}

\begin{fulllineitems}
\phantomsection\label{View/Main/GeneticOperator/GeneticOperatorFrame:View.Main.GeneticOperator.GeneticOperatorFrame.GeneticOperatorFrame}\pysiglinewithargsret{\sphinxstrong{class }\sphinxbfcode{GeneticOperatorFrame}}{\emph{parent}, \emph{features}}{}
Bases: \sphinxcode{Tkinter.Frame}

Reúne y controla las clases SelectionFrame, CrossoverFrame y  
MutationFrame con la finalidad de colocar los elementos gráficos apropiadamente y 
agilizar el intercambio de información con el usuario.
\begin{quote}\begin{description}
\item[{Parameters}] \leavevmode\begin{itemize}
\item {} 
\textbf{\texttt{parent}} (\emph{\texttt{Tkinter.Frame}}) -- Frame padre al que pertenece.

\item {} 
\textbf{\texttt{features}} (\emph{\texttt{Dictionary}}) -- Conjunto de técnicas con sus respectivos parámetros para que
se puedan cargar automáticamente en este Frame \textbf{(véase
Controller/XMLParser.py)}.

\end{itemize}

\item[{Returns}] \leavevmode
Tkinter.Frame

\item[{Return type}] \leavevmode
Instance

\end{description}\end{quote}
\index{get\_information() (GeneticOperatorFrame method)}

\begin{fulllineitems}
\phantomsection\label{View/Main/GeneticOperator/GeneticOperatorFrame:View.Main.GeneticOperator.GeneticOperatorFrame.GeneticOperatorFrame.get_information}\pysiglinewithargsret{\sphinxbfcode{get\_information}}{}{}
Toma la información propiciada en cada Frame y después
la unifica para regresar un sólo conjunto de información.
\begin{quote}\begin{description}
\item[{Returns}] \leavevmode
Un diccionario con la información de SelectionFrame, CrossoverFrame y MutationFrame.

\item[{Return type}] \leavevmode
Dictionary

\end{description}\end{quote}

\end{fulllineitems}

\index{restore\_settings() (GeneticOperatorFrame method)}

\begin{fulllineitems}
\phantomsection\label{View/Main/GeneticOperator/GeneticOperatorFrame:View.Main.GeneticOperator.GeneticOperatorFrame.GeneticOperatorFrame.restore_settings}\pysiglinewithargsret{\sphinxbfcode{restore\_settings}}{}{}
Realiza la restauración de información y contenido en
cada uno de los Frames.

\end{fulllineitems}


\end{fulllineitems}



\paragraph{TemplateGeneticOperatorFrame (clase)}
\label{View/Main/GeneticOperator/TemplateGeneticOperator/TemplateGeneticOperatorFrame:templategeneticoperatorframe-clase}\label{View/Main/GeneticOperator/TemplateGeneticOperator/TemplateGeneticOperatorFrame:module-View.Main.GeneticOperator.TemplateGeneticOperator.TemplateGeneticOperatorFrame}\label{View/Main/GeneticOperator/TemplateGeneticOperator/TemplateGeneticOperatorFrame::doc}\index{View.Main.GeneticOperator.TemplateGeneticOperator.TemplateGeneticOperatorFrame (module)}\index{TemplateGeneticOperatorFrame (class in View.Main.GeneticOperator.TemplateGeneticOperator.TemplateGeneticOperatorFrame)}

\begin{fulllineitems}
\phantomsection\label{View/Main/GeneticOperator/TemplateGeneticOperator/TemplateGeneticOperatorFrame:View.Main.GeneticOperator.TemplateGeneticOperator.TemplateGeneticOperatorFrame.TemplateGeneticOperatorFrame}\pysiglinewithargsret{\sphinxstrong{class }\sphinxbfcode{TemplateGeneticOperatorFrame}}{\emph{parent}, \emph{name}, \emph{features}, \emph{sort\_techniques=False}}{}
Bases: \sphinxcode{Tkinter.Frame}

\begin{DUlineblock}{0em}
\item[] Proporciona la infraestructura gráfica para que el usuario pueda 
elegir técnicas y configurar atributos concernientes a la Selección, Cruza y 
Mutación de Individuos de una Población.
\item[] A grandes rasgos se trata de una plantilla que deberán implementar las clases
SelectionFrame, CrossoverFrame y MutationFrame.
\item[] La clase permite la selección de cada posible técnica disponible y automáticamente 
se muestran los parámetros necesarios \textbf{(si los hay)} para cada una de éstas.
\end{DUlineblock}
\begin{quote}\begin{description}
\item[{Parameters}] \leavevmode\begin{itemize}
\item {} 
\textbf{\texttt{parent}} (\emph{\texttt{Tkinter.Frame}}) -- Frame padre al que pertenece.

\item {} 
\textbf{\texttt{name}} (\emph{\texttt{String}}) -- Identificador \textbf{(único)} que tendrá el Frame.

\item {} 
\textbf{\texttt{features}} (\emph{\texttt{Dictionary}}) -- Conjunto de técnicas con sus respectivos parámetros para que
se puedan cargar automáticamente en este Frame \textbf{(véase
Controller/XMLParser.py)}.

\item {} 
\textbf{\texttt{sort\_techniques}} (\emph{\texttt{Boolean}}) -- Indica si las técnicas disponibles se ordenan alfabéticamente
o no.

\end{itemize}

\item[{Returns}] \leavevmode
Tkinter.Frame

\item[{Return type}] \leavevmode
Instance

\end{description}\end{quote}
\index{\_TemplateGeneticOperatorFrame\_\_create\_dynamic\_widgets() (TemplateGeneticOperatorFrame method)}

\begin{fulllineitems}
\phantomsection\label{View/Main/GeneticOperator/TemplateGeneticOperator/TemplateGeneticOperatorFrame:View.Main.GeneticOperator.TemplateGeneticOperator.TemplateGeneticOperatorFrame.TemplateGeneticOperatorFrame._TemplateGeneticOperatorFrame__create_dynamic_widgets}\pysiglinewithargsret{\sphinxbfcode{\_TemplateGeneticOperatorFrame\_\_create\_dynamic\_widgets}}{}{}~
\begin{notice}{note}{Note:}
Este método es privado.
\end{notice}

Inicializa los elementos dinámicos del Frame, esto es, de acuerdo al tipo 
que lleva cada parámetro se creará un widget diferente.

\end{fulllineitems}

\index{\_TemplateGeneticOperatorFrame\_\_update\_widgets() (TemplateGeneticOperatorFrame method)}

\begin{fulllineitems}
\phantomsection\label{View/Main/GeneticOperator/TemplateGeneticOperator/TemplateGeneticOperatorFrame:View.Main.GeneticOperator.TemplateGeneticOperator.TemplateGeneticOperatorFrame.TemplateGeneticOperatorFrame._TemplateGeneticOperatorFrame__update_widgets}\pysiglinewithargsret{\sphinxbfcode{\_TemplateGeneticOperatorFrame\_\_update\_widgets}}{\emph{event=None}}{}~
\begin{notice}{note}{Note:}
Este método es privado.
\end{notice}

\begin{DUlineblock}{0em}
\item[] Realiza solamente la actualización y colocación de elementos dinámicos 
en el Frame.
\item[] Si el parámetro event es distinto de \textbf{None}, significa que se lanzó 
un evento que provocará que se actualicen los parámetros de acuerdo con
la técnica seleccionada.
\end{DUlineblock}
\begin{quote}\begin{description}
\item[{Parameters}] \leavevmode
\textbf{\texttt{event}} (\emph{\texttt{String}}) -- Contiene el valor del elemento que ejecutó esta función.

\end{description}\end{quote}

\end{fulllineitems}

\index{get\_information() (TemplateGeneticOperatorFrame method)}

\begin{fulllineitems}
\phantomsection\label{View/Main/GeneticOperator/TemplateGeneticOperator/TemplateGeneticOperatorFrame:View.Main.GeneticOperator.TemplateGeneticOperator.TemplateGeneticOperatorFrame.TemplateGeneticOperatorFrame.get_information}\pysiglinewithargsret{\sphinxbfcode{get\_information}}{}{}
Recolecta la información que ha seleccionado e introducido el usuario,
también la organiza para que se pueda utilizar apropiadamente.
\begin{quote}\begin{description}
\item[{Returns}] \leavevmode

\begin{DUlineblock}{0em}
\item[] Un diccionario que contiene:
\item[] \textbf{Clase},
\item[] \textbf{Técnica},
\item[] \textbf{Parámetros}
\end{DUlineblock}


\item[{Return type}] \leavevmode
Dictionary

\end{description}\end{quote}

\end{fulllineitems}

\index{grid\_widgets() (TemplateGeneticOperatorFrame method)}

\begin{fulllineitems}
\phantomsection\label{View/Main/GeneticOperator/TemplateGeneticOperator/TemplateGeneticOperatorFrame:View.Main.GeneticOperator.TemplateGeneticOperator.TemplateGeneticOperatorFrame.TemplateGeneticOperatorFrame.grid_widgets}\pysiglinewithargsret{\sphinxbfcode{grid\_widgets}}{}{}
Permite la colocación adecuada de elementos estáticos y dinámicos, considerando
además el espacio o características necesarias de redimensionamiento para éstos últimos.

\end{fulllineitems}

\index{restore\_settings() (TemplateGeneticOperatorFrame method)}

\begin{fulllineitems}
\phantomsection\label{View/Main/GeneticOperator/TemplateGeneticOperator/TemplateGeneticOperatorFrame:View.Main.GeneticOperator.TemplateGeneticOperator.TemplateGeneticOperatorFrame.TemplateGeneticOperatorFrame.restore_settings}\pysiglinewithargsret{\sphinxbfcode{restore\_settings}}{}{}
Asigna los valores por defecto tanto de las técnicas como de sus 
respectivos parámetros, también limpia aquéllos en donde se hayan 
insertado valores.

\end{fulllineitems}


\end{fulllineitems}


Las clases que implementan esta plantilla son las siguientes:


\subparagraph{SelectionFrame (clase)}
\label{View/Main/GeneticOperator/TemplateGeneticOperator/SelectionFrame:module-View.Main.GeneticOperator.SelectionFrame}\label{View/Main/GeneticOperator/TemplateGeneticOperator/SelectionFrame:selectionframe-clase}\label{View/Main/GeneticOperator/TemplateGeneticOperator/SelectionFrame::doc}\index{View.Main.GeneticOperator.SelectionFrame (module)}\index{SelectionFrame (class in View.Main.GeneticOperator.SelectionFrame)}

\begin{fulllineitems}
\phantomsection\label{View/Main/GeneticOperator/TemplateGeneticOperator/SelectionFrame:View.Main.GeneticOperator.SelectionFrame.SelectionFrame}\pysiglinewithargsret{\sphinxstrong{class }\sphinxbfcode{SelectionFrame}}{\emph{parent}, \emph{name}, \emph{features}}{}
Bases: {\hyperref[View/Main/GeneticOperator/TemplateGeneticOperator/TemplateGeneticOperatorFrame:View.Main.GeneticOperator.TemplateGeneticOperator.TemplateGeneticOperatorFrame.TemplateGeneticOperatorFrame]{\sphinxcrossref{\sphinxcode{View.Main.GeneticOperator.TemplateGeneticOperator.TemplateGeneticOperatorFrame.TemplateGeneticOperatorFrame}}}}

\begin{DUlineblock}{0em}
\item[] Esta clase proporciona la infraestructura gráfica para que el usuario pueda 
elegir métodos y características relacionadas con la selección de Individuos.
\item[] También hereda atributos de la clase TemplateGeneticOperatorFrame para facilitar
la carga de elementos en el Frame y su correspondiente recolección de información.
\end{DUlineblock}
\begin{quote}\begin{description}
\item[{Parameters}] \leavevmode\begin{itemize}
\item {} 
\textbf{\texttt{parent}} (\emph{\texttt{Tkinter.Frame}}) -- Frame padre al que pertenece.

\item {} 
\textbf{\texttt{name}} (\emph{\texttt{String}}) -- Identificador \textbf{(único)} que tendrá el Frame.

\item {} 
\textbf{\texttt{features}} (\emph{\texttt{Dictionary}}) -- Conjunto de técnicas con sus respectivos parámetros para que
se puedan cargar automáticamente en este Frame \textbf{(véase
Controller/XMLParser.py)}.

\end{itemize}

\item[{Returns}] \leavevmode
Tkinter.Frame

\item[{Return type}] \leavevmode
Instance

\end{description}\end{quote}
\index{get\_information() (SelectionFrame method)}

\begin{fulllineitems}
\phantomsection\label{View/Main/GeneticOperator/TemplateGeneticOperator/SelectionFrame:View.Main.GeneticOperator.SelectionFrame.SelectionFrame.get_information}\pysiglinewithargsret{\sphinxbfcode{get\_information}}{}{}
Recolecta la información relativa a esta clase haciendo uso del método
de la clase Padre.
\begin{quote}\begin{description}
\item[{Returns}] \leavevmode
Diccionario con información de los métodos genéricos.

\item[{Return type}] \leavevmode
Dictionary

\end{description}\end{quote}

\end{fulllineitems}

\index{restore\_settings() (SelectionFrame method)}

\begin{fulllineitems}
\phantomsection\label{View/Main/GeneticOperator/TemplateGeneticOperator/SelectionFrame:View.Main.GeneticOperator.SelectionFrame.SelectionFrame.restore_settings}\pysiglinewithargsret{\sphinxbfcode{restore\_settings}}{}{}
Ejecuta el método de la clase Padre, el cual restaura los valores por 
defecto de los elementos dinámicos y estáticos del Frame.

\end{fulllineitems}


\end{fulllineitems}



\subparagraph{CrossoverFrame (clase)}
\label{View/Main/GeneticOperator/TemplateGeneticOperator/CrossoverFrame:module-View.Main.GeneticOperator.CrossoverFrame}\label{View/Main/GeneticOperator/TemplateGeneticOperator/CrossoverFrame:crossoverframe-clase}\label{View/Main/GeneticOperator/TemplateGeneticOperator/CrossoverFrame::doc}\index{View.Main.GeneticOperator.CrossoverFrame (module)}\index{CrossoverFrame (class in View.Main.GeneticOperator.CrossoverFrame)}

\begin{fulllineitems}
\phantomsection\label{View/Main/GeneticOperator/TemplateGeneticOperator/CrossoverFrame:View.Main.GeneticOperator.CrossoverFrame.CrossoverFrame}\pysiglinewithargsret{\sphinxstrong{class }\sphinxbfcode{CrossoverFrame}}{\emph{parent}, \emph{name}, \emph{features}}{}
Bases: {\hyperref[View/Main/GeneticOperator/TemplateGeneticOperator/TemplateGeneticOperatorFrame:View.Main.GeneticOperator.TemplateGeneticOperator.TemplateGeneticOperatorFrame.TemplateGeneticOperatorFrame]{\sphinxcrossref{\sphinxcode{View.Main.GeneticOperator.TemplateGeneticOperator.TemplateGeneticOperatorFrame.TemplateGeneticOperatorFrame}}}}

\begin{DUlineblock}{0em}
\item[] Esta clase proporciona la infraestructura gráfica para que el usuario pueda 
elegir técnicas y características concernientes a la cruza entre Individuos.
\item[] También hereda atributos de la clase TemplateGeneticOperatorFrame para facilitar
la carga de elementos en el Frame y su correspondiente recolección de información.
\end{DUlineblock}
\begin{quote}\begin{description}
\item[{Parameters}] \leavevmode\begin{itemize}
\item {} 
\textbf{\texttt{parent}} (\emph{\texttt{Tkinter.Frame}}) -- Frame padre al que pertenece.

\item {} 
\textbf{\texttt{name}} (\emph{\texttt{String}}) -- Identificador \textbf{(único)} que tendrá el Frame.

\item {} 
\textbf{\texttt{features}} (\emph{\texttt{Dictionary}}) -- Conjunto de técnicas con sus respectivos parámetros para que
se puedan cargar automáticamente en este Frame \textbf{(véase
Controller/XMLParser.py)}.

\end{itemize}

\item[{Returns}] \leavevmode
Tkinter.Frame

\item[{Return type}] \leavevmode
Instance

\end{description}\end{quote}
\index{get\_information() (CrossoverFrame method)}

\begin{fulllineitems}
\phantomsection\label{View/Main/GeneticOperator/TemplateGeneticOperator/CrossoverFrame:View.Main.GeneticOperator.CrossoverFrame.CrossoverFrame.get_information}\pysiglinewithargsret{\sphinxbfcode{get\_information}}{}{}
Recolecta la información genérica \textbf{(usando el método de la clase Padre)}, y también
se le añade aquélla recolectada exclusivamente en esta clase.
\begin{quote}\begin{description}
\item[{Returns}] \leavevmode

\begin{DUlineblock}{0em}
\item[] Un diccionario que contiene:
\item[] \textbf{Métodos genéricos,}
\item[] \textbf{Probabilidad de cruza.}
\end{DUlineblock}


\item[{Return type}] \leavevmode
Dictionary

\end{description}\end{quote}

\end{fulllineitems}

\index{restore\_settings() (CrossoverFrame method)}

\begin{fulllineitems}
\phantomsection\label{View/Main/GeneticOperator/TemplateGeneticOperator/CrossoverFrame:View.Main.GeneticOperator.CrossoverFrame.CrossoverFrame.restore_settings}\pysiglinewithargsret{\sphinxbfcode{restore\_settings}}{}{}
Ejecuta el método de la clase Padre, el cual restaura los valores por 
defecto de los elementos dinámicos y estáticos del Frame.

\end{fulllineitems}


\end{fulllineitems}



\subparagraph{MutationFrame (clase)}
\label{View/Main/GeneticOperator/TemplateGeneticOperator/MutationFrame:module-View.Main.GeneticOperator.MutationFrame}\label{View/Main/GeneticOperator/TemplateGeneticOperator/MutationFrame:mutationframe-clase}\label{View/Main/GeneticOperator/TemplateGeneticOperator/MutationFrame::doc}\index{View.Main.GeneticOperator.MutationFrame (module)}\index{MutationFrame (class in View.Main.GeneticOperator.MutationFrame)}

\begin{fulllineitems}
\phantomsection\label{View/Main/GeneticOperator/TemplateGeneticOperator/MutationFrame:View.Main.GeneticOperator.MutationFrame.MutationFrame}\pysiglinewithargsret{\sphinxstrong{class }\sphinxbfcode{MutationFrame}}{\emph{parent}, \emph{name}, \emph{features}}{}
Bases: {\hyperref[View/Main/GeneticOperator/TemplateGeneticOperator/TemplateGeneticOperatorFrame:View.Main.GeneticOperator.TemplateGeneticOperator.TemplateGeneticOperatorFrame.TemplateGeneticOperatorFrame]{\sphinxcrossref{\sphinxcode{View.Main.GeneticOperator.TemplateGeneticOperator.TemplateGeneticOperatorFrame.TemplateGeneticOperatorFrame}}}}

\begin{DUlineblock}{0em}
\item[] Esta clase proporciona la infraestructura gráfica para que el usuario pueda 
elegir técnicas y características relativas a la mutación de Individuos.
\item[] También hereda atributos de la clase TemplateGeneticOperatorFrame para facilitar
la carga automática de elementos en el Frame y su consecuente recolección de información.
\end{DUlineblock}
\begin{quote}\begin{description}
\item[{Parameters}] \leavevmode\begin{itemize}
\item {} 
\textbf{\texttt{parent}} (\emph{\texttt{Tkinter.Frame}}) -- Frame padre al que pertenece.

\item {} 
\textbf{\texttt{name}} (\emph{\texttt{String}}) -- Identificador \textbf{(único)} que tendrá el Frame.

\item {} 
\textbf{\texttt{features}} (\emph{\texttt{Dictionary}}) -- Conjunto de técnicas con sus respectivos parámetros para que
se puedan cargar automáticamente en este Frame \textbf{(véase
Controller/XMLParser.py)}.

\end{itemize}

\item[{Returns}] \leavevmode
Tkinter.Frame

\item[{Return type}] \leavevmode
Instance

\end{description}\end{quote}
\index{get\_information() (MutationFrame method)}

\begin{fulllineitems}
\phantomsection\label{View/Main/GeneticOperator/TemplateGeneticOperator/MutationFrame:View.Main.GeneticOperator.MutationFrame.MutationFrame.get_information}\pysiglinewithargsret{\sphinxbfcode{get\_information}}{}{}
Recolecta la información genérica \textbf{(usando el método de la clase Padre)}, y también
se le añade aquélla recolectada exclusivamente en esta clase.
\begin{quote}\begin{description}
\item[{Returns}] \leavevmode

\begin{DUlineblock}{0em}
\item[] Un diccionario que contiene:
\item[] \textbf{Métodos genéricos,}
\item[] \textbf{Probabilidad de mutación.}
\end{DUlineblock}


\item[{Return type}] \leavevmode
Dictionary

\end{description}\end{quote}

\end{fulllineitems}

\index{restore\_settings() (MutationFrame method)}

\begin{fulllineitems}
\phantomsection\label{View/Main/GeneticOperator/TemplateGeneticOperator/MutationFrame:View.Main.GeneticOperator.MutationFrame.MutationFrame.restore_settings}\pysiglinewithargsret{\sphinxbfcode{restore\_settings}}{}{}
Ejecuta el método de la clase Padre, el cual restaura los valores por 
defecto de los elementos dinámicos y estáticos del Frame.

\end{fulllineitems}


\end{fulllineitems}



\subsubsection{MOEA (módulo)}
\label{View/Main/MOEA/MOEA::doc}\label{View/Main/MOEA/MOEA:moea-modulo}
Proporciona los elementos gráficos para que el usuario realice
configuraciones concernientes a los M.O.E.A.s \textbf{(Multi-Objective Evolutionary Algorithms
ó Algoritmos Evolutivos Multiobjetivo)} y sus atributos relacionados.

Sus elementos son los siguientes:


\paragraph{MOEAFrame (clase)}
\label{View/Main/MOEA/MOEAFrame:module-View.Main.MOEA.MOEAFrame}\label{View/Main/MOEA/MOEAFrame::doc}\label{View/Main/MOEA/MOEAFrame:moeaframe-clase}\index{View.Main.MOEA.MOEAFrame (module)}\index{MOEAFrame (class in View.Main.MOEA.MOEAFrame)}

\begin{fulllineitems}
\phantomsection\label{View/Main/MOEA/MOEAFrame:View.Main.MOEA.MOEAFrame.MOEAFrame}\pysiglinewithargsret{\sphinxstrong{class }\sphinxbfcode{MOEAFrame}}{\emph{parent}, \emph{features}}{}
Bases: \sphinxcode{Tkinter.Frame}

Unifica los Frames AlgorithmFrame y SharingFunctionFrame, 
la razón de ésto es para facilitar el acomodo de componentes de manera 
individual, para así garantizar un acceso asequible a la información.
\begin{quote}\begin{description}
\item[{Parameters}] \leavevmode\begin{itemize}
\item {} 
\textbf{\texttt{parent}} (\emph{\texttt{Tkinter.Frame}}) -- Frame padre al que pertenece.

\item {} 
\textbf{\texttt{features}} (\emph{\texttt{Dictionary}}) -- Conjunto de técnicas con sus respectivos parámetros para que
se puedan cargar automáticamente en este Frame \textbf{(véase
Controller/XMLParser.py)}.

\end{itemize}

\item[{Returns}] \leavevmode
Tkinter.Frame

\item[{Return type}] \leavevmode
Instance

\end{description}\end{quote}
\index{get\_information() (MOEAFrame method)}

\begin{fulllineitems}
\phantomsection\label{View/Main/MOEA/MOEAFrame:View.Main.MOEA.MOEAFrame.MOEAFrame.get_information}\pysiglinewithargsret{\sphinxbfcode{get\_information}}{}{}
Toma la información solicitada en cada Frame y después
la unifica para regresar un sólo conjunto de información.
\begin{quote}\begin{description}
\item[{Returns}] \leavevmode
Un diccionario con la información de AlgorithmFrame y SharingFunctionFrame.

\item[{Return type}] \leavevmode
Dictionary

\end{description}\end{quote}

\end{fulllineitems}

\index{restore\_settings() (MOEAFrame method)}

\begin{fulllineitems}
\phantomsection\label{View/Main/MOEA/MOEAFrame:View.Main.MOEA.MOEAFrame.MOEAFrame.restore_settings}\pysiglinewithargsret{\sphinxbfcode{restore\_settings}}{}{}
Restaura los valores por defecto en cada Frame.

\end{fulllineitems}


\end{fulllineitems}


La clase actual se apoya de los siguientes elementos:


\subparagraph{AlgorithmFrame (clase)}
\label{View/Main/MOEA/AlgorithmFrame:algorithmframe-clase}\label{View/Main/MOEA/AlgorithmFrame::doc}\label{View/Main/MOEA/AlgorithmFrame:module-View.Main.MOEA.AlgorithmFrame}\index{View.Main.MOEA.AlgorithmFrame (module)}\index{AlgorithmFrame (class in View.Main.MOEA.AlgorithmFrame)}

\begin{fulllineitems}
\phantomsection\label{View/Main/MOEA/AlgorithmFrame:View.Main.MOEA.AlgorithmFrame.AlgorithmFrame}\pysiglinewithargsret{\sphinxstrong{class }\sphinxbfcode{AlgorithmFrame}}{\emph{parent}, \emph{name}, \emph{features}}{}
Bases: \sphinxcode{Tkinter.Frame}

Esta clase proporciona una base gráfica para que el usuario pueda
seleccionar técnicas con sus parámetros correspondientes \textbf{(si es que tienen)}
referentes a los M.O.E.A.'s \textbf{(Multi-Objective Evolutionary Algorithms ó Algoritmos Evolutivos Multiobjetivo)}.
\begin{quote}\begin{description}
\item[{Parameters}] \leavevmode\begin{itemize}
\item {} 
\textbf{\texttt{parent}} (\emph{\texttt{Tkinter.Frame}}) -- Frame padre al que pertenece.

\item {} 
\textbf{\texttt{name}} (\emph{\texttt{String}}) -- Identificador \textbf{(único)} que tendrá el Frame.

\item {} 
\textbf{\texttt{features}} (\emph{\texttt{Dictionary}}) -- Conjunto de técnicas con sus respectivos parámetros para que
se puedan cargar automáticamente en este Frame \textbf{(véase
Controller/XMLParser.py)}.

\end{itemize}

\item[{Returns}] \leavevmode
Tkinter.Frame

\item[{Return type}] \leavevmode
Instance

\end{description}\end{quote}
\index{\_AlgorithmFrame\_\_create\_dynamic\_widgets() (AlgorithmFrame method)}

\begin{fulllineitems}
\phantomsection\label{View/Main/MOEA/AlgorithmFrame:View.Main.MOEA.AlgorithmFrame.AlgorithmFrame._AlgorithmFrame__create_dynamic_widgets}\pysiglinewithargsret{\sphinxbfcode{\_AlgorithmFrame\_\_create\_dynamic\_widgets}}{}{}~
\begin{notice}{note}{Note:}
Este método es privado.
\end{notice}

Inicializa los elementos dinámicos del Frame, esto es, de acuerdo al tipo 
que lleva cada parámetro se creará un widget diferente.

\end{fulllineitems}

\index{\_AlgorithmFrame\_\_grid\_widgets() (AlgorithmFrame method)}

\begin{fulllineitems}
\phantomsection\label{View/Main/MOEA/AlgorithmFrame:View.Main.MOEA.AlgorithmFrame.AlgorithmFrame._AlgorithmFrame__grid_widgets}\pysiglinewithargsret{\sphinxbfcode{\_AlgorithmFrame\_\_grid\_widgets}}{}{}~
\begin{notice}{note}{Note:}
Este método es privado.
\end{notice}

Coloca elementos en el Frame, tanto estáticos como dinámicos.

\end{fulllineitems}

\index{\_AlgorithmFrame\_\_update\_widgets() (AlgorithmFrame method)}

\begin{fulllineitems}
\phantomsection\label{View/Main/MOEA/AlgorithmFrame:View.Main.MOEA.AlgorithmFrame.AlgorithmFrame._AlgorithmFrame__update_widgets}\pysiglinewithargsret{\sphinxbfcode{\_AlgorithmFrame\_\_update\_widgets}}{\emph{event=None}}{}~
\begin{notice}{note}{Note:}
Este método es privado.
\end{notice}

\begin{DUlineblock}{0em}
\item[] Realiza solamente la actualización y colocación de elementos dinámicos 
en el Frame.
\item[] Si el parámetro event es distinto de \textbf{None}, significa que se lanzó 
un evento que provocará que se actualicen los parámetros de acuerdo con
la técnica seleccionada.
\end{DUlineblock}
\begin{quote}\begin{description}
\item[{Parameters}] \leavevmode
\textbf{\texttt{event}} (\emph{\texttt{String}}) -- Contiene el valor del elemento que ejecutó esta función.

\end{description}\end{quote}

\end{fulllineitems}

\index{get\_information() (AlgorithmFrame method)}

\begin{fulllineitems}
\phantomsection\label{View/Main/MOEA/AlgorithmFrame:View.Main.MOEA.AlgorithmFrame.AlgorithmFrame.get_information}\pysiglinewithargsret{\sphinxbfcode{get\_information}}{}{}
Recolecta la información que ha seleccionado e introducido el usuario,
también la organiza para que se pueda utilizar apropiadamente.
\begin{quote}\begin{description}
\item[{Returns}] \leavevmode

\begin{DUlineblock}{0em}
\item[] Un diccionario que contiene:
\item[] \textbf{Clase},
\item[] \textbf{Técnica},
\item[] \textbf{Parámetros.}
\end{DUlineblock}


\item[{Return type}] \leavevmode
Dictionary

\end{description}\end{quote}

\end{fulllineitems}

\index{restore\_settings() (AlgorithmFrame method)}

\begin{fulllineitems}
\phantomsection\label{View/Main/MOEA/AlgorithmFrame:View.Main.MOEA.AlgorithmFrame.AlgorithmFrame.restore_settings}\pysiglinewithargsret{\sphinxbfcode{restore\_settings}}{}{}
Asigna los valores por defecto tanto de las técnicas como de sus 
respectivos parámetros, también limpia aquéllos en donde se hayan 
insertado valores.

\end{fulllineitems}


\end{fulllineitems}



\subparagraph{SharingFunctionFrame (clase)}
\label{View/Main/MOEA/SharingFunctionFrame:module-View.Main.MOEA.SharingFunctionFrame}\label{View/Main/MOEA/SharingFunctionFrame:sharingfunctionframe-clase}\label{View/Main/MOEA/SharingFunctionFrame::doc}\index{View.Main.MOEA.SharingFunctionFrame (module)}\index{SharingFunctionFrame (class in View.Main.MOEA.SharingFunctionFrame)}

\begin{fulllineitems}
\phantomsection\label{View/Main/MOEA/SharingFunctionFrame:View.Main.MOEA.SharingFunctionFrame.SharingFunctionFrame}\pysiglinewithargsret{\sphinxstrong{class }\sphinxbfcode{SharingFunctionFrame}}{\emph{parent}, \emph{name}, \emph{features}}{}
Bases: \sphinxcode{Tkinter.Frame}

\begin{DUlineblock}{0em}
\item[] Esta clase proporciona una base gráfica para que el usuario pueda
seleccionar métodos con sus respectivos parámetros \textbf{(si es que tienen)}
referentes a Sharing Function.          
\item[] Una técnica de Sharing Function sirve para aplicar una selección más intensiva
de Individuos en caso de haber un ``empate'' entre éstos.
\end{DUlineblock}
\begin{quote}\begin{description}
\item[{Parameters}] \leavevmode\begin{itemize}
\item {} 
\textbf{\texttt{parent}} (\emph{\texttt{Tkinter.Frame}}) -- Frame padre al que pertenece.

\item {} 
\textbf{\texttt{name}} (\emph{\texttt{String}}) -- Identificador \textbf{(único)} que tendrá el Frame.

\item {} 
\textbf{\texttt{features}} (\emph{\texttt{Dictionary}}) -- Conjunto de técnicas con sus respectivos parámetros para que
se puedan cargar automáticamente en este Frame \textbf{(véase
Controller/XMLParser.py)}.

\end{itemize}

\item[{Returns}] \leavevmode
Tkinter.Frame

\item[{Return type}] \leavevmode
Instance

\end{description}\end{quote}
\index{\_SharingFunctionFrame\_\_create\_dynamic\_widgets() (SharingFunctionFrame method)}

\begin{fulllineitems}
\phantomsection\label{View/Main/MOEA/SharingFunctionFrame:View.Main.MOEA.SharingFunctionFrame.SharingFunctionFrame._SharingFunctionFrame__create_dynamic_widgets}\pysiglinewithargsret{\sphinxbfcode{\_SharingFunctionFrame\_\_create\_dynamic\_widgets}}{}{}~
\begin{notice}{note}{Note:}
Este método es privado.
\end{notice}

Inicializa los elementos dinámicos del Frame, esto es, de acuerdo al tipo 
que lleva cada parámetro se creará un widget diferente.

\end{fulllineitems}

\index{\_SharingFunctionFrame\_\_grid\_widgets() (SharingFunctionFrame method)}

\begin{fulllineitems}
\phantomsection\label{View/Main/MOEA/SharingFunctionFrame:View.Main.MOEA.SharingFunctionFrame.SharingFunctionFrame._SharingFunctionFrame__grid_widgets}\pysiglinewithargsret{\sphinxbfcode{\_SharingFunctionFrame\_\_grid\_widgets}}{}{}~
\begin{notice}{note}{Note:}
Este método es privado.
\end{notice}

Coloca elementos en el Frame, tanto estáticos como dinámicos.

\end{fulllineitems}

\index{\_SharingFunctionFrame\_\_update\_widgets() (SharingFunctionFrame method)}

\begin{fulllineitems}
\phantomsection\label{View/Main/MOEA/SharingFunctionFrame:View.Main.MOEA.SharingFunctionFrame.SharingFunctionFrame._SharingFunctionFrame__update_widgets}\pysiglinewithargsret{\sphinxbfcode{\_SharingFunctionFrame\_\_update\_widgets}}{\emph{event=None}}{}~
\begin{notice}{note}{Note:}
Este método es privado.
\end{notice}

\begin{DUlineblock}{0em}
\item[] Realiza solamente la actualización y colocación de elementos dinámicos 
en el Frame.
\item[] Si el parámetro event es distinto de \textbf{None}, significa que se lanzó 
un evento que provocará que se actualicen los parámetros de acuerdo con
la técnica seleccionada.
\end{DUlineblock}
\begin{quote}\begin{description}
\item[{Parameters}] \leavevmode
\textbf{\texttt{event}} (\emph{\texttt{String}}) -- Contiene el valor del elemento que ejecutó esta función.

\end{description}\end{quote}

\end{fulllineitems}

\index{get\_information() (SharingFunctionFrame method)}

\begin{fulllineitems}
\phantomsection\label{View/Main/MOEA/SharingFunctionFrame:View.Main.MOEA.SharingFunctionFrame.SharingFunctionFrame.get_information}\pysiglinewithargsret{\sphinxbfcode{get\_information}}{}{}
Recolecta la información que ha seleccionado e introducido el usuario,
también la organiza para que se pueda utilizar apropiadamente.
\begin{quote}\begin{description}
\item[{Returns}] \leavevmode

\begin{DUlineblock}{0em}
\item[] Un diccionario que contiene:
\item[] \textbf{Clase},
\item[] \textbf{Técnica},
\item[] \textbf{Parámetros.}
\end{DUlineblock}


\item[{Return type}] \leavevmode
Dictionary

\end{description}\end{quote}

\end{fulllineitems}

\index{restore\_settings() (SharingFunctionFrame method)}

\begin{fulllineitems}
\phantomsection\label{View/Main/MOEA/SharingFunctionFrame:View.Main.MOEA.SharingFunctionFrame.SharingFunctionFrame.restore_settings}\pysiglinewithargsret{\sphinxbfcode{restore\_settings}}{}{}
Asigna los valores por defecto tanto de las técnicas como de sus 
respectivos parámetros, también limpia aquéllos en donde se hayan 
insertado valores.

\end{fulllineitems}


\end{fulllineitems}



\subsection{Additional (módulo)}
\label{View/Additional/Additional:additional-modulo}\label{View/Additional/Additional::doc}
Proporciona elementos gráficos que, aunque no tienen cabilda en la
ventana principal, sí contienen herramientas auxiliares de importancia.

El módulo consta de los siguientes elementos:


\subsubsection{GenerationSignalToplevel (clase)}
\label{View/Additional/GenerationSignal/GenerationSignal:generationsignaltoplevel-clase}\label{View/Additional/GenerationSignal/GenerationSignal::doc}\label{View/Additional/GenerationSignal/GenerationSignal:module-View.Additional.GenerationSignal.GenerationSignalToplevel}\index{View.Additional.GenerationSignal.GenerationSignalToplevel (module)}\index{GenerationSignalToplevel (class in View.Additional.GenerationSignal.GenerationSignalToplevel)}

\begin{fulllineitems}
\phantomsection\label{View/Additional/GenerationSignal/GenerationSignal:View.Additional.GenerationSignal.GenerationSignalToplevel.GenerationSignalToplevel}\pysiglinewithargsret{\sphinxstrong{class }\sphinxbfcode{GenerationSignalToplevel}}{\emph{parent}, \emph{path\_image\_logo}, \emph{execution\_task\_number}}{}
Bases: \sphinxcode{Tkinter.Toplevel}

\begin{DUlineblock}{0em}
\item[] Se trata de un Toplevel \textbf{(ventana independiente)} que muestra el progreso de las generaciones
al momento de ejecutar un Task.
\item[] Esta ventana aunque es creada y mostrada en los procesos de la capa View, será en \textbf{Model/MOEA}
en donde se utilice y actualice, ya que la idea es crear una ``señal'' que indique al usuario el progreso
del MOEA en ejecución para que se dé una idea del desempeño del algoritmo.
\end{DUlineblock}
\begin{quote}\begin{description}
\item[{Parameters}] \leavevmode\begin{itemize}
\item {} 
\textbf{\texttt{parent}} (\emph{\texttt{Tkinter.Frame}}) -- Frame padre al que pertenece.

\item {} 
\textbf{\texttt{path\_image\_logo}} -- La ruta al logotipo que se usa en esta ventana independiente.

\item {} 
\textbf{\texttt{execution\_task\_number}} -- Número que indica el actual Task en ejecución 
\textbf{(véase View/Additional/ResultsGrapher/ResultsGrapherToplevel.py)}.

\end{itemize}

\end{description}\end{quote}

;type path\_image\_logo: String
:type execution\_task\_number: Integer
:returns: Tkinter.Toplevel
:rtype: Instance
\index{\_GenerationSignalToplevel\_\_center() (GenerationSignalToplevel method)}

\begin{fulllineitems}
\phantomsection\label{View/Additional/GenerationSignal/GenerationSignal:View.Additional.GenerationSignal.GenerationSignalToplevel.GenerationSignalToplevel._GenerationSignalToplevel__center}\pysiglinewithargsret{\sphinxbfcode{\_GenerationSignalToplevel\_\_center}}{}{}~
\begin{notice}{note}{Note:}
Este método es privado.
\end{notice}

Centra la ventana independiente con respecto de la ventana principal.
En otras palabras, la ventana independiente será colocada en el centro de la 
ventana principal.

\end{fulllineitems}

\index{\_GenerationSignalToplevel\_\_do\_nothing() (GenerationSignalToplevel method)}

\begin{fulllineitems}
\phantomsection\label{View/Additional/GenerationSignal/GenerationSignal:View.Additional.GenerationSignal.GenerationSignalToplevel.GenerationSignalToplevel._GenerationSignalToplevel__do_nothing}\pysiglinewithargsret{\sphinxbfcode{\_GenerationSignalToplevel\_\_do\_nothing}}{}{}~
\begin{notice}{note}{Note:}
Este método es privado.
\end{notice}

Simplemente es una función ``dummy'' que no realiza nada. 
Es utilizada como sustituto de la función del ícono ``Cerrar'' y así evitar
que el usario intencionadamente intente ocluir la ventana del número de generaciones.

\end{fulllineitems}

\index{close() (GenerationSignalToplevel method)}

\begin{fulllineitems}
\phantomsection\label{View/Additional/GenerationSignal/GenerationSignal:View.Additional.GenerationSignal.GenerationSignalToplevel.GenerationSignalToplevel.close}\pysiglinewithargsret{\sphinxbfcode{close}}{}{}
Oculta y elimina toda referencia gráfica y lógica de la 
ventana independiente, indicando así que el número de generaciones ha alcanzado
su límite.

\end{fulllineitems}

\index{hide() (GenerationSignalToplevel method)}

\begin{fulllineitems}
\phantomsection\label{View/Additional/GenerationSignal/GenerationSignal:View.Additional.GenerationSignal.GenerationSignalToplevel.GenerationSignalToplevel.hide}\pysiglinewithargsret{\sphinxbfcode{hide}}{}{}
Oculta la ventana independiente de la pantalla pero no la elimina de los registros
gráficos.

\end{fulllineitems}

\index{show() (GenerationSignalToplevel method)}

\begin{fulllineitems}
\phantomsection\label{View/Additional/GenerationSignal/GenerationSignal:View.Additional.GenerationSignal.GenerationSignalToplevel.GenerationSignalToplevel.show}\pysiglinewithargsret{\sphinxbfcode{show}}{}{}
Reactiva la ventana independiente, realizando además durante esta ejecución
un par de consignas más para dar una experiencia de usuario suficiente y concisa.

\end{fulllineitems}

\index{update\_current\_generation() (GenerationSignalToplevel method)}

\begin{fulllineitems}
\phantomsection\label{View/Additional/GenerationSignal/GenerationSignal:View.Additional.GenerationSignal.GenerationSignalToplevel.GenerationSignalToplevel.update_current_generation}\pysiglinewithargsret{\sphinxbfcode{update\_current\_generation}}{\emph{current\_generation}}{}~
\begin{DUlineblock}{0em}
\item[] Actualiza la generación actual en la ventana independiente.
\item[] Típicamente esta función será usada en todos los algoritmos de la capa Model/MOEA, pues
es allí donde se designará el progreso del algoritmo que a su vez se verá 
reflejado en la capa de View.
\end{DUlineblock}
\begin{quote}\begin{description}
\item[{Parameters}] \leavevmode
\textbf{\texttt{current\_generation}} (\emph{\texttt{Integer}}) -- La generación que está corriendo actualmente en el MOEA.s

\item[{Returns}] \leavevmode
1 si se  ha alcanzado la generación límite, 0 en otro caso.

\item[{Return type}] \leavevmode
Integer

\end{description}\end{quote}

\end{fulllineitems}

\index{update\_number\_of\_generations() (GenerationSignalToplevel method)}

\begin{fulllineitems}
\phantomsection\label{View/Additional/GenerationSignal/GenerationSignal:View.Additional.GenerationSignal.GenerationSignalToplevel.GenerationSignalToplevel.update_number_of_generations}\pysiglinewithargsret{\sphinxbfcode{update\_number\_of\_generations}}{\emph{number\_of\_generations}}{}
Actualiza el número total de generaciones. Generalmente esta función
será llamada desde Controller/Controller.py ya que ahí es donde se decide
si las configuraciones iniciales son adecuadas para poder ejecutar el algoritmo.
\begin{quote}\begin{description}
\item[{Parameters}] \leavevmode
\textbf{\texttt{number\_of\_generations}} (\emph{\texttt{Integer.}}) -- El número de generaciones total que tendrá el MOEA.

\end{description}\end{quote}

\end{fulllineitems}


\end{fulllineitems}



\subsubsection{MenuInternalOption (módulo)}
\label{View/Additional/MenuInternalOption/MenuInternalOption:menuinternaloption-modulo}\label{View/Additional/MenuInternalOption/MenuInternalOption::doc}
Contiene elementos gráficos que permiten acceder a configuraciones internas del
programa y también a M.O.P.s \textbf{(Multi-Objective Problems)} previamente cargados
para hacer uso fácil de ellos.
\phantomsection\label{View/Additional/MenuInternalOption/MenuInternalOption:module-View.Additional.MenuInternalOption.MenuInternalOption}\index{View.Additional.MenuInternalOption.MenuInternalOption (module)}\index{MenuInternalOption (class in View.Additional.MenuInternalOption.MenuInternalOption)}

\begin{fulllineitems}
\phantomsection\label{View/Additional/MenuInternalOption/MenuInternalOption:View.Additional.MenuInternalOption.MenuInternalOption.MenuInternalOption}\pysiglinewithargsret{\sphinxstrong{class }\sphinxbfcode{MenuInternalOption}}{\emph{parent}, \emph{path\_image\_logo}, \emph{features}}{}
Bases: \sphinxcode{Tkinter.Menu}

\begin{DUlineblock}{0em}
\item[] Se crea el Menú de Opciones Internas o Menú Secundario.
\item[] Básicamente se trata de una serie de características que, aunque no
forman parte esencial del programa, sí ofrecen alternativas que pueden
facilitar la experiencia de usuario.
\item[] Este menú será atado al Frame Principal y desde allí el usuario podrá
tener acceso a las opciones que aquí se describen.
\end{DUlineblock}
\begin{quote}\begin{description}
\item[{Parameters}] \leavevmode\begin{itemize}
\item {} 
\textbf{\texttt{parent}} (\emph{\texttt{Tkinter.Frame}}) -- El Frame Padre al que pertenece esta implementación.

\item {} 
\textbf{\texttt{path\_image\_logo}} -- La ruta al logotipo que se usa en esta ventana independiente.

\item {} 
\textbf{\texttt{features}} -- Un diccionario con las características que deberá tener cada una
de las opciones listadas.

\end{itemize}

\end{description}\end{quote}

;type path\_image\_logo: String
:type features: Dictionary 
:returns: Tkinter.Menu
:rtype: Instance
\index{\_MenuInternalOption\_\_launch\_about\_toplevel() (MenuInternalOption method)}

\begin{fulllineitems}
\phantomsection\label{View/Additional/MenuInternalOption/MenuInternalOption:View.Additional.MenuInternalOption.MenuInternalOption.MenuInternalOption._MenuInternalOption__launch_about_toplevel}\pysiglinewithargsret{\sphinxbfcode{\_MenuInternalOption\_\_launch\_about\_toplevel}}{}{}~
\begin{notice}{note}{Note:}
Este método es privado.
\end{notice}

Abre la ventana independiente \textbf{(Toplevel)} About.
También verifica que se abra una y sólo una instancia de 
dicha ventana.

\end{fulllineitems}

\index{\_MenuInternalOption\_\_launch\_internal\_option\_toplevel() (MenuInternalOption method)}

\begin{fulllineitems}
\phantomsection\label{View/Additional/MenuInternalOption/MenuInternalOption:View.Additional.MenuInternalOption.MenuInternalOption.MenuInternalOption._MenuInternalOption__launch_internal_option_toplevel}\pysiglinewithargsret{\sphinxbfcode{\_MenuInternalOption\_\_launch\_internal\_option\_toplevel}}{}{}~
\begin{notice}{note}{Note:}
Este método es privado.
\end{notice}

Abre la ventana independiente \textbf{(Toplevel)} Internal Options
\textbf{(o simplemente Options)}.
También verifica que se abra una y sólo una instancia de 
dicha ventana.

\end{fulllineitems}

\index{about\_toplevel\_custom\_close() (MenuInternalOption method)}

\begin{fulllineitems}
\phantomsection\label{View/Additional/MenuInternalOption/MenuInternalOption:View.Additional.MenuInternalOption.MenuInternalOption.MenuInternalOption.about_toplevel_custom_close}\pysiglinewithargsret{\sphinxbfcode{about\_toplevel\_custom\_close}}{}{}
Indica que la única instancia que debe crearse
para la opción About está disponible.

\end{fulllineitems}

\index{internal\_option\_toplevel\_custom\_close() (MenuInternalOption method)}

\begin{fulllineitems}
\phantomsection\label{View/Additional/MenuInternalOption/MenuInternalOption:View.Additional.MenuInternalOption.MenuInternalOption.MenuInternalOption.internal_option_toplevel_custom_close}\pysiglinewithargsret{\sphinxbfcode{internal\_option\_toplevel\_custom\_close}}{}{}
Indica que la única instancia que debe crearse
para la opción Options está disponible.

\end{fulllineitems}


\end{fulllineitems}


El módulo consta de las siguientes características:


\paragraph{InternalOptionToplevel (clase)}
\label{View/Additional/MenuInternalOption/InternalOptionToplevel:internaloptiontoplevel-clase}\label{View/Additional/MenuInternalOption/InternalOptionToplevel::doc}\label{View/Additional/MenuInternalOption/InternalOptionToplevel:module-View.Additional.MenuInternalOption.InternalOptionToplevel}\index{View.Additional.MenuInternalOption.InternalOptionToplevel (module)}\index{InternalOptionToplevel (class in View.Additional.MenuInternalOption.InternalOptionToplevel)}

\begin{fulllineitems}
\phantomsection\label{View/Additional/MenuInternalOption/InternalOptionToplevel:View.Additional.MenuInternalOption.InternalOptionToplevel.InternalOptionToplevel}\pysiglinewithargsret{\sphinxstrong{class }\sphinxbfcode{InternalOptionToplevel}}{\emph{parent}, \emph{path\_image\_logo}, \emph{features}, \emph{custom\_function}}{}
Bases: \sphinxcode{Tkinter.Toplevel}

\begin{DUlineblock}{0em}
\item[] Contiene un Menú pequeño con pestañas que indican las
características internas del sistema a las que puede tener acceso el 
usuario.
\item[] En su mayoría se trata de características que muestran los métodos,
técnicas y sistemas auxiliares que garantizan un manejo más armonioso 
del programa y si así lo desea el usuario, modificarlos para ajustar
su desempeño.
\end{DUlineblock}
\begin{quote}\begin{description}
\item[{Parameters}] \leavevmode\begin{itemize}
\item {} 
\textbf{\texttt{parent}} (\emph{\texttt{Tkinter.Menu}}) -- El elemento Padre al que pertenece la actual
ventana independiente \textbf{(Toplevel)}.

\item {} 
\textbf{\texttt{path\_image\_logo}} -- La ruta al logotipo que se usa en esta ventana 
independiente.

\item {} 
\textbf{\texttt{features}} -- Un diccionario que contiene las características necesarias
que serán mostradas en esta ventana independiente.

\item {} 
\textbf{\texttt{custom\_function}} -- Una variable que contiene una función, la cual
redefinirá más apropiadamente el comportamiento de
la actual ventana principal con respecto de su Frame Padre.

\end{itemize}

\end{description}\end{quote}

;type path\_image\_logo: String
:type features: Dictionary
:type custom\_function: Instance
:returns: La ventana independiente que contiene la información
\begin{quote}

señalada.
\end{quote}
\begin{quote}\begin{description}
\item[{Return type}] \leavevmode
Tkinter.Toplevel

\end{description}\end{quote}
\index{\_InternalOptionToplevel\_\_center() (InternalOptionToplevel method)}

\begin{fulllineitems}
\phantomsection\label{View/Additional/MenuInternalOption/InternalOptionToplevel:View.Additional.MenuInternalOption.InternalOptionToplevel.InternalOptionToplevel._InternalOptionToplevel__center}\pysiglinewithargsret{\sphinxbfcode{\_InternalOptionToplevel\_\_center}}{}{}~
\begin{notice}{note}{Note:}
Este método es privado.
\end{notice}

Centra la ventana independiente con respecto de la ventana principal.
En otras palabras, la ventana independiente será colocada en el centro de la 
ventana principal.

\end{fulllineitems}

\index{close() (InternalOptionToplevel method)}

\begin{fulllineitems}
\phantomsection\label{View/Additional/MenuInternalOption/InternalOptionToplevel:View.Additional.MenuInternalOption.InternalOptionToplevel.InternalOptionToplevel.close}\pysiglinewithargsret{\sphinxbfcode{close}}{}{}~
\begin{notice}{note}{Note:}
Este método es privado.
\end{notice}

Cierra y elimina todo rastro de esta ventana independiente.

\end{fulllineitems}


\end{fulllineitems}



\paragraph{InternalOptionTab (módulo)}
\label{View/Additional/MenuInternalOption/InternalOptionTab/InternalOptionTab:internaloptiontab-modulo}\label{View/Additional/MenuInternalOption/InternalOptionTab/InternalOptionTab::doc}
Contiene las partes gráficas que conformarán cada una de las pestañas
concernientes al Toplevel \textbf{(ventana independiente)} de opciones internas
\textbf{(InternalOptionToplevel)}.

Consta de los siguientes elementos:


\subparagraph{MOPExampleFrame (clase)}
\label{View/Additional/MenuInternalOption/InternalOptionTab/MOPExampleFrame:module-View.Additional.MenuInternalOption.InternalOptionTab.MOPExampleFrame}\label{View/Additional/MenuInternalOption/InternalOptionTab/MOPExampleFrame::doc}\label{View/Additional/MenuInternalOption/InternalOptionTab/MOPExampleFrame:mopexampleframe-clase}\index{View.Additional.MenuInternalOption.InternalOptionTab.MOPExampleFrame (module)}\index{MOPExampleFrame (class in View.Additional.MenuInternalOption.InternalOptionTab.MOPExampleFrame)}

\begin{fulllineitems}
\phantomsection\label{View/Additional/MenuInternalOption/InternalOptionTab/MOPExampleFrame:View.Additional.MenuInternalOption.InternalOptionTab.MOPExampleFrame.MOPExampleFrame}\pysiglinewithargsret{\sphinxstrong{class }\sphinxbfcode{MOPExampleFrame}}{\emph{parent}, \emph{features}}{}
Bases: \sphinxcode{Tkinter.Frame}

\begin{DUlineblock}{0em}
\item[] Unifica dos elementos: Canvas y MOPFrame. La razón de esto es que, en promedio la
información mostrada por MOPFrame rebasará el tamaño de la ventana de la información 
final \textbf{(véase View/Additional/ResultsGrapher/ResultsGrapherTopLevel.py)}, es entonces 
que se deben agregar barras de desplazamiento para poder acceder al contenido que quedaría oculto.
\item[] Uno de los elementos en Tkinter más sencillos que cumplen con este cometido es un
Canvas. Luego entonces esa es la razón de tal fusión.
\end{DUlineblock}
\begin{quote}\begin{description}
\item[{Parameters}] \leavevmode\begin{itemize}
\item {} 
\textbf{\texttt{parent}} (\emph{\texttt{Tkinter.Toplevel}}) -- El elemento Padre al que pertenece el actual
Frame.

\item {} 
\textbf{\texttt{features}} (\emph{\texttt{Dictionary}}) -- Un diccionario que contiene las características necesarias
que serán mostradas en este Frame.

\end{itemize}

\item[{Returns}] \leavevmode
El Frame que contiene la información señalada.

\item[{Return type}] \leavevmode
Tkinter.Frame

\end{description}\end{quote}
\index{\_MOPExampleFrame\_\_update\_scrollbar() (MOPExampleFrame method)}

\begin{fulllineitems}
\phantomsection\label{View/Additional/MenuInternalOption/InternalOptionTab/MOPExampleFrame:View.Additional.MenuInternalOption.InternalOptionTab.MOPExampleFrame.MOPExampleFrame._MOPExampleFrame__update_scrollbar}\pysiglinewithargsret{\sphinxbfcode{\_MOPExampleFrame\_\_update\_scrollbar}}{\emph{event}}{}~
\begin{notice}{note}{Note:}
Este método es privado.
\end{notice}

Actualiza la barra de desplazamiento de acuerdo al número de elementos
existentes en el Frame, esto para poder hacer un recorrido apropiado de 
la barra.
\begin{quote}\begin{description}
\item[{Parameters}] \leavevmode
\textbf{\texttt{event}} (\emph{\texttt{String}}) -- Elemento que ejecutó esta función.

\end{description}\end{quote}

\end{fulllineitems}


\end{fulllineitems}


La clase actual tiene como base elsiguiente elemento:


\subparagraph{MOPFrame (clase)}
\label{View/Additional/MenuInternalOption/InternalOptionTab/MOPFrame:mopframe-clase}\label{View/Additional/MenuInternalOption/InternalOptionTab/MOPFrame:module-View.Additional.MenuInternalOption.InternalOptionTab.MOPFrame}\label{View/Additional/MenuInternalOption/InternalOptionTab/MOPFrame::doc}\index{View.Additional.MenuInternalOption.InternalOptionTab.MOPFrame (module)}\index{MOPFrame (class in View.Additional.MenuInternalOption.InternalOptionTab.MOPFrame)}

\begin{fulllineitems}
\phantomsection\label{View/Additional/MenuInternalOption/InternalOptionTab/MOPFrame:View.Additional.MenuInternalOption.InternalOptionTab.MOPFrame.MOPFrame}\pysiglinewithargsret{\sphinxstrong{class }\sphinxbfcode{MOPFrame}}{\emph{parent}, \emph{grandparent}, \emph{features}}{}
Bases: \sphinxcode{Tkinter.Frame}

Muestra la información relativa a los M.O.P.'s y
provee de métodos que facilitan la carga de éstos en la
Ventana Principal.
Un M.O.P. \textbf{(Multi Objective Problem)} es un conjunto de funciones
y variables bien definidas que ya han sido previamente estudiadas,
así como su comportamiento en conjunto; la idea es proporcionarle al
usuario un ambiente de carga fácil de datos para que pueda probar los
ejemplos ya tratados por muchos autores en los libros que se citarán
en el trabajo escrito.
\begin{quote}\begin{description}
\item[{Parameters}] \leavevmode\begin{itemize}
\item {} 
\textbf{\texttt{parent}} (\emph{\texttt{Tkinter.Frame}}) -- El elemento Padre al que pertenece el actual
Frame.

\item {} 
\textbf{\texttt{grandparent}} (\emph{\texttt{Tkinter.Toplevel}}) -- El elemento Padre del Padre al que pertenece el actual
Frame.

\item {} 
\textbf{\texttt{features}} (\emph{\texttt{Dictionary}}) -- Un diccionario que contiene las características necesarias
que serán mostradas en este Frame.

\end{itemize}

\item[{Returns}] \leavevmode
El Frame que contiene la información señalada.

\item[{Return type}] \leavevmode
Tkinter.Frame

\end{description}\end{quote}
\index{\_MOPFrame\_\_get\_mop\_example() (MOPFrame method)}

\begin{fulllineitems}
\phantomsection\label{View/Additional/MenuInternalOption/InternalOptionTab/MOPFrame:View.Additional.MenuInternalOption.InternalOptionTab.MOPFrame.MOPFrame._MOPFrame__get_mop_example}\pysiglinewithargsret{\sphinxbfcode{\_MOPFrame\_\_get\_mop\_example}}{\emph{event}}{}~
\begin{notice}{note}{Note:}
Este método es privado.
\end{notice}

Con base en la selección de M.O.P.
hecha por el usuario, se carga éste
en la Ventana Principal.
\begin{quote}\begin{description}
\item[{Parameters}] \leavevmode
\textbf{\texttt{event}} (\emph{\texttt{String}}) -- El evento del elemento gráfico que
activa esta función.

\end{description}\end{quote}

\end{fulllineitems}

\index{\_MOPFrame\_\_update\_current\_mop() (MOPFrame method)}

\begin{fulllineitems}
\phantomsection\label{View/Additional/MenuInternalOption/InternalOptionTab/MOPFrame:View.Additional.MenuInternalOption.InternalOptionTab.MOPFrame.MOPFrame._MOPFrame__update_current_mop}\pysiglinewithargsret{\sphinxbfcode{\_MOPFrame\_\_update\_current\_mop}}{\emph{event=None}}{}~
\begin{notice}{note}{Note:}
Este método es privado.
\end{notice}

Despliega la información relacionada con el
M.O.P. seleccionado.
\begin{quote}\begin{description}
\item[{Parameters}] \leavevmode
\textbf{\texttt{event}} (\emph{\texttt{String}}) -- El evento del elemento gráfico que
activa esta función.

\end{description}\end{quote}

\end{fulllineitems}


\end{fulllineitems}



\subparagraph{FeatureFrame (clase)}
\label{View/Additional/MenuInternalOption/InternalOptionTab/FeatureFrame:module-View.Additional.MenuInternalOption.InternalOptionTab.FeatureFrame}\label{View/Additional/MenuInternalOption/InternalOptionTab/FeatureFrame::doc}\label{View/Additional/MenuInternalOption/InternalOptionTab/FeatureFrame:featureframe-clase}\index{View.Additional.MenuInternalOption.InternalOptionTab.FeatureFrame (module)}\index{FeatureFrame (class in View.Additional.MenuInternalOption.InternalOptionTab.FeatureFrame)}

\begin{fulllineitems}
\phantomsection\label{View/Additional/MenuInternalOption/InternalOptionTab/FeatureFrame:View.Additional.MenuInternalOption.InternalOptionTab.FeatureFrame.FeatureFrame}\pysiglinewithargsret{\sphinxstrong{class }\sphinxbfcode{FeatureFrame}}{\emph{parent}, \emph{features}}{}
Bases: \sphinxcode{Tkinter.Frame}

\begin{DUlineblock}{0em}
\item[] Unifica dos elementos: Canvas y CharacteristicFrame. La razón de esto es que, en promedio la
información mostrada por CharacteristicFrame rebasará el tamaño de la ventana de la información 
final \textbf{(véase View/Additional/ResultsGrapher/ResultsGrapherTopLevel.py)}, es entonces 
que se deben agregar barras de desplazamiento para poder acceder al contenido que quedaría oculto.
\item[] Uno de los elementos en Tkinter más sencillos que cumplen con este cometido es un
Canvas. Luego entonces esa es la razón de tal fusión.
\end{DUlineblock}
\begin{quote}\begin{description}
\item[{Parameters}] \leavevmode\begin{itemize}
\item {} 
\textbf{\texttt{parent}} (\emph{\texttt{Tkinter.Toplevel}}) -- El elemento Padre al que pertenece el actual
Frame.

\item {} 
\textbf{\texttt{features}} (\emph{\texttt{Dictionary}}) -- Un diccionario que contiene las características necesarias
que serán mostradas en este Frame.

\end{itemize}

\item[{Returns}] \leavevmode
El Frame que contiene la información señalada.

\item[{Return type}] \leavevmode
Tkinter.Frame

\end{description}\end{quote}
\index{\_FeatureFrame\_\_update\_scrollbar() (FeatureFrame method)}

\begin{fulllineitems}
\phantomsection\label{View/Additional/MenuInternalOption/InternalOptionTab/FeatureFrame:View.Additional.MenuInternalOption.InternalOptionTab.FeatureFrame.FeatureFrame._FeatureFrame__update_scrollbar}\pysiglinewithargsret{\sphinxbfcode{\_FeatureFrame\_\_update\_scrollbar}}{\emph{event}}{}~
\begin{notice}{note}{Note:}
Este método es privado.
\end{notice}

Actualiza la barra de desplazamiento de acuerdo al número de elementos
existentes en el Frame, esto para poder hacer un recorrido apropiado de 
la barra.
\begin{quote}\begin{description}
\item[{Parameters}] \leavevmode
\textbf{\texttt{event}} (\emph{\texttt{String}}) -- Elemento que ejecutó esta función.

\end{description}\end{quote}

\end{fulllineitems}


\end{fulllineitems}


La clase actual se apoya del siguiente elemento:


\subparagraph{CharacteristicFrame (clase)}
\label{View/Additional/MenuInternalOption/InternalOptionTab/CharacteristicFrame:module-View.Additional.MenuInternalOption.InternalOptionTab.CharacteristicFrame}\label{View/Additional/MenuInternalOption/InternalOptionTab/CharacteristicFrame::doc}\label{View/Additional/MenuInternalOption/InternalOptionTab/CharacteristicFrame:characteristicframe-clase}\index{View.Additional.MenuInternalOption.InternalOptionTab.CharacteristicFrame (module)}\index{CharacteristicFrame (class in View.Additional.MenuInternalOption.InternalOptionTab.CharacteristicFrame)}

\begin{fulllineitems}
\phantomsection\label{View/Additional/MenuInternalOption/InternalOptionTab/CharacteristicFrame:View.Additional.MenuInternalOption.InternalOptionTab.CharacteristicFrame.CharacteristicFrame}\pysiglinewithargsret{\sphinxstrong{class }\sphinxbfcode{CharacteristicFrame}}{\emph{parent}, \emph{features}}{}
Bases: \sphinxcode{Tkinter.Frame}

\begin{DUlineblock}{0em}
\item[] Despliega información concerniente a todas las técnicas 
\textbf{(con sus respectivos parámetros)} disponibles para el usuario.
\item[] Se agrupan éstas en las mismas categorías que presenta el programa,
más en concreto, las secciones que conforman a la Ventana Principal 
\textbf{(véase View/MainWindow.py)}.
\item[] 
\item[] También señala someramente las instrucciones necesarias para que el
programa pueda reconocer cualquier técnica que desarrolle el usuario.
\end{DUlineblock}
\begin{quote}\begin{description}
\item[{Parameters}] \leavevmode\begin{itemize}
\item {} 
\textbf{\texttt{parent}} (\emph{\texttt{Tkinter.Toplevel}}) -- El elemento Padre al que pertenece el actual
Frame.

\item {} 
\textbf{\texttt{features}} (\emph{\texttt{Dictionary}}) -- Un diccionario que contiene las características necesarias
que serán mostradas en este Frame.

\end{itemize}

\item[{Returns}] \leavevmode
El Frame que contiene la información señalada.

\item[{Return type}] \leavevmode
Tkinter.Frame

\end{description}\end{quote}

\end{fulllineitems}



\subparagraph{PythonExpressionFrame (clase)}
\label{View/Additional/MenuInternalOption/InternalOptionTab/PythonExpressionFrame::doc}\label{View/Additional/MenuInternalOption/InternalOptionTab/PythonExpressionFrame:module-View.Additional.MenuInternalOption.InternalOptionTab.PythonExpressionFrame}\label{View/Additional/MenuInternalOption/InternalOptionTab/PythonExpressionFrame:pythonexpressionframe-clase}\index{View.Additional.MenuInternalOption.InternalOptionTab.PythonExpressionFrame (module)}\index{PythonExpressionFrame (class in View.Additional.MenuInternalOption.InternalOptionTab.PythonExpressionFrame)}

\begin{fulllineitems}
\phantomsection\label{View/Additional/MenuInternalOption/InternalOptionTab/PythonExpressionFrame:View.Additional.MenuInternalOption.InternalOptionTab.PythonExpressionFrame.PythonExpressionFrame}\pysiglinewithargsret{\sphinxstrong{class }\sphinxbfcode{PythonExpressionFrame}}{\emph{parent}, \emph{features}}{}
Bases: \sphinxcode{Tkinter.Frame}

\begin{DUlineblock}{0em}
\item[] Realiza la fusión de Canvas y ExpressionFrame, debido a que, cuando se agregan 
numerosas variables al ExpressionFrame, se debe insertar una barra de desplazamiento
para poder acceder a aquéllos que se encuentren hasta abajo. Dentro del ambiente
de Tkinter, el elemento más sencillo para lograr este efecto es un Canvas, por ello 
se anida el ExpressionFrame al Canvas.
\end{DUlineblock}
\begin{quote}\begin{description}
\item[{Parameters}] \leavevmode\begin{itemize}
\item {} 
\textbf{\texttt{parent}} (\emph{\texttt{Tkinter.Frame}}) -- Frame padre al que pertenece.

\item {} 
\textbf{\texttt{features}} (\emph{\texttt{Dictionary}}) -- Conjunto de técnicas con sus respectivos parámetros para que
se puedan cargar automáticamente en este Frame \textbf{(véase
Controller/XML/PythonExpressions.xml)}.

\end{itemize}

\item[{Returns}] \leavevmode
Tkinter.Frame

\item[{Return type}] \leavevmode
Instance

\end{description}\end{quote}
\index{\_PythonExpressionFrame\_\_activate\_scroll() (PythonExpressionFrame method)}

\begin{fulllineitems}
\phantomsection\label{View/Additional/MenuInternalOption/InternalOptionTab/PythonExpressionFrame:View.Additional.MenuInternalOption.InternalOptionTab.PythonExpressionFrame.PythonExpressionFrame._PythonExpressionFrame__activate_scroll}\pysiglinewithargsret{\sphinxbfcode{\_PythonExpressionFrame\_\_activate\_scroll}}{\emph{event=None}}{}~
\begin{notice}{note}{Note:}
Este método es privado.
\end{notice}

Actualiza la barra de desplazamiento y con base en esta acción
la activa o desactiva.
\begin{quote}\begin{description}
\item[{Parameters}] \leavevmode
\textbf{\texttt{event}} (\emph{\texttt{String}}) -- Elemento que ejecutó esta función.

\end{description}\end{quote}

\end{fulllineitems}

\index{\_PythonExpressionFrame\_\_update\_scrollbar() (PythonExpressionFrame method)}

\begin{fulllineitems}
\phantomsection\label{View/Additional/MenuInternalOption/InternalOptionTab/PythonExpressionFrame:View.Additional.MenuInternalOption.InternalOptionTab.PythonExpressionFrame.PythonExpressionFrame._PythonExpressionFrame__update_scrollbar}\pysiglinewithargsret{\sphinxbfcode{\_PythonExpressionFrame\_\_update\_scrollbar}}{\emph{event=None}}{}~
\begin{notice}{note}{Note:}
Este método es privado.
\end{notice}

Actualiza la barra de desplazamiento de acuerdo al número de elementos
existentes en el Frame, esto para poder hacer un recorrido apropiado de 
la barra.
\begin{quote}\begin{description}
\item[{Parameters}] \leavevmode
\textbf{\texttt{event}} (\emph{\texttt{String}}) -- Elemento que ejecutó esta función.

\end{description}\end{quote}

\end{fulllineitems}


\end{fulllineitems}


La clase actual toma como referencia el siguiente elemento:


\subparagraph{ExpressionFrame (clase)}
\label{View/Additional/MenuInternalOption/InternalOptionTab/ExpressionFrame::doc}\label{View/Additional/MenuInternalOption/InternalOptionTab/ExpressionFrame:module-View.Additional.MenuInternalOption.InternalOptionTab.ExpressionFrame}\label{View/Additional/MenuInternalOption/InternalOptionTab/ExpressionFrame:expressionframe-clase}\index{View.Additional.MenuInternalOption.InternalOptionTab.ExpressionFrame (module)}\index{ExpressionFrame (class in View.Additional.MenuInternalOption.InternalOptionTab.ExpressionFrame)}

\begin{fulllineitems}
\phantomsection\label{View/Additional/MenuInternalOption/InternalOptionTab/ExpressionFrame:View.Additional.MenuInternalOption.InternalOptionTab.ExpressionFrame.ExpressionFrame}\pysiglinewithargsret{\sphinxstrong{class }\sphinxbfcode{ExpressionFrame}}{\emph{parent}, \emph{features}}{}
Bases: \sphinxcode{Tkinter.Frame}

\begin{DUlineblock}{0em}
\item[] Ofrece opciones simples para mostrar y añadir expresiones
de Python. 
\item[] Lo anterior ocurre ya que al momento de crear y evaluar funciones objetivo 
hay algunas palabras reservadas que no pueden ser usadas en Python 
directamente si no se hace un renombramiento apropiado.
\item[] Dicha información se encuentra en 
\item[] \textbf{Controller/XML/PythonExpressions.xml}
\end{DUlineblock}
\begin{quote}\begin{description}
\item[{Parameters}] \leavevmode\begin{itemize}
\item {} 
\textbf{\texttt{parent}} (\emph{\texttt{Tkinter.Toplevel}}) -- El elemento Padre al que pertenece el actual
Frame.

\item {} 
\textbf{\texttt{features}} (\emph{\texttt{Dictionary}}) -- Un diccionario que contiene las características necesarias
que serán mostradas en este Frame.

\end{itemize}

\item[{Returns}] \leavevmode
El Frame que contiene la información señalada.

\item[{Return type}] \leavevmode
Tkinter.Frame

\end{description}\end{quote}
\index{\_ExpressionFrame\_\_add\_expression() (ExpressionFrame method)}

\begin{fulllineitems}
\phantomsection\label{View/Additional/MenuInternalOption/InternalOptionTab/ExpressionFrame:View.Additional.MenuInternalOption.InternalOptionTab.ExpressionFrame.ExpressionFrame._ExpressionFrame__add_expression}\pysiglinewithargsret{\sphinxbfcode{\_ExpressionFrame\_\_add\_expression}}{\emph{event}}{}~
\begin{notice}{note}{Note:}
Este método es privado.
\end{notice}

Inserta una casilla que conforma una expresión
dentro del Frame.
\begin{quote}\begin{description}
\item[{Parameters}] \leavevmode
\textbf{\texttt{event}} (\emph{\texttt{String}}) -- Identificador del elemento gráfico que activó la función.

\end{description}\end{quote}

\end{fulllineitems}

\index{\_ExpressionFrame\_\_delete\_single\_expression() (ExpressionFrame method)}

\begin{fulllineitems}
\phantomsection\label{View/Additional/MenuInternalOption/InternalOptionTab/ExpressionFrame:View.Additional.MenuInternalOption.InternalOptionTab.ExpressionFrame.ExpressionFrame._ExpressionFrame__delete_single_expression}\pysiglinewithargsret{\sphinxbfcode{\_ExpressionFrame\_\_delete\_single\_expression}}{\emph{event}}{}~
\begin{notice}{note}{Note:}
Este método es privado.
\end{notice}

Elimina una expresión y todos los elementos gráficos que la acompañan.
También elimina todo rastro que se encuentre en las estructuras lógicas.
\begin{quote}\begin{description}
\item[{Parameters}] \leavevmode
\textbf{\texttt{event}} (\emph{\texttt{String}}) -- Identificador del elemento gráfico que activó la función.

\end{description}\end{quote}

\end{fulllineitems}

\index{\_ExpressionFrame\_\_get\_information() (ExpressionFrame method)}

\begin{fulllineitems}
\phantomsection\label{View/Additional/MenuInternalOption/InternalOptionTab/ExpressionFrame:View.Additional.MenuInternalOption.InternalOptionTab.ExpressionFrame.ExpressionFrame._ExpressionFrame__get_information}\pysiglinewithargsret{\sphinxbfcode{\_ExpressionFrame\_\_get\_information}}{}{}~
\begin{notice}{note}{Note:}
Este método es privado.
\end{notice}

Toma la información del Frame \textbf{(en específico de las casillas)} 
y regresa las expresiones con sus respectivos equivalentes en Python.
\begin{quote}\begin{description}
\item[{Returns}] \leavevmode
Una lista que contiene arreglos de dos elementos donde el primero es
la expresión normal mientras que el segundo es la expresión equivalente en
Python.

\item[{Return type}] \leavevmode
List

\end{description}\end{quote}

\end{fulllineitems}

\index{\_ExpressionFrame\_\_insert\_expression() (ExpressionFrame method)}

\begin{fulllineitems}
\phantomsection\label{View/Additional/MenuInternalOption/InternalOptionTab/ExpressionFrame:View.Additional.MenuInternalOption.InternalOptionTab.ExpressionFrame.ExpressionFrame._ExpressionFrame__insert_expression}\pysiglinewithargsret{\sphinxbfcode{\_ExpressionFrame\_\_insert\_expression}}{\emph{expression=None}}{}~
\begin{notice}{note}{Note:}
Este método es privado.
\end{notice}

\begin{DUlineblock}{0em}
\item[] Coloca en el Frame una colección de elementos:
\item[] {[}etiqueta para expresión normal, expresión normal, etiqueta para expresión de Pyrhon, expresión de Python, botón para eliminar{]}
\item[] Si el parámetro expression es \textbf{None}, se añade la casilla vacía, de lo contrario se 
agrega ésta con la información pertinente.
\end{DUlineblock}
\begin{quote}\begin{description}
\item[{Parameters}] \leavevmode
\textbf{\texttt{expression}} (\emph{\texttt{Array}}) -- Un arreglo con dos elementos, el primero contiene la expresión normal
mientras que el segundo maneja la información de la expresión equivalente
en Python.

\end{description}\end{quote}

\end{fulllineitems}

\index{\_ExpressionFrame\_\_load\_expressions() (ExpressionFrame method)}

\begin{fulllineitems}
\phantomsection\label{View/Additional/MenuInternalOption/InternalOptionTab/ExpressionFrame:View.Additional.MenuInternalOption.InternalOptionTab.ExpressionFrame.ExpressionFrame._ExpressionFrame__load_expressions}\pysiglinewithargsret{\sphinxbfcode{\_ExpressionFrame\_\_load\_expressions}}{}{}~
\begin{notice}{note}{Note:}
Este método es privado.
\end{notice}

\begin{DUlineblock}{0em}
\item[] Carga las expresiones a manera de contenido gráfico
en el Frame.
\item[] Dichas expresiones son tomadas del archivo 
\textbf{Controller/XML/PythonExpressions.xml.}
\end{DUlineblock}

\end{fulllineitems}

\index{\_ExpressionFrame\_\_save\_changes() (ExpressionFrame method)}

\begin{fulllineitems}
\phantomsection\label{View/Additional/MenuInternalOption/InternalOptionTab/ExpressionFrame:View.Additional.MenuInternalOption.InternalOptionTab.ExpressionFrame.ExpressionFrame._ExpressionFrame__save_changes}\pysiglinewithargsret{\sphinxbfcode{\_ExpressionFrame\_\_save\_changes}}{\emph{event}}{}~
\begin{notice}{note}{Note:}
Este método es privado.
\end{notice}

Toma la información existente en las casillas y procede a sobreescribir
el archivo \textbf{Controller/XML/PythonExpressions.xml} con la información
recién recabada.
\begin{quote}\begin{description}
\item[{Parameters}] \leavevmode
\textbf{\texttt{event}} (\emph{\texttt{String}}) -- Identificador del elemento gráfico que activó la función.

\end{description}\end{quote}

\end{fulllineitems}

\index{get\_current\_elements() (ExpressionFrame method)}

\begin{fulllineitems}
\phantomsection\label{View/Additional/MenuInternalOption/InternalOptionTab/ExpressionFrame:View.Additional.MenuInternalOption.InternalOptionTab.ExpressionFrame.ExpressionFrame.get_current_elements}\pysiglinewithargsret{\sphinxbfcode{get\_current\_elements}}{}{}
Regresa el número actual de casillas en el Frame.
\begin{quote}\begin{description}
\item[{Returns}] \leavevmode
Cantidad de elementos en la estructura rows, donde se guardan las casillas \textbf{(Entry's)}.

\item[{Return type}] \leavevmode
Int

\end{description}\end{quote}

\end{fulllineitems}


\end{fulllineitems}



\paragraph{AboutToplevel (clase)}
\label{View/Additional/MenuInternalOption/AboutToplevel:abouttoplevel-clase}\label{View/Additional/MenuInternalOption/AboutToplevel::doc}\label{View/Additional/MenuInternalOption/AboutToplevel:module-View.Additional.MenuInternalOption.AboutToplevel}\index{View.Additional.MenuInternalOption.AboutToplevel (module)}\index{AboutToplevel (class in View.Additional.MenuInternalOption.AboutToplevel)}

\begin{fulllineitems}
\phantomsection\label{View/Additional/MenuInternalOption/AboutToplevel:View.Additional.MenuInternalOption.AboutToplevel.AboutToplevel}\pysiglinewithargsret{\sphinxstrong{class }\sphinxbfcode{AboutToplevel}}{\emph{parent}, \emph{path\_image\_logo}, \emph{custom\_function}}{}
Bases: \sphinxcode{Tkinter.Toplevel}

Esta ventana independiente \textbf{(Toplevel)} proporciona
información básica del programa así como de sus 
desarrolladores.
\begin{quote}\begin{description}
\item[{Parameters}] \leavevmode\begin{itemize}
\item {} 
\textbf{\texttt{parent}} (\emph{\texttt{Tkinter.Menu}}) -- El elemento Padre al que pertenece la actual
ventana independiente \textbf{(Toplevel)}.

\item {} 
\textbf{\texttt{path\_image\_logo}} -- La ruta al logotipo que se usa en esta ventana independiente.

\item {} 
\textbf{\texttt{custom\_function}} -- Una variable que contiene una función, la cual
redefinirá más apropiadamente el comportamiento de
la actual ventana principal con respecto de su Frame Padre.

\end{itemize}

\end{description}\end{quote}

;type path\_image\_logo: String
:type custom\_function: Instance
:returns: Tkinter.Toplevel
:rtype: Instance
\index{\_AboutToplevel\_\_center() (AboutToplevel method)}

\begin{fulllineitems}
\phantomsection\label{View/Additional/MenuInternalOption/AboutToplevel:View.Additional.MenuInternalOption.AboutToplevel.AboutToplevel._AboutToplevel__center}\pysiglinewithargsret{\sphinxbfcode{\_AboutToplevel\_\_center}}{}{}~
\begin{notice}{note}{Note:}
Este método es privado.
\end{notice}

Centra la ventana independiente con respecto de la ventana principal.
En otras palabras, la ventana independiente será colocada en el centro de la 
ventana principal.

\end{fulllineitems}

\index{\_AboutToplevel\_\_close() (AboutToplevel method)}

\begin{fulllineitems}
\phantomsection\label{View/Additional/MenuInternalOption/AboutToplevel:View.Additional.MenuInternalOption.AboutToplevel.AboutToplevel._AboutToplevel__close}\pysiglinewithargsret{\sphinxbfcode{\_AboutToplevel\_\_close}}{\emph{custom\_function}}{}~
\begin{notice}{note}{Note:}
Este método es privado.
\end{notice}

Cierra y elimina todo rastro de esta ventana independiente.
\begin{quote}\begin{description}
\item[{Parameters}] \leavevmode
\textbf{\texttt{custom\_function}} (\emph{\texttt{Instance}}) -- Una variable que contiene una función que ha de 
ejecutarse dentro de este método.

\end{description}\end{quote}

\end{fulllineitems}


\end{fulllineitems}



\subsubsection{ResultsGrapher (módulo)}
\label{View/Additional/ResultsGrapher/ResultsGrapher:resultsgrapher-modulo}\label{View/Additional/ResultsGrapher/ResultsGrapher::doc}
Proporcional los elementos gráficos para poder presentar
las gráficas de los resultados que ha arrojado la ejecución
de algún M.O.E.A.

Consta de los siguientes elementos:


\paragraph{ResultsGrapherToplevel (clase)}
\label{View/Additional/ResultsGrapher/ResultsGrapherToplevel:module-View.Additional.ResultsGrapher.ResultsGrapherToplevel}\label{View/Additional/ResultsGrapher/ResultsGrapherToplevel:resultsgraphertoplevel-clase}\label{View/Additional/ResultsGrapher/ResultsGrapherToplevel::doc}\index{View.Additional.ResultsGrapher.ResultsGrapherToplevel (module)}\index{ResultsGrapherToplevel (class in View.Additional.ResultsGrapher.ResultsGrapherToplevel)}

\begin{fulllineitems}
\phantomsection\label{View/Additional/ResultsGrapher/ResultsGrapherToplevel:View.Additional.ResultsGrapher.ResultsGrapherToplevel.ResultsGrapherToplevel}\pysiglinewithargsret{\sphinxstrong{class }\sphinxbfcode{ResultsGrapherToplevel}}{\emph{parent}, \emph{path\_image\_logo}, \emph{execution\_task\_count}, \emph{main\_features}, \emph{gathered\_information}, \emph{final\_results}}{}
Bases: \sphinxcode{Tkinter.Toplevel}

\begin{DUlineblock}{0em}
\item[] Esta clase lanza una ventana independiente que muestra los resultados arrojados por una configuración
previa del usuario.
\item[] 
\item[] Primero que nada es menester mencionar que una ventana independiente es un Toplevel en Tkinter,
la cual es casi ajena a la Ventana Principal \textbf{(véase View/Main/MainWindow.py)}, pero si ésta última es
cerrada, se eliminarán también las ventanas independientes creadas.
\item[] 
\item[] Cada ventana independiente mostrará el número de Task, es decir, el orden en el que fue procesada la información
con respecto de otros Tasks.
\item[] 
\item[] Entiéndase por Task a una ejecución de algún algoritmo MOEA bajo un cierto conjunto de configuraciones iniciales.
Así, los Tasks serán mostrados en una ventana independiente. La numeración de los Tasks irá siempre en orden progresivo,
lo que significa que el número será reinicializado sólamente volviendo a ejecutar el programa principal.
\item[] De esta manera es posible tener varias ventanas independientes abiertas y en cuestiones más generales, es posible
ejecutar varios Tasks simultáneamente, ya que el programa es multi-threading en ese sentido.
\item[] 
\item[] Finalmente, la información será mostrada en dos pestañas: en una \textbf{(SummaryFrame)} se otorga un resumen de todas
las funciones objetivo, variables de decisión, MOEA usado y configuraciones adicionales en el Task.
\item[] En la otra \textbf{(GraphFrame)} se colocan todas las gráficas pertinentes producto de la ejecución del MOEA con las
funciones objectivo, variables de decisión y configuraciones ingresadas \textbf{(véase Model/Community/Community.py)}
\textbf{(véase View/Additional/ResultsGrapher/GraphFrame.py)}.
\item[] 
\item[] Si por cualquier circunstancia llega a haber una falla interna durante la ejecución del proceso, ninguna de las dos
pestañas será mostrada y en su lugar aparecerá una de error \textbf{(ErrorFrame)}, especificando además el tipo de error
y en qué parte de Model \textbf{(ó Modelo)} ocurrió.
\end{DUlineblock}
\begin{quote}\begin{description}
\item[{Parameters}] \leavevmode\begin{itemize}
\item {} 
\textbf{\texttt{parent}} (\emph{\texttt{Tkinter.Frame}}) -- Frame padre al que pertenece.

\item {} 
\textbf{\texttt{path\_image\_logo}} -- La ruta al logotipo que se usa en esta ventana independiente.

\item {} 
\textbf{\texttt{execution\_task\_count}} -- Número que indica el actual Task en ejecución.

\item {} 
\textbf{\texttt{main\_features}} -- Diccionario que contiene, entre otras cosas, los nombres de los
parámetros asociados a cada técnica.

\item {} 
\textbf{\texttt{gathered\_information}} -- Diccionario que contiene todas las configuraciones 
recabadas ingresadas por el usuario \textbf{(véase View/Main/MainWindow.py)}.

\item {} 
\textbf{\texttt{final\_results}} -- Diccionario que contiene la información procesada lista para graficar
\textbf{(véase View/Additional/ResultsGrapher/GraphFrame.py)}.

\end{itemize}

\end{description}\end{quote}

;type path\_image\_logo: String
:type execution\_task\_count: Integer
:type main\_features: Dictionary
:type gathered\_information: Dictionary
:type final\_results: Dictionary
:returns: Tkinter.Toplevel
:rtype: Instance
\index{\_ResultsGrapherToplevel\_\_center() (ResultsGrapherToplevel method)}

\begin{fulllineitems}
\phantomsection\label{View/Additional/ResultsGrapher/ResultsGrapherToplevel:View.Additional.ResultsGrapher.ResultsGrapherToplevel.ResultsGrapherToplevel._ResultsGrapherToplevel__center}\pysiglinewithargsret{\sphinxbfcode{\_ResultsGrapherToplevel\_\_center}}{}{}~
\begin{notice}{note}{Note:}
Este método es privado.
\end{notice}

Centra la ventana independiente con respecto de la ventana principal.
En otras palabras, la ventana independiente será colocada en el centro de la 
ventana principal.

\end{fulllineitems}

\index{\_ResultsGrapherToplevel\_\_create\_renamed\_settings() (ResultsGrapherToplevel method)}

\begin{fulllineitems}
\phantomsection\label{View/Additional/ResultsGrapher/ResultsGrapherToplevel:View.Additional.ResultsGrapher.ResultsGrapherToplevel.ResultsGrapherToplevel._ResultsGrapherToplevel__create_renamed_settings}\pysiglinewithargsret{\sphinxbfcode{\_ResultsGrapherToplevel\_\_create\_renamed\_settings}}{}{}~
\begin{notice}{note}{Note:}
Este método es privado.
\end{notice}

\begin{DUlineblock}{0em}
\item[] Tal como su nombre lo dice, renombra las funciones objetivo y variables de decisión
para posteriormente almacenarlas en una estructura por cada tipo.
\item[] Renombrar una función o variable de decisión es hacer un mapeo que consista en:
\item[] Elemento\_renombrado -\textgreater{} elemento original.
\item[] Para el caso de la función objetivo, el renombramiento se da anteponiendo la letra \textbf{F}
seguido de la posición en la que fue insertada originalmente por el usuario.
\item[] El caso es análogo para la variable de decisión, sólo que la letra es \textbf{V}.
\item[] La idea de renombrar las funciones y variables surge como alternativa al momento de graficar los datos
\textbf{(véase View/Additional/ResultsGrapher/GraphFrame.py)}, ya que el usuario puede ingresar
funciones muy largas o variables con identificadores muy complejos y esto en la parte gráfica 
se vería muy amontonado; por ello fue preferible mostrar la parte renombrada en la sección de 
GraphFrame y colocar la muestra original en el SummaryFrame.         
\end{DUlineblock}

\end{fulllineitems}


\end{fulllineitems}


La clase actual consta de los siguientes elementos:


\subparagraph{GraphFrame (clase)}
\label{View/Additional/ResultsGrapher/GraphFrame:graphframe-clase}\label{View/Additional/ResultsGrapher/GraphFrame::doc}\label{View/Additional/ResultsGrapher/GraphFrame:module-View.Additional.ResultsGrapher.GraphFrame}\index{View.Additional.ResultsGrapher.GraphFrame (module)}\index{GraphFrame (class in View.Additional.ResultsGrapher.GraphFrame)}

\begin{fulllineitems}
\phantomsection\label{View/Additional/ResultsGrapher/GraphFrame:View.Additional.ResultsGrapher.GraphFrame.GraphFrame}\pysiglinewithargsret{\sphinxstrong{class }\sphinxbfcode{GraphFrame}}{\emph{parent}, \emph{execution\_task\_count}, \emph{objective\_functions}, \emph{decision\_variables}, \emph{final\_results}}{}
Bases: \sphinxcode{Tkinter.Frame}

Proporciona un Frame que contiene gráficas alimentadas por los resultados obtenidos
al ejecutar algún MOEA, el cual ha sido refinado por las configuraciones
recabadas de la Ventana Principal \textbf{(véase Model/MOEA)}.
\begin{quote}\begin{description}
\item[{Parameters}] \leavevmode\begin{itemize}
\item {} 
\textbf{\texttt{parent}} (\emph{\texttt{Tkinter.Frame}}) -- Frame padre al que pertenece.

\item {} 
\textbf{\texttt{execution\_task\_count}} (\emph{\texttt{Integer}}) -- Número que indica el actual Task en ejecución.

\item {} 
\textbf{\texttt{objective\_functions}} (\emph{\texttt{List}}) -- Lista que contiene las funciones objetivo renombradas.

\item {} 
\textbf{\texttt{decision\_variables}} (\emph{\texttt{List}}) -- Lista que contiene las variables de decisión renombradas.

\item {} 
\textbf{\texttt{final\_results}} (\emph{\texttt{Dictionary}}) -- Diccionario que contiene la información para graficar. Se divide en dos categorías principales:
Frente de Pareto y Mejor Individuo por Generación.

\end{itemize}

\item[{Returns}] \leavevmode
Tkinter.Frame

\item[{Return type}] \leavevmode
Instance

\end{description}\end{quote}
\index{\_GraphFrame\_\_change\_canvas\_category() (GraphFrame method)}

\begin{fulllineitems}
\phantomsection\label{View/Additional/ResultsGrapher/GraphFrame:View.Additional.ResultsGrapher.GraphFrame.GraphFrame._GraphFrame__change_canvas_category}\pysiglinewithargsret{\sphinxbfcode{\_GraphFrame\_\_change\_canvas\_category}}{}{}~
\begin{notice}{note}{Note:}
Este método es privado.
\end{notice}

Realiza el cambio de Canvas de la categoría de funciones objetivo 
a la de variables y decision y viceversa, tomando en cuenta factores como por
ejemplo si alguna de las dos categorías tiene un OptionMenu asociado \textbf{(para entonces
colocarlo apropiadamente)} e identificando siempre el último Canvas seleccionado de 
la categoría anterior para que cuando sea oportuno se vuelva a colocar.

\end{fulllineitems}

\index{\_GraphFrame\_\_change\_inner\_canvas() (GraphFrame method)}

\begin{fulllineitems}
\phantomsection\label{View/Additional/ResultsGrapher/GraphFrame:View.Additional.ResultsGrapher.GraphFrame.GraphFrame._GraphFrame__change_inner_canvas}\pysiglinewithargsret{\sphinxbfcode{\_GraphFrame\_\_change\_inner\_canvas}}{\emph{event}}{}~
\begin{notice}{note}{Note:}
Este método es privado.
\end{notice}

Realiza el cambio de Canvas dentro de una misma categoría, esto
en caso en que los datos hayan arrojado más de una gráfica. El cambio
se hace con ayuda de su OptionMenu asociado.
\begin{quote}\begin{description}
\item[{Parameters}] \leavevmode
\textbf{\texttt{event}} (\emph{\texttt{String}}) -- Elemento que ejecutó esta función.

\end{description}\end{quote}

\end{fulllineitems}

\index{\_GraphFrame\_\_create\_2d\_canvas() (GraphFrame method)}

\begin{fulllineitems}
\phantomsection\label{View/Additional/ResultsGrapher/GraphFrame:View.Additional.ResultsGrapher.GraphFrame.GraphFrame._GraphFrame__create_2d_canvas}\pysiglinewithargsret{\sphinxbfcode{\_GraphFrame\_\_create\_2d\_canvas}}{\emph{x\_label}, \emph{x\_index}, \emph{y\_label}, \emph{y\_index}, \emph{collection\_points}}{}~
\begin{notice}{note}{Note:}
Este método es privado.
\end{notice}

Crea una gráfica en 2 dimensiones que es envuelta en un Canvas.
\begin{quote}\begin{description}
\item[{Parameters}] \leavevmode\begin{itemize}
\item {} 
\textbf{\texttt{x\_label}} (\emph{\texttt{String}}) -- Nombre para el eje X de la gráfica.

\item {} 
\textbf{\texttt{x\_index}} (\emph{\texttt{Integer}}) -- Posición dentro de collection\_points para los datos del eje X.

\item {} 
\textbf{\texttt{y\_label}} (\emph{\texttt{String}}) -- Nombre para el eje Y de la gráfica.

\item {} 
\textbf{\texttt{y\_index}} (\emph{\texttt{Integer}}) -- Posición dentro de collection\_points para los datos del eje Y.

\item {} 
\textbf{\texttt{collection\_points}} (\emph{\texttt{Dictionary}}) -- Diccionario que contiene los puntos a graficar.

\end{itemize}

\item[{Returns}] \leavevmode
Canvas

\item[{Return type}] \leavevmode
matplotlib.backends.backend\_tkagg.FigureCanvasTkAgg

\end{description}\end{quote}

\end{fulllineitems}

\index{\_GraphFrame\_\_create\_3d\_canvas() (GraphFrame method)}

\begin{fulllineitems}
\phantomsection\label{View/Additional/ResultsGrapher/GraphFrame:View.Additional.ResultsGrapher.GraphFrame.GraphFrame._GraphFrame__create_3d_canvas}\pysiglinewithargsret{\sphinxbfcode{\_GraphFrame\_\_create\_3d\_canvas}}{\emph{x\_label}, \emph{x\_index}, \emph{y\_label}, \emph{y\_index}, \emph{z\_label}, \emph{z\_index}, \emph{collection\_points}}{}~
\begin{notice}{note}{Note:}
Este método es privado.
\end{notice}

Crea una gráfica en 3 dimensiones que es envuelta en un Canvas.
\begin{quote}\begin{description}
\item[{Parameters}] \leavevmode\begin{itemize}
\item {} 
\textbf{\texttt{x\_label}} (\emph{\texttt{String}}) -- Nombre para el eje X de la gráfica.

\item {} 
\textbf{\texttt{x\_index}} (\emph{\texttt{Integer}}) -- Posición dentro de collection\_points para los datos del eje X.

\item {} 
\textbf{\texttt{y\_label}} (\emph{\texttt{String}}) -- Nombre para el eje Y de la gráfica.

\item {} 
\textbf{\texttt{y\_index}} (\emph{\texttt{Integer}}) -- Posición dentro de collection\_points para los datos del eje Y.

\item {} 
\textbf{\texttt{z\_label}} (\emph{\texttt{String}}) -- Nombre para el eje Z de la gráfica.

\item {} 
\textbf{\texttt{z\_index}} (\emph{\texttt{Integer}}) -- Posición dentro de collection\_points para los datos del eje Z.

\item {} 
\textbf{\texttt{collection\_points}} (\emph{\texttt{Dictionary}}) -- Diccionario que contiene los puntos a graficar.

\end{itemize}

\item[{Returns}] \leavevmode
Canvas

\item[{Return type}] \leavevmode
matplotlib.backends.backend\_tkagg.FigureCanvasTkAgg

\end{description}\end{quote}

\end{fulllineitems}

\index{\_GraphFrame\_\_create\_decision\_variables\_canvas() (GraphFrame method)}

\begin{fulllineitems}
\phantomsection\label{View/Additional/ResultsGrapher/GraphFrame:View.Additional.ResultsGrapher.GraphFrame.GraphFrame._GraphFrame__create_decision_variables_canvas}\pysiglinewithargsret{\sphinxbfcode{\_GraphFrame\_\_create\_decision\_variables\_canvas}}{\emph{decision\_variables}, \emph{collection\_points}}{}~
\begin{notice}{note}{Note:}
Este método es privado
\end{notice}

Crea los Canvas para las variables de decisión.
\begin{quote}\begin{description}
\item[{Parameters}] \leavevmode\begin{itemize}
\item {} 
\textbf{\texttt{decision\_variables}} (\emph{\texttt{List}}) -- Lista que contiene las variables de decisión renombradas.

\item {} 
\textbf{\texttt{collection\_points}} (\emph{\texttt{Dictionary}}) -- Diccionario que contiene los valores de las funciones objetivo
de todos los Individuos en la Población final.

\end{itemize}

\end{description}\end{quote}

\end{fulllineitems}

\index{\_GraphFrame\_\_create\_objective\_functions\_canvas() (GraphFrame method)}

\begin{fulllineitems}
\phantomsection\label{View/Additional/ResultsGrapher/GraphFrame:View.Additional.ResultsGrapher.GraphFrame.GraphFrame._GraphFrame__create_objective_functions_canvas}\pysiglinewithargsret{\sphinxbfcode{\_GraphFrame\_\_create\_objective\_functions\_canvas}}{\emph{objective\_functions}, \emph{collection\_points}}{}~
\begin{notice}{note}{Note:}
Este método es privado.
\end{notice}

Crea los Canvas para las funciones objetivo.
\begin{quote}\begin{description}
\item[{Parameters}] \leavevmode\begin{itemize}
\item {} 
\textbf{\texttt{objective\_functions}} (\emph{\texttt{List}}) -- Lista que contiene las funciones objetivo renombradas.

\item {} 
\textbf{\texttt{collection\_points}} (\emph{\texttt{Dictionary}}) -- Diccionario que contiene los valores de las funciones objetivo
de todos los Individuos en la Población final.

\end{itemize}

\end{description}\end{quote}

\end{fulllineitems}


\end{fulllineitems}


La clase actual toma como referencia el siguiente elemento:


\subparagraph{CustomNavigationToolbar2TkAgg (clase)}
\label{View/Additional/ResultsGrapher/CustomNavigationToolbar2TkAgg::doc}\label{View/Additional/ResultsGrapher/CustomNavigationToolbar2TkAgg:module-View.Additional.ResultsGrapher.CustomNavigationToolbar2TkAgg}\label{View/Additional/ResultsGrapher/CustomNavigationToolbar2TkAgg:customnavigationtoolbar2tkagg-clase}\index{View.Additional.ResultsGrapher.CustomNavigationToolbar2TkAgg (module)}\index{CustomNavigationToolbar2TkAgg (class in View.Additional.ResultsGrapher.CustomNavigationToolbar2TkAgg)}

\begin{fulllineitems}
\phantomsection\label{View/Additional/ResultsGrapher/CustomNavigationToolbar2TkAgg:View.Additional.ResultsGrapher.CustomNavigationToolbar2TkAgg.CustomNavigationToolbar2TkAgg}\pysiglinewithargsret{\sphinxstrong{class }\sphinxbfcode{CustomNavigationToolbar2TkAgg}}{\emph{canvas}, \emph{window}, \emph{parent\_frame}, \emph{execution\_task\_count}, \emph{image\_text}}{}
Bases: \sphinxcode{matplotlib.backends.backend\_tkagg.NavigationToolbar2TkAgg}

\begin{DUlineblock}{0em}
\item[] Proporciona una Barra de Navegación \textbf{(ó NavigationToolbar)} que se anexa a cada
una de las gráficas con el fin de facilitar la exploración y almacenamiento de los
datos obtenidos.
\item[] Por defecto la barra de navegación original se encuentra obsoleta a las necesidades
inherentes a este proyecto, por ello es que se crea una barra personalizada que
responde a requerimientos tales como la obtención apropiada de imágenes relativas
a las gráficas así como su correcto funcionamiento sin importar el sistema operativo
empleado.
\end{DUlineblock}
\begin{quote}\begin{description}
\item[{Parameters}] \leavevmode\begin{itemize}
\item {} 
\textbf{\texttt{canvas}} (\emph{\texttt{matplotlib.backends.backend\_tkagg.FigureCanvasTkAgg}}) -- La estructura que contiene tanto a la gráfica como a la Barra de Navegación.

\item {} 
\textbf{\texttt{window}} (\emph{\texttt{Tkinter.Frame}}) -- El Frame que contiene a canvas.

\item {} 
\textbf{\texttt{parent\_frame}} (\emph{\texttt{Tkinter.Frame}}) -- El Frame que contiene a window, en este caso ResultsGrapherToplevel.py.

\item {} 
\textbf{\texttt{execution\_task\_count}} (\emph{\texttt{Integer}}) -- Un identificador que precisa el número de tarea \textbf{(Task)} en ejecución
\textbf{(véase View/Additional/ResultsGrapher/ResultsGrapherToplevel.py)}.

\item {} 
\textbf{\texttt{image\_text}} (\emph{\texttt{String}}) -- El nombre que tendrán por defecto las imágenes resultantes al guardarse en el equipo
de cómputo.

\end{itemize}

\item[{Returns}] \leavevmode
matplotlib.backends.backend\_tkagg.NavigationToolbar2TkAgg

\item[{Rype}] \leavevmode
Instance

\end{description}\end{quote}
\index{save\_figure() (CustomNavigationToolbar2TkAgg method)}

\begin{fulllineitems}
\phantomsection\label{View/Additional/ResultsGrapher/CustomNavigationToolbar2TkAgg:View.Additional.ResultsGrapher.CustomNavigationToolbar2TkAgg.CustomNavigationToolbar2TkAgg.save_figure}\pysiglinewithargsret{\sphinxbfcode{save\_figure}}{\emph{*args}}{}~
\begin{notice}{note}{Note:}
Este método sobreescribe al original.
\end{notice}

\begin{DUlineblock}{0em}
\item[] Arroja una ventana emergente modificada para guardar archivos, en este caso
las gráficas.
\item[] Las modificaciones con respecto de la función original consisten en agregar
un título para tener conocimiento de las imágenes del Task que se van a guardar.
\item[] Además se modifica el comportamiento de la ventana para adherirlo a la ventana
del Task y no a la Ventana Principal.
\end{DUlineblock}
\begin{quote}\begin{description}
\item[{Parameters}] \leavevmode
\textbf{\texttt{args}} (\emph{\texttt{Tuple}}) -- Un listado con parámetros que aunque no se ocupan en el método
se coloca porque así lo estructuraron los desarrolladores originales
de la biblioteca.

\end{description}\end{quote}

\end{fulllineitems}


\end{fulllineitems}



\subparagraph{SummaryFrame (clase)}
\label{View/Additional/ResultsGrapher/SummaryFrame:summaryframe-clase}\label{View/Additional/ResultsGrapher/SummaryFrame:module-View.Additional.ResultsGrapher.SummaryFrame}\label{View/Additional/ResultsGrapher/SummaryFrame::doc}\index{View.Additional.ResultsGrapher.SummaryFrame (module)}\index{SummaryFrame (class in View.Additional.ResultsGrapher.SummaryFrame)}

\begin{fulllineitems}
\phantomsection\label{View/Additional/ResultsGrapher/SummaryFrame:View.Additional.ResultsGrapher.SummaryFrame.SummaryFrame}\pysiglinewithargsret{\sphinxstrong{class }\sphinxbfcode{SummaryFrame}}{\emph{parent}, \emph{renamed\_objective\_functions}, \emph{renamed\_decision\_variables}, \emph{main\_features}, \emph{gathered\_information}}{}
Bases: \sphinxcode{Tkinter.Frame}

\begin{DUlineblock}{0em}
\item[] Unifica dos elementos: Canvas y ContentFrame. La razón de esto es que, en promedio la
información mostrada por ContentFrame rebasará el tamaño de la ventana de la información 
final \textbf{(véase View/Additional/ResultsGrapher/ResultsGrapherTopLevel.py)}, es entonces 
que se deben agregar barras de desplazamiento para poder acceder al contenido que quedaría oculto.
\item[] Uno de los elementos en Tkinter más sencillos que cumplen con este cometido es un
Canvas. Luego entonces esa es la razón de tal fusión.
\end{DUlineblock}
\begin{quote}\begin{description}
\item[{Parameters}] \leavevmode\begin{itemize}
\item {} 
\textbf{\texttt{parent}} (\emph{\texttt{Tkinter.Frame}}) -- Frame padre al que pertenece.

\item {} 
\textbf{\texttt{renamed\_objective\_functions}} -- Diccionario de funciones objetivo renombradas 
\textbf{(véase View/Additional/ResultsGrapher/ResultsGrapherToplevel.py)}.

\item {} 
\textbf{\texttt{renamed\_decision\_variables}} (\emph{\texttt{Dictionary}}) -- Diccionario de variables de decisión renombradas  
\textbf{(véase View/Additional/ResultsGrapher/ResultsGrapherToplevel.py)}.

\item {} 
\textbf{\texttt{main\_features}} (\emph{\texttt{Dictionary}}) -- Diccionario que contiene, entre otras cosas, los nombres de los
parámetros asociados a cada técnica.

\item {} 
\textbf{\texttt{gathered\_information}} (\emph{\texttt{Dictionary}}) -- Diccionario que contiene todas las configuraciones 
recabadas ingresadas por el usuario \textbf{(véase View/Main/MainWindow.py)}.

\end{itemize}

\item[{Returns}] \leavevmode
Tkinter.Frame

\item[{Return type}] \leavevmode
Instance

\end{description}\end{quote}
\index{\_SummaryFrame\_\_update\_scrollbar() (SummaryFrame method)}

\begin{fulllineitems}
\phantomsection\label{View/Additional/ResultsGrapher/SummaryFrame:View.Additional.ResultsGrapher.SummaryFrame.SummaryFrame._SummaryFrame__update_scrollbar}\pysiglinewithargsret{\sphinxbfcode{\_SummaryFrame\_\_update\_scrollbar}}{\emph{event}}{}~
\begin{notice}{note}{Note:}
Este método es privado.
\end{notice}

Actualiza la barra de desplazamiento de acuerdo al número de elementos
existentes en el Frame, esto para poder hacer un recorrido apropiado de 
la barra.
\begin{quote}\begin{description}
\item[{Parameters}] \leavevmode
\textbf{\texttt{event}} (\emph{\texttt{String}}) -- Elemento que ejecutó esta función.

\end{description}\end{quote}

\end{fulllineitems}


\end{fulllineitems}


La clase actual toma como base el siguiente elemento:


\subparagraph{ContentFrame (clase)}
\label{View/Additional/ResultsGrapher/ContentFrame::doc}\label{View/Additional/ResultsGrapher/ContentFrame:contentframe-clase}\label{View/Additional/ResultsGrapher/ContentFrame:module-View.Additional.ResultsGrapher.ContentFrame}\index{View.Additional.ResultsGrapher.ContentFrame (module)}\index{ContentFrame (class in View.Additional.ResultsGrapher.ContentFrame)}

\begin{fulllineitems}
\phantomsection\label{View/Additional/ResultsGrapher/ContentFrame:View.Additional.ResultsGrapher.ContentFrame.ContentFrame}\pysiglinewithargsret{\sphinxstrong{class }\sphinxbfcode{ContentFrame}}{\emph{parent}, \emph{renamed\_objective\_functions}, \emph{renamed\_decision\_variables}, \emph{main\_features}, \emph{gathered\_information}}{}
Bases: \sphinxcode{Tkinter.Frame}

Recaba el contenido de todas las funciones objetivo, variables de decisión y demás parámetros que 
el usuario ingresó para poder ejecutar un Task determinado. Es entonces que plasma toda esta
información en un Frame para que el usuario pueda cotejar los datos ingresados con los resultados
obtenidos \textbf{(véase View/Additional/ResultsGrapher/GraphFrame.py)}.
\begin{quote}\begin{description}
\item[{Parameters}] \leavevmode\begin{itemize}
\item {} 
\textbf{\texttt{parent}} (\emph{\texttt{Tkinter.Frame}}) -- Frame padre al que pertenece.

\item {} 
\textbf{\texttt{renamed\_objective\_functions}} -- Diccionario de funciones objetivo renombradas 
\textbf{(véase View/Additional/ResultsGrapher/ResultsGrapherToplevel.py)}.

\item {} 
\textbf{\texttt{renamed\_decision\_variables}} (\emph{\texttt{Dictionary}}) -- Diccionario de variables de decisión renombradas  
\textbf{(véase View/Additional/ResultsGrapher/ResultsGrapherToplevel.py)}.

\item {} 
\textbf{\texttt{main\_features}} (\emph{\texttt{Dictionary}}) -- Diccionario que contiene, entre otras cosas, los nombres de los
parámetros asociados a cada técnica.

\item {} 
\textbf{\texttt{gathered\_information}} (\emph{\texttt{Dictionary}}) -- Diccionario que contiene todas las configuraciones 
recabadas ingresadas por el usuario \textbf{(véase View/Main/MainWindow.py)}.

\end{itemize}

\item[{Returns}] \leavevmode
Tkinter.Frame

\item[{Return type}] \leavevmode
Instance

\end{description}\end{quote}

\end{fulllineitems}



\subparagraph{ErrorFrame (clase)}
\label{View/Additional/ResultsGrapher/ErrorFrame:errorframe-clase}\label{View/Additional/ResultsGrapher/ErrorFrame::doc}\label{View/Additional/ResultsGrapher/ErrorFrame:module-View.Additional.ResultsGrapher.ErrorFrame}\index{View.Additional.ResultsGrapher.ErrorFrame (module)}\index{ErrorFrame (class in View.Additional.ResultsGrapher.ErrorFrame)}

\begin{fulllineitems}
\phantomsection\label{View/Additional/ResultsGrapher/ErrorFrame:View.Additional.ResultsGrapher.ErrorFrame.ErrorFrame}\pysiglinewithargsret{\sphinxstrong{class }\sphinxbfcode{ErrorFrame}}{\emph{parent}, \emph{final\_results}}{}
Bases: \sphinxcode{Tkinter.Frame}

\begin{DUlineblock}{0em}
\item[] Este Frame surge si durante el proceso interno en el Modelo \textbf{(véase Model/MOEA)}
se suscita algún error del cual el método no se pueda recuperar.
\item[] Entonces aquí se desplegará toda la información relativa a la falla, asímismo
funciona como medida de contingencia para darle una salida al programa y evitar
que se quede atorado.
\end{DUlineblock}
\begin{quote}\begin{description}
\item[{Parameters}] \leavevmode\begin{itemize}
\item {} 
\textbf{\texttt{parent}} (\emph{\texttt{Tkinter.Frame}}) -- Frame padre al que pertenece.

\item {} 
\textbf{\texttt{final\_results}} (\emph{\texttt{Dictionary}}) -- Diccionario que contiene en este caso las características
alusivas a la falla \textbf{(véase Model/MOEA)}.

\end{itemize}

\item[{Returns}] \leavevmode
Tkinter.Frame

\item[{Return type}] \leavevmode
Instance

\end{description}\end{quote}

\end{fulllineitems}



\section{Controller (Sección)}
\label{Controller/Controller:controller-seccion}\label{Controller/Controller::doc}
Su función principal es la de establecer medidas de comunicación entre la Vista \textbf{(ó View)}
y el Modelo \textbf{(ó Model)} de tal manera que el Controller \textbf{(ó Controlador)} recibe los datos
recabados en la Vista y los transfiere al Modelo para que se puedan llevar a cabo las operaciones
pertinentes y una vez concluidas dichas labores los resultados pasan por éste para llegar a la Vista y
desde ahí graficarse apropiadamente.

De manera secundaria el Controlador ofrece métodos de saneamiento de los datos recabados en la Vista,
con la finalidad de evitar al máximo disturbios indeseables en la sección Modelo y que éste opere con total
eficiencia, además de alimentar la Vista con las técnicas \textbf{(y sus respectivos parámetros)} disponibles
en la sección Modelo para así permitirle al usuario operar con éstas de manera expedita.

Dicho almacén se encuentra en la sección \textbf{Controller/XML}, donde se deduce que las técnicas y características
secundarias se encuentran plasmadas en archivos .xml

El proyecto contempla métodos para operar con dichos archivos y el usuario entonces sólo tendrá que preocuparse
por dar de alta la técnica pertinente en el archivo .xml adecuado \textbf{(además de implementarla en Modelo)} para que
ésta sea reconocida en la sección Vista y se pueda hacer uso de ella.

A continuación se muestran los componentes principales de la sección Controller:


\subsection{Controller (clase)}
\label{Controller/ControllerClass::doc}\label{Controller/ControllerClass:module-Controller.Controller}\label{Controller/ControllerClass:controller-clase}\index{Controller.Controller (module)}\index{Controller (class in Controller.Controller)}

\begin{fulllineitems}
\phantomsection\label{Controller/ControllerClass:Controller.Controller.Controller}\pysigline{\sphinxstrong{class }\sphinxbfcode{Controller}}~
\begin{DUlineblock}{0em}
\item[] Proporciona la infraestructura adecuada para poder comunicar la sección Vista
\textbf{(ó View)} con la sección Modelo \textbf{(ó Model)}, apoyándose de las clases XMLParser y Verifier.
\item[] 
\item[] El ciclo normal consiste en proporcionar a la capa Vista \textbf{(ó View)} la información
recabada en los archivos .xml con ayuda de la clase XMLParser con la finalidad de notificar
al usuario de todas las técnicas disponibles.
\item[] 
\item[] Una vez ejecutada la opción de iniciar un proceso genético por el usuario, se recaban los
datos ingresados por el usuario, los cuales pasan por un proceso de verificación y transformación
empleando para ello los métodos de la clase Verifier. 
\item[] 
\item[] En caso de haber al menos una falla en alguno de los procedimientos mencionados anteriormente
se regresa un mensaje de error, en otro caso se pasa la información respectiva a la
capa Model para que pueda operar con ésta.
\item[] 
\item[] En cualquiera de los dos casos anteriores se regresa la información resultante a la Vista.
\end{DUlineblock}
\begin{quote}\begin{description}
\item[{Returns}] \leavevmode
Controller.Controller

\item[{Return type}] \leavevmode
Instance

\end{description}\end{quote}
\index{execute\_procedure() (Controller method)}

\begin{fulllineitems}
\phantomsection\label{Controller/ControllerClass:Controller.Controller.Controller.execute_procedure}\pysiglinewithargsret{\sphinxbfcode{execute\_procedure}}{\emph{execution\_task\_count}, \emph{generations\_queue}, \emph{sanitized\_information}}{}
Realiza la ejecución de algún algoritmo M.O.E.A.
\textbf{(Multi-Objective Evolutionary Algorithm)} y se encarga
de obtener los resultados apropiadamente.
\begin{quote}\begin{description}
\item[{Parameters}] \leavevmode\begin{itemize}
\item {} 
\textbf{\texttt{execution\_task\_count}} (\emph{\texttt{Integer}}) -- Una característica numérica que identifica inequívocamente
a esta función que será ejecutada de las demás, ya que el objetivo
del proyecto es poder ejecutar varios de estos métodos de manera
concurrente \textbf{(véase View/Additional/ResultsGrapher/ResultsGrapherToplevel.py)}.

\item {} 
\textbf{\texttt{generations\_queue}} (\emph{\texttt{Instance}}) -- Una instancia a una cola \textbf{(Queue)}, la cual servirá para
escribir a esa estructura el número actual de generación
por el que cursa el algoritmo. Esta acción es para fines de 
concurrencia \textbf{(véase View/MainWindow.py)}.

\item {} 
\textbf{\texttt{sanitized\_information}} (\emph{\texttt{Dictionary}}) -- Los parámetros que ingresó el usuario
debidamente verificados y saneados.

\end{itemize}

\item[{Returns}] \leavevmode
Un diccionario con información de los resultados de haber
ejecutado el M.O.E.A. seleccionado por el usuario, la
estructura del mismo puede verse en \textbf{Model/Community/Community.py}.

\item[{Return type}] \leavevmode
Dictionary

\end{description}\end{quote}

\end{fulllineitems}

\index{load\_features() (Controller method)}

\begin{fulllineitems}
\phantomsection\label{Controller/ControllerClass:Controller.Controller.Controller.load_features}\pysiglinewithargsret{\sphinxbfcode{load\_features}}{}{}~
\begin{DUlineblock}{0em}
\item[] Regresa los datos correspondientes \textbf{(debidamente verificados)}
a las técnicas disponibles para el usuario, los cuales se mostrarán
en \textbf{View/MainWindow.py}.
\item[] \textbf{(véase View/Additional/MenuInternalOption/InternalOptionTab/FeatureFrame.py)}.
\item[] Esta técnica tiene como base los símiles que se encuentran en 
\textbf{Controller/XMLParser.py} y \textbf{Controller/Verifier.py}.
\end{DUlineblock}
\begin{quote}\begin{description}
\item[{Returns}] \leavevmode
Una estructura con los métodos disponibles para el usuario.

\item[{Return type}] \leavevmode
Dictionary

\end{description}\end{quote}

\end{fulllineitems}

\index{load\_mop\_examples() (Controller method)}

\begin{fulllineitems}
\phantomsection\label{Controller/ControllerClass:Controller.Controller.Controller.load_mop_examples}\pysiglinewithargsret{\sphinxbfcode{load\_mop\_examples}}{}{}~
\begin{DUlineblock}{0em}
\item[] Obtiene los datos correspondietes \textbf{(previamente verificados)}
a los M.O.P.'s \textbf{(Multi-Objective Problems)} que se utilizan en
en \textbf{View/MainWindow.py}
\item[] \textbf{(véase View/Additional/MenuInternalOption/InternalOptionTab/MOPExampleFrame.py)}.
\item[] Esta técnica tiene como base las análogas que se encuentran en 
\textbf{Controller/XMLParser.py} y \textbf{Controller/Verifier.py}.
\end{DUlineblock}
\begin{quote}\begin{description}
\item[{Returns}] \leavevmode
Una estructura con los M.O.P.'s disponibles para el usuario.

\item[{Return type}] \leavevmode
Dictionary

\end{description}\end{quote}

\end{fulllineitems}

\index{load\_python\_expressions() (Controller method)}

\begin{fulllineitems}
\phantomsection\label{Controller/ControllerClass:Controller.Controller.Controller.load_python_expressions}\pysiglinewithargsret{\sphinxbfcode{load\_python\_expressions}}{}{}~
\begin{DUlineblock}{0em}
\item[] Obtiene los datos correspondietes \textbf{(previamente verificados)}
a las expresiones de Python, las cuales se usan para evaluar 
funciones objetivo más eficientemente
\item[] \textbf{(véase View/Additional/MenuInternalOption/InternalOptionTab/PythonExpressionFrame.py)}.
\item[] Esta función se apoya de las homónimas localizadas en 
\textbf{Controller/XMLParser.py} y \textbf{Controller/Verifier.py}.
\end{DUlineblock}
\begin{quote}\begin{description}
\item[{Returns}] \leavevmode
Una estructura con las expresiones de Python disponibles.

\item[{Return type}] \leavevmode
Dictionary

\end{description}\end{quote}

\end{fulllineitems}

\index{sanitize\_settings() (Controller method)}

\begin{fulllineitems}
\phantomsection\label{Controller/ControllerClass:Controller.Controller.Controller.sanitize_settings}\pysiglinewithargsret{\sphinxbfcode{sanitize\_settings}}{\emph{general\_information}, \emph{features}}{}
Lleva a cabo la verificación y saneamiento de todos los datos
que ha ingresado el usuario en la sección View \textbf{(véase View/MainWindow)}.
\begin{quote}\begin{description}
\item[{Parameters}] \leavevmode\begin{itemize}
\item {} 
\textbf{\texttt{general\_information}} (\emph{\texttt{Dictionary}}) -- El conjunto de datos que el usuario ha ingresado o
seleccionado.

\item {} 
\textbf{\texttt{features}} (\emph{\texttt{Dictionary}}) -- Una colección de todos los elementos con sus características
disponibles para el usuario.

\end{itemize}

\item[{Returns}] \leavevmode


\item[{Return type}] \leavevmode
Dictionary

\end{description}\end{quote}

\end{fulllineitems}

\index{save\_python\_expressions() (Controller method)}

\begin{fulllineitems}
\phantomsection\label{Controller/ControllerClass:Controller.Controller.Controller.save_python_expressions}\pysiglinewithargsret{\sphinxbfcode{save\_python\_expressions}}{\emph{data}}{}
Inserta las expresiones de Python que ha ingresado
el usuario en el archivo .xml correspondiente.
\begin{quote}\begin{description}
\item[{Parameters}] \leavevmode
\textbf{\texttt{data}} (\emph{\texttt{List}}) -- Un conjunto de las expresiones que ha ingresado el usuario.
cada elemento es a su vez una lista con dos elementos, el primero
es la expresión original \textbf{(la que es comprensible por el 
usuario)}, mientras que la segunda es la expresión equivalente
en Python.

\item[{Returns}] \leavevmode
Mensaje ``OK'' si la inserción ha sido exitosa, mientras que en caso
de que haya habido un error entonces el mensaje es ``ERROR''.

\item[{Return type}] \leavevmode
String

\end{description}\end{quote}

\end{fulllineitems}


\end{fulllineitems}



\subsection{XMLParser (clase)}
\label{Controller/XMLParser:xmlparser-clase}\label{Controller/XMLParser:module-Controller.XMLParser}\label{Controller/XMLParser::doc}\index{Controller.XMLParser (module)}\index{XMLParser (class in Controller.XMLParser)}

\begin{fulllineitems}
\phantomsection\label{Controller/XMLParser:Controller.XMLParser.XMLParser}\pysigline{\sphinxstrong{class }\sphinxbfcode{XMLParser}}~
\begin{DUlineblock}{0em}
\item[] Permite leer y escribir a archivos .xml \textbf{(los que se localizan en Controller/XML)}, 
los cuales tienen almacenados: 
\end{DUlineblock}
\begin{itemize}
\item {} 
Los nombres de las técnicas con sus parámetros que se encuentran disponibles en la sección Model \textbf{(Features.xml)}.

\item {} 
La colección de palabras reservadas para poder emplear funciones y constantes auxiliares en las funciones objetivo \textbf{(PythonExpressions.xml)}.

\item {} 
El conjunto de M.O.P.'s \textbf{(Multi-Objective Problems, localizados en MOPExamples.xml)}.

\end{itemize}

\begin{DUlineblock}{0em}
\item[] 
\item[] Dichos archivos proporcionan la información necesaria a la interfaz gráfica \textbf{(véase View/MainWindow.py)}.
\end{DUlineblock}
\begin{quote}\begin{description}
\item[{Returns}] \leavevmode
Controller.XMLParser

\item[{Return type}] \leavevmode
Instance

\end{description}\end{quote}
\index{indent() (XMLParser method)}

\begin{fulllineitems}
\phantomsection\label{Controller/XMLParser:Controller.XMLParser.XMLParser.indent}\pysiglinewithargsret{\sphinxbfcode{indent}}{\emph{element}, \emph{level=0}}{}
Indenta \textbf{(coloca espacios)} apropiadamente
en un documento .xml para poder distinguir más
rápidamente los distintos niveles que existen en éste.
\begin{quote}\begin{description}
\item[{Parameters}] \leavevmode\begin{itemize}
\item {} 
\textbf{\texttt{element}} (\emph{\texttt{String}}) -- Una línea del archivo .xml

\item {} 
\textbf{\texttt{level}} (\emph{\texttt{Integer}}) -- El nivel en el que se está haciendo 
el proceso de identado.

\end{itemize}

\end{description}\end{quote}

\end{fulllineitems}

\index{load\_xml\_features() (XMLParser method)}

\begin{fulllineitems}
\phantomsection\label{Controller/XMLParser:Controller.XMLParser.XMLParser.load_xml_features}\pysiglinewithargsret{\sphinxbfcode{load\_xml\_features}}{\emph{features\_filename}}{}
Lee el archivo que contenga el listado de técnicas 
y sus parámetros disponibles \textbf{(véase Model)} y 
carga todos los elementos que se encuentran en éste.
\begin{quote}\begin{description}
\item[{Parameters}] \leavevmode
\textbf{\texttt{features\_filename}} (\emph{\texttt{String}}) -- Nombre del archivo en cuestión.

\item[{Returns}] \leavevmode
Un diccionario que contiene todos los elementos
del archivo.

\item[{Return type}] \leavevmode
Dictionary

\end{description}\end{quote}

\end{fulllineitems}

\index{load\_xml\_mop\_examples() (XMLParser method)}

\begin{fulllineitems}
\phantomsection\label{Controller/XMLParser:Controller.XMLParser.XMLParser.load_xml_mop_examples}\pysiglinewithargsret{\sphinxbfcode{load\_xml\_mop\_examples}}{\emph{features\_filename}}{}~
\begin{DUlineblock}{0em}
\item[] Lee el archivo que contenga el listado de M.O.P.'s \textbf{(Multi-Objective Problems)}
y carga todos los elementos que se encuentran en éste.
\item[] 
\item[] Un M.O.P es una mezcla de variables de decisión y funciones objetivo ya estudiadas,
se utilizan para reproducir su comportamiento y así garantizar, además de un correcto
funcionamiento del programa, una opción rápida para probar las técnicas que se ofrecen.
\end{DUlineblock}
\begin{quote}\begin{description}
\item[{Parameters}] \leavevmode
\textbf{\texttt{features\_filename}} (\emph{\texttt{String}}) -- Nombre del archivo en cuestión.

\item[{Returns}] \leavevmode
Un diccionario que contiene todos los elementos
del archivo.

\item[{Return type}] \leavevmode
Dictionary

\end{description}\end{quote}

\end{fulllineitems}

\index{load\_xml\_python\_expressions() (XMLParser method)}

\begin{fulllineitems}
\phantomsection\label{Controller/XMLParser:Controller.XMLParser.XMLParser.load_xml_python_expressions}\pysiglinewithargsret{\sphinxbfcode{load\_xml\_python\_expressions}}{\emph{features\_filename}}{}~
\begin{DUlineblock}{0em}
\item[] Lee el archivo que contenga el listado de expresiones en Python
y carga todos los elementos que se encuentran en éste.
\item[] 
\item[] La idea detrás de esto es que, al momento de crear y/o evaluar
funciones objetivo existen algunas palabras reservadas que no pueden 
ser usadas directamente como son las funciones trigonométricas, por eso
es que estas expresiones sirven como intermediarias entre el usuario y
el intérprete de Python. 
\item[] 
\item[] En ocasiones a este tipo de expresiones, no sólo en el ámbito actual
sino en general, se les conoce como azúcar sintáctica.
\end{DUlineblock}
\begin{quote}\begin{description}
\item[{Parameters}] \leavevmode
\textbf{\texttt{features\_filename}} (\emph{\texttt{String}}) -- Nombre del archivo en cuestión.

\item[{Returns}] \leavevmode
Un diccionario que contiene todos los elementos
del archivo.

\item[{Return type}] \leavevmode
Dictionary

\end{description}\end{quote}

\end{fulllineitems}

\index{write\_xml\_python\_expressions() (XMLParser method)}

\begin{fulllineitems}
\phantomsection\label{Controller/XMLParser:Controller.XMLParser.XMLParser.write_xml_python_expressions}\pysiglinewithargsret{\sphinxbfcode{write\_xml\_python\_expressions}}{\emph{features\_filename}, \emph{features}}{}~
\begin{DUlineblock}{0em}
\item[] Sobreescribe el archivo donde se encuentra el listado de expresiones
en Python.
\item[] El objetivo es que, una vez ejecutándose el programa y a través del menú
pertinente \textbf{(véase View/Additional/MenuInternalOption/InternalOptionTab/PythonExpressionFrame.py)},
el usuario pueda añadir o eliminar las expresiones de Python que desee.
\item[] 
\item[] En ocasiones a este tipo de expresiones, no sólo en el ámbito actual 
sino en general, se les conoce como azúcar sintáctica.
\end{DUlineblock}
\begin{quote}\begin{description}
\item[{Parameters}] \leavevmode\begin{itemize}
\item {} 
\textbf{\texttt{features\_filename}} (\emph{\texttt{String}}) -- Nombre del archivo en cuestión.

\item {} 
\textbf{\texttt{features}} (\emph{\texttt{List}}) -- La estructura que contiene las expresiones para ser guardadas
en el archivo .xml.

\end{itemize}

\end{description}\end{quote}

\end{fulllineitems}


\end{fulllineitems}



\subsection{Verifier (clase)}
\label{Controller/Verifier:verifier-clase}\label{Controller/Verifier::doc}\label{Controller/Verifier:module-Controller.Verifier}\index{Controller.Verifier (module)}\index{Verifier (class in Controller.Verifier)}

\begin{fulllineitems}
\phantomsection\label{Controller/Verifier:Controller.Verifier.Verifier}\pysigline{\sphinxstrong{class }\sphinxbfcode{Verifier}}~
\begin{DUlineblock}{0em}
\item[] Realiza principalmente la verificación y transformación adecuada de los datos 
que el usuario introduce en \textbf{View/MainWindow.py} para alimentar a los algoritmos 
que se encuentran en la sección Model \textbf{(más en concreto Model/MOEA)}.
\item[] 
\item[] En caso de haber algún error regresa los mensajes de error adecuados para que puedan
ser interpretados por la capa Vista y precisar al usuario el acontecimiento ocurrido.
\item[] 
\item[] Por otra parte si se ha llevado a cabo la verificación correctamente se obtiene la información
transformada apropiadamente.
\item[] 
\item[] De manera secundaria también ofrece métodos de verificación para la extracción
y colocación de datos en los archivos .xml \textbf{(véase XMLParser y el directorio Controller/XML)}.
\end{DUlineblock}
\begin{quote}\begin{description}
\item[{Returns}] \leavevmode
Controller.Verifier

\item[{Return type}] \leavevmode
Instance

\end{description}\end{quote}
\index{\_Verifier\_\_cast\_parameter() (Verifier method)}

\begin{fulllineitems}
\phantomsection\label{Controller/Verifier:Controller.Verifier.Verifier._Verifier__cast_parameter}\pysiglinewithargsret{\sphinxbfcode{\_Verifier\_\_cast\_parameter}}{\emph{parameter\_value}, \emph{parameter\_settings}}{}~
\begin{notice}{note}{Note:}
Este método es privado.
\end{notice}

Verifica un parámetro asociado a alguna técnica.
Primero asegura que el parámetro se pueda evaluar correctamente,
posteriormente convierte apropiadamente el tipo de dato pasando 
de String a Boolean, Integer ó Float según corresponda.
\begin{quote}\begin{description}
\item[{Parameters}] \leavevmode\begin{itemize}
\item {} 
\textbf{\texttt{parameter\_value}} (\emph{\texttt{Float}}) -- El valor actual del parámetro.

\item {} 
\textbf{\texttt{parameter\_settings}} (\emph{\texttt{Dictionary}}) -- Un diccionario que contiene el tipo del parámetro
\textbf{(bool, integer ó float)} y el rango que debe tomar
tanto inferior como superior.

\end{itemize}

\item[{Returns}] \leavevmode
El valor saneado del parámetro si no hay fallas, pero si se encuentra
algún desperfecto entonces se regresa un diccionario con la información
detallada del desperfecto.

\item[{Return type}] \leavevmode
(Boolean, Integer, Float)/Dictionary

\end{description}\end{quote}

\end{fulllineitems}

\index{\_Verifier\_\_verify\_instance() (Verifier method)}

\begin{fulllineitems}
\phantomsection\label{Controller/Verifier:Controller.Verifier.Verifier._Verifier__verify_instance}\pysiglinewithargsret{\sphinxbfcode{\_Verifier\_\_verify\_instance}}{\emph{name\_class}}{}~
\begin{notice}{note}{Note:}
Este método es privado.
\end{notice}

\begin{DUlineblock}{0em}
\item[] Devuelve una instancia del nombre de la clase que se le pase
como parámetro.
\item[] 
\item[] Esta funcionalidad es útil sobre todo para la sección Model ya que
uno de los objetivos es proporcionar al usuario de una infraestructura 
rápida con técnicas fácilmente intercambiables sin necesidad de estar
importando explícitamente cada una de éstas.
\item[] 
\item[] De esta forma con base en una instancia se puede ejecutar cualquier método
de manera dinámica.
\end{DUlineblock}
\begin{quote}\begin{description}
\item[{Parameters}] \leavevmode
\textbf{\texttt{name\_class}} (\emph{\texttt{String}}) -- el nombre de la clase \textbf{(con su ruta)} de la cual se
desea obtener una instancia.

\item[{Returns}] \leavevmode
Una instancia de la clase solicitada si el proceso es exitoso,
en otro caso se obtiene un diccionario con los detalles de la
falla.

\item[{Return type}] \leavevmode
Instance/Dictionary

\end{description}\end{quote}

\end{fulllineitems}

\index{get\_dynamic\_function() (Verifier method)}

\begin{fulllineitems}
\phantomsection\label{Controller/Verifier:Controller.Verifier.Verifier.get_dynamic_function}\pysiglinewithargsret{\sphinxbfcode{get\_dynamic\_function}}{\emph{complete\_function}}{}
Obtiene una instancia de una función en un String 
de la forma \textbf{biblioteca.función}.
Este método se usa para convertir las expresiones de Python
en instancias que serán utilizadas al momento de evaluar  
funciones objetivo \textbf{(véase View/Additional/MenuInternalOption/InternalOptionTab/PythonExpressionFrame.py,
Controller/XML/PythonExpressions.xml)}.
\begin{quote}\begin{description}
\item[{Parameters}] \leavevmode
\textbf{\texttt{complete\_function}} (\emph{\texttt{String}}) -- un String preferentemente de la forma
\textbf{biblioteca.función} \textbf{(el punto debe ir incluido)}.

\item[{Returns}] \leavevmode
Una instancia de la función asociada a la biblioteca.

\item[{Return type}] \leavevmode
Instance

\end{description}\end{quote}

\end{fulllineitems}

\index{sanitize\_decision\_variables() (Verifier method)}

\begin{fulllineitems}
\phantomsection\label{Controller/Verifier:Controller.Verifier.Verifier.sanitize_decision_variables}\pysiglinewithargsret{\sphinxbfcode{sanitize\_decision\_variables}}{\emph{vector\_variables}}{}~
\begin{DUlineblock}{0em}
\item[] Verifica el conjunto de elementos de la categoría ``Decision Variables''
\textbf{(véase View/Main/DecisionVariable/DecisionVariableFrame.py)}, los cuales
son precisamente las variables de decisión.
\item[] Primero se asegura que cada variable de decisión se pueda evaluar
correctamente, posteriormente convierte apropiadamente el tipo
de dato de sus respectivos rangos, pasando de String a Float.
\end{DUlineblock}
\begin{quote}\begin{description}
\item[{Parameters}] \leavevmode
\textbf{\texttt{vector\_variables}} (\emph{\texttt{Dictionary}}) -- El vector que contiene las variables de 
decisión con sus correspondientes rangos.

\item[{Returns}] \leavevmode
Un diccionario con las variables de decisión y sus 
rangos debidamente saneados.

\item[{Return type}] \leavevmode
Dictionary

\end{description}\end{quote}

\end{fulllineitems}

\index{sanitize\_genetic\_operators\_settings() (Verifier method)}

\begin{fulllineitems}
\phantomsection\label{Controller/Verifier:Controller.Verifier.Verifier.sanitize_genetic_operators_settings}\pysiglinewithargsret{\sphinxbfcode{sanitize\_genetic\_operators\_settings}}{\emph{genetic\_operators\_settings}, \emph{features}, \emph{vector\_variables}, \emph{number\_of\_decimals}}{}
Revisa la integridad y sanea los datos que ingresó el usuario 
concernientes a la sección ``Genetic Operators Settings''
\textbf{(véase View/Main/GeneticOperator/GeneticOperatorFrame.py)}.
\begin{quote}\begin{description}
\item[{Parameters}] \leavevmode\begin{itemize}
\item {} 
\textbf{\texttt{genetic\_operators\_settings}} (\emph{\texttt{Dictionary}}) -- El listado de técnicas y sus parámetros que el usuario
eligió en la sección correspondiente.

\item {} 
\textbf{\texttt{features}} (\emph{\texttt{Dictionary}}) -- El conjunto de las opciones disponibles 
para esta sección, así como sus características.

\item {} 
\textbf{\texttt{vector\_variables}} (\emph{\texttt{List}}) -- El vector de variables de decisión.

\item {} 
\textbf{\texttt{number\_of\_decimals}} (\emph{\texttt{Integer}}) -- El número de decimales que llevará cada solución en 
Population.

\end{itemize}

\item[{Returns}] \leavevmode
Un diccionario que, dependiendo de los resultados, puede contener
o información del error encontrado durante el procedimiento o 
todos los datos debidamente verificados y transformados.

\item[{Return type}] \leavevmode
Dictionary

\end{description}\end{quote}

\end{fulllineitems}

\index{sanitize\_moeas\_settings() (Verifier method)}

\begin{fulllineitems}
\phantomsection\label{Controller/Verifier:Controller.Verifier.Verifier.sanitize_moeas_settings}\pysiglinewithargsret{\sphinxbfcode{sanitize\_moeas\_settings}}{\emph{moeas\_settings}, \emph{features}}{}
Verifica integridad y lleva a cabo el saneamiento de 
los datos que ingresó el usuario concernientes a la 
sección ``MOEAs Settings'' \textbf{(véase View/Main/MOEA/MOEAFrame.py)}.
\begin{quote}\begin{description}
\item[{Parameters}] \leavevmode\begin{itemize}
\item {} 
\textbf{\texttt{moeas\_settings}} (\emph{\texttt{Dictionary}}) -- El listado de técnicas y sus parámetros que el usuario
eligió en la sección correspondiente.

\item {} 
\textbf{\texttt{features}} (\emph{\texttt{Dictionary}}) -- El conjunto de las opciones disponibles 
para esta sección, así como sus características.

\end{itemize}

\item[{Returns}] \leavevmode
Un diccionario que, dependiendo de los resultados, puede contener
o información del error encontrado durante el procedimiento o 
todos los datos debidamente verificados y transformados.

\item[{Return type}] \leavevmode
Dictionary

\end{description}\end{quote}

\end{fulllineitems}

\index{sanitize\_objective\_functions() (Verifier method)}

\begin{fulllineitems}
\phantomsection\label{Controller/Verifier:Controller.Verifier.Verifier.sanitize_objective_functions}\pysiglinewithargsret{\sphinxbfcode{sanitize\_objective\_functions}}{\emph{vector\_variables}, \emph{available\_expressions}, \emph{vector\_functions}}{}
Lleva a cabo el saneamiento de los elementos correspondientes a la categoría
``Objective Functions'' \textbf{(véase View/Main/ObjectiveFuncion/ObjectiveFunctionFrame.py)}, 
los cuales son de hecho sólo las funciones objetivo.
\begin{quote}\begin{description}
\item[{Parameters}] \leavevmode\begin{itemize}
\item {} 
\textbf{\texttt{vector\_variables}} (\emph{\texttt{Dictionary}}) -- El vector de variables de decisión que el usuario
ha ingresado.

\item {} 
\textbf{\texttt{available\_expressions}} (\emph{\texttt{Dictionary}}) -- Un listado con las expresiones de Python
disponibles \textbf{(véase Controller/XML/PythonExpressions.xml,
View/Additional/MenuInternalOption/InternalOptionTab/PythonExpressionFrame.py)}.

\item {} 
\textbf{\texttt{vector\_functions}} (\emph{\texttt{Dictionary}}) -- El vector de funciones objetivo ingresados por el usuario.

\end{itemize}

\item[{Returns}] \leavevmode
Si el proceso fue exitoso, se obtiene el mismo vector\_functions,
en otro caso se regresa un diccionario con información detallada
sobre el errror encontrado.

\item[{Return type}] \leavevmode
List/Dictionary

\end{description}\end{quote}

\end{fulllineitems}

\index{sanitize\_population\_settings() (Verifier method)}

\begin{fulllineitems}
\phantomsection\label{Controller/Verifier:Controller.Verifier.Verifier.sanitize_population_settings}\pysiglinewithargsret{\sphinxbfcode{sanitize\_population\_settings}}{\emph{population\_settings}, \emph{features}}{}
Verifica la consistencia y realiza el saneamiento de los datos
que ingresó el usuario concernientes a la sección ``Population Settings''
\textbf{(véase View/Main/Population/PopulationFrame.py)}.
\begin{quote}\begin{description}
\item[{Parameters}] \leavevmode\begin{itemize}
\item {} 
\textbf{\texttt{population\_settings}} (\emph{\texttt{Dictionary}}) -- El listado de técnicas y sus parámetros que el usuario
eligió en la sección correspondiente.

\item {} 
\textbf{\texttt{features}} (\emph{\texttt{Dictionary}}) -- El conjunto de las opciones disponibles 
para esta sección, así como sus características.

\end{itemize}

\item[{Returns}] \leavevmode
Un diccionario que, dependiendo de los resultados, puede contener
o información del error encontrado durante el procedimiento o 
todos los datos debidamente verificados y transformados.

\item[{Return type}] \leavevmode
Dictionary

\end{description}\end{quote}

\end{fulllineitems}

\index{sanitize\_techniques() (Verifier method)}

\begin{fulllineitems}
\phantomsection\label{Controller/Verifier:Controller.Verifier.Verifier.sanitize_techniques}\pysiglinewithargsret{\sphinxbfcode{sanitize\_techniques}}{\emph{general\_information}, \emph{features}}{}~
\begin{DUlineblock}{0em}
\item[] Realiza una verificación adicional concerniente al tipo de representación de todas las técnicas seleccionadas.
\item[] Lo anterior significa que, usando la Representación Cromosómica \textbf{(ó Chromosomal Representation,
véase Model/ChromosomalRepresentation, View/Main/Population/PopulationFrame.py)}, todas las técnicas
deben concordar con el mismo tipo de representación cromosómica que se haya seleccionado.
\item[] Para esta versión sólo están disponibles las representaciones binaria y de punto flotante.
\end{DUlineblock}
\begin{quote}\begin{description}
\item[{Parameters}] \leavevmode\begin{itemize}
\item {} 
\textbf{\texttt{general\_information}} (\emph{\texttt{Dictionary}}) -- El listado de características disponibles \textbf{(véase XMLParser.py)}.

\item {} 
\textbf{\texttt{features}} (\emph{\texttt{Dictionary}}) -- La colección de datos que seleccionó el usuario en la sección View.

\end{itemize}

\item[{Returns}] \leavevmode
Un diccionario el cual, si la verificación es exitosa, es el mismo general\_informacion,
si por el contrario falla, entonces es un diccionario que contiene detalles del error.

\item[{Return type}] \leavevmode
Dictionary

\end{description}\end{quote}

\end{fulllineitems}

\index{verify\_load\_xml\_features() (Verifier method)}

\begin{fulllineitems}
\phantomsection\label{Controller/Verifier:Controller.Verifier.Verifier.verify_load_xml_features}\pysiglinewithargsret{\sphinxbfcode{verify\_load\_xml\_features}}{\emph{data}}{}
Verifica que los datos obtenidos de las técnicas disponibles que alimentan 
a la Ventana Principal \textbf{(véase View/MainWindow.py)} no tengan defectos.
Este método se apoya de \textbf{load\_xml\_features} localizado en 
\textbf{Controller/XMLParser.py}.
\begin{quote}\begin{description}
\item[{Parameters}] \leavevmode
\textbf{\texttt{data}} (\emph{\texttt{Dictionary}}) -- Los datos que son leídos por el método \textbf{load\_xml\_features}
mencionado previamente.

\item[{Returns}] \leavevmode
Si los datos contienen algún error, un diccionario con las características
de la falla, en otro caso los datos mismos.

\item[{Return type}] \leavevmode
Dictionary

\end{description}\end{quote}

\end{fulllineitems}

\index{verify\_load\_xml\_mop\_examples() (Verifier method)}

\begin{fulllineitems}
\phantomsection\label{Controller/Verifier:Controller.Verifier.Verifier.verify_load_xml_mop_examples}\pysiglinewithargsret{\sphinxbfcode{verify\_load\_xml\_mop\_examples}}{\emph{data}}{}~
\begin{DUlineblock}{0em}
\item[] Revisa que los M.O.P.'s \textbf{(Multi-Objective Problems)}
que se muestran en \textbf{View/MainWindow.py} a través de
\item[] \textbf{View/Additional/MenuInternalOption/InternalOptionTab/MOPExampleFrame.py} 
estén libres de errores.
\item[] Este método se apoya de \textbf{load\_mop\_examples} localizado en 
\textbf{Controller/XMLParser.py}.
\end{DUlineblock}
\begin{quote}\begin{description}
\item[{Parameters}] \leavevmode
\textbf{\texttt{data}} (\emph{\texttt{Dictionary}}) -- Los datos que son leídos por el método \textbf{load\_mop\_examples}
mencionado previamente.

\item[{Returns}] \leavevmode
Si los datos contienen algún error, un diccionario con las características
de la falla, en otro caso los datos mismos.

\item[{Return type}] \leavevmode
Dictionary

\end{description}\end{quote}

\end{fulllineitems}

\index{verify\_load\_xml\_python\_expressions() (Verifier method)}

\begin{fulllineitems}
\phantomsection\label{Controller/Verifier:Controller.Verifier.Verifier.verify_load_xml_python_expressions}\pysiglinewithargsret{\sphinxbfcode{verify\_load\_xml\_python\_expressions}}{\emph{data}}{}~
\begin{DUlineblock}{0em}
\item[] Revisa que las expresiones de Python estén libres de errores.
\item[] Este método se apoya de \textbf{load\_python\_expressions} localizado en 
\textbf{Controller/XMLParser.py}.
\end{DUlineblock}
\begin{quote}\begin{description}
\item[{Parameters}] \leavevmode
\textbf{\texttt{data}} (\emph{\texttt{Dictionary}}) -- Los datos que son leídos por el método \textbf{load\_python\_expressions}
mencionado previamente.

\item[{Returns}] \leavevmode
Si los datos contienen algún error, un diccionario con las características
de la falla, en otro caso los datos mismos.

\item[{Return type}] \leavevmode
Dictionary

\end{description}\end{quote}

\end{fulllineitems}

\index{verify\_write\_xml\_python\_expressions() (Verifier method)}

\begin{fulllineitems}
\phantomsection\label{Controller/Verifier:Controller.Verifier.Verifier.verify_write_xml_python_expressions}\pysiglinewithargsret{\sphinxbfcode{verify\_write\_xml\_python\_expressions}}{\emph{data}}{}
Verifica la consistencia de los datos relativos a las
expresiones de Python antes de ser escritos en el archivo .xml
correspondiente.
\begin{quote}\begin{description}
\item[{Parameters}] \leavevmode
\textbf{\texttt{data}} (\emph{\texttt{List}}) -- El conjunto de expresiones que serán almacenadas.

\item[{Returns}] \leavevmode
Un mensaje ``OK'' si la verificación fue satisfactoria,
y ``ERROR'' en caso de aparecer alguna falla.

\item[{Return type}] \leavevmode
String

\end{description}\end{quote}

\end{fulllineitems}


\end{fulllineitems}



\renewcommand{\indexname}{Python Module Index}
\begin{theindex}
\def\bigletter#1{{\Large\sffamily#1}\nopagebreak\vspace{1mm}}
\bigletter{b}
\item {\texttt{Begin}}, \pageref{Begin:module-Begin}
\indexspace
\bigletter{c}
\item {\texttt{Controller.Controller}}, \pageref{Controller/ControllerClass:module-Controller.Controller}
\item {\texttt{Controller.Verifier}}, \pageref{Controller/Verifier:module-Controller.Verifier}
\item {\texttt{Controller.XMLParser}}, \pageref{Controller/XMLParser:module-Controller.XMLParser}
\indexspace
\bigletter{m}
\item {\texttt{Model.ChromosomalRepresentation.BinaryRepresentation}}, \pageref{Model/ChromosomalRepresentation/BinaryRepresentation:module-Model.ChromosomalRepresentation.BinaryRepresentation}
\item {\texttt{Model.ChromosomalRepresentation.FloatPointRepresentation}}, \pageref{Model/ChromosomalRepresentation/FloatPointRepresentation:module-Model.ChromosomalRepresentation.FloatPointRepresentation}
\item {\texttt{Model.Community.Community}}, \pageref{Model/Community/Community:module-Model.Community.Community}
\item {\texttt{Model.Community.Population.Individual.Individual}}, \pageref{Model/Community/Population/Individual/Individual:module-Model.Community.Population.Individual.Individual}
\item {\texttt{Model.Community.Population.Population}}, \pageref{Model/Community/Population/Population:module-Model.Community.Population.Population}
\item {\texttt{Model.Fitness.LinearRankingFitness}}, \pageref{Model/Fitness/LinearRankingFitness:module-Model.Fitness.LinearRankingFitness}
\item {\texttt{Model.Fitness.NonLinearRankingFitness}}, \pageref{Model/Fitness/NonLinearRankingFitness:module-Model.Fitness.NonLinearRankingFitness}
\item {\texttt{Model.Fitness.ProportionalFitness}}, \pageref{Model/Fitness/ProportionalFitness:module-Model.Fitness.ProportionalFitness}
\item {\texttt{Model.MOEA.MOGA}}, \pageref{Model/MOEA/MOGA:module-Model.MOEA.MOGA}
\item {\texttt{Model.MOEA.NSGAII}}, \pageref{Model/MOEA/NSGAII:module-Model.MOEA.NSGAII}
\item {\texttt{Model.MOEA.SPEAII}}, \pageref{Model/MOEA/SPEAII:module-Model.MOEA.SPEAII}
\item {\texttt{Model.MOEA.VEGA}}, \pageref{Model/MOEA/VEGA:module-Model.MOEA.VEGA}
\item {\texttt{Model.Operator.Crossover.NPointsCrossover}}, \pageref{Model/Operator/Crossover/NPointsCrossover:module-Model.Operator.Crossover.NPointsCrossover}
\item {\texttt{Model.Operator.Crossover.UniformCrossover}}, \pageref{Model/Operator/Crossover/UniformCrossover:module-Model.Operator.Crossover.UniformCrossover}
\item {\texttt{Model.Operator.Mutation.BinaryMutation}}, \pageref{Model/Operator/Mutation/BinaryMutation:module-Model.Operator.Mutation.BinaryMutation}
\item {\texttt{Model.Operator.Mutation.FloatPointMutation}}, \pageref{Model/Operator/Mutation/FloatPointMutation:module-Model.Operator.Mutation.FloatPointMutation}
\item {\texttt{Model.Operator.Selection.ProbabilisticTournament}}, \pageref{Model/Operator/Selection/ProbabilisticTournament:module-Model.Operator.Selection.ProbabilisticTournament}
\item {\texttt{Model.Operator.Selection.Roulette}}, \pageref{Model/Operator/Selection/Roulette:module-Model.Operator.Selection.Roulette}
\item {\texttt{Model.Operator.Selection.StochasticUniversalSampling}}, \pageref{Model/Operator/Selection/StochasticUniversalSampling:module-Model.Operator.Selection.StochasticUniversalSampling}
\item {\texttt{Model.SharingFunction.GenotypicSimilarity.HammingDistance}}, \pageref{Model/SharingFunction/GenotypicSimilarity/HammingDistance:module-Model.SharingFunction.GenotypicSimilarity.HammingDistance}
\item {\texttt{Model.SharingFunction.PhenotypicSimilarity.EuclideanDistance}}, \pageref{Model/SharingFunction/PhenotypicSimilarity/EuclideanDistance:module-Model.SharingFunction.PhenotypicSimilarity.EuclideanDistance}
\indexspace
\bigletter{v}
\item {\texttt{View.Additional.GenerationSignal.GenerationSignalToplevel}}, \pageref{View/Additional/GenerationSignal/GenerationSignal:module-View.Additional.GenerationSignal.GenerationSignalToplevel}
\item {\texttt{View.Additional.MenuInternalOption.AboutToplevel}}, \pageref{View/Additional/MenuInternalOption/AboutToplevel:module-View.Additional.MenuInternalOption.AboutToplevel}
\item {\texttt{View.Additional.MenuInternalOption.InternalOptionTab.CharacteristicFrame}}, \pageref{View/Additional/MenuInternalOption/InternalOptionTab/CharacteristicFrame:module-View.Additional.MenuInternalOption.InternalOptionTab.CharacteristicFrame}
\item {\texttt{View.Additional.MenuInternalOption.InternalOptionTab.ExpressionFrame}}, \pageref{View/Additional/MenuInternalOption/InternalOptionTab/ExpressionFrame:module-View.Additional.MenuInternalOption.InternalOptionTab.ExpressionFrame}
\item {\texttt{View.Additional.MenuInternalOption.InternalOptionTab.FeatureFrame}}, \pageref{View/Additional/MenuInternalOption/InternalOptionTab/FeatureFrame:module-View.Additional.MenuInternalOption.InternalOptionTab.FeatureFrame}
\item {\texttt{View.Additional.MenuInternalOption.InternalOptionTab.MOPExampleFrame}}, \pageref{View/Additional/MenuInternalOption/InternalOptionTab/MOPExampleFrame:module-View.Additional.MenuInternalOption.InternalOptionTab.MOPExampleFrame}
\item {\texttt{View.Additional.MenuInternalOption.InternalOptionTab.MOPFrame}}, \pageref{View/Additional/MenuInternalOption/InternalOptionTab/MOPFrame:module-View.Additional.MenuInternalOption.InternalOptionTab.MOPFrame}
\item {\texttt{View.Additional.MenuInternalOption.InternalOptionTab.PythonExpressionFrame}}, \pageref{View/Additional/MenuInternalOption/InternalOptionTab/PythonExpressionFrame:module-View.Additional.MenuInternalOption.InternalOptionTab.PythonExpressionFrame}
\item {\texttt{View.Additional.MenuInternalOption.InternalOptionToplevel}}, \pageref{View/Additional/MenuInternalOption/InternalOptionToplevel:module-View.Additional.MenuInternalOption.InternalOptionToplevel}
\item {\texttt{View.Additional.MenuInternalOption.MenuInternalOption}}, \pageref{View/Additional/MenuInternalOption/MenuInternalOption:module-View.Additional.MenuInternalOption.MenuInternalOption}
\item {\texttt{View.Additional.ResultsGrapher.ContentFrame}}, \pageref{View/Additional/ResultsGrapher/ContentFrame:module-View.Additional.ResultsGrapher.ContentFrame}
\item {\texttt{View.Additional.ResultsGrapher.CustomNavigationToolbar2TkAgg}}, \pageref{View/Additional/ResultsGrapher/CustomNavigationToolbar2TkAgg:module-View.Additional.ResultsGrapher.CustomNavigationToolbar2TkAgg}
\item {\texttt{View.Additional.ResultsGrapher.ErrorFrame}}, \pageref{View/Additional/ResultsGrapher/ErrorFrame:module-View.Additional.ResultsGrapher.ErrorFrame}
\item {\texttt{View.Additional.ResultsGrapher.GraphFrame}}, \pageref{View/Additional/ResultsGrapher/GraphFrame:module-View.Additional.ResultsGrapher.GraphFrame}
\item {\texttt{View.Additional.ResultsGrapher.ResultsGrapherToplevel}}, \pageref{View/Additional/ResultsGrapher/ResultsGrapherToplevel:module-View.Additional.ResultsGrapher.ResultsGrapherToplevel}
\item {\texttt{View.Additional.ResultsGrapher.SummaryFrame}}, \pageref{View/Additional/ResultsGrapher/SummaryFrame:module-View.Additional.ResultsGrapher.SummaryFrame}
\item {\texttt{View.Main.DecisionVariable.DecisionVariableFrame}}, \pageref{View/Main/DecisionVariable/DecisionVariableFrame:module-View.Main.DecisionVariable.DecisionVariableFrame}
\item {\texttt{View.Main.DecisionVariable.VariableFrame}}, \pageref{View/Main/DecisionVariable/VariableFrame:module-View.Main.DecisionVariable.VariableFrame}
\item {\texttt{View.Main.GeneticOperator.CrossoverFrame}}, \pageref{View/Main/GeneticOperator/TemplateGeneticOperator/CrossoverFrame:module-View.Main.GeneticOperator.CrossoverFrame}
\item {\texttt{View.Main.GeneticOperator.GeneticOperatorFrame}}, \pageref{View/Main/GeneticOperator/GeneticOperatorFrame:module-View.Main.GeneticOperator.GeneticOperatorFrame}
\item {\texttt{View.Main.GeneticOperator.MutationFrame}}, \pageref{View/Main/GeneticOperator/TemplateGeneticOperator/MutationFrame:module-View.Main.GeneticOperator.MutationFrame}
\item {\texttt{View.Main.GeneticOperator.SelectionFrame}}, \pageref{View/Main/GeneticOperator/TemplateGeneticOperator/SelectionFrame:module-View.Main.GeneticOperator.SelectionFrame}
\item {\texttt{View.Main.GeneticOperator.TemplateGeneticOperator.TemplateGeneticOperatorFrame}}, \pageref{View/Main/GeneticOperator/TemplateGeneticOperator/TemplateGeneticOperatorFrame:module-View.Main.GeneticOperator.TemplateGeneticOperator.TemplateGeneticOperatorFrame}
\item {\texttt{View.Main.Home.HomeFrame}}, \pageref{View/Main/Home/HomeFrame:module-View.Main.Home.HomeFrame}
\item {\texttt{View.Main.Home.IntroductionFrame}}, \pageref{View/Main/Home/IntroductionFrame:module-View.Main.Home.IntroductionFrame}
\item {\texttt{View.Main.MOEA.AlgorithmFrame}}, \pageref{View/Main/MOEA/AlgorithmFrame:module-View.Main.MOEA.AlgorithmFrame}
\item {\texttt{View.Main.MOEA.MOEAFrame}}, \pageref{View/Main/MOEA/MOEAFrame:module-View.Main.MOEA.MOEAFrame}
\item {\texttt{View.Main.MOEA.SharingFunctionFrame}}, \pageref{View/Main/MOEA/SharingFunctionFrame:module-View.Main.MOEA.SharingFunctionFrame}
\item {\texttt{View.Main.ObjectiveFunction.FunctionFrame}}, \pageref{View/Main/ObjectiveFunction/FunctionFrame:module-View.Main.ObjectiveFunction.FunctionFrame}
\item {\texttt{View.Main.ObjectiveFunction.ObjectiveFunctionFrame}}, \pageref{View/Main/ObjectiveFunction/ObjectiveFunctionFrame:module-View.Main.ObjectiveFunction.ObjectiveFunctionFrame}
\item {\texttt{View.Main.Population.FitnessFrame}}, \pageref{View/Main/Population/TemplatePopulation/FitnessFrame:module-View.Main.Population.FitnessFrame}
\item {\texttt{View.Main.Population.PopulaceFrame}}, \pageref{View/Main/Population/TemplatePopulation/PopulaceFrame:module-View.Main.Population.PopulaceFrame}
\item {\texttt{View.Main.Population.PopulationFrame}}, \pageref{View/Main/Population/PopulationFrame:module-View.Main.Population.PopulationFrame}
\item {\texttt{View.Main.Population.TemplatePopulation.TemplatePopulationFrame}}, \pageref{View/Main/Population/TemplatePopulation/TemplatePopulationFrame:module-View.Main.Population.TemplatePopulation.TemplatePopulationFrame}
\item {\texttt{View.MainWindow}}, \pageref{View/MainWindow:module-View.MainWindow}
\end{theindex}

\renewcommand{\indexname}{Index}
\printindex
\end{document}
