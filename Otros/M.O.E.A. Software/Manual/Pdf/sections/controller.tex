%Autor: Aarón Martín Castillo Medina.
%Asesora: Dra. Katya Rodríguez Vázquez
%Contacto: katya.rodriguez@iimas.unam.mx; amcm329@hotmail.com

%Este archivo contiene información relacionada con la capa Controlador
%(ó Controller), la cual representa tanto física como lógicamente a uno 
%de los componentes que conforman el producto de software y por tanto al 
%Manual Técnico. 


%Se indica que el documento es de tipo reporte bajo el paquete standalone.
\documentclass[class=report, crop=false]{standalone}

%Se cargan los paquetes relacionados con los subapéndices (elementos que
%conforman el Apéndice en su totalidad).
\usepackage{packages_used_section}

%Comienza el documento.
\begin{document}

\section{Controller (sección)}
%Se coloca el vínculo interno procedente de esta misma sección (a_4).
\label{sec:a_4}
Su función principal es la de establecer medidas de comunicación entre 
la Vista \textbf{(ó View)} y el Modelo \textbf{(ó Model)} de tal manera 
que el Controller \textbf{(ó Controlador)} recibe los datos recabados en la Vista 
y los transfiere al Modelo para que se puedan llevar a cabo las operaciones pertinentes 
y una vez concluidas dichas labores los resultados pasan por éste para llegar a la 
Vista y desde ahí graficarse apropiadamente.\break
De manera secundaria el Controlador ofrece métodos de saneamiento de los datos 
recabados en la Vista, con la finalidad de evitar al máximo disturbios indeseables 
en la sección Modelo y que éste opere con total eficiencia, además de alimentar a la 
Vista con las técnicas \textbf{(y sus respectivos parámetros)} disponibles en la 
sección Modelo para así permitirle al usuario operar con éstas de manera expedita.\break
Dicho almacén se encuentra en la sección \textbf{Controller/XML}, donde se deduce 
que las técnicas y características secundarias se encuentran plasmadas en archivos 
.xml.\medskip\break
El proyecto contempla métodos para operar con dichos archivos y el usuario entonces 
sólo tendrá que preocuparse por dar de alta la técnica pertinente en el archivo .xml 
adecuado \textbf{(además de implementarla en Modelo)} para que ésta sea reconocida en 
la sección Vista y se pueda hacer uso de ella.\medskip\break
A continuación se muestran los componentes principales de la sección Controller:

%******* Empieza clase *******
\subsection{Controller (clase)}
%Se coloca el vínculo interno procedente de esta misma sección (a_4_1).
\label{sec:a_4_1}

\begin{fulllineitems}

\begin{DUlineblock}{0em}
\item[] Proporciona la infraestructura adecuada para poder comunicar la 
sección Vista \textbf{(ó View)} con la sección Modelo \textbf{(ó Model)}, 
apoyándose de las clases XMLParser y Verifier.\medskip\break
El ciclo normal consiste en otorgar a la capa Vista \textbf{(ó View)} la 
información recabada en los archivos .xml con ayuda de la clase XMLParser 
con la  finalidad de notificar al usuario de todas las técnicas 
disponibles.\break
Una vez ejecutada la opción de iniciar un proceso genético por el 
usuario, se recaban los datos ingresados por el usuario, los cuales 
pasan por un proceso de verificación y transformación empleando para ello 
los métodos de la clase Verifier.\break
En caso de haber al menos una falla en alguno de los procedimientos mencionados 
anteriormente se regresa un mensaje de error, en otro caso se pasa la información 
respectiva a la capa Model para que pueda operar con ésta.\medskip\break
En cualquiera de los dos casos anteriores se regresa la información resultante a 
la Vista.
\end{DUlineblock}

\begin{quote}\begin{description}
\item[{Returns}] \leavevmode
Controller.Controller
\item[{Return type}] \leavevmode
Instance
\end{description}\end{quote}

%******* Empieza descripción *******
\begin{fulllineitems}

\pysiglinewithargsret{\sphinxbfcode{execute\_procedure}}{\emph{execution\_task\_count}, \emph{generations\_queue}, \emph{sanitized\_information}}{}
Realiza la ejecución de algún algoritmo M.O.E.A. \textbf{(Multi-Objective Evolutionary Algorithm)} 
y se encarga de obtener los resultados apropiadamente.

\begin{quote}\begin{description}
\item[{Parameters}] \leavevmode\begin{itemize}
\item \textbf{\texttt{execution\_task\_count}} (\emph{\texttt{Integer}}) -- Una característica numérica que identifica inequívocamente a esta función que será ejecutada de las demás, ya que el objetivo del proyecto es poder ejecutar varios de estos métodos de manera concurrente \textbf{(véase View/Additional/}\break\textbf{ResultsGrapher/ResultsGrapherToplevel.py)}.
\item \textbf{\texttt{generations\_queue}} (\emph{\texttt{Instance}}) -- Una instancia a una cola \textbf{(Queue)}, la cual servirá para escribir a esa estructura el número actual de generación por el que cursa el algoritmo. Esta acción es para fines de  concurrencia \textbf{(véase View/MainWindow.py)}.
\item \textbf{\texttt{sanitized\_information}} (\emph{\texttt{Dictionary}}) -- Los parámetros que ingresó el usuario debidamente verificados y saneados.
\end{itemize}
\item[{Returns}] \leavevmode
Un diccionario con información de los resultados de haber ejecutado el M.O.E.A. seleccionado por el usuario, la estructura del mismo puede verse en \textbf{Model/Community/}\break\textbf{Community.py}.
\item[{Return type}] \leavevmode
Dictionary
\end{description}\end{quote}

\end{fulllineitems}
%******* Termina descripción *******

%******* Empieza función *******
\begin{fulllineitems}

\pysiglinewithargsret{\sphinxbfcode{load\_features}}{}{}
Regresa los datos correspondientes \textbf{(debidamente verificados)}
a las \break técnicas disponibles para el usuario, los cuales se mostrarán en \break 
\textbf{View/MainWindow.py} \textbf{(véase View/Additional/MenuInternalOption/}\break
\textbf{InternalOptionTab/FeatureFrame.py)}.\break
Esta técnica tiene como base los símiles que se encuentran en 
\textbf{Controller/}\break\textbf{XMLParser.py} y \textbf{Controller/Verifier.py}.

\begin{quote}\begin{description}
\item[{Returns}] \leavevmode
Una estructura con los métodos disponibles para el usuario.
\item[{Return type}] \leavevmode
Dictionary
\end{description}\end{quote}

\end{fulllineitems}
%******* Termina función *******

%******* Empieza función *******
\begin{fulllineitems}

\pysiglinewithargsret{\sphinxbfcode{load\_mop\_examples}}{}{}
Obtiene los datos correspondientes \textbf{(previamente verificados)}
a los M.O.P.'s \textbf{(Multi-Objective Problems)} que se utilizan en 
\textbf{View/}\break\textbf{MainWindow.py} \textbf{(véase View/Additional/MenuInternalOption/}\break
\textbf{InternalOptionTab/MOPExampleFrame.py)}.\break
Esta técnica tiene como base las análogas que se encuentran en 
\textbf{Controller/XMLParser.py} y \textbf{Controller/Verifier.py}.

\begin{quote}\begin{description}
\item[{Returns}] \leavevmode
Una estructura con los M.O.P.'s disponibles para el usuario.
\item[{Return type}] \leavevmode
Dictionary
\end{description}\end{quote}

\end{fulllineitems}
%******* Termina función *******

%******* Empieza función *******
\begin{fulllineitems}

\pysiglinewithargsret{\sphinxbfcode{load\_python\_expressions}}{}{}
Obtiene los datos correspondientes \textbf{(previamente verificados)}
a las expresiones de Python, las cuales se usan para evaluar 
funciones objetivo más eficientemente \textbf{(véase View/Additional/MenuInternalOption/}\break
\textbf{InternalOptionTab/PythonExpressionFrame.py)}.\break
Esta función se apoya de las homónimas localizadas en \break
\textbf{Controller/XMLParser.py} y \textbf{Controller/Verifier.py}.

\begin{quote}\begin{description}
\item[{Returns}] \leavevmode
Una estructura con las expresiones de Python disponibles.
\item[{Return type}] \leavevmode
Dictionary
\end{description}\end{quote}

\end{fulllineitems}
%******* Termina función *******

%******* Empieza función *******
\begin{fulllineitems}

\pysiglinewithargsret{\sphinxbfcode{sanitize\_settings}}{\emph{general\_information}, \emph{features}}{}
Lleva a cabo la verificación y saneamiento de todos los datos
que ha ingresado el usuario en la sección View \textbf{(véase View/MainWindow)}.

\begin{quote}\begin{description}
\item[{Parameters}] \leavevmode\begin{itemize}
\item \textbf{\texttt{general\_information}} (\emph{\texttt{Dictionary}}) -- El conjunto de datos que el usuario ha ingresado o seleccionado.
\item \textbf{\texttt{features}} (\emph{\texttt{Dictionary}}) -- Una colección de todos los elementos con sus características disponibles para el usuario.
\end{itemize}
\item[{Returns}] \leavevmode
El diccionario que contiene todos los datos debidamente saneados.
\item[{Return type}] \leavevmode
Dictionary
\end{description}\end{quote}

\end{fulllineitems}
%******* Termina función *******

%******* Empieza función *******
\begin{fulllineitems}

\pysiglinewithargsret{\sphinxbfcode{save\_python\_expressions}}{\emph{data}}{}
Inserta las expresiones de Python que ha ingresado
el usuario en el archivo .xml correspondiente.

\begin{quote}\begin{description}
\item[{Parameters}] \leavevmode\begin{itemize}
\item \textbf{\texttt{data}} (\emph{\texttt{List}}) -- Un conjunto de las expresiones que ha ingresado el usuario. Cada elemento es a su vez una lista con dos elementos, el primero es la expresión original \textbf{(la que es comprensible por el usuario)}, mientras que la segunda es la expresión equivalente en Python.
\end{itemize}
\item[{Returns}] \leavevmode
Mensaje ``OK'' si la inserción ha sido exitosa, mientras que en caso de que haya habido un error entonces el mensaje es ``ERROR''.
\item[{Return type}] \leavevmode
String
\end{description}\end{quote}

\end{fulllineitems}
%******* Termina función *******

\end{fulllineitems}
%******* Termina clase *******

%******* Empieza clase *******
\subsection{XMLParser (clase)}
%Se coloca el vínculo interno procedente de esta misma sección (a_4_2).
\label{sec:a_4_2}
%******* Empieza descripción *******
\begin{fulllineitems}

\begin{DUlineblock}{0em}
\item[] Permite leer y escribir a archivos .xml 
\textbf{(los que se localizan en Controller/XML)}, los cuales tienen 
almacenados: 

\begin{itemize}
\item Los nombres de las técnicas con sus parámetros que se encuentran 
disponibles en la sección Model \textbf{(Features.xml)}.
\item La colección de palabras reservadas para poder emplear funciones y 
constantes auxiliares en las funciones objetivo \break\textbf{(PythonExpressions.xml)}.
\item El conjunto de M.O.P.'s \textbf{(Multi-Objective Problems, localizados en MOPExamples.xml)}.
\end{itemize}

Dichos archivos proporcionan la información necesaria a la interfaz 
gráfica \textbf{(véase View/MainWindow.py)}.
\end{DUlineblock}

\begin{quote}\begin{description}
\item[{Returns}] \leavevmode
Controller.XMLParser
\item[{Return type}] \leavevmode
Instance
\end{description}\end{quote}

%******* Termina descripción *******

%******* Empieza función *******
\begin{fulllineitems}

\pysiglinewithargsret{\sphinxbfcode{indent}}{\emph{element}, \emph{level=0}}{}
Indenta \textbf{(coloca espacios)} apropiadamente
en un documento .xml para poder distinguir más
rápidamente los distintos niveles que existen en éste.

\begin{quote}\begin{description}
\item[{Parameters}] \leavevmode\begin{itemize}
\item \textbf{\texttt{element}} (\emph{\texttt{String}}) -- Una línea del archivo .xml
\item \textbf{\texttt{level}} (\emph{\texttt{Integer}}) -- El nivel en el que se está haciendo el proceso de identado.
\end{itemize}
\end{description}\end{quote}

\end{fulllineitems}
%******* Termina función *******

%******* Empieza función *******
\begin{fulllineitems}

\pysiglinewithargsret{\sphinxbfcode{load\_xml\_features}}{\emph{features\_filename}}{}
Realiza la lectura del archivo que contenga el listado de técnicas 
y sus parámetros disponibles \textbf{(véase Model)} y 
carga todos los elementos que se encuentran en éste.

\begin{quote}\begin{description}
\item[{Parameters}] \leavevmode\begin{itemize}
\item \textbf{\texttt{features\_filename}} (\emph{\texttt{String}}) -- Nombre del archivo en cuestión.
\end{itemize}
\item[{Returns}] \leavevmode
Un diccionario que contiene todos los elementos del archivo.
\item[{Return type}] \leavevmode
Dictionary
\end{description}\end{quote}

\end{fulllineitems}
%******* Termina función *******

%******* Empieza función *******
\begin{fulllineitems}

\pysiglinewithargsret{\sphinxbfcode{load\_xml\_mop\_examples}}{\emph{features\_filename}}{}
Lleva a cabo la lectura del archivo que contenga el listado de M.O.P.'s 
\textbf{(Multi-Objective Problems)} y carga todos los elementos 
que se encuentran en éste.\break
Un M.O.P es una mezcla de variables de decisión y funciones 
objetivo ya estudiadas, se utilizan para reproducir su comportamiento 
y así garantizar, además de un correcto funcionamiento del programa, 
una opción rápida para probar las técnicas que se ofrecen.

\begin{quote}\begin{description}
\item[{Parameters}] \leavevmode\begin{itemize}
\item \textbf{\texttt{features\_filename}} (\emph{\texttt{String}}) -- Nombre del archivo en cuestión.
\end{itemize}
\item[{Returns}] \leavevmode
Un diccionario que contiene todos los elementos del archivo.
\item[{Return type}] \leavevmode
Dictionary
\end{description}\end{quote}

\end{fulllineitems}
%******* Termina función *******

%******* Empieza función *******
\begin{fulllineitems}

\pysiglinewithargsret{\sphinxbfcode{load\_xml\_python\_expressions}}{\emph{features\_filename}}{}
Realiza la lectura del archivo que contenga el listado de 
expresiones en Python y carga todos los elementos que se 
encuentran en éste.\break
La idea detrás de esto es que, al momento de crear y/o evaluar
funciones objetivo existen algunas palabras reservadas que no pueden 
ser usadas directamente como son las funciones trigonométricas, por eso
es que estas expresiones sirven como intermediarias entre el usuario y
el intérprete de Python.\break
En ocasiones a este tipo de expresiones, no sólo en el ámbito actual
sino en general, se les conoce como azúcar sintáctica.

\begin{quote}\begin{description}
\item[{Parameters}] \leavevmode\begin{itemize}
\item \textbf{\texttt{features\_filename}} (\emph{\texttt{String}}) -- Nombre del archivo en cuestión.
\end{itemize}
\item[{Returns}] \leavevmode
Un diccionario que contiene todos los elementos del archivo.
\item[{Return type}] \leavevmode
Dictionary
\end{description}\end{quote}

\end{fulllineitems}
%******* Termina función *******

%******* Empieza función *******
\begin{fulllineitems}
\pysiglinewithargsret{\sphinxbfcode{write\_xml\_python\_expressions}}{\emph{features\_filename}, \emph{features}}{}
Sobreescribe el archivo donde se encuentra el listado de expresiones
en Python.\break
El objetivo es que, una vez ejecutándose el programa y a través del menú
pertinente \textbf{(véase View/Additional/MenuInternalOption/}\break
\textbf{InternalOptionTab/PythonExpressionFrame.py)}, el usuario pueda 
añadir o eliminar las expresiones de Python que desee.\break
En ocasiones a este tipo de expresiones, no sólo en el ámbito actual 
sino en general, se les conoce como azúcar sintáctica.

\begin{quote}\begin{description}
\item[{Parameters}] \leavevmode\begin{itemize}
\item \textbf{\texttt{features\_filename}} (\emph{\texttt{String}}) -- Nombre del archivo en cuestión.
\item \textbf{\texttt{features}} (\emph{\texttt{List}}) -- La estructura que contiene las expresiones para ser guardadas
en el archivo .xml.
\end{itemize}
\end{description}\end{quote}

\end{fulllineitems}
%******* Termina función *******

\end{fulllineitems}
%******* Termina clase *******

%******* Empieza clase *******
\subsection{Verifier (clase)}
%Se coloca el vínculo interno procedente de esta misma sección (a_4_3).
\label{sec:a_4_3}
%******* Empieza descripción *******
\begin{fulllineitems}

\begin{DUlineblock}{0em}
\item[] Realiza principalmente la verificación y transformación adecuada 
de los datos que el usuario introduce en \textbf{View/MainWindow.py} para 
alimentar a los algoritmos que se encuentran en la sección Model 
\textbf{(más en concreto Model/MOEA)}.\break
En caso de haber algún error regresa los mensajes de error adecuados para 
que puedan ser interpretados por la capa Vista y precisar al usuario el 
acontecimiento ocurrido.\break
Por otra parte si se ha llevado a cabo la verificación correctamente se 
obtiene la información transformada apropiadamente.\medskip\break
De manera secundaria también ofrece métodos de verificación para la 
extracción y colocación de datos en los archivos .xml 
\textbf{(véase XMLParser y el directorio Controller/XML)}.
\end{DUlineblock}

\begin{quote}\begin{description}
\item[{Returns}] \leavevmode
Controller.Verifier
\item[{Return type}] \leavevmode
Instance
\end{description}\end{quote}

%******* Termina descripción *******

%******* Empieza función *******
\begin{fulllineitems}

\pysiglinewithargsret{\sphinxbfcode{cast\_parameter}}{\emph{parameter\_value}, \emph{parameter\_settings}}{}

\begin{notice}{note}{Note:}
Este método es privado.
\end{notice}

Verifica un parámetro asociado a alguna técnica.
Primero asegura que el parámetro se pueda evaluar correctamente,
posteriormente convierte apropiadamente el tipo de dato pasando 
de String a Boolean, Integer ó Float según corresponda.

\begin{quote}\begin{description}
\item[{Parameters}] \leavevmode\begin{itemize}
\item \textbf{\texttt{parameter\_value}} (\emph{\texttt{Float}}) -- El valor actual del parámetro.
\item \textbf{\texttt{parameter\_settings}} (\emph{\texttt{Dictionary}}) -- Un diccionario que contiene el tipo del parámetro \textbf{(bool, integer ó float)} y el rango que debe tomar tanto inferior como superior.
\end{itemize}
\item[{Returns}] \leavevmode
El valor saneado del parámetro si no hay fallas, pero si se encuentra algún desperfecto entonces se regresa un diccionario con la información detallada del desperfecto.
\item[{Return type}] \leavevmode
(Boolean, Integer, Float)/Dictionary
\end{description}\end{quote}

\end{fulllineitems}
%******* Termina función *******

%******* Empieza función *******
\begin{fulllineitems}
\pysiglinewithargsret{\sphinxbfcode{verify\_instance}}{\emph{name\_class}}{}

\begin{notice}{note}{Note:}
Este método es privado.
\end{notice}

Devuelve una instancia del nombre de la clase que se le pase
como parámetro.\break
Esta funcionalidad es útil sobre todo para la sección Model ya que
uno de los objetivos es proporcionar al usuario de una infraestructura 
rápida con técnicas fácilmente intercambiables sin necesidad de estar
importando explícitamente cada una de éstas.\break
De esta forma con base en una instancia se puede ejecutar 
cualquier método de manera dinámica.

\begin{quote}\begin{description}
\item[{Parameters}] \leavevmode\begin{itemize}
\item \textbf{\texttt{name\_class}} (\emph{\texttt{String}}) -- el nombre de la clase \textbf{(con su ruta)} de la cual se desea obtener una instancia.
\end{itemize}
\item[{Returns}] \leavevmode
Una instancia de la clase solicitada si el proceso es exitoso, en otro caso se obtiene un diccionario con los detalles de la falla.
\item[{Return type}] \leavevmode
Instance/Dictionary
\end{description}\end{quote}

\end{fulllineitems}
%******* Termina función *******

%******* Empieza función *******
\begin{fulllineitems}

\pysiglinewithargsret{\sphinxbfcode{get\_dynamic\_function}}{\emph{complete\_function}}{}
Obtiene una instancia de una función en un String 
de la forma \textbf{biblioteca.función}.
Este método se usa para convertir las expresiones de Python
en instancias que serán utilizadas al momento de evaluar  
funciones objetivo \textbf{(véase View/}\break
\textbf{Additional/MenuInternalOption/InternalOptionTab/}\break\textbf{PythonExpressionFrame.py, Controller/XML/PythonExpressions.xml)}

\begin{quote}\begin{description}
\item[{Parameters}] \leavevmode\begin{itemize}
\item \textbf{\texttt{complete\_function}} (\emph{\texttt{String}}) -- un String preferentemente de la forma \textbf{biblioteca.función} \textbf{(el punto debe ir incluido)}.
\end{itemize}
\item[{Returns}] \leavevmode
Una instancia de la función asociada a la biblioteca.
\item[{Return type}] \leavevmode
Instance
\end{description}\end{quote}

\end{fulllineitems}
%******* Termina función *******

%******* Empieza función *******
\begin{fulllineitems}

\pysiglinewithargsret{\sphinxbfcode{sanitize\_decision\_variables}}{\emph{vector\_variables}}{}
Verifica el conjunto de elementos de la categoría ``Decision Variables''
\textbf{(véase View/Main/DecisionVariable/DecisionVariableFrame.py)}, 
los cuales son precisamente las variables de decisión.\break
Primero se asegura que cada variable de decisión se pueda evaluar
correctamente, posteriormente convierte apropiadamente el tipo
de dato de sus respectivos rangos, pasando de String a Float.

\begin{quote}\begin{description}
\item[{Parameters}] \leavevmode\begin{itemize}
\item \textbf{\texttt{vector\_variables}} (\emph{\texttt{Dictionary}}) -- El vector que contiene las variables de decisión con sus correspondientes rangos.
\end{itemize}
\item[{Returns}] \leavevmode
Un diccionario con las variables de decisión y sus rangos debidamente saneados.
\item[{Return type}] \leavevmode
Dictionary
\end{description}\end{quote}

\end{fulllineitems}
%******* Termina función *******

%******* Empieza función *******
\begin{fulllineitems}

\pysiglinewithargsret{\sphinxbfcode{sanitize\_genetic\_operators\_settings}}{\emph{genetic\_operators\_settings},\emph{features}, \emph{vector\_variables}, \emph{number\_of\_decimals}}{}
Revisa la integridad y sanea los datos que ingresó el usuario 
concernientes a la sección ``Genetic Operators Settings''
\textbf{(véase View/Main}\break\textbf{/GeneticOperator/GeneticOperatorFrame.py)}.

\begin{quote}\begin{description}
\item[{Parameters}] \leavevmode\begin{itemize}
\item \textbf{\texttt{genetic\_operators\_settings}} (\emph{\texttt{Dictionary}}) -- El listado de técnicas y sus parámetros que el usuario
eligió en la sección correspondiente.
\item \textbf{\texttt{features}} (\emph{\texttt{Dictionary}}) -- El conjunto de las opciones disponibles para esta sección, así como sus características.
\item \textbf{\texttt{vector\_variables}} (\emph{\texttt{List}}) -- El vector de variables de decisión.
\item \textbf{\texttt{number\_of\_decimals}} (\emph{\texttt{Integer}}) -- El número de decimales que llevará cada solución en Population.
\end{itemize}

\item[{Returns}] \leavevmode
Un diccionario que, dependiendo de los resultados, puede contener
o información del error encontrado durante el procedimiento o 
todos los datos debidamente verificados y transformados.
\item[{Return type}] \leavevmode
Dictionary
\end{description}\end{quote}

\end{fulllineitems}
%******* Termina función *******

%******* Empieza función *******
\begin{fulllineitems}

\pysiglinewithargsret{\sphinxbfcode{sanitize\_moeas\_settings}}{\emph{moeas\_settings}, \emph{features}}{}
Verifica integridad y lleva a cabo el saneamiento de 
los datos que ingresó el usuario concernientes a la 
sección ``MOEAs Settings'' \textbf{(véase View/Main/}\break\textbf{MOEA/MOEAFrame.py)}.

\begin{quote}\begin{description}
\item[{Parameters}] \leavevmode\begin{itemize}
\item \textbf{\texttt{moeas\_settings}} (\emph{\texttt{Dictionary}}) -- El listado de técnicas y sus parámetros que el usuario eligió en la sección correspondiente.
\item \textbf{\texttt{features}} (\emph{\texttt{Dictionary}}) -- El conjunto de las opciones disponibles para esta sección, así como sus características.
\end{itemize}

\item[{Returns}] \leavevmode
Un diccionario que, dependiendo de los resultados, puede contener
o información del error encontrado durante el procedimiento o 
todos los datos debidamente verificados y transformados.
\item[{Return type}] \leavevmode
Dictionary
\end{description}\end{quote}

\end{fulllineitems}
%******* Termina función *******

%******* Empieza función *******
\begin{fulllineitems}

\pysiglinewithargsret{\sphinxbfcode{sanitize\_objective\_functions}}{\emph{vector\_variables},\emph{available\_expressions}, \emph{vector\_functions}}{}
Lleva a cabo el saneamiento de los elementos correspondientes a la categoría
``Objective Functions'' \textbf{(véase View/Main/ObjectiveFuncion/}\break
\textbf{ObjectiveFunctionFrame.py)}, los cuales son de hecho sólo 
las funciones objetivo.

\begin{quote}\begin{description}
\item[{Parameters}] \leavevmode\begin{itemize}
\item \textbf{\texttt{vector\_variables}} (\emph{\texttt{Dictionary}}) -- El vector de variables de decisión que el usuario ha ingresado.
\item \textbf{\texttt{available\_expressions}} (\emph{\texttt{Dictionary}}) -- Un listado con las expresiones de Python disponibles \textbf{(véase Controller/XML/PythonExpressions.xml,
View/Additional/MenuInternalOption/}\break\textbf{InternalOptionTab/PythonExpressionFrame.py)}.
\item \textbf{\texttt{vector\_functions}} (\emph{\texttt{Dictionary}}) -- El vector de funciones objetivo ingresados por el usuario.
\end{itemize}

\item[{Returns}] \leavevmode
Si el proceso fue exitoso, se obtiene el mismo vector\_functions,
en otro caso se regresa un diccionario con información detallada
sobre el errror encontrado.
\item[{Return type}] \leavevmode
List/Dictionary
\end{description}\end{quote}

\end{fulllineitems}
%******* Termina función *******

%******* Empieza función *******
\begin{fulllineitems}

\pysiglinewithargsret{\sphinxbfcode{sanitize\_population\_settings}}{\emph{population\_settings}, \emph{features}}{}
Verifica la consistencia y realiza el saneamiento de los datos
que ingresó el usuario concernientes a la sección ``Population Settings''
\textbf{(véase View/Main/}\break\textbf{Population/PopulationFrame.py)}.
\begin{quote}\begin{description}
\item[{Parameters}] \leavevmode\begin{itemize}
\item \textbf{\texttt{population\_settings}} (\emph{\texttt{Dictionary}}) -- El listado de técnicas y sus parámetros que el usuario eligió en la sección correspondiente.
\item \textbf{\texttt{features}} (\emph{\texttt{Dictionary}}) -- El conjunto de las opciones disponibles para esta sección, así como sus características.
\end{itemize}

\item[{Returns}] \leavevmode
Un diccionario que, dependiendo de los resultados, puede contener
o información del error encontrado durante el procedimiento o 
todos los datos debidamente verificados y transformados.
\item[{Return type}] \leavevmode
Dictionary
\end{description}\end{quote}

\end{fulllineitems}
%******* Termina función *******

%******* Empieza función *******
\begin{fulllineitems}

\pysiglinewithargsret{\sphinxbfcode{sanitize\_techniques}}{\emph{general\_information}, \emph{features}}{}
Realiza una verificación adicional concerniente al tipo \break
de representación de todas las técnicas seleccionadas.\break
Lo anterior significa que, usando la Representación Cromosómica \break
\textbf{(ó Chromosomal Representation, véase Model/}\break
\textbf{ChromosomalRepresentation, View/Main/Population/}\break
\textbf{PopulationFrame.py)}, todas las técnicas deben concordar 
con el mismo tipo de representación cromosómica que se haya 
seleccionado.\break
Para esta versión sólo están disponibles las representaciones binaria 
y de punto flotante.

\begin{quote}\begin{description}
\item[{Parameters}] \leavevmode\begin{itemize}
\item \textbf{\texttt{general\_information}} (\emph{\texttt{Dictionary}}) -- El listado de características disponibles \textbf{(véase XMLParser.py)}.
\item \textbf{\texttt{features}} (\emph{\texttt{Dictionary}}) -- La colección de datos que seleccionó el usuario en la sección View.
\end{itemize}

\item[{Returns}] \leavevmode
Un diccionario el cual, si la verificación es exitosa, es el mismo general\_informacion,
si por el contrario falla, entonces es un diccionario que contiene detalles del error.
\item[{Return type}] \leavevmode
Dictionary
\end{description}\end{quote}

\end{fulllineitems}
%******* Termina función *******

%******* Empieza función *******
\begin{fulllineitems}

\pysiglinewithargsret{\sphinxbfcode{verify\_load\_xml\_features}}{\emph{data}}{}
Verifica que los datos obtenidos de las técnicas disponibles que alimentan 
a la Ventana Principal \textbf{(véase View/MainWindow.py)} no tengan defectos.
Este método se apoya de \textbf{load\_xml\_features} localizado en 
\textbf{Controller/}\break\textbf{XMLParser.py}.

\begin{quote}\begin{description}
\item[{Parameters}] \leavevmode\begin{itemize}
\item \textbf{\texttt{data}} (\emph{\texttt{Dictionary}}) -- Los datos que son leídos por el método \textbf{load\_xml\_features} mencionado previamente.
\end{itemize}
\item[{Returns}] \leavevmode
Si los datos contienen algún error, un diccionario con las características de la falla, en otro caso los datos mismos.
\item[{Return type}] \leavevmode
Dictionary
\end{description}\end{quote}

\end{fulllineitems}
%******* Termina función *******

%******* Empieza función *******
\begin{fulllineitems}

\pysiglinewithargsret{\sphinxbfcode{verify\_load\_xml\_mop\_examples}}{\emph{data}}{}
Revisa que los M.O.P.'s \textbf{(Multi-Objective Problems)}
que se muestran en \textbf{View/MainWindow.py} a través de
\textbf{View/Additional/MenuInternalOption/}\break\textbf{InternalOptionTab/MOPExampleFrame.py} 
estén libres de errores.\break
Este método se apoya de \textbf{load\_mop\_examples} localizado en 
\textbf{Controller/}\break\textbf{XMLParser.py}.

\begin{quote}\begin{description}
\item[{Parameters}] \leavevmode\begin{itemize}
\item \textbf{\texttt{data}} (\emph{\texttt{Dictionary}}) -- Los datos que son leídos por el método \textbf{load\_mop\_examples} mencionado previamente.
\end{itemize}
\item[{Returns}] \leavevmode
Si los datos contienen algún error, un diccionario con las características de la falla, en otro caso los datos mismos.
\item[{Return type}] \leavevmode
Dictionary
\end{description}\end{quote}

\end{fulllineitems}
%******* Termina función *******

%******* Empieza función *******
\begin{fulllineitems}

\pysiglinewithargsret{\sphinxbfcode{verify\_load\_xml\_python\_expressions}}{\emph{data}}{}
Revisa que las expresiones de Python estén libres de errores.\break
Este método se apoya de \textbf{load\_python\_expressions} localizado en 
\textbf{Controller/XMLParser.py}.

\begin{quote}\begin{description}
\item[{Parameters}] \leavevmode\begin{itemize}
\item \textbf{\texttt{data}} (\emph{\texttt{Dictionary}}) -- Los datos que son leídos por el método \textbf{load\_python\_expressions} mencionado previamente.
\end{itemize}
\item[{Returns}] \leavevmode
Si los datos contienen algún error, un diccionario con las características de la falla, en otro caso los datos mismos.
\item[{Return type}] \leavevmode
Dictionary
\end{description}\end{quote}

\end{fulllineitems}
%******* Termina función *******

%******* Empieza función *******
\begin{fulllineitems}

\pysiglinewithargsret{\sphinxbfcode{verify\_write\_xml\_python\_expressions}}{\emph{data}}{}
Verifica la consistencia de los datos relativos a las
expresiones de Python antes de ser escritos en el 
archivo .xml correspondiente.

\begin{quote}\begin{description}
\item[{Parameters}] \leavevmode\begin{itemize}
\item \textbf{\texttt{data}} (\emph{\texttt{List}}) -- El conjunto de expresiones que serán almacenadas.
\end{itemize}
\item[{Returns}] \leavevmode
Un mensaje ``OK'' si la verificación fue satisfactoria, y ``ERROR'' en caso de aparecer alguna falla.
\item[{Return type}] \leavevmode
String
\end{description}\end{quote}

\end{fulllineitems}
%******* Termina función *******

\end{fulllineitems}
%******* Termina clase *******

%Termina el documento
\end{document}
